\chapter{考察}

本章では、第4章で示されたLLMによる対比因子ラベル自動生成の定量的・定性的な結果に基づき、提案手法の学術的な意義、観測された性能差の構造的な原因、および評価戦略の妥当性について、分析的かつ批評的な議論を展開する。本研究は、既存の説明可能AI(XAI)が抱える「個別性」「命名の課題」を解消し、ニューロン発火条件の集合的差分を自然言語化するという新規パラダイムの実現可能性を検証した[1-4]。BERTScoreで達成された約0.551という中適度な意味的関連性[5-7]が、この高難度タスクにおける実用的な命名の可能性を初めて示した点で意義深い。

\section{ドメイン適応性と汎用性:抽象概念への挑戦}

提案手法の重要な貢献の一つは、そのドメイン汎用性をレビューテキストから感情といった抽象概念まで拡張した点にある。本研究では、Steam(ゲームレビュー)、SemEval(レストランレビュー)、GoEmotions(感情分類)という、内容、文体、概念の粒度が大きく異なるデータセットを用いて、提案手法の有効性を検証した[9-11]。

SemEval-2014やSteamレビューといった製品/サービスドメインにおけるアスペクト(例:Food, Price, Gameplay, Technical)の命名は、テキスト中に具体的な語彙的証拠(例:ピザ、高すぎる、バグ)が存在するため、LLMが集合Aに特徴的な内容を比較的容易に抽出し、差分を要約できることが示唆された。これは、LLMが強力な文脈理解能力と生成能力を活用し、集合AとBの差分を推論し、ラベルを生成する「コントラスト生成器」として機能した結果である。

特に、GoEmotionsデータセットを用いた検証は、本手法の適用範囲を単なる製品特徴の記述から、より複雑な抽象概念の解釈へと広げる可能性を示した点で戦略的な意義を持つ。GoEmotionsは、Redditコメントから収集された28の細粒度感情カテゴリ(例:Admiration, Joy, Angerなど)を扱う[14, 15]。感情は、具体的な物理的実体を持たない高度に抽象的な概念であり、その発火条件をテキスト集合の差分から自然言語ラベルとして推論することは、挑戦的なタスクであると位置づけられる。GoEmotionsを用いた実験では、これらの抽象的な概念の命名精度は、具体的なアスペクトと比較して総じて低いBERTScoreを記録したものの、これは、提案手法が製品レビューのような語彙的に安定した領域に限定されず、人間の内在的な状態や複雑な感情的なニュアンスといった、より抽象度の高いニューロンの機能に対しても、命名モジュールとして機能する可能性(またはその限界)を示した。

この検証結果は、本手法が、非教師ありコンセプト抽出(UCBMなど)によって発見された潜在ベクトル[17]や、メカニスティック解釈(MI)におけるAttribution Graphs[18, 19]によってトレースされたニューロン回路に対し、人間が理解可能な意味的な名前を自動で付与し、MI分野のボトルネック解消に貢献する[4, 16, 20]という、学術的な貢献度を確固たるものにする[4, 5]。

\section{具体性と抽象性のギャップの原因分析}

第4章のサブ実験結果において、生成ラベルの品質は、具体的なアスペクト(例:SemEvalの「Food」「Price」)で優位性を示し、抽象的なアスペクト(例:SemEvalの「Atmosphere」、Steamの「Story」、GoEmotionsの「感情」)で性能が劣位となる系統的な傾向が観測された。この具体性/抽象性のギャップは、LLMのコンテキスト理解と、現在のFew-shotプロンプティング戦略の構造的な制約に起因すると考えられる。

\subsection{概念の性質と推論レベルの差異}

具体的なアスペクトが優位性を示す主な原因は、その概念が語彙的に安定しているためである。例えば、「Price」に関するニューロンが発火するテキスト集合Aは、「高すぎる」「割引」「安い」といった、明確かつ数量的な語彙的証拠を含むことが多く、集合B(非発火群)との差分は、これらの語彙の有無として容易に抽出される。このタスクは、LLMにとって本質的に要約またはテキストからの証拠抽出に近い。

一方、「Atmosphere」や「Story」、「Joy」といった抽象概念は、特定の単語やフレーズに集約されるとは限らない。例えば、「Atmosphere」は、照明、音楽、客層、内装といった広範な文脈や比喩的表現の総合的な印象から推論されるものであり、単一の文脈証拠から単純に抽出することはできない。したがって、抽象概念の命名は、LLMに対して、入力されたA群とB群のテキスト群から、単なる要約や抽出を超えた高度な推論(抽象化、帰納、比喩の解釈など)を必要とする。現在のLLM(GPT-4o-mini)は、具体的な語彙的証拠が乏しい集合差分に対して、この高度な推論タスクにおいてロバスト性を欠いたと考えられる。

\subsection{プロンプティング戦略の限界}

この性能劣位には、Few-shot ICL(In-Context Learning)を含む現在のプロンプト設計が影響している可能性が高い。本研究では、Few-shot ICLを、生成ラベルの出力形式の揺らぎや語彙の安定性を確保するための検証手段として位置づけ[24, 25]、1-shot設定が最も高いBERTScoreを示す傾向が観測された。これは、LLMが1つの適切な例からタスクの定義と出力スタイルを効率的に学習したことを示唆する。

しかし、この出力スタイルの矯正を目的としたFew-shotプロンプティングは、結果的にLLMに対し、抽象化や高度な推論を行うのではなく、入力テキストからの直接的な抽出にバイアスをかけてしまった可能性がある。抽象概念を正しく命名するためには、LLMになぜAはBと異なるのかという論理的な思考ステップを明示的に踏ませる必要がある。現在のプロンプトは、単に簡潔に回答するという出力制約を課しているため、LLMが複雑な推論チェーンを内部で構築することを阻害している可能性がある。この制約が、抽象概念を扱う際に不可欠な、広範な文脈の総合的な推論を放棄させ、性能低下を引き起こした構造的な原因であると分析される。

\section{評価指標の妥当性:BERTScoreの優位性の論証}

第4章の結果は、生成ラベルの評価において、BERTScoreが平均約0.551という中適度な意味的関連性を示したのに対し、BLEUスコアは全てのFew-shot設定で極めて低い値(平均約0.007)を示したことを報告している。この結果は、本研究のタスクにおける評価指標の選択と解釈の妥当性について、重要な議論を提供する。

\subsection{BLEUの評価指標としての不適格性}

観測されたBLEUスコアの極めて低い値は、LLMによる自動命名の失敗を意味するのではなく、語彙的重複を測るBLEUが本タスクの性質に根本的に不適合であることを明確に示している[6, 28, 29]。

本研究の正解ラベル(グラウンド・トゥルース)は、SemEval-2014データセットの既存のアスペクトラベル、例えば「food」「price」といった単一の単語または簡潔なフレーズである[28, 29]。これに対し、LLM(GPT-4o-mini)が集合差分を推論した結果生成するラベルは、Few-shot ICLによって出力スタイルが矯正されたとはいえ、「食べ物の品質に関する言及」「価格設定の側面」といった説明的な自然言語フレーズとなる傾向がある。

BLEU(Bilingual Evaluation Understudy)は、n-gram(単語の並び)の一致度、すなわち語彙的重複を測る指標であり[28, 30, 31]、生成文と参照文の間に語彙的な重複がなければ、意味が近くてもスコアは低くなる[29]。Reiter (2018)も指摘するように、BLEUは機械翻訳(MT)以外のタスク、特に意味的な妥当性や多様性を問うタスクの評価には不向きである[29, 32]。したがって、BLEUが極端に低値を示したという事実は、評価戦略を検討する上で、語彙的重複を基準としない意味内容ベースの指標の採用が不可欠であったことを定量的に裏付ける[26, 28, 29]。

\subsection{BERTScoreの妥当性}

本タスクの主要な評価指標として採用されたBERTScoreは、BERTなどの事前学習済み言語モデルによって得られる文脈化埋め込み表現のコサイン類似度に基づき、文脈的意味的な類似性を測る[28, 31, 33]。BERTScoreが約0.551という中適度な値を達成したことは、LLMが生成したラベルが、正解ラベルの単語とは語彙的に異なっていたとしても、その意味的な核(セマンティック・コア)を正確に捉えているという事実を客観的に示している[7, 26]。

この結果は、提案手法が目指すニューロンの発火条件の集合的な差分から、人間が理解できる意味的な核を抽出するという目的[6]が部分的に達成されたことを意味し、BERTScoreの採用が本タスクの性質(短いラベル生成、意味内容の忠実性)に適合しているという結論を裏付ける[21, 28]。BERTScoreの中程度のスコアは、完全な人間レベルの命名(スコア1.0)には遠いが[5]、非教師あり、コントラスティブという高難度のタスク設定[5, 34]においては、実用的な命名の可能性を示す妥当な結果であると評価されるべきである[5, 21]。

\section{今後の展望:抽象概念のロバスト性向上と評価の高度化}

現在の実験結果は、具体的なアスペクトの自動命名における本手法の有効性を立証した一方で、抽象的な概念の命名精度向上という重要な課題を残した。また、BERTScoreを補完し、より多角的な評価を行う必要性も明らかになった[29]。今後の研究は、これらの限界を克服するための具体的な方向性を定める。

\subsection{抽象概念命名精度の向上のためのCoT導入}

抽象概念の命名精度が劣位となった原因は、LLMが単純な抽出ではなく、高度な推論を必要とするタスクにおいて、現在のFew-shotプロンプティングが抽出にバイアスをかけている可能性が高いことにある。この限界を克服するため、今後の研究では、Chain-of-Thought (CoT)プロンプティングの導入が最も有効な改善策の一つとなる[4, 22, 23, 36]。

CoTを導入することで、LLMに「集合Aの主要なテーマ」「集合Bの主要なテーマ」「両者の差分となる論理的な推論(推論チェーン)」「結論として導かれる簡潔なラベル」といった思考のステップを明示的に記述させる。このプロセスをFew-shot例としてプロンプトに組み込むことにより、LLMが抽象化や帰納的な推論を内部で省略することなく実行するように誘導できる。これにより、広範な文脈の総合的な理解に基づく抽象概念の命名精度を向上させることが期待される。

また、CoTの検証と並行して、LLMの出力を特定の構造(例:JSON形式)に制約する構造化プロンプティングを導入することで、モデルが従うべき論理構造を標準化し、出力の安定性と信頼性を高めることも重要である。

\subsection{人間評価との整合性を高める自動評価指標の検討}

BLEUが不適であったことが判明し、BERTScoreが意味的妥当性の主要な指標として機能したものの、BERTScoreのみでは生成ラベルの流暢性や忠実性(Faithfulness)[37]、および語彙的多様性といった多面的な品質を完全に評価することは困難である[28, 31]。

したがって、今後の研究では、より人間評価との相関が高い学習ベースの評価指標を導入し、評価戦略の高度化を図る必要がある[29, 38]。具体的な候補としては、以下の指標が挙げられる。

\begin{enumerate}
  \item BLEURT:数千の人手評価データで事前学習されており、人間の判断をモデル化する[39-43]。BERTScoreを補完し、生成ラベルの自然さやパラフレーズへの頑健性を評価するために適している[39, 43]。
  \item BARTScore:生成テキストの評価をテキスト生成問題として定式化する指標であり[39, 44, 45]、特にコントラスト生成の忠実性(生成ラベルがソーステキストの差分にどれだけ基づいているか)の評価において、高い親和性を示す[29, 39]。
  \item MoverScore:文脈化埋め込みとEarth Mover's Distanceを用いて語彙非依存のセマンティック距離を測り[31, 39, 42, 45]、生成ラベル群の語彙多様性や、BLEUでは捉えられない意味的な近さを評価するために併用されるべきである[39, 42]。
\end{enumerate}

これらの学習ベース指標を導入することで、BERTScore約0.551という結果の妥当性を確立すると同時に、より複雑なXAIタスクに進むための定量的な評価基盤を構築する[29, 38]。最終的には、自動指標のバイアスを補正するため、妥当性、網羅性、可読性といった軸に基づく人間評価プロトコルを設計し、複数評価者による検証を行うことが不可欠となる[22, 38, 39, 46, 47]。

結論として、提案手法は、従来のXAIが満たせなかった集合差分説明というギャップを埋め、特に具体的な概念の自動命名において実用的な性能を示した[4, 5]。しかし、抽象概念へのロバスト性を高めるためには、CoTによる推論能力の強化が急務であり[22]、評価戦略においても、BLEUの限界を認識し、BLEURTやBARTScoreといった学習ベースの指標へと移行することが、本分野の進展に不可欠である[29, 39]。これは、自動命名が、ブラックボックスモデルの解釈可能性をスケーラブルに実現するための、重要な一歩であることを意味する。

たとえるなら、これまでのXAI手法が、個々の自動車のエンジン音から故障の原因を推測する(LIME/SHAPの個別性)作業だったとすれば、本研究は、特定の車種(発火群A)と他の車種(非発火群B)の設計図の差分を、自然言語で「スポーツ走行向きのサスペンションシステムに関する言及」のように自動でラベリングする試みである。具体的な部品の差分は捉えやすいが(具体性)、乗り心地やデザインといった抽象的な要素の差分(抽象性)を簡潔なラベルにするためには、より高度な工学的な推論(CoT)が必要となる、ということである。
