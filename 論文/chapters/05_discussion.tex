\chapter{総合考察}

本章では,第4章で示した実験結果に基づき,対比因子ラベル自動生成手法の妥当性,性能の決定要因,および限界について整理して考察する.

\section{本手法の限界と課題}

実験結果全体から,本手法にはいくつかの限界が明確になった.
第一に,対比因子ラベルは具体的アスペクトや語彙的一貫性が高いデータセットでは高い BERTScore を達成する一方で,Steam のようなノイズが大きくマルチアスペクトなドメインでは性能が低下する.
これは,単一ニューロンの発火パターンが,必ずしも既存の人手アスペクト体系と一対一に対応しないこと,そしてレビュー言語の多義性により「何についての差分か」が曖昧になることを反映している.

第二に,BLEU のような語彙一致指標は,本タスクでは一貫して有用性を示さなかった.
正解ラベルが短いアスペクト名であり,生成ラベルが説明的フレーズであるという構造的なギャップが存在する以上,n-gram 重複に基づく指標は性能評価に適さない.
これは,対比因子ラベリングの評価設計において,意味分布ベースおよび学習ベースの指標を組み合わせる必要があることを意味する.

第三に,Few-shot ICL やアスペクト説明文は出力の安定化と性能向上に寄与するが,その効果は一様ではない.
特に \textit{Visual} のように説明文が抽象的で他アスペクトと重なりやすい場合,説明文がかえってアスペクト固有性を弱めることがある.
したがって,Few-shot 例や説明文は,対象アスペクトの境界を明確に切り出すように慎重に設計する必要がある.

第四に,COCO 実験で明らかになったように,テキストキャプションに依存した対比因子生成は,キャプション側の頻度バイアスや記述の抜け漏れに影響を受け,視覚的現実とずれたラベルを生成するリスクを持つ.
画像との整合性評価を組み込むこと,および画像特徴量と生成ラベルとの対応関係を明示的に検証することは,精度向上と信頼性確保に向けて望ましい改善方策であると考えられる.

これらの限界は,提案手法が「ニューロン発火条件に対する対比因子ラベルを自動生成する」という目標に対して,どの条件で有効に機能し,どの条件で性能が制約されるかを具体的に示している.
特に,概念の具体性,データセット設計,Few-shot 設定,評価指標設計,テキストと視覚のギャップといった要因が,今後の改良において重点的に扱うべき課題であることが明らかになった.
総じて,本手法は概念が具体的でテキスト分布とラベル体系が整合的な条件では有効に機能する一方,ノイズの大きいマルチアスペクト環境やテキストと視覚表現の乖離が大きい条件では,対比因子ラベルが人間の直観とずれやすいという制約を持つ.


