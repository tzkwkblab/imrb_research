\chapter{総合考察}

本章では,第5章で示した実験結果に基づき,対比因子ラベル自動生成手法の妥当性,性能の決定要因,および限界について整理して考察する.

\section{本手法の限界と課題}

実験結果全体から,本手法にはいくつかの限界が明確になった.
第一に,対比因子ラベルは具体的アスペクトや語彙的一貫性が高いデータセットでは高い SBERT類似度 を達成する一方で,Steam のように話題の混在が大きくマルチアスペクトなドメインでは性能が低下する傾向が見られた(第5章\ref{tab:main_dataset_results},\ref{tab:concreteness_comparison}).
これは,レビュー言語の多義性や同時言及により集合差分が単一アスペクトに対応しにくいこと(第5章\ref{sec:concreteness}節),および本実験が,人手ラベル等にもとづいて構成した集合 $A/B$ を入力として評価しているという前提(第4章\ref{sec:experiment_setup}節)にも関係すると考えられる.

第二に,BLEU のような語彙一致指標は,本研究の設定では参考指標にとどまった.
正解ラベルが短いアスペクト名であり,生成ラベルが説明的フレーズとなりやすいというタスク特性上,n-gram 重複に基づく指標は低値に張り付きやすい(第4章\ref{sec:evaluation_metrics}節,第5章\ref{subsec:main_experiment_discussion}節).
したがって,評価は SBERT類似度 を主要指標としつつ,LLM 評価等の補助指標を併用する設計が妥当である.

第三に,Few-shot ICL やアスペクト説明文は出力の安定化に寄与し得る一方で,効果はアスペクトにより一様ではなかった.
例えば Steam の説明文比較では,\textit{Visual} において説明文あり条件で SBERT類似度 が低下しており(第5章\ref{tab:aspect_desc_aspect}),抽象的な説明が注意を拡散させる可能性が示唆された.
したがって,Few-shot 例や説明文は,対象アスペクトの境界を明確に切り出すように設計する必要がある.

第四に,COCO Retrieved Concepts 実験が示すように,テキストキャプションに依存した対比因子生成は,キャプション側の頻度バイアスや記述の欠落の影響を受け,画像の本質的特徴とずれたラベルを生成するリスクがある(第5章\ref{sec:coco_experiment}節,\ref{tab:coco_results}).
この点は,画像特徴量や物体ラベルなど視覚側の中間表現を併用し,テキストと視覚の両方に整合する形で対比因子を検証する枠組みが必要であることを示唆する(第5章\ref{sec:coco_experiment}節).

これらの限界は,提案手法が「ニューロン発火条件に対する対比因子ラベルを自動生成する」という目標に対して,どの条件で有効に機能し,どの条件で性能が制約されるかを具体的に示している.
特に,概念の具体性,データセット設計,Few-shot 設定,評価指標設計,テキストと視覚のギャップといった要因が,今後の改良において重点的に扱うべき課題であることが明らかになった.
総じて,本手法は概念が具体的でテキスト分布とラベル体系が整合的な条件では有効に機能する一方,ノイズの大きいマルチアスペクト環境やテキストと視覚表現の乖離が大きい条件では,対比因子ラベルが人間の直観とずれやすいという制約を持つ.

\section{有効条件と適用範囲}
第5章の結果から,本手法は「集合差分に安定した統計的パターンが現れる」条件で相対的に有効である.具体的には,SemEval のように語彙的一貫性が高く,人手ラベルとテキスト分布が整合的なベンチマークでは高い SBERT類似度 が得られた一方(第5章\ref{tab:main_dataset_results}),Steam のように長文・多義的で複数アスペクトが混在しやすい環境では難易度が上がる(第5章\ref{sec:concreteness}節).また,第5章の統計的検定では条件差が有意ではなく,ここで述べた差は平均値レベルの傾向に基づく整理である点に留意が必要である(第5章\ref{sec:statistical_analysis}節).

\section{改善方針と今後の課題}
改善の方向性としては,(i) グルーピング精度の向上(集合構成のノイズ低減),(ii) プロンプト設計の精密化(Few-shot 例・説明文の粒度と境界の明確化),(iii) 評価の多面的設計(SBERT類似度 を主要指標としつつ LLM 評価等を補助に用いる)を優先課題として位置づけるのが妥当である.特に視覚ドメインでは,キャプションだけに依存せず視覚特徴との整合性評価を組み込む必要がある(第5章\ref{sec:coco_experiment}節).

\section{研究の含意}
本研究は,集合差分 $(A,B)$ から自然言語ラベルを生成する枠組みを,複数ドメインで統一的に評価し,その有効条件と破綻条件を具体化した点に意義がある.とりわけ,ラベル付きベンチマークでの定量評価と,正解ラベルを持たない設定(COCO)での定性的検証を併置することで,「命名(naming)モジュール」としての適用可能性と限界を整理した.


