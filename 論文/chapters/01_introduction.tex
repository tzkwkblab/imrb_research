\chapter{序章}

% 序章の構成:
% 1. 研究背景・先行研究の課題(XAIの重要性、事後説明手法の限界、C-XAI等の課題)
% 2. 動機(概念の自然言語命名が高コストな人手作業に依存している点を緩和したい)
% 3. 目的(LLMを用いて概念命名作業の自動化を実現する)
% 4. 問題定義(対比因子生成タスクの定義)
% 5. 手法(LLMを用いた対比因子ラベル自動生成手法の提案)
% 6. 実験(アスペクト付きレビューを用いた評価実験)
% 7. 貢献(対比因子生成タスクの新規定義とLLMによる自動化の可能性検証)
% 8. 論文構成

% 研究背景:XAI の重要性と普及
近年の深層学習モデルは,医療,金融,自動運転といった社会的重要性の高いドメインにおいて優れた予測性能を達成している.しかし,その内部はブラックボックスであり、深層学習モデルの意思決定プロセスが人間にとって解釈不能であるという点は、モデルの信頼性・公平性の担保において大きな問題として存在し続けている.この問題の解決に向けて,モデルの判断根拠を説明する XAI(Explainable AI,説明可能 AI)の研究が活発化してきた.例えば,Mersha ら~\cite{mersha2024survey}は,XAI のニーズ・技術・応用を網羅的に整理し,医療,金融,自動運転などの高責任ドメインにおいて説明可能性への要求が急速に高まっていることを示している.また,Vilone と Longo~\cite{vilone2020systematic}は,2010 年以降の XAI 関連文献を体系的にレビューし,論文数の増加と応用分野の拡大を踏まえて,XAI が独立した研究トピックとして確立しつつあることを報告している.

% 意義:従来分野(事後説明手法)の取り組みと課題
従来,XAI の中心的な手法として広く普及したのは,LIME(Local Interpretable Model-agnostic Explanations)~\cite{ribeiro2016should}や SHAP(SHapley Additive exPlanations)~\cite{lundberg2017unified}に代表される事後説明(post-hoc explanation)手法である.これらは,ブラックボックスモデルの入出力関係をもとに個々の予測に対する特徴量寄与を可視化する.しかし,説明の忠実性やロバストネスが解析条件に大きく依存しうること,説明提供者にとって都合の良い説明だけを選択できてしまうこと,そしてピクセルやトークンといった低レベル特徴にとどまり集合レベルの意味構造を説明できないことなど,根本的な限界が指摘されている.これらの詳細は第二章で整理する.

% 先行研究の取り組みと課題(C-XAI・メカニスティック解釈・非教師ありコンセプト発見)
こうした事後説明手法の限界を背景に,近年は,モデルの内部表現やニューロンの振る舞いそのものを解析対象とするコンセプトベース XAI,非教師ありコンセプト発見,メカニスティック解釈といったアプローチが登場している.コンセプトベース XAI の代表例である TCAV(Kim et al., 2018)~\cite{kim2018interpretability}やコンセプトボトルネックモデル(CBM)は,概念レベルでの解釈を可能にする手法として提案されている.非教師ありコンセプト発見の研究では,UCBM~\cite{schrodi2024unsupervised} や CCE~\cite{stein2024towards} のように,人手による事前定義なしに潜在概念ベクトルを抽出する手法が提案されている.また,メカニスティック解釈の文脈では,Attribution Graphs~\cite{ameisen2025attribution} のように,LLMの内部計算プロセスをトレースし,特徴量間の相互作用をグラフとして可視化することで計算構造(回路)を発見する手法~\cite{anthropic2025biology}が進展している.しかし,これらの手法は,潜在概念ベクトルや内部回路といった候補を自動的に抽出し,近年では LLM や視覚言語モデルを用いた自動命名パイプラインも提案されているものの,命名結果の忠実性や,人間が扱う具体的タスクにおける有用性を外部ラベルに基づいて体系的に評価した研究はまだ限られている.この問題は,Kim らが TCAV や CBM の文脈で指摘した高コストな概念キュレーション(concept curation)の課題と本質的に同じであり,非教師ありコンセプト発見やメカニスティック解釈においても,概念や特徴量に対する自然言語での命名・説明をどのように信頼し,実タスクに接続するかという点は未解決の課題として残されている.

% 本研究の動機
本研究の動機は,このような課題,すなわちモデル内部で発見された概念に対する自然言語での命名・説明が高コストであり,その有効性の評価も容易ではないという点を緩和することである.この課題を解決するためには,各概念特徴に対して,その概念特徴が何を表しているのかを特定し,人間が理解できる形で出力する機構に加えて,その説明がタスク固有のラベルとどの程度整合的かを測る評価枠組みが必要である.近年,Network Dissection~\cite{bau2017networkdissection} や CLIP-Dissect~\cite{oikarinen2022clipdissect},Label-Free CBM~\cite{oikarinen2023labelfree},Discover-then-Name~\cite{rao2024discoverthenname},Sparse Autoencoder と LLM を組み合わせた自動解釈手法~\cite{openai2023neurons,bills2023automatedinterp} など,モデル内部の概念特徴に自然言語ラベルを自動付与する試みが相次いでいるが,多くは画像モデルや LLM の内部アクティベーションを対象とし,Simulation Score など内部挙動に基づく指標による評価が中心である.
%% ここまでやった。

%% ここから。以下の内容はスコープ絞りすぎてそらそうだろとなる。論述の流れを変更する。抽象的に「実務的なタスクでの検証が少ない。モデル解釈可能性を内部的に評価しているパターンが多い」あと動機その2として「外部出力だけをみてブラックボックス内の概念に命名する取り組みへの検証がされていない(ほんまか?確認必要)」
アスペクト付きレビューのような実務的テキストデータに対し,概念を持つサンプル群と持たないサンプル群の差分から,人間が付与したアスペクトラベルと対応する説明をどの程度自動生成できるかを,外部ゴールドラベルにもとづき体系的に評価した研究は十分ではない.


この課題に対処するための一つのアプローチとして,大規模言語モデル(LLM)を活用することが考えられる.LLM は強力な文脈理解能力を持ち,多数のテキストサンプルから共通する意味パターンを抽出し,それを自然言語で要約する能力に優れている.実際,LLM は近年,複数の文書群を比較してその差分を自然言語で記述するコントラスティブ要約やグループ差分要約といったタスクにおいて高い性能を示しており,例えば STRUM-LLM~\cite{saha2024strumllm}のように,LLM を用いた対比的要約手法が提案されている.しかし,これらの先行研究は主に一般的な文書比較タスクに焦点を当てており,モデル内部で発見された概念特徴に対する命名という,XAI の文脈における概念キュレーションタスクにおいて,LLM がどの程度有効であるかは検証されていない.

% 本論文の目的
これを踏まえて,本論文の目的は,概念特徴を持つテキスト集合と持たないテキスト集合の差分から自然言語ラベルを自動生成するという概念命名タスクにおいて,大規模言語モデル(LLM)の文脈理解能力がどの程度有効であるかを検証することである.

% 対比因子生成タスクの定義
本研究では,この課題に対処するために,概念抽出手法(UCBM や CCE など)によって抽出された概念特徴を持つテキスト集合と,持たないテキスト集合の差分から意味的な要因を自然言語ラベルとして抽出するタスクを扱う.本論文では,このテキスト集合の差分から抽出した自然言語ラベルを対比因子ラベルと呼び,このラベルを生成するタスクを対比因子生成と定義する.対比因子生成タスクは,概念特徴を持つサンプル群と持たないサンプル群の集合差分から,両者を分ける意味的な要因を自然言語で要約することを目的とする.

% 提案手法
対比因子生成タスクを解決するために,本研究では大規模言語モデル(本論文では主に GPT-4o-mini)を用いた対比因子ラベル自動生成手法を提案する(詳細は第三章で述べる).

% 実験設定
手法の効果を検証するため,あらかじめアスペクトが付与された商品レビュー等のデータセットを用い,例えば「価格」に言及するレビュー群とそうでないレビュー群といったテキスト集合を構成し,これらを LLM に入力して対比因子ラベルを自動生成させる.生成されたラベルが元のアスペクトラベルとどの程度意味的に一致しているかを定量的に検証することで,LLM による対比因子ラベル生成の性能を評価する.

% 本研究の貢献
本研究の貢献は次の三点に要約される.第一に,概念特徴を持つテキスト集合と持たないテキスト集合の差分から自然言語ラベルを生成する対比因子生成タスクを,アスペクト付きレビューにもとづくベンチマークとして定義する.第二に,UCBM や CCE など任意の概念抽出手法によって得られた「概念特徴を持つ/持たない」テキスト集合を入力とし,大規模言語モデルを汎用的な命名インターフェースとして利用するプロンプト設計を提示する.第三に,複数の LLM や Few-shot 設定に対して,生成されたラベルと既存アスペクトラベルとの意味的一致度を BERT スコアや BLEU スコアにより体系的に評価し,実務的条件における自動概念命名の達成度と限界を明らかにする.

% 論文全体の構成
本稿の構成は次のとおりである。第二章では従来の XAI 手法や非教師ありコンセプト発見,コントラスティブ要約などの関連研究を整理し,本研究の位置づけを明らかにする。第三章では,対比因子命名タスクを定式化し,大規模言語モデルを用いた対比因子ラベル自動生成手法と Few-shot 設計について述べる。第四章では,使用したデータセットと評価指標,LLM のパラメータ設定を説明し,各種実験の内容と結果を報告する。第五章では,得られた結果にもとづき,提案手法の妥当性や性能の決定要因,限界について考察する。第六章では,本研究の結論と今後の課題・展望をまとめる。