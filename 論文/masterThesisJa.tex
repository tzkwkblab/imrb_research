\documentclass[a4paper]{jreport}

\usepackage{masterThesisJa}
\usepackage{amsmath}
\usepackage{url}
\DeclareUnicodeCharacter{2248}{\ensuremath{\approx}}
\setcounter{tocdepth}{3}
\setcounter{page}{-1}

\maintitle{大規模言語モデルを用いた対比因子ラベル生成手法に関する研究}{A Study on Generating Contrastive Factor Labels for Explainable AI}

\publish{2025}{12}

\student{202421675}{清野 駿}{Seino Shun}

\jabst{
 大規模言語モデル(LLM)の実運用では、判断の妥当性を説明可能にすることが重要であるが、既存手法は未知データごとに人手で解釈ラベルを付与する負担が大きい。本研究は、二つのテキスト集合(A: ニューロン発火群, B: 非発火群)間の差異を自然言語で要約し、発火条件を表す「対比因子ラベル」を自動生成する枠組みの実現可能性を検証する。手法は、A/B の代表テキストを入力として、Few-shot 例示(0/1/3-shot)付きプロンプトで LLM に差分説明を生成させる。評価は SemEval レストランレビュー、Steam ゲームレビュー、GoEmotions、COCO Captions を含む多様なドメインのデータセットを用い、各グループ 300 件の条件で GPT-4o-mini を含む複数の LLM を用いて対比因子ラベルを生成し、その品質を BERT スコア、BLEU スコア、および LLM による意味的類似度評価を用いて多角的に測定した。結果として、全体平均で BERT ≈ 0.551、BLEU ≈ 0.007 を得て、意味レベルの一致は中程度だが語彙一致は低いことが示された。また Few-shot では 1-shot が最も良く、出力スタイルが「説明的叙述」から「一意に特定する語彙」へ矯正される傾向が確認された。アスペクト別には gameplay/food 等の語彙的に安定な概念で高く、recommended/suggestion 等の抽象概念で低い。以上より、本枠組みは一定の条件で有効であり、人手ラベリングの一部代替となる可能性がある一方、抽象概念の扱いと評価指標の高度化(人手・LLM 補助、非語彙的類似)が今後の課題である。
}

\eabst{
This study examines the feasibility of generating contrastive factor labels that explain neuron activation conditions by comparing two text groups (A: activated, B: non-activated). We prompt an LLM (GPT-4o-mini) with 0/1/3-shot examples to produce concise differences, and evaluate semantic and lexical alignment against gold aspect labels using BERT and BLEU on SemEval restaurant and Steam game reviews (300 samples per group). Results show moderate semantic similarity (BERT ≈ 0.551) but very low lexical overlap (BLEU ≈ 0.007), with 1-shot giving the best performance by shifting outputs toward uniquely identifying terms. Performance is higher for lexically stable aspects (e.g., gameplay, food) and lower for abstract ones (e.g., recommended, suggestion). The approach can partially reduce manual labeling, while handling abstract concepts and improving evaluation beyond lexical matches remain as future work.
}

\advisors{若林 啓}{伊藤 寛祥}

\begin{document}

\makecover

\addtolength{\textheight}{-5mm}
\setlength{\footskip}{15mm}
\fontsize{11pt}{15pt}\selectfont

\pagebreak\setcounter{page}{1}
\pagenumbering{roman}
\pagestyle{plain}
\tableofcontents
\listoffigures
\listoftables

\parindent=1zw
\pagebreak\setcounter{page}{1}
\pagenumbering{arabic}
\pagestyle{plain}

% 各章を読み込む
\chapter{序章}

近年の深層学習(DNN)モデルは、医療、金融、自動運転といった社会的重要性の高いドメインにおいて優れた予測性能を達成している。しかし、その意思決定プロセスが人間にとって解釈不能な「ブラックボックス」であるという根本的な問題は、モデルの信頼性や公平性の担保、そして法的・倫理的な規制要件の遵守を妨げる最大の障壁となっている~\cite{bordt2022posthoc,kim2018interpretability}。この課題に対応するため、モデルの判断根拠を事後的に説明する説明可能 AI(XAI)の研究が活発化してきた。

従来、XAI の中心的な手法として広く普及したのは、LIME(Local Interpretable Model-agnostic Explanations)や SHAP(SHapley Additive exPlanations)に代表される、\textbf{事後説明(Post-hoc Explanation)}手法である。これらの手法は、入力データ(例:画像ピクセル、テキストトークン)の摂動やゲーム理論に基づく貢献度(Shapley 値)の計算を通じて、個別インスタンスの予測に対する特徴量の寄与を可視化する~\cite{ribeiro2016should,lundberg2017unified}。

しかし、事後説明手法には、以下のような根本的な信頼性の問題が指摘されている。第一に、これらの説明は本質的な曖昧さ(high degree of ambiguity)を抱えており、特に共線性を持つ特徴量に対して不安定な結果を示すことがあり、同等の性能を持つ異なるモデルが全く異なる説明を生成する不安定性が問題となる。第二に、Bordt ら~\cite{bordt2022posthoc}は、事後説明アルゴリズムが、規制や倫理が求める「透明性」の目的を達成する上で\textbf{「不適切 (unsuitable)」}であると結論付けている。これは、説明提供者と受け手の利害が対立する「敵対的文脈 (adversarial contexts)」において、事後説明が容易に操作可能であり、モデル開発者が都合の良い説明を選択的に提示できてしまうためである~\cite{bordt2022posthoc}。

さらに重要な点として、LIME や SHAP は、あくまで個別インスタンスの局所的な特徴寄与に焦点を当てており、モデルが特定のデータ集合(例:特定のニューロンが発火するすべてのサンプル)に対して系統的に何を学習しているか、というグループレベルの差異や集合間のコントラストを自然言語で説明する能力を欠いている~\cite{ribeiro2016should,lundberg2017unified}。

事後説明の限界を克服するため、XAI の研究コミュニティは、低レベルの特徴量(ピクセル、トークン ID)から、人間が理解可能な\textbf{「コンセプト(概念)」}レベル(例:「縞模様」や「価格に関する言及」)で説明を提供するパラダイムへと移行している~\cite{kim2018interpretability}。コンセプトボトルネックモデル(CBM)や TCAV(Concept Activation Vectors)といったコンセプトベース XAI(C-XAI)は、モデルの内部状態を高レベルの概念に関連付けることで解釈可能性を向上させた~\cite{kim2018interpretability,schrodi2024unsupervised,stein2024towards}。

しかし、このアプローチは新たな、かつ重大なボトルネックを生み出した。それは、どの概念を監視・学習すべきかという概念の定義(ラベリング)が、依然として人間に依存している点である。このプロセスは Kim らによって\textbf{「高コストな概念キュレーション (expensive concept curation)」}と呼ばれ、特に専門知識が不可欠なドメイン(例:医療)において、C-XAI の導入を阻む最大の障壁となっていた~\cite{kim2018interpretability,schrodi2024unsupervised,stein2024towards}。既存のベンチマークである SemEval-2014 においても、アスペクト(概念)ラベルは人手によるアノテーションに依存している~\cite{pontiki-EtAl:2014:SemEval2014Task4}。

モデルの内部動作を回路図レベルで徹底的に理解しようとする最先端のメカニスティック解釈(MI)分野においても、解釈の「意味付け」は手作業に依存する。Anthropic による Attribution Graphs のような研究は、モデルの計算プロセスをトレースし、特徴量間の相互作用をグラフとして可視化することで、ニューロン間の回路を発見する。

しかし、Attribution Graphs などが発見する個々の「特徴量」やノードが、人間にとって何を意味するのかを自然言語で特定するプロセスは自動化されていない。研究者らは、解釈を容易にするために、関連する意味を持つ特徴量を手動でグループ化し\textbf{「スーパーノード」としてまとめているが、この手動ステップは「労働集約的 (labor-intensive)」}であり、情報の欠損を伴うことが指摘されている~\cite{anthropic2025biology,ameisen2025attribution}。

本研究の動機は、これら従来の XAI アプローチが直面する、「個別インスタンス中心の説明の限界」、「高コストな人手ラベリング依存」、そして\textbf{「発見された内部構造への命名の自動化の欠如」という 3 つの課題の交差点に存在する}点にある~\cite{bordt2022posthoc,kim2018interpretability,ameisen2025attribution}。信頼性が高く、低コストで、スケーラブルな解釈可能性を実現するためには、モデルの内部状態に基づき、かつ集合間の差分を自然言語で記述する自動命名手法が不可欠となる。

本研究が取り組む主要な課題は、非教師ありコンセプト抽出などで発見されたモデル内部の解釈可能な構造、すなわちニューロン発火条件の対比因子ラベルを、2 つのテキスト集合(発火群 A vs 非発火群 B)の差分からスケーラブルに自動生成することである~\cite{schrodi2024unsupervised,stein2024towards}。このタスクは、従来の XAI 研究における「概念の発見」と「命名」を統合する初の試みであると位置づけられる~\cite{schrodi2024unsupervised,stein2024towards}。

従来の非教師ありコンセプト発見手法(UCBM や CCE など)は、人間による事前定義なしにモデル内部から概念(潜在ベクトル)を抽出する点で大きな進歩を遂げた。しかし、これらの手法が発見するのは、あくまでも潜在的なベクトル表現であり、そのベクトルに人間が理解できる「自然言語の名前」を自動で付与する機能は欠如している。命名プロセスは、発火サンプルを見た研究者による手動分析に依存していた~\cite{schrodi2024unsupervised,stein2024towards}。

本研究は、この\textbf{「発見された概念の手動での意味付け」という C-XAI の最大のボトルネックを直接解消する}ことを目指す。具体的には、モデル内部の特定のニューロン(またはコンセプト)が強く発火したテキスト集合 A と、発火しなかったテキスト集合 B を入力とし、集合 A に特有で集合 B に見られない意味的差分を、自然言語ラベル $L$ として生成するタスクを定式化する。このタスクは、NLP における「コントラスティブ要約 (Contrastive Summarization)」~\cite{kardale2023contrastive,saha2024strumllm}の系譜に属するが、これを XAI ドメイン(モデル内部状態の解釈)に初めて適用する点で新規性を有する~\cite{kardale2023contrastive,saha2024strumllm}。

このタスクは、単一インスタンスに対する反事実的説明(例:CEM)とは異なり、集合間の一般的・代表的な差分を記述するものであり、ニューロンが何を計算しているかを集合レベルで説明することを可能にする。

本研究は、この高難度な「集合差分からの自然言語命名」タスクの解決策として、大規模言語モデル(LLM)の強力な文脈理解能力と生成能力を活用する。LLM(特に GPT-4o-mini)をコントラスト生成器として利用し、入力された A 群と B 群のテキスト群から、A 群に特徴的で B 群に欠如している意味的な側面を推論させ、簡潔なラベルを自動生成させる。

このアプローチは、人手によるアノテーションや複雑な事後分析のステップを排除し、非教師ありで抽出された概念(UCBM や Attribution Graphs が発見した特徴量)がもつ意味を、スケーラブルかつ自動で人間が理解できる「名前」として付与する命名モジュールとして機能する。この仮説の妥当性を、定量的な実験を通じて検証することが、本論文の主要な目標となる。

本研究は、LLM(GPT-4o-mini)を用いたコントラスティブ要約を核とする。手法の実現にあたっては、以下のステップを踏む。

\begin{enumerate}
  \item データ収集とグルーピング:モデル内部の特定のニューロンの発火度に基づき、発火が強いテキスト群 A と、発火が弱い(または非発火の)テキスト群 B を抽出する(グループサイズ \texttt{group\_size} はパラメータとして調整可能)。
  \item Few-shot ICL(In-Context Learning)によるプロンプト設計:LLM に A 群と B 群のテキストを入力として提示し、「A 群に特有の内容的・意味的差分」を抽出するよう指示する。Few-shot ICL は、本タスクにおける LLM の出力形式の揺らぎや語彙の安定性を確保するための\textbf{検証手段(サブ実験)}として導入する~\cite{saha2024strumllm}。
  \item 対比因子ラベルの生成:LLM に、集合差分を簡潔に要約した自然言語の「対比因子ラベル」を出力させる。
\end{enumerate}

本研究では、人手アノテーションされたアスペクトラベルを持つ SemEval-2014 Restaurant/Laptop データセットを評価ベンチマークとして使用し、LLM が生成したラベルと正解ラベルとの意味的類似性を評価した~\cite{pontiki-EtAl:2014:SemEval2014Task4}。

評価指標には、語彙的一致度を測る BLEU と、文脈的意味的類似度を測る BERTScore を採用した~\cite{papineni-etal-2002-bleu,zhang2019bertscore,reiter2018structured}。その結果、以下の重要な知見が得られた。

\begin{itemize}
  \item 意味的関連性の達成:生成されたラベルの品質を BERTScore で測定したところ、約 0.551 という中適度な意味的関連性を達成した~\cite{zhang2019bertscore}。この結果は、LLM がニューロンの発火群と非発火群という集合的な差分から、人間が理解できる意味的な核を抽出できること、すなわち「LLM の文脈理解能力を活用すれば、概念の発見と命名を同時に行うプロセスを実現できる」という主要な仮説(Main Hypothesis)が部分的に正しいことを示す。
  \item 語彙的一致の限界:一方、BLEU スコアは極めて低値(およそ 0.007)を示した~\cite{papineni-etal-2002-bleu,reiter2018structured}。これは、正解ラベルが「food」のような単一の単語であるのに対し、LLM が「食べ物の品質に関する言及」のような説明的なフレーズを生成するため、語彙的重複を測る BLEU が本タスクの性質に必ずしも適していないことを示唆している~\cite{reiter2018structured}。
  \item 概念の具体性による優位性:生成ラベルの品質は、「food」「price」のような語彙的に安定した具体的なアスペクトにおいて特に高い優位性を示すことが確認された~\cite{kim2018interpretability,schrodi2024unsupervised,stein2024towards}。対照的に、「story」「atmosphere」のような抽象的な概念の命名は、LLM が単純な要約ではなく高度な推論を必要とするため、性能が劣位となるという課題も特定された~\cite{kim2018interpretability,schrodi2024unsupervised,stein2024towards}。
\end{itemize}

本研究は、XAI と NLP の境界領域において、以下の点で重要な貢献を果たす。

第一に、新規タスクの提案と実現可能性の検証である。XAI における「概念の発見」(UCBM, CCE)と「概念の命名」が分離していた構造的な課題を認識し、LLM によるコントラスティブ要約によって、この二つのプロセスを統合する\textbf{「集合差分によるニューロン対比因子命名」}という新規タスクを提案し、その実現可能性を定量的に検証した~\cite{schrodi2024unsupervised,stein2024towards}。

第二に、XAI パラダイムのギャップ解消である。個別インスタンス中心の事後説明(LIME/SHAP)~\cite{ribeiro2016should,lundberg2017unified}の限界と、命名機能を持たない非教師ありコンセプト抽出(UCBM/CCE)~\cite{schrodi2024unsupervised,stein2024towards}のボトルネックを同時に解消する、スケーラブルなハイブリッドアプローチを提供する~\cite{schrodi2024unsupervised,stein2024towards}。

第三に、メカニスティック解釈の実用化への寄与である。Attribution Graphs~\cite{anthropic2025biology,ameisen2025attribution}などによって発見されたニューロン回路や特徴量に対し、人間が理解可能な意味的な名前を自動で付与する手段を提供し、手動ラベリングに依存していた MI 分野のボトルネック解消に貢献する~\cite{ameisen2025attribution}。

第四に、Few-shot ICL の応用と最適化に関する知見の提供である。Few-shot ICL を活用し、生成ラベルの品質とスタイルを矯正するアプローチを提案し、Few-shot 数の変動による影響を分析することで、LLM を用いた自動命名の安定性向上に向けた知見を提供する~\cite{saha2024strumllm}。


\chapter{関連研究}

本研究は,深層学習モデルの解釈可能性(XAI)において,ニューロンの発火条件を自然言語で自動命名するという,従来の手法が満たせていなかったギャップを埋めることを目的としている.
本章では,関連する先行研究を体系的に整理し,それぞれの限界を明確にすることで,本研究が提案する「集合差分による自動命名」の新規性を位置づける.
本研究は,(1) 個別インスタンス中心の事後説明手法,(2) 概念の発見に留まる非教師ありコンセプト抽出,(3) 一般的な NLP タスクに特化していたコントラスティブ要約,という三つの主要な研究領域の交差点に位置づけられる.

\section{従来の XAI 手法:個別インスタンス中心の解釈}

深層学習モデルのブラックボックス性に対処するため,初期の XAI 研究は主に,特定の予測に対する入力特徴量の寄与度を事後的に(Post-hoc)説明する手法に焦点を当ててきた.

\subsection{特徴量帰属手法の個別性}

最も広く採用されてきた手法として,LIME(Local Interpretable Model-agnostic Explanations, Ribeiro et al., 2016)~\cite{ribeiro2016should}や SHAP(SHapley Additive exPlanations, Lundberg \& Lee, 2017)~\cite{lundberg2017unified}が挙げられる.
LIME は,ターゲットとする予測の周囲でデータを摂動させ,局所的に解釈可能な代理モデル(サロゲートモデル)を構築することで,個々のインスタンスの予測根拠を可視化する.
SHAP は,ゲーム理論に基づく Shapley 値を用いて,特徴量の貢献度を統一的な枠組みで定量化する.

これらの特徴量帰属手法は,その性質上,個別インスタンスの局所的な説明に特化している.
LIME や SHAP の出力は,画像における「重要なピクセル」やテキストにおける「重要な単語」といった低レベルな特徴量の寄与スコアであり,モデルが特定のデータ集合に対して系統的に何を学習しているかを,人間が理解できる自然言語の概念(コンセプト)として説明する能力を欠いている.
本研究の目的は,特定のニューロンが発火する集合 A と発火しない集合 B の間の意味的な差分を抽出することであり,個別予測の寄与度を問う従来の XAI 手法では,このグループレベルの差異の説明というタスクを満たすことができない.

さらに,事後説明手法は信頼性の問題も指摘されている.
特に Bordt et al. (2022)~\cite{bordt2022posthoc}は,事後説明が本質的な曖昧さを持ち,敵対的な文脈において容易に操作可能であるため,法的・倫理的な透明性の目的を達成する上で「不適切 (unsuitable)」であると結論付けている.
本研究がモデルの内部状態(ニューロン発火)に直接着目する動機の一つは,この事後説明の曖昧さを回避し,より忠実な説明を提供することにある.

\subsection{反事実的説明の限界}

反事実的説明(Counterfactual Explanation)もまた,対比的(Contrastive)な要素を持つ XAI 手法として注目を集めてきた~\cite{wachter2017counterfactual}.
CEM(Contrastive Explanations Method, Dhurandhar et al., 2018)や Wachter et al. (2017) の手法は,「もし入力 $x$ の一部が $x'$ に変わったら,予測 $y$ はどう変わるか」という,単一事例に対する最小限の入力変更を特定する.
これにより,ユーザーは「この予測を覆すために何をすべきか」という行動可能な洞察を得る.

反事実的説明は,その定義上,単一インスタンスの局所的な反転に限定される.
本研究が対象とするのは,特定のニューロンの発火パターンが,データ集合全体でどのような意味的な特性を持つかを記述すること,すなわち集合間の一般的・代表的な差分を自然言語で要約するタスクである.
反事実的説明は,この集合レベルでのコントラスト記述という目標を達成できない.

\section{非教師ありコンセプト発見と命名の課題}

事後説明の限界を受け,XAI 研究の焦点は,低レベルな特徴量から,人間が理解できる高レベルの「コンセプト(概念)」に基づいた説明へと移行した~\cite{kim2018interpretability}.
コンセプトベース XAI(C-XAI)の代表例である TCAV(Kim et al., 2018)は,概念活性化ベクトルを用いてモデルが特定の概念にどれだけ敏感であるかを定量化する.

\subsection{高コストな人手ラベリング依存}

従来の C-XAI は,解釈の基礎となる概念の定義(ラベリング)を人間に依存しているという構造的な問題を抱えてきた~\cite{kim2018interpretability}.
このプロセスは,同じく Kim らによって「高コストな概念キュレーション (expensive concept curation)」と呼ばれ,特に医療や科学といった専門知識が必要なドメインでの XAI 導入の最大の障壁となっている.
SemEval-2014 のような既存の ABSA ベンチマークでさえ,アスペクト(概念)ラベルは人手によるアノテーションに依存している~\cite{pontiki-EtAl:2014:SemEval2014Task4}.

\subsection{非教師あり概念抽出の進歩と「命名」の欠如}

近年,この人手依存の課題を克服するため,非教師ありでモデル内部から概念を抽出する手法が大きく進展している~\cite{schrodi2024unsupervised,stein2024towards}.
Unsupervised CBM (UCBM, Schrodi et al., 2024) は,人間による事前定義なしに,モデルの内部表現から概念(潜在ベクトル)を自動抽出することを可能にした.
また,Compositional Concept Extraction (CCE, Stein et al., 2024)~\cite{stein2024towards}は,より構成的な概念の表現を抽出する.

UCBM や CCE の進歩にもかかわらず,これらの手法が「発見」するのは,依然として潜在的なベクトル表現である.
そのベクトルが人間にとって何を意味するのか(例:「コンセプト X」が「縞模様」を意味すること)という自然言語での「命名(ラベリング)」機能は,これらの研究には含まれていない.
命名プロセスは,発見された概念が強く発火するサンプルを研究者が手動で分析し,ラベルを付与するという,非教師あり C-XAI の「最後のワンマイル問題」として残されていた~\cite{schrodi2024unsupervised}.

本研究は,この手動命名というボトルネックに対し,UCBM や CCE が発見した概念(に対応する発火/非発火サンプル群)を LLM に入力し,その意味を直接「対比因子ラベル」として生成するスケーラブルな自動命名モジュールとして機能する.

\subsection{メカニスティック解釈における手動作業}

最先端のメカニスティック解釈(Mechanistic Interpretability, MI)分野においても,同様の課題が存在する.
Anthropic による Attribution Graphs(2025)~\cite{anthropic2025biology}は,LLM(例:Claude 3.5 Haiku)の内部計算プロセスをトレースし,特徴量間の相互作用をグラフとして可視化することで回路を発見する.
しかし,同研究においても,グラフのノードとして発見される個々の「特徴量」の意味付け,すなわちノードが具体的に何を検出しているかを自然言語で特定するプロセスは自動化されていない.

同論文の著者ら自身も,解釈を容易にするために,関連する意味を持つ特徴量を手動で「スーパーノード」としてグループ化しており,この手動ステップが「労働集約的 (labor-intensive)」であり,情報の欠損を引き起こすと認めている~\cite{ameisen2025attribution}.
本研究の提案手法は,この MI によって発見された特徴量や回路に対し,その発火条件の差分に基づき,人間が理解できる対比因子ラベルを自動で付与する手段を提供し,MI の実用化と統合的 XAI の実現に寄与する.

\section{LLM を用いた自動ラベル生成とコントラスティブ要約}

大規模言語モデル(LLM)は,その強力な文脈理解能力と自然言語生成能力により,様々なラベリングや要約タスクに応用されている.

\subsection{LLM による命名と正規化}

LLM は,テキストクラスタリングの結果に対して,クラスタ内のサンプル群の共通するテーマを要約し,クラスタラベルを自動生成するために活用されている~\cite{wang2024llmcluster}.
これにより,Embedding ベースのクラスタリングを Few-shot 学習を用いた分類タスクに変換する新たなパラダイムが提案されている.
また,アスペクトベースド感情分析(ABSA)の領域においても,LLM は Few-shot プロンプトを用いてアスペクトの抽出や,ノイズの多い合成ラベルの正規化(Normalization)に利用されている~\cite{luo2024chatabsa}.

\subsection{本研究のタスクとの乖離}

LLM による従来の命名研究の多くは,単一のデータ集合(クラスタ)内の共通点を記述することに焦点を当てている~\cite{wang2024llmcluster}.
一方,本研究は,ニューロンの発火群 A と非発火群 B の間の差分,すなわちコントラスト(対比)を記述することを目的としている.
また,ABSA における LLM の適用例(例:ChatABSA)は,主に教師ありの環境下で特定のアスペクトを抽出するタスクであり,非教師ありのコントラスティブな設定で未知の概念を自動発見・命名する本研究とはタスク設定が異なる~\cite{luo2024chatabsa}.

\subsection{コントラスティブ要約との関係}

本研究のタスク設定に概念的に最も近いのは,自然言語処理(NLP)分野における「コントラスティブ要約 (Contrastive Summarization)」または「グループ差分要約 (Group-Difference Summarization)」の系譜である~\cite{kardale2023contrastive}.
このタスクは,2 つ以上の文書群を比較し,そのうちの 1 つの集合に特有で,かつ関連性の高い差異をハイライトする要約を生成することを目的とする.

LLM を用いた先行研究として,STRUM-LLM (Saha et al., 2024)~\cite{saha2024strumllm}が挙げられる.
これは,2 つの比較対象(例:製品 A vs 製品 B)の差分を,LLM を用いた多段階パイプラインで属性付きの構造化要約として生成する.

また,Luss et al. (2024)~\cite{luss2024cell}は,CELL(Contrastive Explanations for Large Language Models)を提案し,LLM の出力に対する対比的な説明(なぜその出力が選ばれ,他の出力が選ばれなかったのか)を生成するという点で,本研究と問題意識を共有している.
一方で,CELL はあくまで個々の出力インスタンスに対する説明に焦点を当てており,本研究が対象とするような,ニューロン発火群 $A$ と非発火群 $B$ の\textbf{集合レベルの一般的・代表的な差分}を自然言語ラベルとして要約するタスクとは目的と対象が異なる.

\subsection{本研究の新規性}

STRUM-LLM は,Web 検索を含む複雑なパイプラインを構築し,一般的な製品比較タスクに特化している.
対照的に,本研究は,この「コントラスティブ要約」のフレームワークを XAI ドメイン(モデル内部状態)に初めて適用する点に新規性がある.
本研究は,ニューロンの発火群と非発火群という,より抽象的なテキスト集合の差分抽出に対し,Few-shot プロンプティングという,より簡潔なアプローチで,スケーラブルな自然言語ラベリングの実現可能性を検証するものである.
これは,LLM 命名の知見と,コントラスティブ要約のタスク定義を,XAI の文脈で融合した新規な交差点に位置づけられる.

本研究は,個別インスタンスの説明に留まる LIME/SHAP の限界,命名の課題を残す UCBM/CCE のボトルネック,そして一般 NLP タスクに留まっていたコントラスティブ要約の知見を統合することで,スケーラブルな内部状態の説明を実現する新たな手法を提案するものである.


\chapter{提案手法}

本章では,大規模言語モデル(LLM)を用いて,ニューラルネットワークの特定の内部状態(ニューロンの発火条件)に対応するテキスト集合間の意味的差異を,自然言語の「対比因子ラベル」として自動生成する手法を提案する.本手法は,既存のコンセプトベース XAI(C-XAI)が抱える「人手による命名依存」という最大のボトルネックを解消することを目的としており,ニューロン発火条件の集合的差分を自然言語で記述するタスク設定に特化している.本論文では,集合 $A$ と $B$ の差分を表現するこの自然言語ラベルを一貫して「対比因子ラベル」と呼ぶ.

\section{対比因子命名タスクの定式化}

本研究が提案する対比因子命名タスクは,従来の XAI 手法(個別インスタンスの説明や概念分類)とは根本的に異なるタスクを定式化する~\cite{ribeiro2016should,kim2018interpretability}.

\subsection{タスクの定義}

モデル内部の特定のニューロン $N$(または非教師あり抽出されたコンセプト $C$)に着目する.本研究では,モデル内部で抽出されるこれらの潜在特徴を総称して「ニューロン(またはコンセプト)」と呼び,一方で SemEval などのデータセット側の人手ラベルを「アスペクト」と呼ぶ.学習データセット $\mathcal{D}$ から,ニューロン $N$ が強く活性化する入力テキストの集合を $A$(発火群),そうではない,あるいは対象とするアスペクトを含まない入力テキストの集合を $B$(非発火群)とする.以下,数学的定式化では「集合 $A$」「集合 $B$」と表記し,プロンプトや実験手順の記述では「グループA」「グループB」と表記する.

このタスクの目的は,集合 $A$ に含まれるテキスト群の意味的・内容的な特徴のうち,集合 $B$ には含まれない差分を抽出・推論し,それを簡潔な自然言語ラベル $L$ として自動的に生成することである.

ここで,
\[
 A = \{ x \in \mathcal{D} \mid \text{activation}(x, N) > \tau_A \}, \quad
 B = \{ x \in \mathcal{D} \mid \text{activation}(x, N) < \tau_B \}
\]
と定義し,$\tau_A > \tau_B$ とする.実験章では,この定義をアスペクトラベルや類似度スコアに基づくグループ A/B の抽出として具体化する.$L$ は,この差分を要約する自然言語フレーズ(例:「価格に関する言及」)である.

このタスク全体は,集合入力からラベルへの写像
\[
 (A, B) \mapsto L
\]
として表される.以下では,この写像を便宜的に \texttt{textContrastiveNaming} と呼ぶ.

\subsection{集合差分によるコントラストの必要性}

上記の定式化は,以下の点で従来の XAI と一線を画す.

\begin{enumerate}
 \item \textbf{個別性からの脱却}:
 従来の LIME や SHAP は,単一のインスタンス $x$ に対して,その予測 $y$ に寄与した特徴量(ピクセル,トークン)を可視化するものである.一方,本タスクは,集合 $A$ と $B$ の間の一般的・代表的な意味的コントラストを抽出することに焦点を当てる.
 \item \textbf{潜在表現への命名}:
 Unsupervised CBM(UCBM)や Compositional Concept Extraction(CCE)といった非教師あり概念抽出手法は,概念を潜在ベクトル $v$ として発見するが,そのベクトルに自然言語ラベル $L$ を付与する機能を持たない.本定式化 $\text{textContrastiveNaming}(A,B) \to L$ は,UCBM などが抽出した潜在概念(ニューロンの発火条件)に対し,自動で人間が理解できる「名前」を付与する命名モジュールとして機能する.
\end{enumerate}

このように,対比因子命名タスクは,ニューロンが何を計算しているかというメカニズムの理解を,集合間の意味的なコントラストとして自然言語化する,新しい XAI のタスクである.

\section{LLM によるコントラスティブ要約の実行}

本研究では,定式化された対比因子命名タスクを,大規模言語モデル(LLM)の強力な文脈理解能力と自然言語生成能力を活用して解決する.LLM(本実験では GPT-4o-mini を想定)を,集合 $A$ と $B$ の差分を推論し,ラベルを生成するコントラスト生成器として利用する.

\subsection{処理フロー}

提案手法は,概念図(図~\ref{fig:proposal-overview})に対応する,以下の 3 段階の処理フローを持つ.図が参照できない場合でも,以下の記述のみで流れが理解できるように構成する.

\begin{enumerate}
 \item \textbf{データ抽出とグルーピング(Activation Extraction and Grouping)}:
 解釈対象のニューラルネットワーク $M$ と,特定のニューロン $N$ を選択する.評価データセット $\mathcal{D}$ の各テキスト $x$ を $M$ に入力し,ニューロン $N$ の活性化値 $\text{activation}(x, N)$ を測定する.測定された活性化値に基づき,ハイパーパラメータ \texttt{group\_size} を用いて,活性化値が最も高いテキスト群 $A$ と,活性化値が最も低い(またはランダムな)テキスト群 $B$ を抽出する.すなわち,
 \[
   A = \{ x_1, \dots, x_k \}, \quad
   B = \{ x'_1, \dots, x'_k \}, \quad
   k = \texttt{group\_size}
 \]
 とする.\texttt{group\_size} は,プロンプトのコンテキスト長制限やニューロン活性化のスパース性に応じて決定される.第4章のグループサイズ比較実験では,group\_size を 50〜300 の範囲で変化させても BERTScore の変動は小さく,性能に与える影響は限定的であることが確認された.そのため,本研究ではコンテキスト長と計算量のトレードオフを踏まえ,デフォルト値として \texttt{group\_size} = 100 を採用する.
 \item \textbf{プロンプト設計と差分推論(Prompt Engineering and Contrast Inference)}:
 抽出されたテキスト集合 $A$ と $B$ の内容を,LLM の入力プロンプトに組み込む.プロンプトは,LLM に対し,単なる要約ではなく「グループ $A$ に特徴的でグループ $B$ には見られない主要な違いを特定し,簡潔に回答する」というコントラスティブな推論タスクとして明確に指示する.
 特に,実験で用いるプロンプトは第4章の実験設定と整合するよう,次の要素から構成される.
 \begin{itemize}
  \item \textbf{タスク説明}:まず「2つのデータグループを比較して,グループAに特徴的でグループBには見られない表現パターンや内容の特徴を特定してください」といった指示文を提示する.
  \item \textbf{Few-shot 例(オプション)}:Few-shot 設定が 0 より大きい場合,\texttt{examples\_section} に「【例題$N$】グループA: [...] グループB: [...] 回答: [正解ラベル]」という形式の例を 1 件または 3 件挿入する.
  \item \textbf{集合 $A$ のテキスト群}:【グループA】の見出しの下に,各テキストを「- [テキスト内容]」の形式で最大 \texttt{group\_size} 件列挙する.
  \item \textbf{集合 $B$ のテキスト群}:【グループB】の見出しの下に,同様の形式でテキストを列挙する.
  \item \textbf{出力制約}:プロンプト末尾で「英語で5-10単語程度で,グループAに特徴的でグループBには見られない主要な違いを簡潔に回答してください」と指示し,短い対比因子ラベルの生成を求める.
 \end{itemize}
 LLM は,このプロンプトを入力として受け取り,集合 $A$ のテキスト群には頻出するが,集合 $B$ のテキスト群には見られない語彙,文脈,意味的構造を推論する.この推論能力が,人間による手動分析なしに意味的な差分抽出を可能にする鍵となる.
 \item \textbf{対比因子ラベルの生成(Contrastive Factor Label Generation)}:
 LLM は,推論結果に基づき,集合 $A$ の意味的特性を簡潔に表現した自然言語ラベル $L$ を生成する.例えば,集合 $A$ が「価格が高すぎる」といったレビューを含み,集合 $B$ がレビューを含むが価格には言及しない場合,生成されるラベル $L$ は「価格に関する言及」となる.
\end{enumerate}

生成された対比因子ラベル $L$ の妥当性は,第4章で述べるように,主に BERTScore を主要指標,BLEU を参考指標,さらに LLM による意味的類似度評価を補助指標として定量的に評価する.本章で述べたコントラスティブ要約の枠組みは,第2章で整理した既存のコントラスティブ要約研究(例:STRUM-LLM)を,モデル内部状態の解釈という XAI 文脈に適用したものに相当する.

\begin{figure}[htbp]
  \centering
  % \includegraphics[width=0.8\linewidth]{fig_proposal_overview.png}
  \caption{提案手法の概念図(データ抽出・プロンプト設計・ラベル生成の 3 段階)}
  \label{fig:proposal-overview}
\end{figure}

\section{Few-shot ICL による出力安定性の検証}

本研究では,大規模言語モデルの活用において,Few-shot インコンテキスト・ラーニング(ICL)を,生成される対比因子ラベルの出力形式と安定性を制御するための補助的な手段として導入する.同時に,第4章で示すように,Few-shot 設定とくに 1-shot は BERTScore の向上にも寄与しており,単なる検証手段にとどまらず性能面でも重要な役割を果たす.

\subsection{ICL の役割:安定性とスタイルの確保}

LLM の Few-shot ICL は,プロンプト内にタスクの入力と出力の例(デモンストレーション)を含めることで,モデルがタスクの形式や文体,語彙の傾向を模倣する特性を持つ.この特性を活用し,本研究では,生成される「対比因子ラベル」の出力形式の揺らぎや語彙の安定性を確保するために ICL を用いる.

特に,SemEval-2014 データセットの正解ラベル(例:「food」「price」)は単語や簡潔なフレーズであることが多いため,LLM が出力するラベルをこの正解ラベルのスタイルに近づけ,比較可能性を高めるために ICL を検証する.

\subsection{Few-shot バリエーションの検証}

実験では,Few-shot ICL のバリエーションとして,以下の設定を定量的に検証する.

\begin{enumerate}
 \item 0-shot(Zero-shot):プロンプトにデモンストレーションを含めず,指示文のみで LLM にラベル生成を要求する.これにより,LLM がタスク定義のみに基づいてどれだけ命名できるか,そのベースライン性能を測定する.
 \item 1-shot(One-shot):プロンプトに,正解ラベルが既知の $(A_{\mathrm{ex}}, B_{\mathrm{ex}}, L_{\mathrm{ex}})$ のペアを 1 組含める.
 \item 3-shot(Three-shot):プロンプトに,正解ラベルが既知の $(A_{\mathrm{ex}}, B_{\mathrm{ex}}, L_{\mathrm{ex}})$ のペアを 3 組含める.
\end{enumerate}
Few-shot 例 $(A_{\mathrm{ex}}, B_{\mathrm{ex}}, L_{\mathrm{ex}})$ は,各データセットのアスペクトラベルに基づき,正解ラベルが明確で代表性の高い組み合わせから選定する.例の形式は「【例題$N$】グループA: [...] グループB: [...] 回答: [正解ラベル]」とし,タスク説明の直後に \texttt{examples\_section} として挿入する.

\subsection{検証の目的}

Few-shot ICL は,LLM の生成が入力例のスタイルまで強く模倣する特性を持つため,生成ラベルが「food」という単語(語彙的安定性が高い)にどの程度収束するか,あるいは「食べ物の品質に関する言及」といった説明的なフレーズ(語彙的多様性が高い)になるか,という出力スタイルの影響を定量的に分析する目的で実施される.

Few-shot 実験の詳細な結果は第4章で報告するが,Steam データセットにおける実験では,1-shot 設定が平均 BERTScore 0.6530 と最も高い意味的関連性を示した.この結果は,1-shot の Few-shot 例がラベルのスタイルと意味的妥当性の両面でバランスが良いことを示唆しており,本研究におけるデフォルト設定の根拠となる.

\chapter{評価実験}

\section{実験の目的と概要}
本章では,提案手法の評価実験の目的,使用データセット,実験設定,評価指標を述べる.実験結果と考察は第5章で報告する.
本実験の目的は,外部ラベルにもとづいて定義した 2 つのテキスト集合 $A/B$ の差分から,LLM(GPT-4o-mini 等)が対比因子ラベルをどの程度生成できるか,また生成ラベルが人手アノテーションによる正解ラベル(またはその説明文)とどの程度意味的に一致するかを,定量的に評価することである.

提案手法のドメイン汎用性を検証するために,レビュー,感情分類,画像キャプションという多様なドメインに属する 4 種類のデータセットを用いた.
これらのデータセットは,それぞれが持つ正解アスペクトラベルを,LLM が生成する対比因子ラベルの妥当性を評価するためのグラウンド・トゥルースとして使用した.
使用したデータセットの概要を表~\ref{tab:dataset_overview}に示す.
\begin{table}[htbp]
  \centering
  \caption{使用データセットの概要}
  \label{tab:dataset_overview}
  \begin{tabular}{lll}
    \toprule
    データセット & ドメイン & 特徴 \\
    \midrule
    SemEval-2014 ABSA & レビュー & 具体的アスペクト(Food,Service等) \\
    GoEmotions & 感情分類 & 抽象的概念(28感情カテゴリ) \\
    Steam Review Aspect Dataset & レビュー & ドメイン固有で抽象度の高いアスペクト(Gameplay等) \\
    COCO Retrieved Concepts & 画像キャプション & 視覚的概念記述 \\
    \bottomrule
  \end{tabular}
\end{table}

本実験は,以下の 6 つの実験カテゴリで構成された.
データセット別比較では,SemEval-2014 ABSA,GoEmotions,Steam Review Aspect Dataset の 3 データセットを用いて,提案手法の基本性能を評価した.
Few-shot 設定による性能比較実験では,0-shot,1-shot,3-shot の 3 つの設定を比較した.
グループサイズの影響分析実験では,group\_size を 50,100,150,200,300 の 5 段階で変化させた.
モデル比較実験では,GPT-4o-mini と GPT-5.1 の 2 モデルを比較した.
アスペクト説明文の効果検証実験では,アスペクト説明文の有無による性能差を検証した.
COCO Retrieved Concepts 実験では,正解ラベルがない画像キャプションデータセットに対する対比因子生成を検証した.
各実験カテゴリの概要を表~\ref{tab:experiment_overview}に示す.
\begin{table}[htbp]
  \centering
  \caption{実験カテゴリの概要}
  \label{tab:experiment_overview}
  \begin{tabular}{ll}
    \toprule
    実験カテゴリ & 目的・検証内容 \\
    \midrule
    データセット別比較 & 基本性能評価 3データセット \\
    Few-shot実験 & 0/1/3-shot設定の比較 \\
    グループサイズ比較 & group\_size(50--300)の影響分析 \\
    モデル比較 & GPT-4o-mini vs GPT-5.1 \\
    アスペクト説明文比較 & 説明文の有無による性能差 \\
    COCO実験 & 正解ラベルなしデータセットでの検証 \\
    \bottomrule
  \end{tabular}
\end{table}

評価の観点として,以下の 3 つの指標を用いた(各指標の詳細な定義は\ref{subsec:evaluation_metrics}節を参照).
有効性の評価には,Sentence-BERT(SBERT)文埋め込みのコサイン類似度を 0.0〜1.0 に正規化した値(以降,SBERT類似度)を主要指標として使用し,生成ラベルと正解ラベルの意味的類似度を測定した.
汎用性の評価には,複数のドメイン(レビュー,感情分類,画像キャプション)と複数のアスペクトタイプ(具体的アスペクト,抽象的概念)に対する性能を測定した.
性能の評価には,BLEU スコアを参考指標として使用し,語彙レベルの一致度を補助的に確認した.
また,LLM による意味的類似度評価を補助指標として用い,GPT-4o-mini による 5 段階評価を実施した.
用いた評価指標の概要を表~\ref{tab:evaluation_metrics_overview}に示す.
\begin{table}[htbp]
  \centering
  \caption{評価指標の概要}
  \label{tab:evaluation_metrics_overview}
  \begin{tabular}{lll}
    \toprule
    評価指標 & 役割 & 位置づけ \\
    \midrule
    SBERT類似度 & 意味的類似度測定 & 主要指標 \\
    BLEU & 語彙レベル一致度確認 & 参考指標 \\
    LLM評価 & 意味的類似度評価(5段階) & 補助指標 \\
    \bottomrule
  \end{tabular}
\end{table}

\section{データセット}
\label{sec:dataset}

\subsection{データセットの選定理由}
本実験では,ドメインの多様性を確保するため,以下の 4 種類のデータセットを選定した.
SemEval-2014 ABSA は,アスペクトベース感情分析の標準的なベンチマークとして広く使用されており,正解ラベルが人手でアノテーションされている~\cite{pontiki-EtAl:2014:SemEval2014Task4}.
GoEmotions は,感情という抽象的な概念を扱うデータセットであり,具体的な物理的実体を持たない概念の命名精度を測るために選定した~\cite{demszky2020goemotions}.
Steam Review Aspect Dataset は,ゲームという特定の製品ドメインに特化したデータセットであり,専門性の高いテキスト集合間の意味的差分を抽出する能力を検証するために選定した~\cite{srec:steam-review-aspect-dataset}.
Retrieved Concepts(COCO Captions)は,画像キャプションデータセットであり,視覚的概念記述の生成能力を検証するために選定した~\cite{lin2014microsoft}.

ドメインの多様性として,レビューテキスト(SemEval-2014,Steam),感情分類テキスト(GoEmotions),画像キャプション(COCO)という異なるテキストタイプを網羅した.
また,具体的なアスペクト(Food,Price,Gameplay,Technical)と抽象的な概念(Atmosphere,Story,感情カテゴリ)を含むデータセットを扱うことで,概念の性質やデータセット特性と性能の関係を考察できるようにした.

以降の考察でデータセット特性を定量的に参照するため,テキスト統計の概要を表~\ref{tab:text_stats_overview}に示す(語数は小文字化後の空白分割による).
\begin{table}[htbp]
  \centering
  \caption{テキスト統計の概要(train/dev/test 合算)}
  \label{tab:text_stats_overview}
  \begin{tabular}{lrrrrrr}
    \toprule
    データセット & 件数 & 平均語数 & 中央語数 & 最大語数 & 平均ラベル数 & 複数ラベル率 \\
    \midrule
    GoEmotions & 54,263 & 12.83 & 12 & 33 & 1.18 & 16.25\% \\
    Steam Review Aspect & 1,100 & 244.70 & 156 & 1,473 & 3.27 & 90.27\% \\
    SemEval-2014 restaurant14 & 4,728 & 19.41 & 18 & 79 & 1.00 & 0.00\% \\
    SemEval-2014 laptop14 & 2,966 & 20.75 & 18 & 83 & 1.00 & 0.00\% \\
    \bottomrule
  \end{tabular}
\end{table}

\subsection{Steam Review Aspect Dataset}
Steam Review Aspect Dataset は,Steam ゲームレビューから収集されたテキストデータであり,特定のゲームアスペクトに関する言及を含むデータセットである~\cite{srec:steam-review-aspect-dataset}.
本データセットは英語の Steam ゲームレビュー 1,100 件(学習用 900 件,テスト用 200 件)で構成され,データ収集は SRec データベースのスナップショットに基づき行われた.レビューを特徴づける 8 種類のアスペクトが人手でアノテーションされており,Recommended(推奨),Story(物語),Gameplay(ゲームプレイ),Visual(視覚),Audio(聴覚),Technical(技術),Price(価格),Suggestion(提案・要望)からなる.ゲームという特定の製品ドメインに特化しており,特に \texttt{Gameplay} や \texttt{Technical} といったゲーム固有のメカニクスや技術的側面に関するアスペクトを含む.テストセットにおけるアスペクト件数の例として,Gameplay が 154 件,Recommended が 148 件,Story が 89 件である.
レビュー長は平均 244.70 語(中央値 156 語,最大 1,473 語)であり,語彙タイプ数は 29,056 語である.また,1 サンプルあたりの平均ラベル数は 3.27 で,複数アスペクトを併記するレビューは 90.27\% を占める.

実験での使用方法として,各アスペクトについて,そのアスペクトを含むテキスト群をグループ A,含まないテキスト群をグループ B として抽出した.
分割タイプは \texttt{aspect\_vs\_others} を用い,特定のアスペクトが含まれるテキストと含まれないテキストを比較した.
グループ A と B の抽出は,各アスペクトのラベルに基づいて行い,group\_size パラメータに応じてサンプル数を調整した.
データセット別比較では,group\_size = 100 を用い,各グループから最大 100 件のテキストを抽出した.

% データセット別比較で使用したアスペクト一覧は第\ref{sec:experiment_setup}節でまとめて述べる.

\subsection{SemEval-2014 ABSA(Restaurants)}
SemEval-2014 ABSA(Restaurants)は,アスペクトベース感情分析(ABSA)の標準的なベンチマークとして広く使用されるレストランレビューのデータセットである~\cite{pontiki-EtAl:2014:SemEval2014Task4}.
レストランレビューのテキストを含み,各文に対してアスペクト(観点)とそれに対する感情極性が人手でアノテーションされている.本研究では主に Food(食べ物),Service(サービス),Price(価格),Atmosphere(雰囲気)の 4 種類のアスペクトを用いる.Food や Price は具体的な名詞や数値に関連する言及が中心となる一方,Atmosphere は広範な文脈や比喩的表現からの推論を必要とし,データセット特性と命名性能の関係を考察する際の比較対象となる.
本データセットは 1 文あたり 1 アスペクトを前提としており,restaurant14 は平均 19.41 語,laptop14 は平均 20.75 語と短文である(表~\ref{tab:text_stats_overview}).

実験での使用方法として,SemEval-2014 データセットにおいて,Restaurant ドメインから Food と Service,Laptop ドメインから Battery と Screen の 4 アスペクトを用いた.
各アスペクトについて,そのアスペクトを含むテキスト群をグループ A,含まないテキスト群をグループ B として抽出し,split\_type = \texttt{aspect\_vs\_others} で分割した.

使用したアスペクトの選定理由として,Food は具体的な名詞や数値に関連する言及が中心となる具体的なアスペクトとして,Service,Battery,Screen は製品の属性に関する具体的なアスペクトとして選定した.
これにより,具体的なアスペクトにおける命名性能を評価できるようにした.

\subsection{GoEmotions}
GoEmotions は,細粒度感情分類タスクのために Reddit コメントから収集されたデータセットである~\cite{demszky2020goemotions}.
Demszky らによって構築された総レコード数 63,812 件から成るマルチラベル形式のデータセットであり,28 の感情カテゴリ(27 感情 + neutral)でラベル付けされている.主要なカテゴリには Joy(喜び),Anger(怒り),Admiration(称賛),Neutral(中立)などが含まれる.感情という抽象的概念をテキスト集合差分から推論する必要があるため,具体的な物理的実体を持たない概念の命名精度を検証する目的で用いた.実験では任意の感情アスペクト(例:joy)を指定し,その感情を含むテキスト群 $A$ とその他のアスペクトを含むテキスト群 $B$ を比較する設定とした.
平均 12.83 語の短文が中心であり,複数感情ラベルを併記するサンプルは 16.25\% である(表~\ref{tab:text_stats_overview}).

実験での使用方法として,GoEmotions データセットから全 28 感情カテゴリを用いた.
各感情カテゴリについて,その感情を含むテキスト群をグループ A,含まないテキスト群をグループ B として抽出し,split\_type = \texttt{aspect\_vs\_others} で分割した.

28 感情カテゴリの選定理由として,感情は物理的な実体を持たない高度に抽象的な概念であり,具体的なアスペクトと比較して命名精度が低下する傾向があるかを検証するために,全カテゴリを対象とした.
これにより,抽象的な概念における命名性能を包括的に評価できるようにした.

\subsection{Retrieved Concepts(COCO Captions)}
Retrieved Concepts(COCO Captions)は,視覚的概念記述の生成能力を検証するために,画像キャプションデータセット COCO に基づいて構築されたデータセットである~\cite{lin2014microsoft}.
MS-COCO 2017 train split の画像に対して,実験協力者のFarnoosh Javar によって訓練された非教師ありコンセプト発見モデルから得られた 300 個の潜在コンセプト埋め込みと,CLIP(ViT-B/32)による画像埋め込みとのコサイン類似度を計算し,各コンセプトについて類似度が高い画像 Top-100 と低い画像 Bottom-100 を取得している.この潜在コンセプト埋め込みは,UCBM~\cite{schrodi2024unsupervised} に代表される辞書学習型の非教師ありコンセプト発見手法と同様に,既存モデルの中間表現から自動的に概念ベクトルを抽出するタイプのモデルにより学習されているが,本論文では上流モデルの詳細には立ち入らず,得られた concept embeddings と,それに対して CLIP 類似度にもとづき取得された Top/Bottom 画像およびその COCO 由来キャプションのみを利用する.
各画像には COCO 由来の人手キャプションが 5 つ付与されており,本研究ではこれらのキャプションのみを集合 $A$, $B$ の要素として利用する.
非教師ありに学習された 300 の潜在コンセプト(concept\_0 ~ concept\_299)に対し,CLIP 類似度にもとづき取得された Top-100/Bottom-100 画像とそのキャプションからなる.各コンセプトに対して人手の正解ラベル(アスペクト名)は与えられておらず,潜在コンセプトとその Top/Bottom 例のみが提供されるため,正解アスペクト名が与えられない設定において,集合差分から対比因子ラベルを生成する挙動を補助的に検証できる.

実験では,300 コンセプトのうち concept\_0,concept\_1,concept\_2,concept\_10,concept\_50 の 5 コンセプトを用いた.
各コンセプトについて,潜在コンセプト埋め込みと画像埋め込みとの CLIP(ViT-B/32)コサイン類似度に基づき,類似度が高い順に Top-100,低い順に Bottom-100 の画像を選び,それらに付与されたキャプションをグループ A(Top 側),グループ B(Bottom 側)として用いた.
分割タイプは \texttt{aspect\_vs\_bottom100} を用いた.

正解ラベルが存在しないため,SBERT類似度 と BLEU スコアは参考値として記録するにとどめ,主に生成された対比因子と対応する画像群との整合性に基づく定性的評価を行った.

\section{実験設定}
\label{sec:experiment_setup}

\subsection{実験パイプラインの概要}
本実験では,コントラスティブ要約に基づく対比因子ラベル生成器として,GPT-4o-mini を含む複数の大規模言語モデルを用いた.
提案手法は,統一されたパイプラインとして構築され,その目的は,特定の概念に対応するテキスト集合 $A$(グループA)と,そうでないテキスト集合 $B$(グループB)の差分から,意味的な対比因子ラベル $L$ を LLM に生成させることである.本章の実験では,第\ref{sec:dataset}節で述べたとおり,SemEval,GoEmotions,Steam では人手アスペクトラベルにもとづき,COCO Retrieved Concepts では Top/Bottom 構造にもとづき,グループA/Bを構成する.

\begin{itemize}
  \item \textbf{タスク定式化}: 第3章で定義したとおり,集合 $A$ と $B$ の差分から自然言語ラベル $L$ を生成する写像 $\text{textContrastiveNaming}(A,B) \to L$ を用いる.本章の実験では,各データセットごとに定義されたグルーピング規則にもとづき構成したグループA/Bを入力として用いる.
  \item \textbf{グルーピング}: 各データセットごとに定義された規則に従い,ハイパーパラメータ $group\_size$ を用いて集合 $A$ と $B$ を抽出する.SemEval,GoEmotions,Steam ではアスペクトラベルの有無にもとづき,COCO Retrieved Concepts では各潜在コンセプトに対する Top-100/Bottom-100 画像のキャプションにもとづいてグループA/Bを構成する.
  データセット別比較では,プロンプトのコンテキスト長制限を考慮し,$group\_size = 100$ を用いた.
  \item \textbf{Few-shot ICL の検証}: LLM の出力形式の揺らぎや語彙の安定性を確保するための検証手段として,Few-shot インコンテキスト・ラーニング(ICL)のバリエーション(0-shot, 1-shot, 3-shot)を定量的に検証した.
  LLM は,プロンプト内でグループAとグループBを比較し,グループAに特徴的でグループBには欠如している意味的側面を推論するよう指示された.
\end{itemize}

実験パイプラインの全体フローを以下に示す.

\begin{enumerate}
  \item \textbf{データセット読み込み}:各データセットからテキストを読み込む.
\item \textbf{グループA/B抽出}:各データセットで定義されたグルーピング規則(アスペクトラベル,あるいは Retrieved Concepts における Top/Bottom 構造)に基づき,グループA(特定概念を含むテキスト群)とグループB(含まないテキスト群)を抽出する.
  \item \textbf{プロンプト生成}:グループAとグループBのテキストリストをプロンプトに組み込む.Few-shot例が設定されている場合は,プロンプトにFew-shot例を挿入する.
  \item \textbf{LLMによる対比因子ラベル生成}:GPT-4o-mini等のLLMにプロンプトを入力し,対比因子ラベルを生成する.
  \item \textbf{評価}:生成されたラベルと正解ラベルの意味的類似度をSBERT類似度,BLEU,LLM評価により測定する.
\end{enumerate}

データセット別比較で対象としたアスペクトは,SemEval-2014 から Food,Service,Battery,Screen の 4 種類,GoEmotions から全 28 感情カテゴリ,Steam から Gameplay,Visual,Story,Audio の 4 種類である.

\subsection{LLMモデルとパラメータ設定}
本実験では,対比因子ラベル生成器として GPT-4o-mini を主要モデルとして用いた.
モデル比較実験では,GPT-5.1 も使用した.
GPT-4o-mini を選択した理由は,コスト効率が高く,かつ十分な性能を発揮することが既存研究で確認されているためである.

各実験カテゴリでのモデル選択として,データセット別比較,Few-shot 実験,グループサイズ比較実験では GPT-4o-mini を使用した.
アスペクト説明文比較実験では GPT-4o を使用した.
モデル比較実験では,GPT-4o-mini と GPT-5.1 の 2 モデルを比較した.
COCO Retrieved Concepts 実験では,GPT-4o-mini を使用した.

温度パラメータ(temperature)の設定として,データセット別比較では temperature = 0.0 を用いた.
この設定により,決定論的な出力が得られ,実験の再現性が確保される.
Few-shot 実験,グループサイズ比較実験,モデル比較実験,アスペクト説明文比較実験においても temperature = 0.0 を用いた.
COCO Retrieved Concepts 実験においても temperature = 0.0 を用いた.

最大トークン数(max\_tokens)の設定として,データセット別比較では max\_tokens = 2000 に設定した.
Few-shot 実験,モデル比較実験では max\_tokens = 100 に設定した.
グループサイズ比較実験,アスペクト説明文比較実験,COCO Retrieved Concepts 実験では max\_tokens = 2000 に設定した.

その他の生成パラメータとして,top\_p や frequency\_penalty はデフォルト値を使用した.

\subsection{プロンプト設計}
プロンプトテンプレートは,以下の構造を持つ.
まず,タスクの説明として \texttt{2つのデータグループを比較して,グループAに特徴的でグループBには見られない表現パターンや内容の特徴を特定してください} という指示を提示する.
次に,Few-shot 例が存在する場合は \texttt{examples\_section} に挿入される.
Few-shot 例の形式は \texttt{【例題N】グループA: [...] グループB: [...] 回答: [正解ラベル]} である.
その後,実際のデータとして【グループA】と【グループB】のテキストリストが提示される.
各テキストは \texttt{- [テキスト内容]} の形式で列挙され,コンテキスト長制限を考慮して最大 100 件に制限される.
最後に,出力形式の指示として \texttt{英語で5-10単語程度で,グループAに特徴的でグループBには見られない主要な違いを簡潔に回答してください} が追加される.

グループA/Bの提示方法として,各テキストは \texttt{- [テキスト内容]} の形式で列挙され,グループAとグループBはそれぞれ \texttt{【グループA】} と \texttt{【グループB】} の見出しで区別された.
コンテキスト長制限を考慮し,各グループから最大 100 件のテキストを抽出した.

出力形式の指示方法として,\texttt{英語で5-10単語程度で,グループAに特徴的でグループBには見られない主要な違いを簡潔に回答してください} という指示をプロンプトの末尾に追加した.

Few-shot 例の挿入方法として,Few-shot 設定が 0 より大きい場合,プロンプトのタスク説明の後に \texttt{examples\_section} を挿入した.
Few-shot 例の形式は \texttt{【例題N】グループA: [...] グループB: [...] 回答: [正解ラベル]} であり,$N$ は例題番号である.

アスペクト説明文の使用方法として,アスペクト説明文比較実験では,アスペクトの説明文をプロンプトの冒頭に追加した.
例えば,Food アスペクトの場合,\texttt{Food refers to mentions of food quality, taste, menu items, or dining experience} といった説明文を挿入した.
説明文ありの条件と説明文なしの条件を比較することで,アスペクト説明文の効果を検証した.

\subsection{データの前処理と分割方法}
テキストの前処理手順として,各データセットからテキストを読み込み,アスペクトラベルに基づいてグループ A とグループ B に分割した.
テキストの前処理として,特殊文字の処理や正規化は行わず,データセットの生のテキストをそのまま使用した.

グループA/Bの抽出方法として,各アスペクトについて,そのアスペクトを含むテキスト群をグループA,含まないテキスト群をグループBとして抽出した.
分割タイプは \texttt{aspect\_vs\_others} を用い,特定のアスペクトが含まれるテキストと含まれないテキストを比較した.
COCO Retrieved Concepts 実験では,分割タイプとして \texttt{aspect\_vs\_bottom100} を用い,Top-100のキャプションをグループA,Bottom-100のキャプションをグループBとして抽出した.

グループサイズ(group\_size)の決定方法として,データセット別比較では group\_size = 100 を用いた.
この値は,プロンプトのコンテキスト長制限を考慮して決定された.
グループサイズ比較実験では,group\_size を 50,100,150,200,300 の 5 段階で変化させた.

サンプリングは,各グループの候補集合から group\_size 件になるようランダムにサブサンプリングし,候補数が group\_size 未満の場合は重複を許して補完した.重み付きサンプリングは使用しなかった.

コンテキスト長制限への対応として,各グループから最大 100 件のテキストを抽出し,プロンプトのコンテキスト長を制限内に収めた.
データセット別比較では group\_size = 100 を用い,グループサイズ比較実験では最大 300 まで検証したが,コンテキスト長超過エラーを回避するため,実際のプロンプトでは必要に応じてテキスト数を制限した.

\subsection{Few-shot例の作成方法}
Few-shot例の選定基準として,各データセットのアスペクトラベルに基づき,正解ラベルが明確な例を選定した.
Few-shot例は,グループAとグループBのテキストリストと,それに対応する正解ラベルで構成された.

例の品質管理方法として,Few-shot 例は,各データセットのアスペクトラベルに基づいて作成し,正解ラベルが明確であることを確認した.
Few-shot 例の形式は \texttt{【例題N】グループA: [...] グループB: [...] 回答: [正解ラベル]} であり,$N$ は例題番号である.

0-shot,1-shot,3-shot の違いと設定方法として,0-shot 設定では Few-shot 例を挿入せず,タスク説明のみを提示した.
1-shot 設定では,1 つの Few-shot 例を \texttt{examples\_section} に挿入した.
3-shot 設定では,3 つの Few-shot 例を \texttt{examples\_section} に挿入した.
Few-shot 実験では,0-shot,1-shot,3-shot の 3 つの設定を比較した.

\subsection{実験カテゴリの定義}
各実験カテゴリのパラメータ設定を表~\ref{tab:experiment_config}に示す.

temperature は全条件で 0.0 とし,top\_p や frequency\_penalty などその他の生成パラメータはデフォルト値を使用した.
LLM 評価は gpt-4o-mini(temperature = 0.0)で実施し,COCO 実験のみ無効とした.

\begin{table}[htbp]
\centering
\caption{実験カテゴリ別パラメータ設定(変動項目のみ)}
\label{tab:experiment_config}
\small
\begin{tabular}{lllll}
  \toprule
  実験カテゴリ & max\_tokens & few\_shot & group\_size & 生成モデル \\
  \midrule
  データセット別比較 & 2000 & 0 & 100 & \texttt{4o-mini} \\
  Few-shot実験 & 100 & 0/1/3 & 100 & \texttt{4o-mini} \\
  グループサイズ比較 & 2000 & 0 & 50--300 & \texttt{4o-mini} \\
  モデル比較 & 100 & 0 & 100 & \texttt{4o-mini, 5.1} \\
  アスペクト説明文比較 & 2000 & 0 & 100 & \texttt{4o} \\
  COCO実験 & 2000 & 0 & 100 & \texttt{4o-mini} \\
  \bottomrule
\end{tabular}
\end{table}

\subsection{比較手法とベンチマーク}
LLMによって生成された対比因子ラベルの品質を,人手アノテーションされた既存の ABSA ベンチマーク(SemEval-2014 Restaurant/Laptop,Steam レビューなど)の正解ラベルとの意味的類似性と比較することで評価した.

\section{評価指標}
\label{sec:evaluation_metrics}
生成された自然言語ラベル $L$ の品質を評価するために,SBERT類似度,BLEU,および LLM による意味的類似度評価を用いた.

\subsection{評価指標の選定理由}
\label{subsec:evaluation_metrics}
本実験では,SBERT類似度,BLEU,LLM 評価の 3 つの指標を用いた.
SBERT類似度 を選んだ理由は,生成ラベルと正解ラベルの意味的類似度を測定するためである.
本タスクでは,LLM が生成するラベルが \texttt{食べ物の品質に関する言及} のような説明的なフレーズとなるのに対し,正解ラベルは \texttt{food} のような単一の単語であるため,語彙レベルの一致度を超えたセマンティックな評価が必要である.
BLEU を選んだ理由は,語彙レベルの一致度を補助的に確認するためである.
LLM 評価を選んだ理由は,LLM による意味的類似度評価を補助指標として用いるためである.

各指標の役割と位置づけとして,SBERT類似度 は主要指標として位置づけられ,生成ラベルと正解ラベルの意味的類似度を測定した.
BLEU は参考指標として位置づけられ,語彙レベルの一致度を補助的に確認した.
LLM 評価は補助指標として位置づけられ,GPT-4o-mini による 5 段階評価を実施した.

\subsection{BERTスコア(Sentence-BERT類似度)}
BERT系モデルにより文を埋め込み表現に変換し,文間の意味的な近さを類似度として数値化する指標である.

本実験では,Sentence-BERT~\cite{reimers2019sentence} により生成ラベル $L$ と正解ラベル $L_{\mathrm{ref}}$(アスペクト説明文を用いる条件ではその説明文)を文埋め込みに変換し,コサイン類似度を計算する.実装では,SentenceTransformer(\texttt{all-MiniLM-L6-v2})の文埋め込みを用い,コサイン類似度($-1$〜$1$)を $(\mathrm{cos}+1)/2$ により 0.0〜1.0 に正規化した値を BERTスコアとして扱う.

本指標により,生成ラベルが正解ラベル(またはその説明文)と意味的にどの程度一致しているかを連続値で測り,語彙一致に依存しない有効性評価を行うことを狙う.

値が 1.0 に近いほど意味的に類似していることを示す.

\subsection{BLEU(Bilingual Evaluation Understudy)}
BLEU は,生成文と参照文の間の n-gram の重複にもとづき,語彙レベルの一致度を数値化する指標である~\cite{papineni-etal-2002-bleu}.

本実験では,NLTK~\cite{bird2009natural} の \texttt{sentence\_bleu} により,参照側に正解ラベル(またはその説明文),候補側に生成ラベルを与えて BLEU を算出し,SmoothingFunction.method1 を適用した.評価範囲は 0.0 から 1.0 であり,1.0 に近いほど一致度が高いことを示す.

本指標は,語彙レベルの一致度を補助的に確認するために用いた.ただし,本タスクでは正解ラベルが \texttt{food} や \texttt{price} のように 1 語の名詞句で与えられることが多い一方,生成ラベルは説明的フレーズになりやすく,n-gram が成立しにくいため BLEU が低値になりやすい.したがって,BLEU は参考値として解釈する.

\subsection{LLM評価スコア}
LLM 評価は,LLM を評価器として用い,参照テキストと候補テキストの意味的類似度を段階評価する指標である.

本実験では,参照側に正解ラベル(またはその説明文),候補側に生成ラベルを与え,GPT 系モデルに 5 段階(1--5)で評価させた.データセット別比較,Few-shot 実験,グループサイズ比較実験,アスペクト説明文比較実験では GPT-4o-mini を用い,temperature = 0.0 に設定した.モデル比較実験では GPT-4o を用い,temperature = 0.0 に設定した.COCO Retrieved Concepts 実験では LLM 評価を無効化した.

本指標により,埋め込み類似度では捉えにくい意味的一致/不一致を補助的に確認し,BERTスコアやBLEUの結果の解釈を支えることを狙う.ただし,単一モデルを評価器として用いる自動評価であるため,評価器バイアスが混入し得る点に留意し,補助指標として解釈する.

評価プロンプトの設計として,以下のプロンプトを使用した.
\begin{quote}
\ttfamily
参照テキストと候補テキストの意味的類似度を5段階(1-5)で評価してください.\par
参照テキスト: \{reference\_text\}\par
候補テキスト: \{candidate\_text\}\par
評価基準:\par
- 5: 完全に同じ意味\par
- 4: ほぼ同じ意味(細かい違いのみ)\par
- 3: 類似しているが一部異なる\par
- 2: 部分的に類似している\par
- 1: ほとんど異なる\par
出力形式(JSON形式):\par
\{\par
    "score": 4,\par
    "normalized\_score": 0.8,\par
    "reasoning": "評価理由を簡潔に説明"\par
\}\par
\end{quote}

評価基準(5段階評価の詳細)として,1 から 5 の整数で評価し,5 が最も類似度が高く,1 が最も類似度が低い.
正規化方法として,5 段階評価(1-5)を 0.0-1.0 に正規化し,normalized\_score = (score - 1) / 4 として計算した.

SBERT類似度 との関係として,LLM 評価スコアは SBERT類似度 を補完する補助指標として位置づけられ,両指標を併用することで生成ラベルの品質を多角的に評価した.

\section{統計的分析}
\label{sec:statistical_tests}
追加実験では,条件差がアスペクトに依らず一貫して観測されるかを補足的に確認するため,統計的検定を行った.有意水準は \( \alpha=0.05 \)(両側)とした.検定対象は,生成ラベルと参照ラベル(または説明文)との意味的一致を表す主要指標である SBERT類似度(BERTスコア)とした.検定統計量の計算には SciPy を用いた~\cite{virtanen2020scipy}.

\subsection{繰り返し測定(ブロック)と多水準比較}
Few-shot(0/1/3-shot)および group\_size(50/100/150/200/300)の比較では,Steam の 4 アスペクトをブロック(繰り返し測定の単位)とみなし,条件を要因とする Friedman 検定を実施した~\cite{friedman1937use,demsar2006statistical}.補足として条件ペアごとの対応のある Wilcoxon 検定(両側)も算出し~\cite{wilcoxon1945individual},多重比較の補正には Holm 法を用いた~\cite{holm1979simple}(主検定が非有意の場合,事後比較は参考として扱う).

\subsection{2条件比較}
モデル比較およびアスペクト説明文あり/なしの比較では,Steam の 4 アスペクトをブロックとする対応のある Wilcoxon 検定(両側)を用いた~\cite{wilcoxon1945individual}.

\subsection{指標間の整合(補足)}
SBERT類似度と LLM 評価の序列がどの程度一致するかを補足的に示すため,全 36 条件の順位に対して Spearman の順位相関(両側)を算出した~\cite{spearman1904association}.

\input{chapters/05_results}
\chapter{総合考察}

本章では,第5章で示した実験結果に基づき,LLM を命名モジュールとして用いた場合の性能傾向,性能に影響する要因,および限界について整理して考察する.
結論として,本手法の成否は,集合 $A/B$ の差分が単一の概念軸として立ち上がるか(単軸性)と,その差分が対象ドメインにとって妥当な表現空間の差分であるか(妥当性)に大きく依存する.

\section{横断的知見:集合差分が単一軸としてまとまるか}
\label{sec:concreteness}

まず,データセット別比較のデータセット別結果は,総合的な性能傾向を与える.表\ref{tab:main_dataset_results}が示すように,SemEval-2014 と GoEmotions では SBERT 類似度が高く,Steam では低い.この差は,単に語彙的一貫性の有無として片付けるより,入力として与える集合 $A/B$ の差分が,単一の概念軸としてまとまっているかどうかとして理解するのが自然である.

第5章\ref{subsec:main_experiment_discussion}節の議論に沿えば,SemEval-2014 や GoEmotions ではアスペクトや感情カテゴリに対応する典型表現が相対的に揃っており,$A/B$ の差分が特定の意味領域に集中しやすい.その結果,LLM は差分の中心を短いラベルへ圧縮しやすく,命名が安定しやすいと解釈できる.一方 Steam は,平均 244.70 語の長文レビューであり,複数アスペクト併記が 90.27\% を占める(第4章 表~\ref{tab:text_stats_overview})ため,差分が複数の観点に分散しやすい.このとき,生成ラベルは対比として一定に妥当でも,単一アスペクト名への対応が弱まり,スコア低下として現れやすい.

このように,個別結果をまとめると,本手法の成否は,データセットの名前そのものより,集合差分が単一軸として立ち上がる条件かどうかに大きく依存する,という知見として整理できる.

\section{介入条件(Few-shot,説明文)の効き方:単軸化を助けるか崩すか}

次に,Few-shot ICL やアスペクト説明文といった介入は,出力の安定化に寄与し得る一方で,一様な改善としては観測されなかった.例えば Steam の Few-shot 比較では,アスペクトごとに最も良い設定が揃わず(第5章 表\ref{tab:fewshot_aspect}),説明文付与でも \textit{Visual} のみ低下するなど効果の向きが一致しない(第5章\ref{tab:aspect_desc_aspect}).また,第5章で実施した検定は非有意のものが多く,平均値レベルでは安定した差として断定しにくい.

この結果から得られる知見は,介入が性能を単調に押し上げるというより,差分の見え方を変える操作だという点である.すなわち,Few-shot 例や説明文が,対象アスペクトの境界を明確にし,差分を単一軸としてまとめる方向に働く場合には改善が期待できる.一方で,例や説明が抽象的で対象範囲が広い場合には,注意が分散して差分の単軸性を崩し,アスペクト名との対応が弱まる可能性がある.したがって,介入の設計では,対象アスペクトの境界を狭く明確に切り出すことが重要である.

\section{評価指標の解釈:何が測れて何が測れないか}

評価の観点では,BLEU のような語彙一致指標は,多くの条件で低値に張り付いており(第4章\ref{sec:evaluation_metrics}節,第5章\ref{subsec:main_experiment_discussion}節),本研究の設定では主要な差を捉えにくい.これは,正解ラベルが 1 語の名詞句で与えられることが多い一方,生成ラベルは説明的になりやすく,表層一致が成立しにくいためである.一方 SBERT 類似度は,語彙が一致しなくても意味的な近さを捉えやすく,命名候補が正解概念とどの程度整合するかを広く把握できる.

ただし,意味的に近い表現であっても,対比因子として焦点が合っているか,外部スキーマとしてのアスペクト名に対応する概念として読めるかは別問題である.この点で LLM 評価は,意味的一致に加えて命名としての妥当性や簡潔性をより保守的に反映し得る.一方で,単一モデルを評価器として用いる自動評価であり,評価器バイアスが混入し得る点に留意が必要である.以上をまとめると,本研究では SBERT 類似度を主要指標として広い一致を測りつつ,LLM 評価で対比因子ラベルとしての読みやすさと焦点を補助的に確認する,という位置づけが妥当である.

\section{視覚ドメインの含意:差分の妥当性と表現空間のギャップ}

COCO Retrieved Concepts 実験は,別の制約を明確にする.第5章\ref{sec:coco_experiment}節が示すように,テキストキャプション集合だけにもとづく対比因子生成は,キャプション側の頻度バイアスや記述の欠落の影響を受け,画像の本質的特徴とずれたラベルを生成するリスクがある.ここで重要なのは,差分が単一軸としてまとまっているかに加え,その差分が何の表現空間の差分なのかという妥当性である.

すなわち,本実験では入力がテキストである以上,LLM はキャプション分布の差分を忠実に要約できても,視覚側の差分と必ずしも一致しない.このギャップを縮めるには,画像特徴量や物体ラベルなど視覚側の中間表現を併用し,テキストと視覚の両方に整合する形で差分を検証する枠組みが必要である.

\section{本手法の限界と課題(総括)}

以上の考察を踏まえると,本手法の限界は,結果の羅列ではなく,次の二点に集約できる.第一に,入力集合 $A/B$ の差分が複数軸に分散する条件では,短い対比因子ラベルへの圧縮が不安定になりやすい.これは Steam のようなマルチアスペクト環境で顕在化した.第二に,入力が特定の表現空間に偏る場合,生成ラベルがその空間の差分を要約しても,別の表現空間の本質とずれる可能性がある.これは COCO 実験で顕在化した.

したがって,本手法を命名モジュールとして運用する際には,(i) $A/B$ の構成段階で差分を単一軸に近づける工夫と,(ii) 対象ドメインに応じて差分の妥当性を担保する観測設計が,性能以前に前提条件として重要になる.

\section{改善方針と今後の課題}
改善の方向性としては,(i) グルーピング精度の向上(集合構成のノイズ低減)により差分の単軸性を高めること,(ii) プロンプト設計の精密化(Few-shot 例・説明文の粒度と境界の明確化)により単軸化を助ける介入へ寄せること,(iii) 評価の多面的設計(SBERT 類似度を主要指標としつつ LLM 評価等を補助に用いる)により命名としての妥当性を確認すること,を優先課題として位置づける.特に視覚ドメインでは,キャプションだけに依存せず視覚側の中間表現との整合性評価を組み込む必要がある(第5章\ref{sec:coco_experiment}節).

\section{研究の含意}
本研究は,集合差分 $(A,B)$ から自然言語ラベルを生成する枠組みを複数ドメインで統一的に評価し,有効条件と破綻条件を具体化した点に意義がある.とりわけ,ラベル付きベンチマークでの定量評価と,正解ラベルを持たない設定(COCO)での定性的検証を併置することで,命名モジュールとしての適用可能性と限界を整理した.

以上より,本手法は「単軸な差分」を適切な表現空間から観測できる条件で有効に機能し,いずれかが崩れる条件では命名が不安定化しやすい,という形で総括できる.



\chapter{まとめ}
\section{結論}

本研究は、大規模言語モデル(LLM)の強力な文脈理解能力と生成能力を活用し、ニューロン発火条件に対応するテキスト集合間の意味的差分を自然言語で記述する「対比因子命名」という新規タスクの実現可能性を検証した~[1--3]。従来の XAI 手法が抱えていた構造的なボトルネックを解消するため、LLM(GPT-4o-mini)をコントラスト生成器として利用するアプローチを提案し、その有効性を定量的に実証した~[1, 4, 5]。

その有効性を示すために、本研究では性質の異なる 4 つの多様なデータセット――SemEval-2014(レストランレビュー)、Steam レビュー、GoEmotions(感情分類)、Retrieved Concepts(視覚的概念記述)――を対象に検証を行った~[6--10]。これにより、本手法が製品レビュー、感情データ、視覚概念テキストといった、内容と粒度の異なるドメインに対しても一貫した枠組みとして適用可能であることを示した。

主要な定量評価では、生成ラベルと人手ラベルとの意味的類似性を BERTScore を用いて測定した結果、平均約 0.551 という中適度な意味的関連性を達成した~[1, 11--14]。この結果は、LLM がニューロンの発火パターンという集合的な差分から、その意味的な核を抽出できるという主要な仮説~[4] が、一定の条件の下で妥当であることを定量的に裏付けるものである~[1, 14]。

また、生成ラベルの品質は、「Food」「Price」のような語彙的に安定した具体的アスペクトにおいて特に高い傾向を示し、LLM が明示的なテキスト証拠に基づく「抽出」タスクに強みを持つことを確認した~[12, 13, 15, 16]。一方で、語彙的一致度を測る BLEU スコアは極めて低値(約 0.007)であり、本タスクにおいては語彙的重複よりも意味的妥当性を重視する BERTScore のような指標が不可欠であることが明らかになった~[1, 14, 18, 19]。

\section{本研究の貢献}

本研究の第一の貢献は、従来は人手に依存していた「概念の命名」プロセスを、LLM を用いて自動化するスケーラブルな基盤を提示した点である~[3, 20, 21]。LIME や SHAP~[22--24] に代表される局所的な特徴可視化手法や、非教師ありコンセプト抽出(UCBM/CCE)~[25--27] のように概念ベクトルの発見に留まっていた従来手法と異なり、本手法は「概念の発見」と「命名」を一体として扱う XAI の新たなワークフローを提案した~[3, 12, 28]。

第二に、本手法は人手アノテーションが困難あるいは高コストであるドメインにおいても適用可能な汎用性を有することを示した。具体的には、物理的実体を持たない抽象概念である Joy や Anger など 28 種類の感情カテゴリを含む GoEmotions データセット~[10, 29, 30]、および「Gameplay」「Technical」といった専門性の高いアスペクトを含む Steam レビューデータセット~[7, 8, 31] を対象とし、抽象的・専門的な概念に対しても一定水準の対比因子命名が可能であることを示した。

第三に、本研究は、メカニスティック解釈(MI)分野~[28, 32] における Attribution Graphs~[26, 33] などが発見した特徴量に対し、人間が理解できる自然言語ラベルを自動付与する命名モジュールとして機能し得ることを示した点である。これにより、「労働集約的 (labor-intensive)」と指摘されてきた手動ラベリングのボトルネック~[3, 32, 34] を緩和し、ブラックボックスモデルの構造理解をより実務的なレベルへと押し上げる足掛かりを提供した。

\section{今後の課題}

一方で、本研究にはいくつかの限界も存在する。まず、BERTScore における平均スコア 0.551 は中程度の意味的類似度を示すものの、高精度な命名が常に達成されているわけではなく、とりわけ抽象的な感情概念や、複数アスペクトが絡み合うケースにおいては、生成ラベルと人手ラベルの解釈がずれる事例も確認された~[10, 30]。また、評価指標として BERTScore と BLEU に依存しており、人間評価との整合性を直接的に検証できていない点も課題として残る~[1, 14, 18, 19]。

これらの課題に対して、本研究の結果は今後の具体的な改善方向性も示唆している。第一に、抽象概念命名の精度向上に向けて、Chain-of-Thought(CoT)プロンプティングを導入し、LLM による逐次的な思考過程の明示化を通じて、集合差分の解釈プロセスを段階的に構造化することが有望である~[35, 36]。第二に、BLEURT や BARTScore など、人間評価との整合性が高いとされる学習ベースの評価指標~[36, 37] を導入し、単一指標に依存しない多面的な評価設計へ移行することが求められる~[35, 37]。

最後に、本研究で提案した対比因子命名は、これまでの非教師あり概念抽出技術が広大な地層から「未知の鉱石(概念ベクトル)」を発見する作業であったとすれば~[38]、その鉱石の「成分と特性の差分」(集合 A vs.~集合 B)を LLM という高性能な分析装置にかけ、「希少な合金に関する言及」のような、人間に理解可能な名前(対比因子ラベル)を瞬時に自動付与する試みであると言える~[38, 39]。この枠組みは、ブラックボックスモデルの解釈可能性をスケーラブルに高めるための重要な一歩であり、今後の XAI 研究および実応用に向けた基盤として発展が期待される。

\chapter*{謝辞}
\addcontentsline{toc}{chapter}{謝辞}

% TODO: 謝辞の内容を記述


% 参考文献(References)
\newpage
\addcontentsline{toc}{chapter}{参考文献}
\renewcommand{\bibname}{参考文献}

%% 参考文献に bibtex を使う場合
%\bibliographystyle{junsrt}
%\bibliography{hoge}

%% 参考文献を直接ファイルに含めて書く場合
\begin{thebibliography}{99}

\bibitem{pontiki-EtAl:2014:SemEval2014Task4}
M.~Pontiki, D.~Galanis, J.~Pavlopoulos, H.~Papageorgiou, I.~Androutsopoulos, and S.~Manandhar:
``SemEval-2014 Task 4: Aspect Based Sentiment Analysis,''
in \textit{Proceedings of the 8th International Workshop on Semantic Evaluation (SemEval 2014)}, pp.~27--35, 2014.

\bibitem{ribeiro2016should}
M.~T. Ribeiro, S.~Singh, and C.~Guestrin:
``Why should I trust you?: Explaining the predictions of any classifier,''
in \textit{Proceedings of the 22nd ACM SIGKDD International Conference on Knowledge Discovery and Data Mining}, pp.~1135--1144, 2016.

\bibitem{lundberg2017unified}
S.~M. Lundberg and S.-I. Lee:
``A unified approach to interpreting model predictions,''
in \textit{Advances in Neural Information Processing Systems}, vol.~30, 2017.

\bibitem{wachter2017counterfactual}
S.~Wachter, B.~Mittelstadt, and C.~Russell:
``Counterfactual explanations without opening the black box: Automated decisions and the GDPR,''
\textit{Harvard Journal of Law \& Technology}, vol.~31, no.~2, p.~841, 2017.

\bibitem{kim2018interpretability}
B.~Kim, M.~Wattenberg, J.~Gilmer, C.~Cai, J.~Wexler, F.~Viegas, et al.:
``Interpretability beyond feature attribution: Quantitative testing with concept activation vectors (TCAV),''
in \textit{Proceedings of the 35th International Conference on Machine Learning}, pp.~2673--2682, 2018.

\bibitem{luss2024cell}
R.~Luss:
``CELL your Model: Contrastive Explanations for Large Language Models,''
arXiv:2406.11785, 2024.

\bibitem{bucinca2024contrastive}
Z.~Bu{\c{c}}inca:
``Contrastive Explanations That Anticipate Human Misconceptions Can Improve Human Decision-Making Skills,''
arXiv:2410.04253, 2024.

\bibitem{xu2024concept}
Y.~Xu et al.:
``Concept Bottleneck Models Without Predefined Concepts,''
arXiv:2407.03921, 2024.

\bibitem{anthropic2025biology}
Anthropic:
``On the Biology of a Large Language Model,''
Transformer Circuits, 2025. Available at \url{https://transformer-circuits.pub/2025/attribution-graphs/biology.html}.

\bibitem{alghamdi2024dynamic}
M.~Alghamdi et al.:
``Dynamic Sentiment Analysis with Local Large Language Models using Majority Voting,''
arXiv:2407.13069, 2024.

\bibitem{demszky2020goemotions}
D.~Demszky, D.~Movshovitz-Attias, J.~Ko, A.~Cowen, G.~Nemade, and S.~Ravi:
``GoEmotions: A Dataset of Fine-Grained Emotions,''
in \textit{Proceedings of the 58th Annual Meeting of the Association for Computational Linguistics}, 2020.

\bibitem{srec:steam-review-aspect-dataset}
S.~Khosasi:
``Steam review aspect dataset,''
2024. Available at \url{https://srec.ai/blog/steam-review-aspect-dataset}.

\bibitem{papineni-etal-2002-bleu}
K.~Papineni, S.~Roukos, T.~Ward, and W.-J. Zhu:
``BLEU: a Method for Automatic Evaluation of Machine Translation,''
in \textit{Proceedings of the 40th Annual Meeting of the Association for Computational Linguistics}, pp.~311--318, 2002.

\bibitem{devlin2018bert}
J.~Devlin, M.-W. Chang, K.~Lee, and K.~Toutanova:
``BERT: Pre-training of Deep Bidirectional Transformers for Language Understanding,''
arXiv:1810.04805, 2018.

\bibitem{schrodi2024unsupervised}
S.~Schrodi, M.~R{\"u}ckl, T.~Wirth, M.~B{\"o}hm, and D.~R{\"u}gamer:
``Concept Bottleneck Models Without Predefined Concepts,''
arXiv:2407.03921, 2024.

\bibitem{ameisen2025attribution}
E.~Ameisen, J.~Lindsey, A.~Pearce, W.~Gurnee, N.~L. Turner, B.~Chen, C.~Citro, D.~Abrahams, S.~Carter, B.~Hosmer, J.~Marcus, M.~Sklar, A.~Templeton, T.~Bricken, C.~McDougall, H.~Cunningham, T.~Henighan, A.~Jermyn, A.~Jones, A.~Persic, Z.~Qi, T.~B. Thompson, S.~Zimmerman, K.~Rivoire, T.~Conerly, C.~Olah, and J.~Batson:
``Circuit Tracing: Revealing Computational Graphs in Language Models,''
Transformer Circuits, 2025. Available at \url{https://transformer-circuits.pub/2025/attribution-graphs/methods.html}.

\bibitem{bordt2022posthoc}
S.~Bordt, M.~Finck, E.~Raidl, and U.~von Luxburg:
``Post-Hoc Explanations Fail to Achieve their Purpose in Adversarial Contexts,''
in \textit{Proceedings of the 2022 ACM Conference on Fairness, Accountability, and Transparency (FAccT '22)}, pp.~1495--1515, 2022.

\bibitem{kardale2023contrastive}
A.~Kardale:
``Contrastive text summarization: a survey,''
\textit{International Journal of Information and Computation}, vol.~12, no.~3, pp.~1--10, 2023.

\bibitem{saha2024strumllm}
A.~Saha, B.~P. Majumder, H.~Jhamtani, S.~Subramanian, S.~Sreedhar, S.~Chakrabarti, and P.~Kankar:
``STRUM-LLM: Attributed and Structured Contrastive Summarization for User-Oriented Comparison,''
arXiv:2403.19710, 2024.

\bibitem{luo2024chatabsa}
Z.~Luo, Z.~Feng, Y.~Zhang, and H.~Liu:
``ChatABSA: A Novel Framework for Aspect-based Sentiment Analysis using Large Language Models,''
arXiv:2401.08226, 2024.

\bibitem{wang2024llmcluster}
J.~Wang, J.~Song, X.~Sun, C.~Chen, W.~Liu, and Y.~Liu:
``Improving Clustering Performance by Leveraging Large Language Models,''
arXiv:2410.00927, 2024.

\bibitem{sellam-etal-2020-bleurt}
T.~Sellam, D.~Das, and A.~Parikh:
``BLEURT: Learning Robust Metrics for Text Generation,''
in \textit{Proceedings of the 58th Annual Meeting of the Association for Computational Linguistics}, pp.~7881--7892, 2020.

\bibitem{yuan2021bartscore}
W.~Yuan, G.~Neubig, and P.~Liu:
``BARTScore: Evaluating Generated Text as Text Generation,''
in \textit{Advances in Neural Information Processing Systems}, vol.~34, pp.~27263--27277, 2021.

\bibitem{reiter2018structured}
E.~Reiter:
``A Structured Review of the Validity of BLEU,''
\textit{Computational Linguistics}, vol.~44, no.~3, pp.~393--401, 2018.

\bibitem{holtzman2020curious}
A.~Holtzman, J.~Buys, L.~Du, M.~Forbes, and Y.~Choi:
``The Curious Case of Neural Text Degeneration,''
in \textit{International Conference on Learning Representations (ICLR)}, 2020.

\bibitem{vaswani2017attention}
A.~Vaswani, N.~Shazeer, N.~Parmar, J.~Uszkoreit, L.~Jones, A.~N. Gomez, {\L}.~Kaiser, and I.~Polosukhin:
``Attention is All you Need,''
in \textit{Advances in Neural Information Processing Systems (NIPS)}, 2017.

\bibitem{zhang2019bertscore}
T.~Zhang, V.~Kishore, F.~Wu, K.~Q. Weinberger, and Y.~Artzi:
``BERTScore: Evaluating Text Generation with BERT,''
arXiv:1904.09675, 2019.

\bibitem{stein2024towards}
A.~Stein, A.~Naik, Y.~Wu, M.~Naik, and E.~Wong:
``Towards Compositionality in Concept Learning,''
in \textit{Proceedings of the International Conference on Machine Learning (ICML)}, 2024.

\end{thebibliography}



\end{document}
