\documentclass[a4paper]{jreport}

\usepackage{masterThesisJa}
\usepackage{amsmath}
\usepackage{url}
\DeclareUnicodeCharacter{2248}{\ensuremath{\approx}}
\setcounter{tocdepth}{3}
\setcounter{page}{-1}

\maintitle{大規模言語モデルを用いた対比因子ラベル生成手法に関する研究}{A Study on Generating Contrastive Factor Labels for Explainable AI}

\publish{2025}{12}

\student{202421675}{清野 駿}{Seino Shun}

\jabst{
    大規模言語モデル(LLM)は多様なタスクで高い性能を示している一方で,その内部処理は依然としてブラックボックス性が高く,理解が難しい。この問題を解決する試みとして,LLM 内のニューロンの発火条件を解析する研究が行われてきたが,こうした手法には,人手によるラベル付けに大きく依存するという課題がある。モデル内部でどのような情報が利用されているかを明らかにするためには,未知データごとに人間が解釈ラベルを付与する必要があり,膨大な時間とコストがかかる。もしモデル内部表現に対応する概念や特徴を自然言語で自動的に記述できれば,この負担を大きく軽減でき,説明可能 AI の実現に向けた重要な一歩となる。
    本研究では,二つのテキスト集合 A/B の差異を自然言語で要約する「対比因子ラベル」を自動生成する枠組みの実現可能性を検証する.
    本研究の目的は,LLM がこの対比因子ラベル生成タスクに対してどの程度の性能を示しうるかを明らかにすることである。具体的には,既存のアスペクトラベルにもとづいてあらかじめ二つのテキスト集合 A/B を構成し,LLM によって生成された対比因子ラベルと,集合を分割した元のアスペクトとの対応関係を比較することで,本枠組みの実現可能性を評価する。
    手法としては,A/B の代表テキストを LLM に入力し,Few-shot 例示(0/1/3-shot)を与えたプロンプトにより,両集合を分ける特徴を自然言語で生成させる。評価には SemEval-2014 ABSA(Restaurants/Laptop),Steam ゲームレビュー,GoEmotions,COCO Retrieved Concepts など複数ドメインを含むデータセットを用い,各グループ100件の条件で GPT-4o-mini を含む複数モデルを比較した。生成されたラベルの品質は,Sentence-BERT による埋め込み類似度(BERTScore),BLEU,および LLM による意味的類似度評価を組み合わせ,多角的に測定した。
    実験の結果,全 36 条件の平均で BERTScore は約 0.698 と中程度の意味的一致を示した一方,BLEU は 0.008 と語彙レベルの一致は低かった。また Few-shot では 1-shot が最も良好で,出力が「説明的な文章」から「より一意に特定可能な語彙」へと安定する傾向が見られた。さらに gameplay や food のような語彙的に具体的・安定した概念では高いスコアを示し,recommended や suggestion といった抽象概念では低下する傾向が確認された。
    これらの結果から,提案手法は一定の条件下において、人手による解釈ラベリングを部分的に代替し得ることが示された。ただし,概念が抽象的な場合の説明の難しさや、語彙が一致しない場合の文脈を汲み取った評価ができない問題など,今後解決すべき課題も明らかになった。
}


\advisors{若林 啓}{伊藤 寛祥}

\begin{document}

\makecover

\addtolength{\textheight}{-5mm}
\setlength{\footskip}{15mm}
\fontsize{11pt}{15pt}\selectfont

\pagebreak\setcounter{page}{1}
\pagenumbering{roman}
\pagestyle{plain}
\tableofcontents
\listoffigures
\listoftables

\parindent=1zw
\pagebreak\setcounter{page}{1}
\pagenumbering{arabic}
\pagestyle{plain}

% 各章を読み込む
\chapter{序章}

近年の深層学習(DNN)モデルは、医療、金融、自動運転といった社会的重要性の高いドメインにおいて優れた予測性能を達成している。しかし、その意思決定プロセスが人間にとって解釈不能な「ブラックボックス」であるという根本的な問題は、モデルの信頼性や公平性の担保、そして法的・倫理的な規制要件の遵守を妨げる最大の障壁となっている~\cite{bordt2022posthoc,kim2018interpretability}。この課題に対応するため、モデルの判断根拠を事後的に説明する説明可能 AI(XAI)の研究が活発化してきた。

従来、XAI の中心的な手法として広く普及したのは、LIME(Local Interpretable Model-agnostic Explanations)や SHAP(SHapley Additive exPlanations)に代表される、\textbf{事後説明(Post-hoc Explanation)}手法である。これらの手法は、入力データ(例:画像ピクセル、テキストトークン)の摂動やゲーム理論に基づく貢献度(Shapley 値)の計算を通じて、個別インスタンスの予測に対する特徴量の寄与を可視化する~\cite{ribeiro2016should,lundberg2017unified}。

しかし、事後説明手法には、以下のような根本的な信頼性の問題が指摘されている。第一に、これらの説明は本質的な曖昧さ(high degree of ambiguity)を抱えており、特に共線性を持つ特徴量に対して不安定な結果を示すことがあり、同等の性能を持つ異なるモデルが全く異なる説明を生成する不安定性が問題となる。第二に、Bordt ら~\cite{bordt2022posthoc}は、事後説明アルゴリズムが、規制や倫理が求める「透明性」の目的を達成する上で\textbf{「不適切 (unsuitable)」}であると結論付けている。これは、説明提供者と受け手の利害が対立する「敵対的文脈 (adversarial contexts)」において、事後説明が容易に操作可能であり、モデル開発者が都合の良い説明を選択的に提示できてしまうためである~\cite{bordt2022posthoc}。

さらに重要な点として、LIME や SHAP は、あくまで個別インスタンスの局所的な特徴寄与に焦点を当てており、モデルが特定のデータ集合(例:特定のニューロンが発火するすべてのサンプル)に対して系統的に何を学習しているか、というグループレベルの差異や集合間のコントラストを自然言語で説明する能力を欠いている~\cite{ribeiro2016should,lundberg2017unified}。

事後説明の限界を克服するため、XAI の研究コミュニティは、低レベルの特徴量(ピクセル、トークン ID)から、人間が理解可能な\textbf{「コンセプト(概念)」}レベル(例:「縞模様」や「価格に関する言及」)で説明を提供するパラダイムへと移行している~\cite{kim2018interpretability}。コンセプトボトルネックモデル(CBM)や TCAV(Concept Activation Vectors)といったコンセプトベース XAI(C-XAI)は、モデルの内部状態を高レベルの概念に関連付けることで解釈可能性を向上させた~\cite{kim2018interpretability,schrodi2024unsupervised,stein2024towards}。

しかし、このアプローチは新たな、かつ重大なボトルネックを生み出した。それは、どの概念を監視・学習すべきかという概念の定義(ラベリング)が、依然として人間に依存している点である。このプロセスは Kim らによって\textbf{「高コストな概念キュレーション (expensive concept curation)」}と呼ばれ、特に専門知識が不可欠なドメイン(例:医療)において、C-XAI の導入を阻む最大の障壁となっていた~\cite{kim2018interpretability,schrodi2024unsupervised,stein2024towards}。既存のベンチマークである SemEval-2014 においても、アスペクト(概念)ラベルは人手によるアノテーションに依存している~\cite{pontiki-EtAl:2014:SemEval2014Task4}。

モデルの内部動作を回路図レベルで徹底的に理解しようとする最先端のメカニスティック解釈(MI)分野においても、解釈の「意味付け」は手作業に依存する。Anthropic による Attribution Graphs のような研究は、モデルの計算プロセスをトレースし、特徴量間の相互作用をグラフとして可視化することで、ニューロン間の回路を発見する。

しかし、Attribution Graphs などが発見する個々の「特徴量」やノードが、人間にとって何を意味するのかを自然言語で特定するプロセスは自動化されていない。研究者らは、解釈を容易にするために、関連する意味を持つ特徴量を手動でグループ化し\textbf{「スーパーノード」としてまとめているが、この手動ステップは「労働集約的 (labor-intensive)」}であり、情報の欠損を伴うことが指摘されている~\cite{anthropic2025biology,ameisen2025attribution}。

本研究の動機は、これら従来の XAI アプローチが直面する、「個別インスタンス中心の説明の限界」、「高コストな人手ラベリング依存」、そして\textbf{「発見された内部構造への命名の自動化の欠如」という 3 つの課題の交差点に存在する}点にある~\cite{bordt2022posthoc,kim2018interpretability,ameisen2025attribution}。信頼性が高く、低コストで、スケーラブルな解釈可能性を実現するためには、モデルの内部状態に基づき、かつ集合間の差分を自然言語で記述する自動命名手法が不可欠となる。

本研究が取り組む主要な課題は、非教師ありコンセプト抽出などで発見されたモデル内部の解釈可能な構造、すなわちニューロン発火条件の対比因子ラベルを、2 つのテキスト集合(発火群 A vs 非発火群 B)の差分からスケーラブルに自動生成することである~\cite{schrodi2024unsupervised,stein2024towards}。このタスクは、従来の XAI 研究における「概念の発見」と「命名」を統合する初の試みであると位置づけられる~\cite{schrodi2024unsupervised,stein2024towards}。

従来の非教師ありコンセプト発見手法(UCBM や CCE など)は、人間による事前定義なしにモデル内部から概念(潜在ベクトル)を抽出する点で大きな進歩を遂げた。しかし、これらの手法が発見するのは、あくまでも潜在的なベクトル表現であり、そのベクトルに人間が理解できる「自然言語の名前」を自動で付与する機能は欠如している。命名プロセスは、発火サンプルを見た研究者による手動分析に依存していた~\cite{schrodi2024unsupervised,stein2024towards}。

本研究は、この\textbf{「発見された概念の手動での意味付け」という C-XAI の最大のボトルネックを直接解消する}ことを目指す。具体的には、モデル内部の特定のニューロン(またはコンセプト)が強く発火したテキスト集合 A と、発火しなかったテキスト集合 B を入力とし、集合 A に特有で集合 B に見られない意味的差分を、自然言語ラベル $L$ として生成するタスクを定式化する。このタスクは、NLP における「コントラスティブ要約 (Contrastive Summarization)」~\cite{kardale2023contrastive,saha2024strumllm}の系譜に属するが、これを XAI ドメイン(モデル内部状態の解釈)に初めて適用する点で新規性を有する~\cite{kardale2023contrastive,saha2024strumllm}。

このタスクは、単一インスタンスに対する反事実的説明(例:CEM)とは異なり、集合間の一般的・代表的な差分を記述するものであり、ニューロンが何を計算しているかを集合レベルで説明することを可能にする。

本研究は、この高難度な「集合差分からの自然言語命名」タスクの解決策として、大規模言語モデル(LLM)の強力な文脈理解能力と生成能力を活用する。LLM(特に GPT-4o-mini)をコントラスト生成器として利用し、入力された A 群と B 群のテキスト群から、A 群に特徴的で B 群に欠如している意味的な側面を推論させ、簡潔なラベルを自動生成させる。

このアプローチは、人手によるアノテーションや複雑な事後分析のステップを排除し、非教師ありで抽出された概念(UCBM や Attribution Graphs が発見した特徴量)がもつ意味を、スケーラブルかつ自動で人間が理解できる「名前」として付与する命名モジュールとして機能する。この仮説の妥当性を、定量的な実験を通じて検証することが、本論文の主要な目標となる。

本研究は、LLM(GPT-4o-mini)を用いたコントラスティブ要約を核とする。手法の実現にあたっては、以下のステップを踏む。

\begin{enumerate}
  \item データ収集とグルーピング:モデル内部の特定のニューロンの発火度に基づき、発火が強いテキスト群 A と、発火が弱い(または非発火の)テキスト群 B を抽出する(グループサイズ \texttt{group\_size} はパラメータとして調整可能)。
  \item Few-shot ICL(In-Context Learning)によるプロンプト設計:LLM に A 群と B 群のテキストを入力として提示し、「A 群に特有の内容的・意味的差分」を抽出するよう指示する。Few-shot ICL は、本タスクにおける LLM の出力形式の揺らぎや語彙の安定性を確保するための\textbf{検証手段(サブ実験)}として導入する~\cite{saha2024strumllm}。
  \item 対比因子ラベルの生成:LLM に、集合差分を簡潔に要約した自然言語の「対比因子ラベル」を出力させる。
\end{enumerate}

本研究では、人手アノテーションされたアスペクトラベルを持つ SemEval-2014 Restaurant/Laptop データセットを評価ベンチマークとして使用し、LLM が生成したラベルと正解ラベルとの意味的類似性を評価した~\cite{pontiki-EtAl:2014:SemEval2014Task4}。

評価指標には、語彙的一致度を測る BLEU と、文脈的意味的類似度を測る BERTScore を採用した~\cite{papineni-etal-2002-bleu,zhang2019bertscore,reiter2018structured}。その結果、以下の重要な知見が得られた。

\begin{itemize}
  \item 意味的関連性の達成:生成されたラベルの品質を BERTScore で測定したところ、約 0.551 という中適度な意味的関連性を達成した~\cite{zhang2019bertscore}。この結果は、LLM がニューロンの発火群と非発火群という集合的な差分から、人間が理解できる意味的な核を抽出できること、すなわち「LLM の文脈理解能力を活用すれば、概念の発見と命名を同時に行うプロセスを実現できる」という主要な仮説(Main Hypothesis)が部分的に正しいことを示す。
  \item 語彙的一致の限界:一方、BLEU スコアは極めて低値(およそ 0.007)を示した~\cite{papineni-etal-2002-bleu,reiter2018structured}。これは、正解ラベルが「food」のような単一の単語であるのに対し、LLM が「食べ物の品質に関する言及」のような説明的なフレーズを生成するため、語彙的重複を測る BLEU が本タスクの性質に必ずしも適していないことを示唆している~\cite{reiter2018structured}。
  \item 概念の具体性による優位性:生成ラベルの品質は、「food」「price」のような語彙的に安定した具体的なアスペクトにおいて特に高い優位性を示すことが確認された~\cite{kim2018interpretability,schrodi2024unsupervised,stein2024towards}。対照的に、「story」「atmosphere」のような抽象的な概念の命名は、LLM が単純な要約ではなく高度な推論を必要とするため、性能が劣位となるという課題も特定された~\cite{kim2018interpretability,schrodi2024unsupervised,stein2024towards}。
\end{itemize}

本研究は、XAI と NLP の境界領域において、以下の点で重要な貢献を果たす。

第一に、新規タスクの提案と実現可能性の検証である。XAI における「概念の発見」(UCBM, CCE)と「概念の命名」が分離していた構造的な課題を認識し、LLM によるコントラスティブ要約によって、この二つのプロセスを統合する\textbf{「集合差分によるニューロン対比因子命名」}という新規タスクを提案し、その実現可能性を定量的に検証した~\cite{schrodi2024unsupervised,stein2024towards}。

第二に、XAI パラダイムのギャップ解消である。個別インスタンス中心の事後説明(LIME/SHAP)~\cite{ribeiro2016should,lundberg2017unified}の限界と、命名機能を持たない非教師ありコンセプト抽出(UCBM/CCE)~\cite{schrodi2024unsupervised,stein2024towards}のボトルネックを同時に解消する、スケーラブルなハイブリッドアプローチを提供する~\cite{schrodi2024unsupervised,stein2024towards}。

第三に、メカニスティック解釈の実用化への寄与である。Attribution Graphs~\cite{anthropic2025biology,ameisen2025attribution}などによって発見されたニューロン回路や特徴量に対し、人間が理解可能な意味的な名前を自動で付与する手段を提供し、手動ラベリングに依存していた MI 分野のボトルネック解消に貢献する~\cite{ameisen2025attribution}。

第四に、Few-shot ICL の応用と最適化に関する知見の提供である。Few-shot ICL を活用し、生成ラベルの品質とスタイルを矯正するアプローチを提案し、Few-shot 数の変動による影響を分析することで、LLM を用いた自動命名の安定性向上に向けた知見を提供する~\cite{saha2024strumllm}。


\chapter{関連研究}

本研究は,深層学習モデルの解釈可能性(XAI)において,ニューロンの発火条件を自然言語で自動命名するという,従来の手法が満たせていなかったギャップを埋めることを目的としている.
本章では,関連する先行研究を体系的に整理し,それぞれの限界を明確にすることで,本研究が提案する「集合差分による自動命名」の新規性を位置づける.
本研究は,(1) 個別インスタンス中心の事後説明手法,(2) 概念の発見に留まる非教師ありコンセプト抽出,(3) 一般的な NLP タスクに特化していたコントラスティブ要約,という三つの主要な研究領域の交差点に位置づけられる.

\section{従来の XAI 手法:個別インスタンス中心の解釈}

深層学習モデルのブラックボックス性に対処するため,初期の XAI 研究は主に,特定の予測に対する入力特徴量の寄与度を事後的に(Post-hoc)説明する手法に焦点を当ててきた.

\subsection{特徴量帰属手法の個別性}

最も広く採用されてきた手法として,LIME(Local Interpretable Model-agnostic Explanations, Ribeiro et al., 2016)~\cite{ribeiro2016should}や SHAP(SHapley Additive exPlanations, Lundberg \& Lee, 2017)~\cite{lundberg2017unified}が挙げられる.
LIME は,ターゲットとする予測の周囲でデータを摂動させ,局所的に解釈可能な代理モデル(サロゲートモデル)を構築することで,個々のインスタンスの予測根拠を可視化する.
SHAP は,ゲーム理論に基づく Shapley 値を用いて,特徴量の貢献度を統一的な枠組みで定量化する.

これらの特徴量帰属手法は,その性質上,個別インスタンスの局所的な説明に特化している.
LIME や SHAP の出力は,画像における「重要なピクセル」やテキストにおける「重要な単語」といった低レベルな特徴量の寄与スコアであり,モデルが特定のデータ集合に対して系統的に何を学習しているかを,人間が理解できる自然言語の概念(コンセプト)として説明する能力を欠いている.
本研究の目的は,特定のニューロンが発火する集合 A と発火しない集合 B の間の意味的な差分を抽出することであり,個別予測の寄与度を問う従来の XAI 手法では,このグループレベルの差異の説明というタスクを満たすことができない.

さらに,事後説明手法は信頼性の問題も指摘されている.
特に Bordt et al. (2022)~\cite{bordt2022posthoc}は,事後説明が本質的な曖昧さを持ち,敵対的な文脈において容易に操作可能であるため,法的・倫理的な透明性の目的を達成する上で「不適切 (unsuitable)」であると結論付けている.
本研究がモデルの内部状態(ニューロン発火)に直接着目する動機の一つは,この事後説明の曖昧さを回避し,より忠実な説明を提供することにある.

\subsection{反事実的説明の限界}

反事実的説明(Counterfactual Explanation)もまた,対比的(Contrastive)な要素を持つ XAI 手法として注目を集めてきた~\cite{wachter2017counterfactual}.
CEM(Contrastive Explanations Method, Dhurandhar et al., 2018)や Wachter et al. (2017) の手法は,「もし入力 $x$ の一部が $x'$ に変わったら,予測 $y$ はどう変わるか」という,単一事例に対する最小限の入力変更を特定する.
これにより,ユーザーは「この予測を覆すために何をすべきか」という行動可能な洞察を得る.

反事実的説明は,その定義上,単一インスタンスの局所的な反転に限定される.
本研究が対象とするのは,特定のニューロンの発火パターンが,データ集合全体でどのような意味的な特性を持つかを記述すること,すなわち集合間の一般的・代表的な差分を自然言語で要約するタスクである.
反事実的説明は,この集合レベルでのコントラスト記述という目標を達成できない.

\section{非教師ありコンセプト発見と命名の課題}

事後説明の限界を受け,XAI 研究の焦点は,低レベルな特徴量から,人間が理解できる高レベルの「コンセプト(概念)」に基づいた説明へと移行した~\cite{kim2018interpretability}.
コンセプトベース XAI(C-XAI)の代表例である TCAV(Kim et al., 2018)は,概念活性化ベクトルを用いてモデルが特定の概念にどれだけ敏感であるかを定量化する.

\subsection{高コストな人手ラベリング依存}

従来の C-XAI は,解釈の基礎となる概念の定義(ラベリング)を人間に依存しているという構造的な問題を抱えてきた~\cite{kim2018interpretability}.
このプロセスは,同じく Kim らによって「高コストな概念キュレーション (expensive concept curation)」と呼ばれ,特に医療や科学といった専門知識が必要なドメインでの XAI 導入の最大の障壁となっている.
SemEval-2014 のような既存の ABSA ベンチマークでさえ,アスペクト(概念)ラベルは人手によるアノテーションに依存している~\cite{pontiki-EtAl:2014:SemEval2014Task4}.

\subsection{非教師あり概念抽出の進歩と「命名」の欠如}

近年,この人手依存の課題を克服するため,非教師ありでモデル内部から概念を抽出する手法が大きく進展している~\cite{schrodi2024unsupervised,stein2024towards}.
Unsupervised CBM (UCBM, Schrodi et al., 2024) は,人間による事前定義なしに,モデルの内部表現から概念(潜在ベクトル)を自動抽出することを可能にした.
また,Compositional Concept Extraction (CCE, Stein et al., 2024)~\cite{stein2024towards}は,より構成的な概念の表現を抽出する.

UCBM や CCE の進歩にもかかわらず,これらの手法が「発見」するのは,依然として潜在的なベクトル表現である.
そのベクトルが人間にとって何を意味するのか(例:「コンセプト X」が「縞模様」を意味すること)という自然言語での「命名(ラベリング)」機能は,これらの研究には含まれていない.
命名プロセスは,発見された概念が強く発火するサンプルを研究者が手動で分析し,ラベルを付与するという,非教師あり C-XAI の「最後のワンマイル問題」として残されていた~\cite{schrodi2024unsupervised}.

本研究は,この手動命名というボトルネックに対し,UCBM や CCE が発見した概念(に対応する発火/非発火サンプル群)を LLM に入力し,その意味を直接「対比因子ラベル」として生成するスケーラブルな自動命名モジュールとして機能する.

\subsection{メカニスティック解釈における手動作業}

最先端のメカニスティック解釈(Mechanistic Interpretability, MI)分野においても,同様の課題が存在する.
Anthropic による Attribution Graphs(2025)~\cite{anthropic2025biology}は,LLM(例:Claude 3.5 Haiku)の内部計算プロセスをトレースし,特徴量間の相互作用をグラフとして可視化することで回路を発見する.
しかし,同研究においても,グラフのノードとして発見される個々の「特徴量」の意味付け,すなわちノードが具体的に何を検出しているかを自然言語で特定するプロセスは自動化されていない.

同論文の著者ら自身も,解釈を容易にするために,関連する意味を持つ特徴量を手動で「スーパーノード」としてグループ化しており,この手動ステップが「労働集約的 (labor-intensive)」であり,情報の欠損を引き起こすと認めている~\cite{ameisen2025attribution}.
本研究の提案手法は,この MI によって発見された特徴量や回路に対し,その発火条件の差分に基づき,人間が理解できる対比因子ラベルを自動で付与する手段を提供し,MI の実用化と統合的 XAI の実現に寄与する.

\section{LLM を用いた自動ラベル生成とコントラスティブ要約}

大規模言語モデル(LLM)は,その強力な文脈理解能力と自然言語生成能力により,様々なラベリングや要約タスクに応用されている.

\subsection{LLM による命名と正規化}

LLM は,テキストクラスタリングの結果に対して,クラスタ内のサンプル群の共通するテーマを要約し,クラスタラベルを自動生成するために活用されている~\cite{wang2024llmcluster}.
これにより,Embedding ベースのクラスタリングを Few-shot 学習を用いた分類タスクに変換する新たなパラダイムが提案されている.
また,アスペクトベースド感情分析(ABSA)の領域においても,LLM は Few-shot プロンプトを用いてアスペクトの抽出や,ノイズの多い合成ラベルの正規化(Normalization)に利用されている~\cite{luo2024chatabsa}.

\subsection{本研究のタスクとの乖離}

LLM による従来の命名研究の多くは,単一のデータ集合(クラスタ)内の共通点を記述することに焦点を当てている~\cite{wang2024llmcluster}.
一方,本研究は,ニューロンの発火群 A と非発火群 B の間の差分,すなわちコントラスト(対比)を記述することを目的としている.
また,ABSA における LLM の適用例(例:ChatABSA)は,主に教師ありの環境下で特定のアスペクトを抽出するタスクであり,非教師ありのコントラスティブな設定で未知の概念を自動発見・命名する本研究とはタスク設定が異なる~\cite{luo2024chatabsa}.

\subsection{コントラスティブ要約との関係}

本研究のタスク設定に概念的に最も近いのは,自然言語処理(NLP)分野における「コントラスティブ要約 (Contrastive Summarization)」または「グループ差分要約 (Group-Difference Summarization)」の系譜である~\cite{kardale2023contrastive}.
このタスクは,2 つ以上の文書群を比較し,そのうちの 1 つの集合に特有で,かつ関連性の高い差異をハイライトする要約を生成することを目的とする.

LLM を用いた先行研究として,STRUM-LLM (Saha et al., 2024)~\cite{saha2024strumllm}が挙げられる.
これは,2 つの比較対象(例:製品 A vs 製品 B)の差分を,LLM を用いた多段階パイプラインで属性付きの構造化要約として生成する.

また,Luss et al. (2024)~\cite{luss2024cell}は,CELL(Contrastive Explanations for Large Language Models)を提案し,LLM の出力に対する対比的な説明(なぜその出力が選ばれ,他の出力が選ばれなかったのか)を生成するという点で,本研究と問題意識を共有している.
一方で,CELL はあくまで個々の出力インスタンスに対する説明に焦点を当てており,本研究が対象とするような,ニューロン発火群 $A$ と非発火群 $B$ の\textbf{集合レベルの一般的・代表的な差分}を自然言語ラベルとして要約するタスクとは目的と対象が異なる.

\subsection{本研究の新規性}

STRUM-LLM は,Web 検索を含む複雑なパイプラインを構築し,一般的な製品比較タスクに特化している.
対照的に,本研究は,この「コントラスティブ要約」のフレームワークを XAI ドメイン(モデル内部状態)に初めて適用する点に新規性がある.
本研究は,ニューロンの発火群と非発火群という,より抽象的なテキスト集合の差分抽出に対し,Few-shot プロンプティングという,より簡潔なアプローチで,スケーラブルな自然言語ラベリングの実現可能性を検証するものである.
これは,LLM 命名の知見と,コントラスティブ要約のタスク定義を,XAI の文脈で融合した新規な交差点に位置づけられる.

本研究は,個別インスタンスの説明に留まる LIME/SHAP の限界,命名の課題を残す UCBM/CCE のボトルネック,そして一般 NLP タスクに留まっていたコントラスティブ要約の知見を統合することで,スケーラブルな内部状態の説明を実現する新たな手法を提案するものである.


\chapter{提案手法}

本章では,大規模言語モデル(LLM)を用いて,ニューラルネットワークの特定の内部状態(ニューロンの発火条件)に対応するテキスト集合間の意味的差異を,自然言語の「対比因子ラベル」として自動生成する手法を提案する.本手法は,既存のコンセプトベース XAI(C-XAI)が抱える「人手による命名依存」という最大のボトルネックを解消することを目的としており,ニューロン発火条件の集合的差分を自然言語で記述するタスク設定に特化している.本論文では,集合 $A$ と $B$ の差分を表現するこの自然言語ラベルを一貫して「対比因子ラベル」と呼ぶ.

\section{対比因子命名タスクの定式化}

本研究が提案する対比因子命名タスクは,従来の XAI 手法(個別インスタンスの説明や概念分類)とは根本的に異なるタスクを定式化する~\cite{ribeiro2016should,kim2018interpretability}.

\subsection{タスクの定義}

モデル内部の特定のニューロン $N$(または非教師あり抽出されたコンセプト $C$)に着目する.本研究では,モデル内部で抽出されるこれらの潜在特徴を総称して「ニューロン(またはコンセプト)」と呼び,一方で SemEval などのデータセット側の人手ラベルを「アスペクト」と呼ぶ.学習データセット $\mathcal{D}$ から,ニューロン $N$ が強く活性化する入力テキストの集合を $A$(発火群),そうではない,あるいは対象とするアスペクトを含まない入力テキストの集合を $B$(非発火群)とする.以下,数学的定式化では「集合 $A$」「集合 $B$」と表記し,プロンプトや実験手順の記述では「グループA」「グループB」と表記する.

このタスクの目的は,集合 $A$ に含まれるテキスト群の意味的・内容的な特徴のうち,集合 $B$ には含まれない差分を抽出・推論し,それを簡潔な自然言語ラベル $L$ として自動的に生成することである.

ここで,
\[
 A = \{ x \in \mathcal{D} \mid \text{activation}(x, N) > \tau_A \}, \quad
 B = \{ x \in \mathcal{D} \mid \text{activation}(x, N) < \tau_B \}
\]
と定義し,$\tau_A > \tau_B$ とする.実験章では,この定義をアスペクトラベルや類似度スコアに基づくグループ A/B の抽出として具体化する.$L$ は,この差分を要約する自然言語フレーズ(例:「価格に関する言及」)である.

このタスク全体は,集合入力からラベルへの写像
\[
 (A, B) \mapsto L
\]
として表される.以下では,この写像を便宜的に \texttt{textContrastiveNaming} と呼ぶ.

\subsection{集合差分によるコントラストの必要性}

上記の定式化は,以下の点で従来の XAI と一線を画す.

\begin{enumerate}
 \item \textbf{個別性からの脱却}:
 従来の LIME や SHAP は,単一のインスタンス $x$ に対して,その予測 $y$ に寄与した特徴量(ピクセル,トークン)を可視化するものである.一方,本タスクは,集合 $A$ と $B$ の間の一般的・代表的な意味的コントラストを抽出することに焦点を当てる.
 \item \textbf{潜在表現への命名}:
 Unsupervised CBM(UCBM)や Compositional Concept Extraction(CCE)といった非教師あり概念抽出手法は,概念を潜在ベクトル $v$ として発見するが,そのベクトルに自然言語ラベル $L$ を付与する機能を持たない.本定式化 $\text{textContrastiveNaming}(A,B) \to L$ は,UCBM などが抽出した潜在概念(ニューロンの発火条件)に対し,自動で人間が理解できる「名前」を付与する命名モジュールとして機能する.
\end{enumerate}

このように,対比因子命名タスクは,ニューロンが何を計算しているかというメカニズムの理解を,集合間の意味的なコントラストとして自然言語化する,新しい XAI のタスクである.

\section{LLM によるコントラスティブ要約の実行}

本研究では,定式化された対比因子命名タスクを,大規模言語モデル(LLM)の強力な文脈理解能力と自然言語生成能力を活用して解決する.LLM(本実験では GPT-4o-mini を想定)を,集合 $A$ と $B$ の差分を推論し,ラベルを生成するコントラスト生成器として利用する.

\subsection{処理フロー}

提案手法は,概念図(図~\ref{fig:proposal-overview})に対応する,以下の 3 段階の処理フローを持つ.図が参照できない場合でも,以下の記述のみで流れが理解できるように構成する.

\begin{enumerate}
 \item \textbf{データ抽出とグルーピング(Activation Extraction and Grouping)}:
 解釈対象のニューラルネットワーク $M$ と,特定のニューロン $N$ を選択する.評価データセット $\mathcal{D}$ の各テキスト $x$ を $M$ に入力し,ニューロン $N$ の活性化値 $\text{activation}(x, N)$ を測定する.測定された活性化値に基づき,ハイパーパラメータ \texttt{group\_size} を用いて,活性化値が最も高いテキスト群 $A$ と,活性化値が最も低い(またはランダムな)テキスト群 $B$ を抽出する.すなわち,
 \[
   A = \{ x_1, \dots, x_k \}, \quad
   B = \{ x'_1, \dots, x'_k \}, \quad
   k = \texttt{group\_size}
 \]
 とする.\texttt{group\_size} は,プロンプトのコンテキスト長制限やニューロン活性化のスパース性に応じて決定される.第4章のグループサイズ比較実験では,group\_size を 50〜300 の範囲で変化させても BERTScore の変動は小さく,性能に与える影響は限定的であることが確認された.そのため,本研究ではコンテキスト長と計算量のトレードオフを踏まえ,デフォルト値として \texttt{group\_size} = 100 を採用する.
 \item \textbf{プロンプト設計と差分推論(Prompt Engineering and Contrast Inference)}:
 抽出されたテキスト集合 $A$ と $B$ の内容を,LLM の入力プロンプトに組み込む.プロンプトは,LLM に対し,単なる要約ではなく「グループ $A$ に特徴的でグループ $B$ には見られない主要な違いを特定し,簡潔に回答する」というコントラスティブな推論タスクとして明確に指示する.
 特に,実験で用いるプロンプトは第4章の実験設定と整合するよう,次の要素から構成される.
 \begin{itemize}
  \item \textbf{タスク説明}:まず「2つのデータグループを比較して,グループAに特徴的でグループBには見られない表現パターンや内容の特徴を特定してください」といった指示文を提示する.
  \item \textbf{Few-shot 例(オプション)}:Few-shot 設定が 0 より大きい場合,\texttt{examples\_section} に「【例題$N$】グループA: [...] グループB: [...] 回答: [正解ラベル]」という形式の例を 1 件または 3 件挿入する.
  \item \textbf{集合 $A$ のテキスト群}:【グループA】の見出しの下に,各テキストを「- [テキスト内容]」の形式で最大 \texttt{group\_size} 件列挙する.
  \item \textbf{集合 $B$ のテキスト群}:【グループB】の見出しの下に,同様の形式でテキストを列挙する.
  \item \textbf{出力制約}:プロンプト末尾で「英語で5-10単語程度で,グループAに特徴的でグループBには見られない主要な違いを簡潔に回答してください」と指示し,短い対比因子ラベルの生成を求める.
 \end{itemize}
 LLM は,このプロンプトを入力として受け取り,集合 $A$ のテキスト群には頻出するが,集合 $B$ のテキスト群には見られない語彙,文脈,意味的構造を推論する.この推論能力が,人間による手動分析なしに意味的な差分抽出を可能にする鍵となる.
 \item \textbf{対比因子ラベルの生成(Contrastive Factor Label Generation)}:
 LLM は,推論結果に基づき,集合 $A$ の意味的特性を簡潔に表現した自然言語ラベル $L$ を生成する.例えば,集合 $A$ が「価格が高すぎる」といったレビューを含み,集合 $B$ がレビューを含むが価格には言及しない場合,生成されるラベル $L$ は「価格に関する言及」となる.
\end{enumerate}

生成された対比因子ラベル $L$ の妥当性は,第4章で述べるように,主に BERTScore を主要指標,BLEU を参考指標,さらに LLM による意味的類似度評価を補助指標として定量的に評価する.本章で述べたコントラスティブ要約の枠組みは,第2章で整理した既存のコントラスティブ要約研究(例:STRUM-LLM)を,モデル内部状態の解釈という XAI 文脈に適用したものに相当する.

\begin{figure}[htbp]
  \centering
  % \includegraphics[width=0.8\linewidth]{fig_proposal_overview.png}
  \caption{提案手法の概念図(データ抽出・プロンプト設計・ラベル生成の 3 段階)}
  \label{fig:proposal-overview}
\end{figure}

\section{Few-shot ICL による出力安定性の検証}

本研究では,大規模言語モデルの活用において,Few-shot インコンテキスト・ラーニング(ICL)を,生成される対比因子ラベルの出力形式と安定性を制御するための補助的な手段として導入する.同時に,第4章で示すように,Few-shot 設定とくに 1-shot は BERTScore の向上にも寄与しており,単なる検証手段にとどまらず性能面でも重要な役割を果たす.

\subsection{ICL の役割:安定性とスタイルの確保}

LLM の Few-shot ICL は,プロンプト内にタスクの入力と出力の例(デモンストレーション)を含めることで,モデルがタスクの形式や文体,語彙の傾向を模倣する特性を持つ.この特性を活用し,本研究では,生成される「対比因子ラベル」の出力形式の揺らぎや語彙の安定性を確保するために ICL を用いる.

特に,SemEval-2014 データセットの正解ラベル(例:「food」「price」)は単語や簡潔なフレーズであることが多いため,LLM が出力するラベルをこの正解ラベルのスタイルに近づけ,比較可能性を高めるために ICL を検証する.

\subsection{Few-shot バリエーションの検証}

実験では,Few-shot ICL のバリエーションとして,以下の設定を定量的に検証する.

\begin{enumerate}
 \item 0-shot(Zero-shot):プロンプトにデモンストレーションを含めず,指示文のみで LLM にラベル生成を要求する.これにより,LLM がタスク定義のみに基づいてどれだけ命名できるか,そのベースライン性能を測定する.
 \item 1-shot(One-shot):プロンプトに,正解ラベルが既知の $(A_{\mathrm{ex}}, B_{\mathrm{ex}}, L_{\mathrm{ex}})$ のペアを 1 組含める.
 \item 3-shot(Three-shot):プロンプトに,正解ラベルが既知の $(A_{\mathrm{ex}}, B_{\mathrm{ex}}, L_{\mathrm{ex}})$ のペアを 3 組含める.
\end{enumerate}
Few-shot 例 $(A_{\mathrm{ex}}, B_{\mathrm{ex}}, L_{\mathrm{ex}})$ は,各データセットのアスペクトラベルに基づき,正解ラベルが明確で代表性の高い組み合わせから選定する.例の形式は「【例題$N$】グループA: [...] グループB: [...] 回答: [正解ラベル]」とし,タスク説明の直後に \texttt{examples\_section} として挿入する.

\subsection{検証の目的}

Few-shot ICL は,LLM の生成が入力例のスタイルまで強く模倣する特性を持つため,生成ラベルが「food」という単語(語彙的安定性が高い)にどの程度収束するか,あるいは「食べ物の品質に関する言及」といった説明的なフレーズ(語彙的多様性が高い)になるか,という出力スタイルの影響を定量的に分析する目的で実施される.

Few-shot 実験の詳細な結果は第4章で報告するが,Steam データセットにおける実験では,1-shot 設定が平均 BERTScore 0.6530 と最も高い意味的関連性を示した.この結果は,1-shot の Few-shot 例がラベルのスタイルと意味的妥当性の両面でバランスが良いことを示唆しており,本研究におけるデフォルト設定の根拠となる.

\chapter{評価実験と結果}

\section{実験の目的と概要}
本章では、提案する大規模言語モデル(LLM)を用いた対比因子ラベル自動生成手法の有効性と汎用性を定量的に検証した結果を報告する。
本実験は、モデル内部の特定のニューロン発火条件に対応する集合的な意味的差分を、LLM(GPT-4o-mini 等)によるコントラスティブ要約によって自然言語ラベルとして抽出可能であるか、また生成されたラベルが人手アノテーションによる正解ラベルとどの程度意味的に類似するかを、定量的指標および LLM による評価を用いて客観的に測定することを目的とする。

提案手法のドメイン汎用性を検証するために、レビュー、感情分類、画像キャプションという多様なドメインに属する 4 種類のデータセットを用いた。
これらのデータセットは、それぞれが持つ正解アスペクトラベルを、LLM が生成する対比因子ラベルの妥当性を評価するためのグラウンド・トゥルースとして使用した。

% TODO: 実験の全体像(実験カテゴリの概要)
% TODO: 評価の観点(有効性、汎用性、性能など)

\section{データセット}

\subsection{データセットの選定理由}
% TODO: なぜこれらのデータセットを選んだか
% TODO: ドメインの多様性の説明

\subsection{Steam Review Aspect Dataset}
Steam Review Aspect Dataset は、Steam ゲームレビューから収集されたテキストデータであり、特定のゲームアスペクトに関する言及を含むデータセットである \mbox{[6, 8]}。

\begin{itemize}
  \item \textbf{概要}: 英語の Steam ゲームレビュー 1,100 件(学習用 900 件、テスト用 200 件)で構成される。
  データ収集は SRec データベースのスナップショットに基づき行われた。
  \item \textbf{アスペクト}: レビューを特徴づける 8 種類のアスペクトが人手でアノテーションされている。
  内訳は Recommended(推奨)、Story(物語)、Gameplay(ゲームプレイ)、Visual(視覚)、Audio(聴覚)、Technical(技術)、Price(価格)、Suggestion(提案・要望)である。
  \item \textbf{特徴}: ゲームという特定の製品ドメインに特化しており、特に「Gameplay」や「Technical」といったゲーム固有のメカニクスや技術的側面に関するアスペクトを含む。
  テストセットにおけるアスペクト件数の例として、Gameplay が 154 件、Recommended が 148 件、Story が 89 件である。
  これにより、LLM が専門性の高いテキスト集合間の意味的差分を抽出する能力を検証するためのベンチマークとして機能する。
\end{itemize}

% TODO: 実験での使用方法(グループA/Bの作成方法)
% TODO: 使用したアスペクトとその理由

\subsection{SemEval-2014 ABSA(Restaurants)}
SemEval-2014 ABSA(Restaurants)は、アスペクトベース感情分析(ABSA)の標準的なベンチマークとして広く使用されるレストランレビューのデータセットである \mbox{[2, 6, 12--14]}。

\begin{itemize}
  \item \textbf{概要}: レストランレビューのテキストを含み、各文に対してアスペクト(観点)とそれに対する感情極性が人手でアノテーションされている。
  \item \textbf{アスペクト}: 本研究では、主に Food(食べ物)、Service(サービス)、Price(価格)、Atmosphere(雰囲気)の 4 種類のアスペクトを採用した。
  \item \textbf{特徴}: このデータセットは、LLM による自動命名の性能が概念の具体性に依存するかを検証するための鍵となるデータを含む。
  「Food」や「Price」は具体的な名詞や数値に関連する言及が中心となる具体的なアスペクトである一方、「Atmosphere」は広範な文脈や比喩的表現からの高度な推論を必要とする抽象的なアスペクトに分類される。
  本研究の主要な定量評価ベンチマークとして使用された。
\end{itemize}

% TODO: 実験での使用方法
% TODO: 使用したアスペクトとその理由

\subsection{GoEmotions}
GoEmotions は、細粒度感情分類タスクのために Reddit コメントから収集されたデータセットである \mbox{[7, 19, 20]}。

\begin{itemize}
  \item \textbf{概要}: Demszky らによって構築されたもので、総レコード数 63,812 件から成るマルチラベル形式のデータセットである。
  \item \textbf{アスペクト}: 28 の感情カテゴリ(27 感情 + neutral)でラベル付けされている。
  主要なカテゴリには Joy(喜び)、Anger(怒り)、Admiration(称賛)、Neutral(中立)などが含まれる。
  \item \textbf{特徴}: 感情という極めて抽象的な概念を対比因子として扱えるかを検証するための挑戦的なデータセットとして位置づけられる。
  このデータセットでは、感情という内在的な状態をテキストの集合差分から推論する必要があり、具体的な物理的実体を持たない概念の命名精度を測るために使用された。
  実験では、任意の感情アスペクト(例:joy)を指定し、「その感情を含むテキスト群 $A$」と「その他のアスペクトを含むテキスト群 $B$」を比較する設定が採用された。
\end{itemize}

% TODO: 実験での使用方法
% TODO: 28感情カテゴリの選定理由

\subsection{Retrieved Concepts(COCO Captions)}
Retrieved Concepts(COCO Captions)は、視覚的概念記述の生成能力を検証するために、画像キャプションデータセット COCO に基づいて構築されたデータセットである \mbox{[7]}。

\begin{itemize}
  \item \textbf{概要}: COCO Captions に基づく 300 の概念(concept\_0 ~ concept\_299)を扱う。
  データには、Top-100/Bottom-100 の類似度順キャプションデータが含まれる。
  \item \textbf{特徴}: LLM がテキストの集合差分から視覚的な特徴を抽象化した概念記述を生成できるかを確認するために使用された。
  この検証は、本手法が将来的に画像モデルのニューロン解釈(例:縞模様、空の色といった視覚的概念)に適用可能であるかを探るための基礎的なデータを提供する。
\end{itemize}

% TODO: 実験での使用方法
% TODO: Top-100/Bottom-100の選定方法
% TODO: 正解ラベルがない場合の評価方法

\section{実験設定}

\subsection{実験パイプラインの概要}
本実験では、コントラスティブ要約に基づく対比因子ラベル生成器として、GPT-4o-mini を含む複数の大規模言語モデルを採用した。
提案手法は、統一されたパイプラインとして構築され、その目的は、特定の概念に対応するテキスト集合 $A$(特徴あり群)と、そうでないテキスト集合 $B$(特徴なし群)の差分から、意味的な対比因子ラベル $L$ を LLM に生成させることである。

\begin{itemize}
  \item \textbf{タスク定式化}: ニューロン $N$ が強く活性化するテキスト集合 $A$ と、活性化しない集合 $B$ を入力とし、集合 $A$ に特有で $B$ には見られない意味的差分を $L$ として生成する写像 $(A, B) \to L$ を定式化した。
  \item \textbf{グルーピング}: 活性化値に基づき、ハイパーパラメータ $group\_size$ を用いて集合 $A$ と $B$ を抽出する。
  メイン実験では、プロンプトのコンテキスト長制限を考慮し、$group\_size = 100$ を採用した。
  \item \textbf{Few-shot ICL の検証}: LLM の出力形式の揺らぎや語彙の安定性を確保するための検証手段として、Few-shot インコンテキスト・ラーニング(ICL)のバリエーション(0-shot, 1-shot, 3-shot)を定量的に検証した \mbox{[2, 32--34]}。
  LLM は、プロンプト内で集合 $A$ と $B$ を比較し、$A$ に特徴的で $B$ には欠如している意味的側面を推論するよう指示された。
\end{itemize}

% TODO: パイプラインの全体フロー図(オプション)

\subsection{LLMモデルとパラメータ設定}
% TODO: 使用モデル(GPT-4o-mini等)の詳細
% TODO: 各実験カテゴリでのモデル選択理由
% TODO: 温度パラメータ(temperature)の設定と理由
% TODO: 最大トークン数(max_tokens)の設定と理由
% TODO: その他の生成パラメータ

\subsection{プロンプト設計}
% TODO: プロンプトテンプレートの構造
% TODO: グループA/Bの提示方法
% TODO: 出力形式の指示方法
% TODO: Few-shot例の挿入方法
% TODO: アスペクト説明文の使用方法(該当実験のみ)
% TODO: プロンプトの具体例

\subsection{データの前処理と分割方法}
% TODO: テキストの前処理手順
% TODO: グループA/Bの抽出方法
  % TODO: グループサイズ(group_size)の決定方法
  % TODO: サンプリング方法(ランダム/順序/重み付き)
% TODO: 分割タイプの説明(aspect_vs_others等)
% TODO: コンテキスト長制限への対応

\subsection{Few-shot例の作成方法}
% TODO: Few-shot例の選定基準
% TODO: 例の品質管理方法
% TODO: 0-shot, 1-shot, 3-shotの違いと設定方法

\subsection{実験カテゴリの定義}
% TODO: メイン実験の定義と目的
% TODO: Few-shot実験の定義と目的
% TODO: グループサイズ比較実験の定義と目的
% TODO: モデル比較実験の定義と目的
% TODO: アスペクト説明文比較実験の定義と目的
% TODO: COCO実験の定義と目的

\subsection{比較手法とベンチマーク}
本研究は、非教師ありコンセプト抽出(UCBM や CCE など)が発見した潜在ベクトルに名前を付与する「命名モジュール」としての機能に特化している \mbox{[29, 35]}。
そのため、生成されたラベルの品質を、人手アノテーションされた既存の ABSA ベンチマーク(SemEval-2014 Restaurant/Laptop、Steam レビューなど)の正解ラベルとの意味的類似性と比較することで評価した。
このアプローチにより、本手法の命名性能が、高コストな人手ラベリングによって確立された基準に対してどの程度妥当であるかを検証した。

\section{評価指標}
生成された自然言語ラベル $L$ の品質を評価するために、BERTScore、BLEU、および LLM による意味的類似度評価を採用した。

\subsection{評価指標の選定理由}
% TODO: なぜBERTScore, BLEU, LLM評価を選んだか
% TODO: 各指標の役割と位置づけ

\subsection{BERTScore}
\begin{itemize}
  \item \textbf{定義と役割}: BERTScore は、生成されたラベル $L$ と人手アノテーションされた正解ラベル $L_{\mathrm{ref}}$(またはその説明文)との間の文脈的意味的な類似性を測る主要な指標として採用された \mbox{[1, 2, 17, 37]}。
  この指標は、BERT などの事前学習済み言語モデルによって得られる文脈化埋め込み表現のコサイン類似度に基づき、語彙レベルの一致度を超えたセマンティックな評価を提供する。
  \item \textbf{位置づけ}: 本タスクにおいては、LLM が集合差分という複雑な推論タスクの結果を要約した自然言語フレーズを生成するため、意味的な妥当性を定量的に示す BERTScore が最も重要な評価基準として位置づけられた。
\end{itemize}

% TODO: 計算方法の詳細(モデル、正規化方法)
% TODO: 評価範囲と解釈方法

\subsection{BLEU(Bilingual Evaluation Understudy)}
\begin{itemize}
  \item \textbf{定義と役割}: BLEU は、生成ラベルと正解ラベルとの間の語彙レベルの一致度(n-gram の重複)を確認するために補助的に使用された。
  \item \textbf{本タスクにおける制約}: 本タスクの性質上、BLEU スコアは極めて低い値を示すことが前提とされた。
  これは、正解ラベルが「food」「price」のような単一の言葉または簡潔なフレーズであるのに対し、
  LLM が生成するラベルはしばしば「食べ物の品質に関する言及」「価格設定の側面」といった説明的なフレーズとなるため、語彙的な重複(n-gram overlap)が本質的に生じにくいためである。
  したがって、BLEU スコアの低さはモデルの命名失敗を意味するものではなく、単に語彙的一致度を測る指標が本タスクの性質に適合していないことを示す参考値として扱われた。
\end{itemize}

% TODO: 計算方法の詳細(スムージング関数等)
% TODO: 低値が予想される理由の詳細

\subsection{LLM評価スコア}
% TODO: LLM評価の目的と位置づけ
% TODO: 評価に使用するモデル
% TODO: 評価プロンプトの設計
% TODO: 評価基準(5段階評価の詳細)
% TODO: 正規化方法
% TODO: BERTScoreとの関係

\section{実験結果}

\subsection{メイン実験結果}
% TODO: 実験設定の詳細(パラメータ一覧)
% TODO: 総実験数と成功/失敗数
% TODO: データセット別の結果
  % TODO: SemEvalの結果(平均、最小、最大)
  % TODO: GoEmotionsの結果
  % TODO: Steamの結果
% TODO: アスペクト別の結果(主要なアスペクト)
% TODO: 主要な発見と考察

\subsection{主要実験結果:自動命名の定量分析}
SemEval-2014 データセット(レストランレビュー)における Few-shot 設定ごとの平均 BERTScore および BLEU スコアは、以下の定量的な傾向を示した。

\begin{itemize}
  \item \textbf{BERTScore の達成値}: LLM によるコントラスティブ要約の結果、生成された対比因子ラベルは、正解ラベルとの間で平均約 0.551 という中適度な意味的関連性を達成した。
  この値は、LLM がニューロンの発火群と非発火群というテキストの集合差分から、その集合の本質的な意味的核を抽出できる能力を有していることを示唆する。
  \item \textbf{Few-shot ICL の影響}: Few-shot ICL のバリエーション(0-shot, 1-shot, 3-shot)を比較した結果、1-shot 設定が他の設定と比較して最も高い BERTScore を示す傾向が観測された。
  これは、LLM がタスクの定義と出力スタイルを学習する上で、少数の適切に選定された例(1 組)が最も効率的かつ効果的に機能することを示している。
  \item \textbf{BLEU スコアの傾向}: 一方、BLEU スコアは全ての Few-shot 設定において極めて低値(平均約 0.007)を示した。
  この低値は、生成ラベルと正解ラベルとの間で語彙的な重複がほとんど存在しないという、本タスクの性質を裏付ける結果である。
\end{itemize}

\subsection{Few-shot設定による性能比較}
% TODO: 実験設定(0-shot, 1-shot, 3-shot)
% TODO: Few-shot別の平均スコア
% TODO: アスペクト別のFew-shot効果
% TODO: 最適なFew-shot設定の考察

\subsection{グループサイズの影響分析}
% TODO: 実験設定(50, 100, 150, 200, 300)
% TODO: グループサイズ別の性能
% TODO: 最適なグループサイズの考察
% TODO: コンテキスト長との関係

\subsection{モデル比較実験}
% TODO: GPT-4o-mini vs GPT-5.1の比較
% TODO: モデル別の性能差
% TODO: アスペクトによる性能差の違い
% TODO: モデル選択の示唆

\subsection{アスペクト説明文の効果検証}
% TODO: 実験設定(説明文あり/なし)
% TODO: 説明文の有無による性能差
% TODO: アスペクト別の効果の違い
% TODO: 説明文の有用性の考察

\subsection{COCO Retrieved Concepts実験}
% TODO: 実験設定(正解ラベルなし)
% TODO: 生成された対比因子の例
% TODO: 画像との整合性確認方法
% TODO: 視覚的概念記述の生成能力の検証結果

\subsection{概念の具体性による性能比較}
LLM による対比因子ラベル生成の性能は、対象となる概念の具体性(Concrete vs.\ Abstract)によって明確な差異を示す傾向が観測された。

\begin{itemize}
  \item \textbf{具体的なアスペクトにおける優位性}: SemEval-2014 における「Food」や「Price」といった語彙的に安定した具体的なアスペクトの命名において、本手法は相対的に高い BERTScore を達成した。
  これらの概念は、具体的な製品や属性に関する明確な語彙的証拠(例:ピザ、高すぎる、割引)がテキスト集合 $A$ に含まれやすく、LLM が差分を容易に抽出できたことを示唆する。
  Steam Review Aspect Dataset における「Technical」や「Gameplay」といった具体的特性も、同様に比較的高い意味的類似性を示す傾向があった。
  \item \textbf{抽象的な概念における性能劣位}: 対照的に、SemEval-2014 における「Atmosphere」や Steam Review Dataset における「Story」といった抽象的なアスペクトの命名精度は、具体的なアスペクトと比較して劣位となる傾向が確認された。
  さらに、GoEmotions データセットで検証された 28 の感情カテゴリ(例:Joy, Anger, Admiration など)は物理的な実体を持たない高度に抽象的な概念であり、これらの概念に対する対比因子ラベルの生成は、具体的なアスペクトと比較して総じて低い BERTScore を記録した。
  \item \textbf{傾向の対比}: これらの結果は、LLM が集合差分を推論する際、具体的な語彙や構造に強く依存する「抽出」タスクに近い性能を確保できる一方で、
  広範な文脈や感情的なニュアンスといった抽象的な要素を要約し、簡潔なラベルとして命名するタスク(高度な「推論」を必要とする)においては、性能が相対的に低下する傾向があることを定量的に示した。
\end{itemize}

% TODO: 定量的な数値による裏付け

\subsection{エラー分析と限界}
% TODO: 失敗した実験の分析
% TODO: エラーの種類と原因
% TODO: 手法の限界の考察

\subsection{補足分析}
Few-shot ICL の導入は、LLM の生成ラベルのスタイルを正解ラベルの簡潔なスタイルに近づけることを目的として検証された。
具体的には、Few-shot の例をプロンプトの \texttt{examples\_section} として挿入し、モデルがその出力形式を模倣する特性を利用した。
1-shot 設定が最適であったことは、LLM がタスク定義と 1 つの高品質な例から、集合差分を命名するための有効な生成戦略を迅速に確立したことを示す。

また、SemEval-2014 や Steam レビューデータは、LLM 命名の評価ベンチマークとして使用された。
GoEmotions データセットは、総レコード数 63,812 件に及ぶ大規模な Reddit コメントから収集されており、
28 の感情カテゴリは、本手法のロバスト性を抽象概念の集合に拡張するための重要な挑戦的要素として機能した。
実験パイプラインでは、グループ $A$ と $B$ を抽出する際、コンテキスト長超過エラーを回避するため、$group\_size$ が最大 100 件に制限された設定が用いられた。

さらに、BLEU スコア(約 0.007)の低さは、生成ラベルが「~に関する言及」といった説明文であり、正解ラベル(例:「food」)と直接的な語彙重複を持たないという、
評価指標とタスクの性質とのミスマッチによって生じている。
この観測結果は、語彙的重複を測る指標が本タスクに不適であり、文脈的意味類似度を測る BERTScore の採用が妥当であるという評価戦略の選択を裏付ける定量的証拠となる。
BERTScore が約 0.551 という中適度な値を示したことは、語彙レベルでは一致しないが意味レベルでは関連性が保持されているという事実を客観的に示す。

最後に、実験結果は、LLM(GPT-4o-mini)によるコントラスティブ要約が XAI におけるニューロン対比因子命名タスクの実現可能性を示し、
SemEval-2014 ベンチマークにおいて BERTScore で約 0.551 という妥当な意味的関連性を達成したことを定量的に示した。
また、概念の具体性が命名精度に系統的に影響を与え、具体的なアスペクトにおいて優位性を示す傾向が確認された。

\section{統計的分析}
% TODO: 統計的有意性の検定(必要に応じて)
% TODO: 信頼区間の計算
% TODO: 外れ値の分析
% TODO: 実験間の一貫性の確認

\chapter{総合考察}

本章では,第4章で示した実験結果に基づき,対比因子ラベル自動生成手法の妥当性,性能の決定要因,および限界について整理して考察する.

\section{本手法の限界と課題}

実験結果全体から,本手法にはいくつかの限界が明確になった.
第一に,対比因子ラベルは具体的アスペクトや語彙的一貫性が高いデータセットでは高い BERTScore を達成する一方で,Steam のようなノイズが大きくマルチアスペクトなドメインでは性能が低下する.
これは,単一ニューロンの発火パターンが,必ずしも既存の人手アスペクト体系と一対一に対応しないこと,そしてレビュー言語の多義性により「何についての差分か」が曖昧になることを反映している.

第二に,BLEU のような語彙一致指標は,本タスクでは一貫して有用性を示さなかった.
正解ラベルが短いアスペクト名であり,生成ラベルが説明的フレーズであるという構造的なギャップが存在する以上,n-gram 重複に基づく指標は性能評価に適さない.
これは,対比因子ラベリングの評価設計において,意味分布ベースおよび学習ベースの指標を組み合わせる必要があることを意味する.

第三に,Few-shot ICL やアスペクト説明文は出力の安定化と性能向上に寄与するが,その効果は一様ではない.
特に \textit{Visual} のように説明文が抽象的で他アスペクトと重なりやすい場合,説明文がかえってアスペクト固有性を弱めることがある.
したがって,Few-shot 例や説明文は,対象アスペクトの境界を明確に切り出すように慎重に設計する必要がある.

第四に,COCO 実験で明らかになったように,テキストキャプションに依存した対比因子生成は,キャプション側の頻度バイアスや記述の抜け漏れに影響を受け,視覚的現実とずれたラベルを生成するリスクを持つ.
画像との整合性評価を組み込むこと,および画像特徴量と生成ラベルとの対応関係を明示的に検証することは,精度向上と信頼性確保に向けて望ましい改善方策であると考えられる.

これらの限界は,提案手法が「ニューロン発火条件に対する対比因子ラベルを自動生成する」という目標に対して,どの条件で有効に機能し,どの条件で性能が制約されるかを具体的に示している.
特に,概念の具体性,データセット設計,Few-shot 設定,評価指標設計,テキストと視覚のギャップといった要因が,今後の改良において重点的に扱うべき課題であることが明らかになった.
総じて,本手法は概念が具体的でテキスト分布とラベル体系が整合的な条件では有効に機能する一方,ノイズの大きいマルチアスペクト環境やテキストと視覚表現の乖離が大きい条件では,対比因子ラベルが人間の直観とずれやすいという制約を持つ.



\chapter{まとめ}
\section{結論}

本研究は、大規模言語モデル(LLM)の強力な文脈理解能力と生成能力を活用し、ニューロン発火条件に対応するテキスト集合間の意味的差分を自然言語で記述する「対比因子命名」という新規タスクの実現可能性を検証した。従来の XAI 手法が抱えていた構造的なボトルネックを解消するため、LLM(GPT-4o-mini)をコントラスト生成器として利用するアプローチを提案し、その有効性を定量的に実証した。

その有効性を示すために、本研究では性質の異なる 4 つの多様なデータセット――SemEval-2014(レストランレビュー)、Steam レビュー、GoEmotions(感情分類)、Retrieved Concepts(視覚的概念記述)――を対象に検証を行った。これにより、本手法が製品レビュー、感情データ、視覚概念テキストといった、内容と粒度の異なるドメインに対しても一貫した枠組みとして適用可能であることを示した。

主要な定量評価では、生成ラベルと人手ラベルとの意味的類似性を BERTScore を用いて測定した結果、平均約 0.551 という中適度な意味的関連性を達成した。この結果は、LLM がニューロンの発火パターンという集合的な差分から、その意味的な核を抽出できるという主要な仮説が、一定の条件の下で妥当であることを定量的に裏付けるものである。

また、生成ラベルの品質は、「Food」「Price」のような語彙的に安定した具体的アスペクトにおいて特に高い傾向を示し、LLM が明示的なテキスト証拠に基づく「抽出」タスクに強みを持つことを確認した。一方で、語彙的一致度を測る BLEU スコアは極めて低値(約 0.007)であり、本タスクにおいては語彙的重複よりも意味的妥当性を重視する BERTScore のような指標が不可欠であることが明らかになった。

\section{本研究の貢献}

本研究の第一の貢献は、従来は人手に依存していた「概念の命名」プロセスを、LLM を用いて自動化するスケーラブルな基盤を提示した点である。LIMEや SHAPに代表される局所的な特徴可視化手法や、非教師ありコンセプト抽出(UCBM/CCE)のように概念ベクトルの発見に留まっていた従来手法と異なり、本手法は「概念の発見」と「命名」を一体として扱う XAI の新たなワークフローを提案した。

第二に、本手法は人手アノテーションが困難あるいは高コストであるドメインにおいても適用可能な汎用性を有することを示した。具体的には、物理的実体を持たない抽象概念である Joy や Anger など 28 種類の感情カテゴリを含む GoEmotions データセット、および「Gameplay」「Technical」といった専門性の高いアスペクトを含む Steam レビューデータセットを対象とし、抽象的・専門的な概念に対しても一定水準の対比因子命名が可能であることを示した。

第三に、本研究は、メカニスティック解釈(MI)分野における Attribution Graphsなどが発見した特徴量に対し、人間が理解できる自然言語ラベルを自動付与する命名モジュールとして機能し得ることを示した点である。Attribution Graphsによって発見された特徴量に対しても、本手法は自動命名を可能にする。これにより、「労働集約的 (labor-intensive)」と指摘されてきた手動ラベリングのボトルネックを緩和し、ブラックボックスモデルの構造理解をより実務的なレベルへと押し上げる足掛かりを提供した。

\section{今後の課題}

一方で、本研究にはいくつかの限界も存在する。まず、BERTScore における平均スコア 0.551 は中程度の意味的類似度を示すものの、高精度な命名が常に達成されているわけではなく、とりわけ抽象的な感情概念や、複数アスペクトが絡み合うケースにおいては、生成ラベルと人手ラベルの解釈がずれる事例も確認された。また、評価指標として BERT と BLEU に依存しており、人間評価との整合性を直接的に検証できていない点も課題として残る。

これらの課題に対して、本研究の結果は今後の具体的な改善方向性も示唆している。第一に、抽象概念命名の精度向上に向けて、Chain-of-Thought(CoT)プロンプティング~\cite{holtzman2020curious}を導入し、LLM による逐次的な思考過程の明示化を通じて、集合差分の解釈プロセスを段階的に構造化することが有望である。第二に、BLEURT~\cite{sellam-etal-2020-bleurt}や BARTScore~\cite{yuan2021bartscore}など、人間評価との整合性が高いとされる学習ベースの評価指標を導入し、単一指標に依存しない多面的な評価設計へ移行することが求められる。

最後に、本研究で提案した対比因子命名は、非教師あり概念抽出によって得られた概念ベクトルが表す意味の違い(集合 A と集合 B の差分)を、LLM を用いて人間に分かりやすい自然言語ラベルとして自動的に付与する手法である。この枠組みは、モデル内部の構造を直感的に理解可能な形で提示し、ブラックボックスモデルの解釈可能性をスケーラブルに高めるための一つの方向性であり、今後の XAI 研究および実応用に向けた基盤として発展が期待される。

\input{chapters/07_acknowledgments}
% 参考文献(References)
\newpage
\addcontentsline{toc}{chapter}{参考文献}
\renewcommand{\bibname}{参考文献}

%% 参考文献に bibtex を使う場合
%\bibliographystyle{junsrt}
%\bibliography{hoge}

%% 参考文献を直接ファイルに含めて書く場合
\begin{thebibliography}{99}

\bibitem{pontiki-EtAl:2014:SemEval2014Task4}
M.~Pontiki, D.~Galanis, J.~Pavlopoulos, H.~Papageorgiou, I.~Androutsopoulos, and S.~Manandhar:
``SemEval-2014 Task 4: Aspect Based Sentiment Analysis,''
in \textit{Proceedings of the 8th International Workshop on Semantic Evaluation (SemEval 2014)}, Dublin, Ireland, August 23--24, 2014, pp.~27--35, doi:10.3115/v1/S14-2004. Available at \url{https://aclanthology.org/S14-2004/}.
% 研究概要: Aspect Based Sentiment Analysis(アスペクトベース感情分析)の標準ベンチマークタスクを提案。レビュー文からアスペクト(観点)とその極性(ポジティブ/ネガティブ/ニュートラル)を抽出するタスクを定義し、評価データセットを提供。
% データセット: Restaurant(レストラン)とLaptop(ノートPC)の2ドメインで構成。Restaurantドメインではfood, service, price, atmosphereなどのアスペクト、Laptopドメインではbattery, screen, keyboard, performanceなどのアスペクトが定義されている。
% タスク構成: 複数のサブタスク(アスペクト抽出、極性分類、アスペクトカテゴリ分類など)を含む包括的な評価フレームワークを提供。
% 本研究での使用: 対比因子生成実験の主要データセットとして使用。Restaurantドメインからfood/service、Laptopドメインからbattery/screenの4アスペクトを採用し、アスペクトを含むテキスト群と含まないテキスト群の対比分析に活用。
% 意義: ABSA研究の標準ベンチマークとして広く採用され、本研究における対比因子生成手法の評価基盤を提供。

\bibitem{ribeiro2016should}
M.~T. Ribeiro, S.~Singh, and C.~Guestrin:
``Why should I trust you?: Explaining the predictions of any classifier,''
in \textit{Proceedings of the 22nd ACM SIGKDD International Conference on Knowledge Discovery and Data Mining}, San Francisco, CA, USA, August 13--17, 2016, pp.~1135--1144, doi:10.1145/2939672.2939778.
% 研究概要: ブラックボックス機械学習モデルの予測を説明するための手法LIME(Local Interpretable Model-agnostic Explanations)を提案。任意の分類器に対して、個々の予測に対する局所的な説明を生成する。
% 手法: モデル非依存(model-agnostic)のアプローチで、対象インスタンスの近傍で擬似データを生成し、距離カーネルで重み付けしたLASSO回帰(線形モデル)を学習して各特徴量の重要度を算出。テキスト、画像、表形式データなど複数のモダリティに対応。
% 主な貢献: 説明可能性の評価指標としてlocal fidelity(忠実性:局所近傍で説明モデルが元のモデルをどれだけ再現できているか)とinterpretability(可読性)を導入し、人間実験により説明の有用性(予測精度向上や欠陥発見)を実証。SP-LIME(submodular pick)により、サブモジュラー最適化を用いて冗長性の少ない多様な代表的説明セットを選択する手法も提案。
% 課題: 局所的な説明に限定され、モデル全体の動作原理は説明しない。後続研究(Slack et al. 2020、Alvarez-Melis & Jaakkola 2018)により、敵対的攻撃に対する脆弱性やロバスト性の問題が指摘されている。なお、説明の安定性(stability)は本論文の中心的な評価指標ではなく、主に後続研究で議論される概念である。

\bibitem{lundberg2017unified}
S.~M. Lundberg and S.-I. Lee:
``A unified approach to interpreting model predictions,''
in \textit{Advances in Neural Information Processing Systems}, vol.~30, Long Beach, CA, USA, December 4--9, 2017, pp.~4765--4774, arXiv:1705.07874, doi:10.5555/3295222.3295230. Available at \url{https://proceedings.neurips.cc/paper/2017/file/8a20a8621978632d76c43dfd28b67767-Paper.pdf}.
% 研究概要: 協力ゲーム理論のShapley値に基づき、機械学習モデルの予測に対する各特徴量の寄与度を説明するSHAP (SHapley Additive exPlanations) を提案。
% 手法: 既存の複数の説明手法(LIME、DeepLIFT、Layer-Wise Relevance Propagation、Shapley regression valuesなど)を統一的に解釈できる理論的フレームワークを提供。
% SHAP値は、特徴量のすべての可能な組み合わせに対する予測への寄与度を平均することで計算され、加法性(各特徴量のSHAP値の合計が予測値と基準値の差に等しい)を満たす。
% 主な貢献: 異なる説明手法間の比較を可能にし、モデル解釈の統一的な評価基準を確立。線形モデル、ツリーモデル、深層学習モデルなど、様々なモデルタイプに適用可能。
% 課題: 計算コストが高く、特に特徴量数が多い場合には近似手法が必要。また、後付け説明(post-hoc explanation)の限界として、モデル自体の解釈可能性を高めるわけではない。

\bibitem{wachter2017counterfactual}
S.~Wachter, B.~Mittelstadt, and C.~Russell:
``Counterfactual explanations without opening the black box: Automated decisions and the GDPR,''
\textit{Harvard Journal of Law \& Technology}, vol.~31, no.~2, pp.~841--887, 2017. Available at \url{https://jolt.law.harvard.edu/assets/articlePDFs/v31/Counterfactual-Explanations-without-Opening-the-Black-Box-Sandra-Wachter-et-al.pdf}.

\bibitem{dhurandhar2018cem}
A.~Dhurandhar, P.~Chen, R.~Luss, C.~Tu, P.~Shanmugam, K.~Das, Y.~Liu, and P.~Tambe:
``Explanations based on the Missing: Towards Contrastive Explanations with Pertinent Negatives,''
in \textit{Advances in Neural Information Processing Systems}, vol.~31, 2018 (NeurIPS 2018), Montr\'eal, Canada, December 3--8, 2018, arXiv:1802.07623. Available at \url{https://arxiv.org/abs/1802.07623}.
% 研究概要: ブラックボックス分類器に対して、入力に含まれるべき最小限の特徴(Pertinent Positives)と含まれてはいけない最小限の特徴(Pertinent Negatives)を同時に求めるCEM(Contrastive Explanation Method)を提案し、対比的な説明を与える枠組みを示す。
% 手法: 元の入力に対してL1/L2正則化付き最適化問題を解き、決定クラスを維持しつつ不要な特徴を削除・必要な特徴の追加を行うことでPP/PNを導出する。MNIST、調達不正データ、脳活動データなどで有効性を検証。
% 本研究での位置づけ: 「何があるか/ないか」に基づく対比的説明の代表的先行研究として参照し、本研究で扱うテキストベースの対比因子ラベル生成との関係と違い(数値特徴前提・連続最適化ベース)を整理する際に用いる。

\bibitem{kim2018interpretability}
B.~Kim, M.~Wattenberg, J.~Gilmer, C.~Cai, J.~Wexler, F.~Viegas, et al.:
``Interpretability beyond feature attribution: Quantitative testing with concept activation vectors (TCAV),''
in \textit{Proceedings of the 35th International Conference on Machine Learning (ICML 2018)}, Stockholm, Sweden, July 10--15, 2018, pp.~2673--2682, arXiv:1711.11279. Available at \url{https://proceedings.mlr.press/v80/kim18d.html}.

\bibitem{luss2024cell}
R.~Luss, E.~Miehling, and A.~Dhurandhar:
``CELL your Model: Contrastive Explanations for Large Language Models,''
arXiv preprint arXiv:2406.11785, June 2024. Available at \url{https://arxiv.org/abs/2406.11785}.

\bibitem{bucinca2024contrastive}
Z.~Bu{\c{c}}inca, S.~Swaroop, A.~E. Paluch, F.~Doshi-Velez, and K.~Z. Gajos:
``Contrastive Explanations That Anticipate Human Misconceptions Can Improve Human Decision-Making Skills,''
in \textit{Proceedings of the 2024 CHI Conference on Human Factors in Computing Systems (CHI '24)}, April 2024, arXiv preprint arXiv:2410.04253. Available at \url{https://arxiv.org/abs/2410.04253}.

\bibitem{anthropic2025biology}
Anthropic:
``On the Biology of a Large Language Model,''
Transformer Circuits, 2025. Available at \url{https://transformer-circuits.pub/2025/attribution-graphs/biology.html}.
% 研究概要: Claude 3.5 Haikuを対象に、Circuit Tracing(回路トレース)手法を用いてモデル内部の計算プロセスを解析。
% 手法: 出力から逆方向に、各特徴量(features)がどの順序で使われたかを追跡し、最終出力への影響を数値化。
% 元の巨大モデルを直接可視化するのではなく、あるプロンプトに対する振る舞いを模倣するより単純な「置き換えモデル(replacement model)」を構築し、その上で特徴量間の相互作用をグラフとして可視化することで、影響の大きい計算経路(attribution graph)を抽出。
% 主な発見: 多言語処理における共通概念空間、詩生成における計画性、推論プロセスと自己説明の乖離など。
% 注意: 解析対象は主に「特徴量(features)」とその相互作用であり、個々の物理的ニューロンではなく人間が意味づけしやすい中間表現(superfeatures/human-interpretable features)を扱う。
% 課題: Attribution Graphが描くのは「replacement model上のfeaturesの流れ」であり、必ずしもオリジナルのすべての内部挙動を忠実に再現したものとは限らない。
% また、紹介されている回路や計算経路はモデルが出力を生成する過程のごく一部を捉えたものであり、複雑な長文や多段推論、リアルタイムの文脈変化などをすべてトレースするのは現状では困難。
% グラフのノード(特徴量)の意味付けは自動化されておらず、手動で「スーパーノード」としてグループ化する必要がある。また、attribution graphを因果(causal)な説明と見るか相関(correlational)な説明と見るかには慎重さが求められる。

\bibitem{demszky2020goemotions}
D.~Demszky, D.~Movshovitz-Attias, J.~Ko, A.~Cowen, G.~Nemade, and S.~Ravi:
``GoEmotions: A Dataset of Fine-Grained Emotions,''
in \textit{Proceedings of the 58th Annual Meeting of the Association for Computational Linguistics (ACL 2020)}, pp.~4040--4054, 2020. Available at \url{https://aclanthology.org/2020.acl-main.372/}.

\bibitem{srec:steam-review-aspect-dataset}
S.~Khosasi:
``Steam review aspect dataset,''
2024. Available at \url{https://srec.ai/blog/steam-review-aspect-dataset}.

\bibitem{papineni-etal-2002-bleu}
K.~Papineni, S.~Roukos, T.~Ward, and W.-J. Zhu:
``BLEU: a Method for Automatic Evaluation of Machine Translation,''
in \textit{Proceedings of the 40th Annual Meeting of the Association for Computational Linguistics (ACL 2002)}, pp.~311--318, 2002. Available at \url{https://aclanthology.org/P02-1040.pdf}.

\bibitem{devlin2018bert}
J.~Devlin, M.-W. Chang, K.~Lee, and K.~Toutanova:
``BERT: Pre-training of Deep Bidirectional Transformers for Language Understanding,''
arXiv preprint arXiv:1810.04805, 2018. Available at \url{https://arxiv.org/abs/1810.04805}.

\bibitem{bau2017networkdissection}
D.~Bau, B.~Zhou, A.~Khosla, A.~Oliva, and A.~Torralba:
``Network Dissection: Quantifying Interpretability of Deep Visual Representations,''
in \textit{Proceedings of the IEEE Conference on Computer Vision and Pattern Recognition (CVPR 2017)}, Honolulu, HI, USA, July 2017, pp.~6541--6549, arXiv:1704.05796. Available at \url{https://arxiv.org/abs/1704.05796}.
% 研究概要: CNN の各ユニットと Broden データセットの概念マスクとの IoU を用いて,「どのユニットがどの概念を検出しているか」を定量化する手法 Network Dissection を提案し,深層視覚表現の解釈可能性を測定。
% 本研究での位置づけ: 画像モデル内部の概念特徴に対する「固定語彙ベースの自動ラベリング」手法として参照し,対比因子生成タスクとの違い(内部アクティベーション前提・画像中心)を説明する際の代表例として用いる。

\bibitem{oikarinen2022clipdissect}
T.~Oikarinen and T.-W. Weng:
``CLIP-Dissect: Automatic Description of Neuron Visual Features with Language Models,''
arXiv preprint arXiv:2204.10965, 2022. Available at \url{https://arxiv.org/abs/2204.10965}.
% 研究概要: CLIP の画像・テキスト埋め込み空間を用い,高活性化画像と多数のテキスト候補との類似度を計算することで,任意の視覚モデルのユニットにオープンエンドな概念ラベルを自動付与する CLIP-Dissect を提案。
% 本研究での位置づけ: CLIP を用いた検索ベースの自動概念命名の代表例として参照し,本研究のテキスト集合ベース・外部ラベル評価との対比に用いる。

\bibitem{schrodi2024unsupervised}
S.~Schrodi, M.~R{\"u}ckl, T.~Wirth, M.~B{\"o}hm, and D.~R{\"u}gamer:
``Concept Bottleneck Models Without Predefined Concepts,''
arXiv preprint arXiv:2407.03921, 2024. Available at \url{https://arxiv.org/abs/2407.03921}.

\bibitem{oikarinen2023labelfree}
T.~Oikarinen, S.~Tripathi, T.~M. Mitchell, and D.~Alvarez\mbox{-}Melis:
``Label-Free Concept Bottleneck Models,''
in \textit{Proceedings of the 11th International Conference on Learning Representations (ICLR 2023)}, Kigali, Rwanda, May 2023, arXiv:2304.06129. Available at \url{https://arxiv.org/abs/2304.06129}.
% 研究概要: GPT-3 や CLIP を用いてタスクに関連する概念候補リスト(概念バンク)を自動生成し,ラベル付き概念を用いずに Concept Bottleneck Model を構築する Label-Free CBM を提案。
% 本研究での位置づけ: 事前定義概念なしに概念ボトルネックを構築する既存手法として引用し,Discover-then-Name との関係や,本研究の「テキスト集合+外部アスペクトラベル」という評価設定との違いを説明する。

\bibitem{rao2024discoverthenname}
S.~Rao, Y.~Zhao, M.~Sachan, and A.~Gupta:
``Discover-then-Name: Task-Agnostic Concept Bottlenecks via Automated Concept Discovery,''
arXiv preprint arXiv:2407.14499, 2024. Available at \url{https://arxiv.org/abs/2407.14499}.
% 研究概要: CLIP 特徴に対して Sparse Autoencoder や NMF を適用して概念方向を教師なしで発見し,高活性化画像と CLIP/LLM を用いて概念に名前を付ける Discover-then-Name (DN-CBM) を提案。
% 本研究での位置づけ: 「発見してから名付ける」コンセプトボトルネックの代表例として参照し,本研究の対比因子生成タスクとのタスク設計の違いを議論する。

\bibitem{ameisen2025attribution}
E.~Ameisen, J.~Lindsey, A.~Pearce, W.~Gurnee, N.~L. Turner, B.~Chen, C.~Citro, D.~Abrahams, S.~Carter, B.~Hosmer, J.~Marcus, M.~Sklar, A.~Templeton, T.~Bricken, C.~McDougall, H.~Cunningham, T.~Henighan, A.~Jermyn, A.~Jones, A.~Persic, Z.~Qi, T.~B. Thompson, S.~Zimmerman, K.~Rivoire, T.~Conerly, C.~Olah, and J.~Batson:
``Circuit Tracing: Revealing Computational Graphs in Language Models,''
\textit{Transformer Circuits}, 2025. Available at \url{https://transformer-circuits.pub/2025/attribution-graphs/methods.html}.
% 研究概要: Attribution Graphsの手法論を提案。LLMの内部計算プロセスをトレースし、特徴量間の相互作用をグラフとして可視化することで計算構造(回路)を発見する。
% 手法: ある出力が計算されるまでにネットワーク内部でどのニューロンや重みがどのような順序で使われたかを、出力側からさかのぼって追跡。
% 各ニューロンや結合が最終出力にどの程度影響したかを数値化し、影響の大きい経路どうしを結んだグラフとして可視化することで、計算グラフを抽出。
% 課題: グラフのノードとして発見される「特徴量(features)」は必ずしも自然言語的に意味のある概念(semantic concept)とは限らず、ノードが具体的に何を検出しているかを自然言語で特定するプロセスは自動化されていない。
% また、Attribution Graphはモデルの振る舞いの一部を可視化できるが、すべての振る舞いやグローバルなアルゴリズムを保証するものではない。特に注意(attention)による経路は追えない場合があり、「dark matter」と呼ばれる未説明部分が残る。
% 可視化された回路は多くのノード・エッジを含みうるため、手作業での解釈や簡潔な説明への落とし込みが困難で、feature-splitting/absorptionやsupernodesによる手動グルーピングが必要となる。
% 関連研究: anthropic2025biologyと同一シリーズの研究で、Attribution Graphsの手法を具体的に適用した実証研究。

\bibitem{bordt2022posthoc}
S.~Bordt, M.~Finck, E.~Raidl, and U.~von Luxburg:
``Post-Hoc Explanations Fail to Achieve their Purpose in Adversarial Contexts,''
in \textit{Proceedings of the 2022 ACM Conference on Fairness, Accountability, and Transparency (FAccT '22)}, Seoul, Republic of Korea, June 21-24, 2022, pp.~1495--1515, doi:10.1145/3531146.3533153.

\bibitem{slack2020fooling}
D.~Slack, S.~Hilgard, E.~Jia, S.~Singh, and H.~Lakkaraju:
``Fooling LIME and SHAP: Adversarial Attacks on Post hoc Explanation Methods,''
in \textit{Proceedings of the 2020 AAAI/ACM Conference on AI, Ethics, and Society (AIES 2020)}, New York, NY, USA, February 7--8, 2020, pp.~180--186, arXiv:1911.02508, doi:10.1145/3375627.3375830.

\bibitem{alvarez2018robustness}
D.~Alvarez\mbox{-}Melis and T.~S. Jaakkola:
``On the Robustness of Interpretability Methods,''
in \textit{Proceedings of the 32nd Conference on Neural Information Processing Systems (NeurIPS 2018) Workshops}, 2018, arXiv:1806.08049. Available at \url{https://arxiv.org/abs/1806.08049}.

\bibitem{mersha2024survey}
M.~Mersha, K.~Lam, J.~Wood, A.~AlShami, and J.~Kalita:
``Explainable AI: A Survey of Needs, Techniques, Applications, and Future Direction,''
arXiv preprint arXiv:2409.00265, 2024. Available at \url{https://arxiv.org/abs/2409.00265}.

\bibitem{rudin2019stop}
C.~Rudin:
``Stop explaining black box machine learning models for high stakes decisions and use interpretable models instead,''
\textit{Nature Machine Intelligence}, vol.~1, no.~5, pp.~206--215, 2019, doi:10.1038/s42256-019-0048-x.

\bibitem{vilone2020systematic}
G.~Vilone and L.~Longo:
``Explainable Artificial Intelligence: a Systematic Review,''
arXiv preprint arXiv:2006.00093, 2020, doi:10.48550/arXiv.2006.00093. Available at \url{https://arxiv.org/abs/2006.00093}.

\bibitem{haedecke2025conceptClusters}
E.~Haedecke, M.~Akila, and L.~von Rueden:
``Global Properties from Local Explanations with Concept Explanation Clusters,''
in \textit{World Conference on eXplainable Artificial Intelligence (xAI 2025)}, Springer, Cham, 2025, pp.~3--24, doi:10.1007/978-3-031-41510-0\_1.

\bibitem{hu2024interpretableClustering}
L.~Hu, M.~Jiang, J.~Dong, X.~Liu, and Z.~He:
``Interpretable Clustering: A Survey,''
arXiv preprint arXiv:2409.00743, 2024. Available at \url{https://arxiv.org/abs/2409.00743}.

\bibitem{kardale2023contrastive}
A.~Kardale:
``Contrastive text summarization: a survey,''
\textit{International Journal of Information and Computation}, vol.~12, no.~3, pp.~1--10, 2023.

\bibitem{saha2024strumllm}
A.~Saha, B.~P. Majumder, H.~Jhamtani, S.~Subramanian, S.~Sreedhar, S.~Chakrabarti, and P.~Kankar:
``STRUM-LLM: Attributed and Structured Contrastive Summarization for User-Oriented Comparison,''
arXiv preprint arXiv:2403.19710, 2024. Available at \url{https://arxiv.org/abs/2403.19710}.

\bibitem{luo2024chatabsa}
Z.~Luo, Z.~Feng, Y.~Zhang, and H.~Liu:
``ChatABSA: A Novel Framework for Aspect-based Sentiment Analysis using Large Language Models,''
arXiv preprint arXiv:2401.08226, 2024. Available at \url{https://arxiv.org/abs/2401.08226}.

\bibitem{wang2024llmcluster}
J.~Wang, J.~Song, X.~Sun, C.~Chen, W.~Liu, and Y.~Liu:
``Improving Clustering Performance by Leveraging Large Language Models,''
arXiv preprint arXiv:2410.00927, 2024. Available at \url{https://arxiv.org/abs/2410.00927}.

\bibitem{sellam-etal-2020-bleurt}
T.~Sellam, D.~Das, and A.~Parikh:
``BLEURT: Learning Robust Metrics for Text Generation,''
in \textit{Proceedings of the 58th Annual Meeting of the Association for Computational Linguistics (ACL 2020)}, pp.~7881--7892, 2020. Available at \url{https://aclanthology.org/2020.acl-main.704/}.

\bibitem{yuan2021bartscore}
W.~Yuan, G.~Neubig, and P.~Liu:
``BARTScore: Evaluating Generated Text as Text Generation,''
in \textit{Advances in Neural Information Processing Systems (NeurIPS)}, vol.~34, pp.~27263--27277, 2021. Available at \url{https://arxiv.org/abs/2106.11520}.

\bibitem{reiter2018structured}
E.~Reiter:
``A Structured Review of the Validity of BLEU,''
\textit{Computational Linguistics}, vol.~44, no.~3, pp.~393--401, 2018. Available at \url{https://aclanthology.org/J18-3002/}.

\bibitem{holtzman2020curious}
A.~Holtzman, J.~Buys, L.~Du, M.~Forbes, and Y.~Choi:
``The Curious Case of Neural Text Degeneration,''
in \textit{International Conference on Learning Representations (ICLR)}, 2020. Available at \url{https://openreview.net/forum?id=rygGQyrFvH}.

\bibitem{zhang2019bertscore}
T.~Zhang, V.~Kishore, F.~Wu, K.~Q. Weinberger, and Y.~Artzi:
``BERTScore: Evaluating Text Generation with BERT,''
arXiv preprint arXiv:1904.09675, 2019. Available at \url{https://arxiv.org/abs/1904.09675}.

\bibitem{openai2023neurons}
OpenAI:
``Language models can explain neurons in language models,''
OpenAI Blog, 2023. Available at \url{https://openai.com/index/language-models-can-explain-neurons-in-language-models/}.
% 研究概要: GPT-2 の多数のニューロンに対し,トップ発火トークン列を GPT-4 に入力して自然言語説明を生成し,Simulation Score により説明の妥当性を自動評価する「自動解釈可能性」パイプラインを提示。
% 本研究での位置づけ: LLM を用いたニューロン説明と自動スコアリングの代表例として参照し,本研究が扱うテキスト集合レベルのタスクとの違いを説明する。

\bibitem{bills2023automatedinterp}
S.~Bills, N.~Muennighoff, R.~Hoang, N.~Mu, et al.:
``Automatically Interpreting Millions of Features in Large Language Models,''
arXiv preprint arXiv:2310.13052, 2023. Available at \url{https://arxiv.org/abs/2310.13052}.
% 研究概要: Sparse Autoencoder によって LLM の中間表現から数百万規模のスパース特徴を抽出し,各特徴に対して LLM による説明文生成と Simulation Score による自動評価を行うことで,大規模な自動概念命名・自動スコアリングを実現。
% 本研究での位置づけ: 「OpenAI/Anthropic 型」の自動解釈パイプラインの代表として参照し,自動命名の達成度と限界に関する議論の背景とする。

\bibitem{stein2024towards}
A.~Stein, A.~Naik, Y.~Wu, M.~Naik, and E.~Wong:
``Towards Compositionality in Concept Learning,''
in \textit{Proceedings of the International Conference on Machine Learning (ICML)}, 2024. Available at \url{https://arxiv.org/abs/2406.18534}.

\bibitem{lin2014microsoft}
T.-Y.~Lin, M.~Maire, S.~Belongie, J.~Hays, P.~Perona, D.~Ramanan, P.~Doll{\'a}r, and C.~L. Zitnick:
``Microsoft COCO: Common Objects in Context,''
in \textit{Proceedings of the 13th European Conference on Computer Vision (ECCV 2014)}, Zurich, Switzerland, September 6--12, 2014, pp.~740--755, arXiv:1405.0312, doi:10.1007/978-3-319-10602-1\_48. Available at \url{https://arxiv.org/abs/1405.0312}.

\bibitem{koh2020concept}
P.~W. Koh, T.~Nguyen, Y.~S. Tang, S.~Mussmann, E.~Pierson, B.~Kim, and P.~Liang:
``Concept Bottleneck Models,''
in \textit{Proceedings of the 37th International Conference on Machine Learning (ICML 2020)}, vol.~119, pp.~5338--5348, 2020, arXiv:2007.04612. Available at \url{https://arxiv.org/abs/2007.04612}.

\bibitem{radford2021learning}
A.~Radford, J.~W. Kim, C.~Hallacy, A.~Ramesh, G.~Goh, S.~Agarwal, G.~Sastry, A.~Askell, P.~Mishkin, J.~Clark, G.~Krueger, and I.~Sutskever:
``Learning Transferable Visual Models From Natural Language Supervision,''
in \textit{Proceedings of the 38th International Conference on Machine Learning (ICML 2021)}, vol.~139, pp.~8748--8763, 2021, arXiv:2103.00020. Available at \url{https://arxiv.org/abs/2103.00020}.

\bibitem{reimers2019sentence}
N.~Reimers and I.~Gurevych:
``Sentence-BERT: Sentence Embeddings using Siamese BERT-Networks,''
in \textit{Proceedings of the 2019 Conference on Empirical Methods in Natural Language Processing and the 9th International Joint Conference on Natural Language Processing (EMNLP-IJCNLP)}, Hong Kong, China, November 3--7, 2019, pp.~3982--3992, arXiv:1908.10084. Available at \url{https://arxiv.org/abs/1908.10084}.

\bibitem{bird2009natural}
S.~Bird, E.~Klein, and E.~Loper:
\textit{Natural Language Processing with Python: Analyzing Text with the Natural Language Toolkit},
O'Reilly Media, 2009. Available at \url{https://www.nltk.org/book/}.

\bibitem{patricio2025cbvlm}
M.~Patr{\'\i}cio, et al.:
``CBVLM: Training-free explainable concept-based Large Vision Language Models for medical image classification,''
2025.

\bibitem{deng2019annotation}
Y.~Deng, et al.:
``Efforts estimation of doctors annotating medical image,''
in \textit{Proceedings of the 2019 IEEE International Conference on Bioinformatics and Biomedicine (BIBM)}, 2019.

\bibitem{li2024wsi}
X.~Li, et al.:
``Deep learning quantifies pathologists' visual patterns for whole slide image diagnosis,''
2024.

\bibitem{tseng2025expertllm}
Y.-H.~Tseng, et al.:
``Evaluating Large Language Models as Expert Annotators,''
2025.

\bibitem{zhong2022describing}
R.~Zhong, C.~Snell, D.~Klein, and J.~Steinhardt:
``Describing Differences between Text Distributions with Natural Language,''
in \textit{Proceedings of the 39th International Conference on Machine Learning (ICML 2022)}, 
vol.~162, pp.~27099--27116, 2022.
Available at \url{https://proceedings.mlr.press/v162/zhong22a.html}.
% 研究概要: 2つのテキスト分布の違いを自然言語で説明するタスクを定式化。GPT-3をファインチューニングして候補説明を生成し、再ランキングにより最適な説明を選択するProposer-Verifierフレームワークを提案。54の実世界バイナリ分類タスクで評価し、人間アノテーションとの76%の類似度を達成。
% 本研究での位置づけ: ブラックボックスなLLMを用いたテキスト分布間の差分説明の先駆的研究として参照。本研究の対比因子生成タスクとの違い(ABSAスキーマとの意味的一致評価の欠如)を説明する際の代表例として用いる。

\bibitem{dunlap2024visdiff}
L.~Dunlap, Y.~Zhang, X.~Wang, R.~Zhong, T.~Darrell, J.~Steinhardt, J.~E. Gonzalez, and S.~Yeung-Levy:
``Describing Differences in Image Sets with Natural Language,''
in \textit{Proceedings of the IEEE/CVF Conference on Computer Vision and Pattern Recognition (CVPR 2024)}, Seattle, WA, USA, June 17--21, 2024, pp.~17952--17961, arXiv:2312.02974. Available at \url{https://openaccess.thecvf.com/content/CVPR2024/html/Dunlap_Describing_Differences_in_Image_Sets_with_Natural_Language_CVPR_2024_paper.html}.
% 研究概要: 画像セット間の差分を自然言語で説明するSet Difference Captioningタスクを提案。VisDiffは2段階アプローチを採用し、第1段階で画像キャプションとLLMを用いて候補説明を生成、第2段階でCLIPを用いて再ランキングを行う。VisDiffBenchデータセット(187組の画像セットペア)で評価。
% 本研究での位置づけ: Zhong et al. (2022)の画像領域への拡張として参照。本研究のテキスト集合ベース・ABSAスキーマ評価との違いを説明する際の代表例として用いる。

\bibitem{pham2024topicgpt}
C.~M. Pham, A.~Hoyle, S.~Sun, P.~Resnik, and M.~Iyyer:
``TopicGPT: A Prompt-based Topic Modeling Framework,''
in \textit{Proceedings of the 2024 Conference of the North American Chapter of the Association for Computational Linguistics: Human Language Technologies (NAACL-HLT 2024)}, Mexico City, Mexico, June 16--21, 2024, pp.~2956--2984. Available at \url{https://aclanthology.org/2024.naacl-long.164/}.
% 研究概要: LLMを用いたトピックモデリングフレームワークTopicGPTを提案。従来のLDAのような曖昧な単語バッグ表現ではなく、自然言語ラベルと自由形式の説明を持つトピックを生成し、解釈可能性を向上。ユーザーが制約を指定し、モデルを再訓練せずにトピックを修正可能。Wikipediaトピックに対する調和平均純度0.74を達成。
% 本研究での位置づけ: LLMを用いたクラスタ・トピックラベリングの代表例として参照。本研究の対比的概念命名タスクとの違い(差分説明の欠如、ABSAスキーマ評価の欠如)を説明する際の代表例として用いる。

\bibitem{lam2024lloom}
M.~S. Lam, J.~Teoh, J.~Landay, J.~Heer, and M.~S. Bernstein:
``Concept Induction: Analyzing Unstructured Text with High-Level Concepts Using LLooM,''
in \textit{Proceedings of the 2024 CHI Conference on Human Factors in Computing Systems (CHI '24)}, Honolulu, HI, USA, May 11--16, 2024, Art.~No.~632, pp.~1--17, doi:10.1145/3613904.3642607.
% 研究概要: 非構造化テキストから高レベルで人間が解釈可能な概念を抽出するLLooMアルゴリズムを提案。従来のトピックモデリングやクラスタリングが低レベルのキーワードに焦点を当てるのに対し、LLooMはLLMを活用して反復的にサンプルテキストを合成し、より一般的な概念を提案。LLooM Workbenchは対話型テキスト分析ツールとして、自動生成された概念を可視化し、カスタム概念の作成を可能にする。
% 本研究での位置づけ: LLMを用いた高レベル概念帰納の代表例として参照。本研究の対比的概念命名タスクとの違い(単一コーパスからの概念帰納 vs 対比的テキスト集合間の差分説明)を説明する際の代表例として用いる。

\bibitem{anthropic2024monosemantic}
A.~Templeton, T.~Conerly, J.~Marcus, J.~Lindsey, T.~Bricken, B.~Chen, A.~Pearce, et al.:
``Scaling Monosemanticity: Extracting Interpretable Features from Claude 3 Sonnet,''
\textit{Transformer Circuits Thread}, 2024. Available at \url{https://transformer-circuits.pub/2024/scaling-monosemanticity/}.
% 研究概要: Claude 3 Sonnetの中間層活性をSparse Autoencoder (SAE)を用いて分解し、より解釈可能なコンポーネントに変換する手法を提案。SAEにより、モデル内から数百万の解釈可能な特徴を抽出し、言語やモダリティを超えた抽象的概念に対応する特徴を特定。Golden Gate Bridgeのような具体的実体から、「裏切り」「自己認識」などの高度に抽象的な概念までを含む特徴を発見。
% 本研究での位置づけ: LLM内部活性ベースの自動解釈可能性の代表例として参照。本研究のブラックボックス設定(内部活性アクセス不要)との違いを説明する際の代表例として用いる。

\end{thebibliography}



\end{document}
