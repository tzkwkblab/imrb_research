\documentclass[a4paper]{jreport}

\usepackage{masterThesisJa}
\usepackage{amsmath}
\usepackage{url}
\DeclareUnicodeCharacter{2248}{\ensuremath{\approx}}
\setcounter{tocdepth}{3}
\setcounter{page}{-1}

\maintitle{大規模言語モデルを用いた対比因子ラベル生成手法に関する研究}{A Study on Contrastive Factor Label Generation Using Large Language Models for Explainable AI}

\publish{2025}{12}

\student{202421675}{清野 駿}{Seino Shun}

\jabst{
    深層学習モデルは多様なタスクで高い性能を示している一方で,その内部処理は依然としてブラックボックス性が高く,意思決定の根拠や内部で表現される概念を人間が理解することは容易ではない.この問題を解決する試みとして,モデル内部の中間表現への直接アクセスを前提に,内部要素に自然言語ラベルや説明文を自動で命名する研究が進展している,一方で,モデル内部に立ち入らず外部から観測できるテキスト集合の差分だけにもとづいて概念を命名する設定では,LLMが実務的なアスペクトスキーマの語彙をどの程度再発見できるかが体系的に検証されていない.
    本研究の目的は,モデル内部への中間表現への直接アクセスなしに概念命名を行う設定でのLLMの性能を定量的に評価することである.
    そのために本研究の問題設定として,既存のアスペクトラベルを含む・含まないの条件で構成されたテキスト集合A/Bに対して,Aに含まれてBに含まれない特徴を自然言語ラベルで生成するという対比因子ラベル生成タスクを提案し,LLMが対比因子ラベル生成タスクにおいて妥当な自然言語ラベルを生成できるかを複数データセットを用いて横断的に検証した.
    手法としては,A/B の代表テキストを LLM に入力し,Few-shot 例示(0/1/3-shot)を与えたプロンプトにより,両集合を分ける特徴を自然言語で生成させる.評価には SemEval-2014 ABSA(Restaurants/Laptop),Steam ゲームレビュー,GoEmotions,COCO Retrieved Concepts など複数ドメインを含むデータセットを用い,各グループ 100 件の条件で GPT-4o-mini を含む複数モデルを比較した.生成されたラベルの品質は,Sentence-BERT(SBERT)による埋め込み類似度(SBERT類似度),BLEU,および LLM による意味的類似度評価を組み合わせ,多角的に測定した.
    実験の結果,全 36 条件の平均で SBERT類似度 は約 0.698 と中程度の意味的一致を示した一方,BLEU は 0.008 と語彙レベルの一致は低かった.また Few-shot では 1-shot が最も良好で,出力が「説明的な文章」から「より一意に特定可能な語彙」へと安定する傾向が見られた.さらに gameplay や food のような語彙的に具体的・安定した概念では高いスコアを示し,抽象度の高い概念では低下する傾向が確認された.
    これらの結果から,提案手法は一定の条件下において,人手による解釈ラベリングを部分的に代替し得ることが示された.ただし,概念が抽象的な場合の説明の難しさや,語彙が一致しない場合の文脈を汲み取った評価ができない問題など,今後解決すべき課題も明らかになった.
}


\advisors{若林 啓}{伊藤 寛祥}

\begin{document}

\makecover

\addtolength{\textheight}{-5mm}
\setlength{\footskip}{15mm}
\fontsize{11pt}{15pt}\selectfont

\pagebreak\setcounter{page}{1}
\pagenumbering{roman}
\pagestyle{plain}
\tableofcontents
\listoffigures
\listoftables

\parindent=1zw
\pagebreak\setcounter{page}{1}
\pagenumbering{arabic}
\pagestyle{plain}

% 各章を読み込む
\chapter{序章}

近年の深層学習(DNN)モデルは、医療、金融、自動運転といった社会的重要性の高いドメインにおいて優れた予測性能を達成している。しかし、その意思決定プロセスが人間にとって解釈不能な「ブラックボックス」であるという根本的な問題は、モデルの信頼性や公平性の担保、そして法的・倫理的な規制要件の遵守を妨げる最大の障壁となっている~\cite{bordt2022posthoc,kim2018interpretability}。この課題に対応するため、モデルの判断根拠を事後的に説明する説明可能 AI(XAI)の研究が活発化してきた。

従来、XAI の中心的な手法として広く普及したのは、LIME(Local Interpretable Model-agnostic Explanations)や SHAP(SHapley Additive exPlanations)に代表される、\textbf{事後説明(Post-hoc Explanation)}手法である。これらの手法は、入力データ(例:画像ピクセル、テキストトークン)の摂動やゲーム理論に基づく貢献度(Shapley 値)の計算を通じて、個別インスタンスの予測に対する特徴量の寄与を可視化する~\cite{ribeiro2016should,lundberg2017unified}。

しかし、事後説明手法には、以下のような根本的な信頼性の問題が指摘されている。第一に、これらの説明は本質的な曖昧さ(high degree of ambiguity)を抱えており、特に共線性を持つ特徴量に対して不安定な結果を示すことがあり、同等の性能を持つ異なるモデルが全く異なる説明を生成する不安定性が問題となる。第二に、Bordt ら~\cite{bordt2022posthoc}は、事後説明アルゴリズムが、規制や倫理が求める「透明性」の目的を達成する上で\textbf{「不適切 (unsuitable)」}であると結論付けている。これは、説明提供者と受け手の利害が対立する「敵対的文脈 (adversarial contexts)」において、事後説明が容易に操作可能であり、モデル開発者が都合の良い説明を選択的に提示できてしまうためである~\cite{bordt2022posthoc}。

さらに重要な点として、LIME や SHAP は、あくまで個別インスタンスの局所的な特徴寄与に焦点を当てており、モデルが特定のデータ集合(例:特定のニューロンが発火するすべてのサンプル)に対して系統的に何を学習しているか、というグループレベルの差異や集合間のコントラストを自然言語で説明する能力を欠いている~\cite{ribeiro2016should,lundberg2017unified}。

事後説明の限界を克服するため、XAI の研究コミュニティは、低レベルの特徴量(ピクセル、トークン ID)から、人間が理解可能な\textbf{「コンセプト(概念)」}レベル(例:「縞模様」や「価格に関する言及」)で説明を提供するパラダイムへと移行している~\cite{kim2018interpretability}。コンセプトボトルネックモデル(CBM)や TCAV(Concept Activation Vectors)といったコンセプトベース XAI(C-XAI)は、モデルの内部状態を高レベルの概念に関連付けることで解釈可能性を向上させた~\cite{kim2018interpretability,schrodi2024unsupervised,stein2024towards}。

しかし、このアプローチは新たな、かつ重大なボトルネックを生み出した。それは、どの概念を監視・学習すべきかという概念の定義(ラベリング)が、依然として人間に依存している点である。このプロセスは Kim らによって\textbf{「高コストな概念キュレーション (expensive concept curation)」}と呼ばれ、特に専門知識が不可欠なドメイン(例:医療)において、C-XAI の導入を阻む最大の障壁となっていた~\cite{kim2018interpretability,schrodi2024unsupervised,stein2024towards}。既存のベンチマークである SemEval-2014 においても、アスペクト(概念)ラベルは人手によるアノテーションに依存している~\cite{pontiki-EtAl:2014:SemEval2014Task4}。

モデルの内部動作を回路図レベルで徹底的に理解しようとする最先端のメカニスティック解釈(MI)分野においても、解釈の「意味付け」は手作業に依存する。Anthropic による Attribution Graphs のような研究は、モデルの計算プロセスをトレースし、特徴量間の相互作用をグラフとして可視化することで、ニューロン間の回路を発見する。

しかし、Attribution Graphs などが発見する個々の「特徴量」やノードが、人間にとって何を意味するのかを自然言語で特定するプロセスは自動化されていない。研究者らは、解釈を容易にするために、関連する意味を持つ特徴量を手動でグループ化し\textbf{「スーパーノード」としてまとめているが、この手動ステップは「労働集約的 (labor-intensive)」}であり、情報の欠損を伴うことが指摘されている~\cite{anthropic2025biology,ameisen2025attribution}。

本研究の動機は、これら従来の XAI アプローチが直面する、「個別インスタンス中心の説明の限界」、「高コストな人手ラベリング依存」、そして\textbf{「発見された内部構造への命名の自動化の欠如」という 3 つの課題の交差点に存在する}点にある~\cite{bordt2022posthoc,kim2018interpretability,ameisen2025attribution}。信頼性が高く、低コストで、スケーラブルな解釈可能性を実現するためには、モデルの内部状態に基づき、かつ集合間の差分を自然言語で記述する自動命名手法が不可欠となる。

本研究が取り組む主要な課題は、非教師ありコンセプト抽出などで発見されたモデル内部の解釈可能な構造、すなわちニューロン発火条件の対比因子ラベルを、2 つのテキスト集合(発火群 A vs 非発火群 B)の差分からスケーラブルに自動生成することである~\cite{schrodi2024unsupervised,stein2024towards}。このタスクは、従来の XAI 研究における「概念の発見」と「命名」を統合する初の試みであると位置づけられる~\cite{schrodi2024unsupervised,stein2024towards}。

従来の非教師ありコンセプト発見手法(UCBM や CCE など)は、人間による事前定義なしにモデル内部から概念(潜在ベクトル)を抽出する点で大きな進歩を遂げた。しかし、これらの手法が発見するのは、あくまでも潜在的なベクトル表現であり、そのベクトルに人間が理解できる「自然言語の名前」を自動で付与する機能は欠如している。命名プロセスは、発火サンプルを見た研究者による手動分析に依存していた~\cite{schrodi2024unsupervised,stein2024towards}。

本研究は、この\textbf{「発見された概念の手動での意味付け」という C-XAI の最大のボトルネックを直接解消する}ことを目指す。具体的には、モデル内部の特定のニューロン(またはコンセプト)が強く発火したテキスト集合 A と、発火しなかったテキスト集合 B を入力とし、集合 A に特有で集合 B に見られない意味的差分を、自然言語ラベル $L$ として生成するタスクを定式化する。このタスクは、NLP における「コントラスティブ要約 (Contrastive Summarization)」~\cite{kardale2023contrastive,saha2024strumllm}の系譜に属するが、これを XAI ドメイン(モデル内部状態の解釈)に初めて適用する点で新規性を有する~\cite{kardale2023contrastive,saha2024strumllm}。

このタスクは、単一インスタンスに対する反事実的説明(例:CEM)とは異なり、集合間の一般的・代表的な差分を記述するものであり、ニューロンが何を計算しているかを集合レベルで説明することを可能にする。

本研究は、この高難度な「集合差分からの自然言語命名」タスクの解決策として、大規模言語モデル(LLM)の強力な文脈理解能力と生成能力を活用する。LLM(特に GPT-4o-mini)をコントラスト生成器として利用し、入力された A 群と B 群のテキスト群から、A 群に特徴的で B 群に欠如している意味的な側面を推論させ、簡潔なラベルを自動生成させる。

このアプローチは、人手によるアノテーションや複雑な事後分析のステップを排除し、非教師ありで抽出された概念(UCBM や Attribution Graphs が発見した特徴量)がもつ意味を、スケーラブルかつ自動で人間が理解できる「名前」として付与する命名モジュールとして機能する。この仮説の妥当性を、定量的な実験を通じて検証することが、本論文の主要な目標となる。

本研究は、LLM(GPT-4o-mini)を用いたコントラスティブ要約を核とする。手法の実現にあたっては、以下のステップを踏む。

\begin{enumerate}
  \item データ収集とグルーピング:モデル内部の特定のニューロンの発火度に基づき、発火が強いテキスト群 A と、発火が弱い(または非発火の)テキスト群 B を抽出する(グループサイズ \texttt{group\_size} はパラメータとして調整可能)。
  \item Few-shot ICL(In-Context Learning)によるプロンプト設計:LLM に A 群と B 群のテキストを入力として提示し、「A 群に特有の内容的・意味的差分」を抽出するよう指示する。Few-shot ICL は、本タスクにおける LLM の出力形式の揺らぎや語彙の安定性を確保するための\textbf{検証手段(サブ実験)}として導入する~\cite{saha2024strumllm}。
  \item 対比因子ラベルの生成:LLM に、集合差分を簡潔に要約した自然言語の「対比因子ラベル」を出力させる。
\end{enumerate}

本研究では、人手アノテーションされたアスペクトラベルを持つ SemEval-2014 Restaurant/Laptop データセットを評価ベンチマークとして使用し、LLM が生成したラベルと正解ラベルとの意味的類似性を評価した~\cite{pontiki-EtAl:2014:SemEval2014Task4}。

評価指標には、語彙的一致度を測る BLEU と、文脈的意味的類似度を測る BERTScore を採用した~\cite{papineni-etal-2002-bleu,zhang2019bertscore,reiter2018structured}。その結果、以下の重要な知見が得られた。

\begin{itemize}
  \item 意味的関連性の達成:生成されたラベルの品質を BERTScore で測定したところ、約 0.551 という中適度な意味的関連性を達成した~\cite{zhang2019bertscore}。この結果は、LLM がニューロンの発火群と非発火群という集合的な差分から、人間が理解できる意味的な核を抽出できること、すなわち「LLM の文脈理解能力を活用すれば、概念の発見と命名を同時に行うプロセスを実現できる」という主要な仮説(Main Hypothesis)が部分的に正しいことを示す。
  \item 語彙的一致の限界:一方、BLEU スコアは極めて低値(およそ 0.007)を示した~\cite{papineni-etal-2002-bleu,reiter2018structured}。これは、正解ラベルが「food」のような単一の単語であるのに対し、LLM が「食べ物の品質に関する言及」のような説明的なフレーズを生成するため、語彙的重複を測る BLEU が本タスクの性質に必ずしも適していないことを示唆している~\cite{reiter2018structured}。
  \item 概念の具体性による優位性:生成ラベルの品質は、「food」「price」のような語彙的に安定した具体的なアスペクトにおいて特に高い優位性を示すことが確認された~\cite{kim2018interpretability,schrodi2024unsupervised,stein2024towards}。対照的に、「story」「atmosphere」のような抽象的な概念の命名は、LLM が単純な要約ではなく高度な推論を必要とするため、性能が劣位となるという課題も特定された~\cite{kim2018interpretability,schrodi2024unsupervised,stein2024towards}。
\end{itemize}

本研究は、XAI と NLP の境界領域において、以下の点で重要な貢献を果たす。

第一に、新規タスクの提案と実現可能性の検証である。XAI における「概念の発見」(UCBM, CCE)と「概念の命名」が分離していた構造的な課題を認識し、LLM によるコントラスティブ要約によって、この二つのプロセスを統合する\textbf{「集合差分によるニューロン対比因子命名」}という新規タスクを提案し、その実現可能性を定量的に検証した~\cite{schrodi2024unsupervised,stein2024towards}。

第二に、XAI パラダイムのギャップ解消である。個別インスタンス中心の事後説明(LIME/SHAP)~\cite{ribeiro2016should,lundberg2017unified}の限界と、命名機能を持たない非教師ありコンセプト抽出(UCBM/CCE)~\cite{schrodi2024unsupervised,stein2024towards}のボトルネックを同時に解消する、スケーラブルなハイブリッドアプローチを提供する~\cite{schrodi2024unsupervised,stein2024towards}。

第三に、メカニスティック解釈の実用化への寄与である。Attribution Graphs~\cite{anthropic2025biology,ameisen2025attribution}などによって発見されたニューロン回路や特徴量に対し、人間が理解可能な意味的な名前を自動で付与する手段を提供し、手動ラベリングに依存していた MI 分野のボトルネック解消に貢献する~\cite{ameisen2025attribution}。

第四に、Few-shot ICL の応用と最適化に関する知見の提供である。Few-shot ICL を活用し、生成ラベルの品質とスタイルを矯正するアプローチを提案し、Few-shot 数の変動による影響を分析することで、LLM を用いた自動命名の安定性向上に向けた知見を提供する~\cite{saha2024strumllm}。


\chapter{関連研究}

本研究は,深層学習モデルの解釈可能性(XAI)において,ニューロンの発火条件を自然言語で自動命名するという,従来の手法が満たせていなかったギャップを埋めることを目的としている.
本章では,関連する先行研究を体系的に整理し,それぞれの限界を明確にすることで,本研究が提案する「集合差分による自動命名」の新規性を位置づける.
本研究は,(1) 個別インスタンス中心の事後説明手法,(2) 概念の発見に留まる非教師ありコンセプト抽出,(3) 一般的な NLP タスクに特化していたコントラスティブ要約,という三つの主要な研究領域の交差点に位置づけられる.

\section{従来の XAI 手法:個別インスタンス中心の解釈}

深層学習モデルのブラックボックス性に対処するため,初期の XAI 研究は主に,特定の予測に対する入力特徴量の寄与度を事後的に(Post-hoc)説明する手法に焦点を当ててきた.

\subsection{特徴量帰属手法の個別性}

最も広く採用されてきた手法として,LIME(Local Interpretable Model-agnostic Explanations, Ribeiro et al., 2016)~\cite{ribeiro2016should}や SHAP(SHapley Additive exPlanations, Lundberg \& Lee, 2017)~\cite{lundberg2017unified}が挙げられる.
LIME は,ターゲットとする予測の周囲でデータを摂動させ,局所的に解釈可能な代理モデル(サロゲートモデル)を構築することで,個々のインスタンスの予測根拠を可視化する.
SHAP は,ゲーム理論に基づく Shapley 値を用いて,特徴量の貢献度を統一的な枠組みで定量化する.

これらの特徴量帰属手法は,その性質上,個別インスタンスの局所的な説明に特化している.
LIME や SHAP の出力は,画像における「重要なピクセル」やテキストにおける「重要な単語」といった低レベルな特徴量の寄与スコアであり,モデルが特定のデータ集合に対して系統的に何を学習しているかを,人間が理解できる自然言語の概念(コンセプト)として説明する能力を欠いている.
本研究の目的は,特定のニューロンが発火する集合 A と発火しない集合 B の間の意味的な差分を抽出することであり,個別予測の寄与度を問う従来の XAI 手法では,このグループレベルの差異の説明というタスクを満たすことができない.

さらに,事後説明手法は信頼性の問題も指摘されている.
特に Bordt et al. (2022)~\cite{bordt2022posthoc}は,事後説明が本質的な曖昧さを持ち,敵対的な文脈において容易に操作可能であるため,法的・倫理的な透明性の目的を達成する上で「不適切 (unsuitable)」であると結論付けている.
本研究がモデルの内部状態(ニューロン発火)に直接着目する動機の一つは,この事後説明の曖昧さを回避し,より忠実な説明を提供することにある.

\subsection{反事実的説明の限界}

反事実的説明(Counterfactual Explanation)もまた,対比的(Contrastive)な要素を持つ XAI 手法として注目を集めてきた~\cite{wachter2017counterfactual}.
CEM(Contrastive Explanations Method, Dhurandhar et al., 2018)や Wachter et al. (2017) の手法は,「もし入力 $x$ の一部が $x'$ に変わったら,予測 $y$ はどう変わるか」という,単一事例に対する最小限の入力変更を特定する.
これにより,ユーザーは「この予測を覆すために何をすべきか」という行動可能な洞察を得る.

反事実的説明は,その定義上,単一インスタンスの局所的な反転に限定される.
本研究が対象とするのは,特定のニューロンの発火パターンが,データ集合全体でどのような意味的な特性を持つかを記述すること,すなわち集合間の一般的・代表的な差分を自然言語で要約するタスクである.
反事実的説明は,この集合レベルでのコントラスト記述という目標を達成できない.

\section{非教師ありコンセプト発見と命名の課題}

事後説明の限界を受け,XAI 研究の焦点は,低レベルな特徴量から,人間が理解できる高レベルの「コンセプト(概念)」に基づいた説明へと移行した~\cite{kim2018interpretability}.
コンセプトベース XAI(C-XAI)の代表例である TCAV(Kim et al., 2018)は,概念活性化ベクトルを用いてモデルが特定の概念にどれだけ敏感であるかを定量化する.

\subsection{高コストな人手ラベリング依存}

従来の C-XAI は,解釈の基礎となる概念の定義(ラベリング)を人間に依存しているという構造的な問題を抱えてきた~\cite{kim2018interpretability}.
このプロセスは,同じく Kim らによって「高コストな概念キュレーション (expensive concept curation)」と呼ばれ,特に医療や科学といった専門知識が必要なドメインでの XAI 導入の最大の障壁となっている.
SemEval-2014 のような既存の ABSA ベンチマークでさえ,アスペクト(概念)ラベルは人手によるアノテーションに依存している~\cite{pontiki-EtAl:2014:SemEval2014Task4}.

\subsection{非教師あり概念抽出の進歩と「命名」の欠如}

近年,この人手依存の課題を克服するため,非教師ありでモデル内部から概念を抽出する手法が大きく進展している~\cite{schrodi2024unsupervised,stein2024towards}.
Unsupervised CBM (UCBM, Schrodi et al., 2024) は,人間による事前定義なしに,モデルの内部表現から概念(潜在ベクトル)を自動抽出することを可能にした.
また,Compositional Concept Extraction (CCE, Stein et al., 2024)~\cite{stein2024towards}は,より構成的な概念の表現を抽出する.

UCBM や CCE の進歩にもかかわらず,これらの手法が「発見」するのは,依然として潜在的なベクトル表現である.
そのベクトルが人間にとって何を意味するのか(例:「コンセプト X」が「縞模様」を意味すること)という自然言語での「命名(ラベリング)」機能は,これらの研究には含まれていない.
命名プロセスは,発見された概念が強く発火するサンプルを研究者が手動で分析し,ラベルを付与するという,非教師あり C-XAI の「最後のワンマイル問題」として残されていた~\cite{schrodi2024unsupervised}.

本研究は,この手動命名というボトルネックに対し,UCBM や CCE が発見した概念(に対応する発火/非発火サンプル群)を LLM に入力し,その意味を直接「対比因子ラベル」として生成するスケーラブルな自動命名モジュールとして機能する.

\subsection{メカニスティック解釈における手動作業}

最先端のメカニスティック解釈(Mechanistic Interpretability, MI)分野においても,同様の課題が存在する.
Anthropic による Attribution Graphs(2025)~\cite{anthropic2025biology}は,LLM(例:Claude 3.5 Haiku)の内部計算プロセスをトレースし,特徴量間の相互作用をグラフとして可視化することで回路を発見する.
しかし,同研究においても,グラフのノードとして発見される個々の「特徴量」の意味付け,すなわちノードが具体的に何を検出しているかを自然言語で特定するプロセスは自動化されていない.

同論文の著者ら自身も,解釈を容易にするために,関連する意味を持つ特徴量を手動で「スーパーノード」としてグループ化しており,この手動ステップが「労働集約的 (labor-intensive)」であり,情報の欠損を引き起こすと認めている~\cite{ameisen2025attribution}.
本研究の提案手法は,この MI によって発見された特徴量や回路に対し,その発火条件の差分に基づき,人間が理解できる対比因子ラベルを自動で付与する手段を提供し,MI の実用化と統合的 XAI の実現に寄与する.

\section{LLM を用いた自動ラベル生成とコントラスティブ要約}

大規模言語モデル(LLM)は,その強力な文脈理解能力と自然言語生成能力により,様々なラベリングや要約タスクに応用されている.

\subsection{LLM による命名と正規化}

LLM は,テキストクラスタリングの結果に対して,クラスタ内のサンプル群の共通するテーマを要約し,クラスタラベルを自動生成するために活用されている~\cite{wang2024llmcluster}.
これにより,Embedding ベースのクラスタリングを Few-shot 学習を用いた分類タスクに変換する新たなパラダイムが提案されている.
また,アスペクトベースド感情分析(ABSA)の領域においても,LLM は Few-shot プロンプトを用いてアスペクトの抽出や,ノイズの多い合成ラベルの正規化(Normalization)に利用されている~\cite{luo2024chatabsa}.

\subsection{本研究のタスクとの乖離}

LLM による従来の命名研究の多くは,単一のデータ集合(クラスタ)内の共通点を記述することに焦点を当てている~\cite{wang2024llmcluster}.
一方,本研究は,ニューロンの発火群 A と非発火群 B の間の差分,すなわちコントラスト(対比)を記述することを目的としている.
また,ABSA における LLM の適用例(例:ChatABSA)は,主に教師ありの環境下で特定のアスペクトを抽出するタスクであり,非教師ありのコントラスティブな設定で未知の概念を自動発見・命名する本研究とはタスク設定が異なる~\cite{luo2024chatabsa}.

\subsection{コントラスティブ要約との関係}

本研究のタスク設定に概念的に最も近いのは,自然言語処理(NLP)分野における「コントラスティブ要約 (Contrastive Summarization)」または「グループ差分要約 (Group-Difference Summarization)」の系譜である~\cite{kardale2023contrastive}.
このタスクは,2 つ以上の文書群を比較し,そのうちの 1 つの集合に特有で,かつ関連性の高い差異をハイライトする要約を生成することを目的とする.

LLM を用いた先行研究として,STRUM-LLM (Saha et al., 2024)~\cite{saha2024strumllm}が挙げられる.
これは,2 つの比較対象(例:製品 A vs 製品 B)の差分を,LLM を用いた多段階パイプラインで属性付きの構造化要約として生成する.

また,Luss et al. (2024)~\cite{luss2024cell}は,CELL(Contrastive Explanations for Large Language Models)を提案し,LLM の出力に対する対比的な説明(なぜその出力が選ばれ,他の出力が選ばれなかったのか)を生成するという点で,本研究と問題意識を共有している.
一方で,CELL はあくまで個々の出力インスタンスに対する説明に焦点を当てており,本研究が対象とするような,ニューロン発火群 $A$ と非発火群 $B$ の\textbf{集合レベルの一般的・代表的な差分}を自然言語ラベルとして要約するタスクとは目的と対象が異なる.

\subsection{本研究の新規性}

STRUM-LLM は,Web 検索を含む複雑なパイプラインを構築し,一般的な製品比較タスクに特化している.
対照的に,本研究は,この「コントラスティブ要約」のフレームワークを XAI ドメイン(モデル内部状態)に初めて適用する点に新規性がある.
本研究は,ニューロンの発火群と非発火群という,より抽象的なテキスト集合の差分抽出に対し,Few-shot プロンプティングという,より簡潔なアプローチで,スケーラブルな自然言語ラベリングの実現可能性を検証するものである.
これは,LLM 命名の知見と,コントラスティブ要約のタスク定義を,XAI の文脈で融合した新規な交差点に位置づけられる.

本研究は,個別インスタンスの説明に留まる LIME/SHAP の限界,命名の課題を残す UCBM/CCE のボトルネック,そして一般 NLP タスクに留まっていたコントラスティブ要約の知見を統合することで,スケーラブルな内部状態の説明を実現する新たな手法を提案するものである.


\chapter{提案手法}

本章では,大規模言語モデル(LLM)を用いて,ニューラルネットワークの特定の内部状態(ニューロンの発火条件)に対応するテキスト集合間の意味的差異を,自然言語の「対比因子ラベル」として自動生成する手法を提案する.本手法は,既存のコンセプトベース XAI(C-XAI)が抱える「人手による命名依存」という最大のボトルネックを解消することを目的としており,ニューロン発火条件の集合的差分を自然言語で記述するタスク設定に特化している.本論文では,集合 $A$ と $B$ の差分を表現するこの自然言語ラベルを一貫して「対比因子ラベル」と呼ぶ.

\section{対比因子命名タスクの定式化}

本研究が提案する対比因子命名タスクは,従来の XAI 手法(個別インスタンスの説明や概念分類)とは根本的に異なるタスクを定式化する~\cite{ribeiro2016should,kim2018interpretability}.

\subsection{タスクの定義}

モデル内部の特定のニューロン $N$(または非教師あり抽出されたコンセプト $C$)に着目する.本研究では,モデル内部で抽出されるこれらの潜在特徴を総称して「ニューロン(またはコンセプト)」と呼び,一方で SemEval などのデータセット側の人手ラベルを「アスペクト」と呼ぶ.学習データセット $\mathcal{D}$ から,ニューロン $N$ が強く活性化する入力テキストの集合を $A$(発火群),そうではない,あるいは対象とするアスペクトを含まない入力テキストの集合を $B$(非発火群)とする.以下,数学的定式化では「集合 $A$」「集合 $B$」と表記し,プロンプトや実験手順の記述では「グループA」「グループB」と表記する.

このタスクの目的は,集合 $A$ に含まれるテキスト群の意味的・内容的な特徴のうち,集合 $B$ には含まれない差分を抽出・推論し,それを簡潔な自然言語ラベル $L$ として自動的に生成することである.

ここで,
\[
 A = \{ x \in \mathcal{D} \mid \text{activation}(x, N) > \tau_A \}, \quad
 B = \{ x \in \mathcal{D} \mid \text{activation}(x, N) < \tau_B \}
\]
と定義し,$\tau_A > \tau_B$ とする.実験章では,この定義をアスペクトラベルや類似度スコアに基づくグループ A/B の抽出として具体化する.$L$ は,この差分を要約する自然言語フレーズ(例:「価格に関する言及」)である.

このタスク全体は,集合入力からラベルへの写像
\[
 (A, B) \mapsto L
\]
として表される.以下では,この写像を便宜的に \texttt{textContrastiveNaming} と呼ぶ.

\subsection{集合差分によるコントラストの必要性}

上記の定式化は,以下の点で従来の XAI と一線を画す.

\begin{enumerate}
 \item \textbf{個別性からの脱却}:
 従来の LIME や SHAP は,単一のインスタンス $x$ に対して,その予測 $y$ に寄与した特徴量(ピクセル,トークン)を可視化するものである.一方,本タスクは,集合 $A$ と $B$ の間の一般的・代表的な意味的コントラストを抽出することに焦点を当てる.
 \item \textbf{潜在表現への命名}:
 Unsupervised CBM(UCBM)や Compositional Concept Extraction(CCE)といった非教師あり概念抽出手法は,概念を潜在ベクトル $v$ として発見するが,そのベクトルに自然言語ラベル $L$ を付与する機能を持たない.本定式化 $\text{textContrastiveNaming}(A,B) \to L$ は,UCBM などが抽出した潜在概念(ニューロンの発火条件)に対し,自動で人間が理解できる「名前」を付与する命名モジュールとして機能する.
\end{enumerate}

このように,対比因子命名タスクは,ニューロンが何を計算しているかというメカニズムの理解を,集合間の意味的なコントラストとして自然言語化する,新しい XAI のタスクである.

\section{LLM によるコントラスティブ要約の実行}

本研究では,定式化された対比因子命名タスクを,大規模言語モデル(LLM)の強力な文脈理解能力と自然言語生成能力を活用して解決する.LLM(本実験では GPT-4o-mini を想定)を,集合 $A$ と $B$ の差分を推論し,ラベルを生成するコントラスト生成器として利用する.

\subsection{処理フロー}

提案手法は,概念図(図~\ref{fig:proposal-overview})に対応する,以下の 3 段階の処理フローを持つ.図が参照できない場合でも,以下の記述のみで流れが理解できるように構成する.

\begin{enumerate}
 \item \textbf{データ抽出とグルーピング(Activation Extraction and Grouping)}:
 解釈対象のニューラルネットワーク $M$ と,特定のニューロン $N$ を選択する.評価データセット $\mathcal{D}$ の各テキスト $x$ を $M$ に入力し,ニューロン $N$ の活性化値 $\text{activation}(x, N)$ を測定する.測定された活性化値に基づき,ハイパーパラメータ \texttt{group\_size} を用いて,活性化値が最も高いテキスト群 $A$ と,活性化値が最も低い(またはランダムな)テキスト群 $B$ を抽出する.すなわち,
 \[
   A = \{ x_1, \dots, x_k \}, \quad
   B = \{ x'_1, \dots, x'_k \}, \quad
   k = \texttt{group\_size}
 \]
 とする.\texttt{group\_size} は,プロンプトのコンテキスト長制限やニューロン活性化のスパース性に応じて決定される.第4章のグループサイズ比較実験では,group\_size を 50〜300 の範囲で変化させても BERTScore の変動は小さく,性能に与える影響は限定的であることが確認された.そのため,本研究ではコンテキスト長と計算量のトレードオフを踏まえ,デフォルト値として \texttt{group\_size} = 100 を採用する.
 \item \textbf{プロンプト設計と差分推論(Prompt Engineering and Contrast Inference)}:
 抽出されたテキスト集合 $A$ と $B$ の内容を,LLM の入力プロンプトに組み込む.プロンプトは,LLM に対し,単なる要約ではなく「グループ $A$ に特徴的でグループ $B$ には見られない主要な違いを特定し,簡潔に回答する」というコントラスティブな推論タスクとして明確に指示する.
 特に,実験で用いるプロンプトは第4章の実験設定と整合するよう,次の要素から構成される.
 \begin{itemize}
  \item \textbf{タスク説明}:まず「2つのデータグループを比較して,グループAに特徴的でグループBには見られない表現パターンや内容の特徴を特定してください」といった指示文を提示する.
  \item \textbf{Few-shot 例(オプション)}:Few-shot 設定が 0 より大きい場合,\texttt{examples\_section} に「【例題$N$】グループA: [...] グループB: [...] 回答: [正解ラベル]」という形式の例を 1 件または 3 件挿入する.
  \item \textbf{集合 $A$ のテキスト群}:【グループA】の見出しの下に,各テキストを「- [テキスト内容]」の形式で最大 \texttt{group\_size} 件列挙する.
  \item \textbf{集合 $B$ のテキスト群}:【グループB】の見出しの下に,同様の形式でテキストを列挙する.
  \item \textbf{出力制約}:プロンプト末尾で「英語で5-10単語程度で,グループAに特徴的でグループBには見られない主要な違いを簡潔に回答してください」と指示し,短い対比因子ラベルの生成を求める.
 \end{itemize}
 LLM は,このプロンプトを入力として受け取り,集合 $A$ のテキスト群には頻出するが,集合 $B$ のテキスト群には見られない語彙,文脈,意味的構造を推論する.この推論能力が,人間による手動分析なしに意味的な差分抽出を可能にする鍵となる.
 \item \textbf{対比因子ラベルの生成(Contrastive Factor Label Generation)}:
 LLM は,推論結果に基づき,集合 $A$ の意味的特性を簡潔に表現した自然言語ラベル $L$ を生成する.例えば,集合 $A$ が「価格が高すぎる」といったレビューを含み,集合 $B$ がレビューを含むが価格には言及しない場合,生成されるラベル $L$ は「価格に関する言及」となる.
\end{enumerate}

生成された対比因子ラベル $L$ の妥当性は,第4章で述べるように,主に BERTScore を主要指標,BLEU を参考指標,さらに LLM による意味的類似度評価を補助指標として定量的に評価する.本章で述べたコントラスティブ要約の枠組みは,第2章で整理した既存のコントラスティブ要約研究(例:STRUM-LLM)を,モデル内部状態の解釈という XAI 文脈に適用したものに相当する.

\begin{figure}[htbp]
  \centering
  % \includegraphics[width=0.8\linewidth]{fig_proposal_overview.png}
  \caption{提案手法の概念図(データ抽出・プロンプト設計・ラベル生成の 3 段階)}
  \label{fig:proposal-overview}
\end{figure}

\section{Few-shot ICL による出力安定性の検証}

本研究では,大規模言語モデルの活用において,Few-shot インコンテキスト・ラーニング(ICL)を,生成される対比因子ラベルの出力形式と安定性を制御するための補助的な手段として導入する.同時に,第4章で示すように,Few-shot 設定とくに 1-shot は BERTScore の向上にも寄与しており,単なる検証手段にとどまらず性能面でも重要な役割を果たす.

\subsection{ICL の役割:安定性とスタイルの確保}

LLM の Few-shot ICL は,プロンプト内にタスクの入力と出力の例(デモンストレーション)を含めることで,モデルがタスクの形式や文体,語彙の傾向を模倣する特性を持つ.この特性を活用し,本研究では,生成される「対比因子ラベル」の出力形式の揺らぎや語彙の安定性を確保するために ICL を用いる.

特に,SemEval-2014 データセットの正解ラベル(例:「food」「price」)は単語や簡潔なフレーズであることが多いため,LLM が出力するラベルをこの正解ラベルのスタイルに近づけ,比較可能性を高めるために ICL を検証する.

\subsection{Few-shot バリエーションの検証}

実験では,Few-shot ICL のバリエーションとして,以下の設定を定量的に検証する.

\begin{enumerate}
 \item 0-shot(Zero-shot):プロンプトにデモンストレーションを含めず,指示文のみで LLM にラベル生成を要求する.これにより,LLM がタスク定義のみに基づいてどれだけ命名できるか,そのベースライン性能を測定する.
 \item 1-shot(One-shot):プロンプトに,正解ラベルが既知の $(A_{\mathrm{ex}}, B_{\mathrm{ex}}, L_{\mathrm{ex}})$ のペアを 1 組含める.
 \item 3-shot(Three-shot):プロンプトに,正解ラベルが既知の $(A_{\mathrm{ex}}, B_{\mathrm{ex}}, L_{\mathrm{ex}})$ のペアを 3 組含める.
\end{enumerate}
Few-shot 例 $(A_{\mathrm{ex}}, B_{\mathrm{ex}}, L_{\mathrm{ex}})$ は,各データセットのアスペクトラベルに基づき,正解ラベルが明確で代表性の高い組み合わせから選定する.例の形式は「【例題$N$】グループA: [...] グループB: [...] 回答: [正解ラベル]」とし,タスク説明の直後に \texttt{examples\_section} として挿入する.

\subsection{検証の目的}

Few-shot ICL は,LLM の生成が入力例のスタイルまで強く模倣する特性を持つため,生成ラベルが「food」という単語(語彙的安定性が高い)にどの程度収束するか,あるいは「食べ物の品質に関する言及」といった説明的なフレーズ(語彙的多様性が高い)になるか,という出力スタイルの影響を定量的に分析する目的で実施される.

Few-shot 実験の詳細な結果は第4章で報告するが,Steam データセットにおける実験では,1-shot 設定が平均 BERTScore 0.6530 と最も高い意味的関連性を示した.この結果は,1-shot の Few-shot 例がラベルのスタイルと意味的妥当性の両面でバランスが良いことを示唆しており,本研究におけるデフォルト設定の根拠となる.

\chapter{評価実験}

\section{実験の目的と概要}
本章では,提案手法の評価実験の目的,使用データセット,実験設定,評価指標を述べる.実験結果と考察は第5章で報告する.
本実験の目的は,外部ラベルにもとづいて定義した 2 つのテキスト集合 $A/B$ の差分から,LLM(GPT-4o-mini 等)が対比因子ラベルをどの程度生成できるか,また生成ラベルが人手アノテーションによる正解ラベル(またはその説明文)とどの程度意味的に一致するかを,定量的に評価することである.

提案手法のドメイン汎用性を検証するために,レビュー,感情分類,画像キャプションという多様なドメインに属する 4 種類のデータセットを用いた.
これらのデータセットは,それぞれが持つ正解アスペクトラベルを,LLM が生成する対比因子ラベルの妥当性を評価するためのグラウンド・トゥルースとして使用した.
使用したデータセットの概要を表~\ref{tab:dataset_overview}に示す.
\begin{table}[htbp]
  \centering
  \caption{使用データセットの概要}
  \label{tab:dataset_overview}
  \begin{tabular}{lll}
    \toprule
    データセット & ドメイン & 特徴 \\
    \midrule
    SemEval-2014 ABSA & レビュー & 具体的アスペクト(Food,Service等) \\
    GoEmotions & 感情分類 & 抽象的概念(28感情カテゴリ) \\
    Steam Review Aspect Dataset & レビュー & ドメイン固有で抽象度の高いアスペクト(Gameplay等) \\
    COCO Retrieved Concepts & 画像キャプション & 視覚的概念記述 \\
    \bottomrule
  \end{tabular}
\end{table}

本実験は,以下の 6 つの実験カテゴリで構成された.
データセット別比較では,SemEval-2014 ABSA,GoEmotions,Steam Review Aspect Dataset の 3 データセットを用いて,提案手法の基本性能を評価した.
Few-shot 設定による性能比較実験では,0-shot,1-shot,3-shot の 3 つの設定を比較した.
グループサイズの影響分析実験では,group\_size を 50,100,150,200,300 の 5 段階で変化させた.
モデル比較実験では,GPT-4o-mini と GPT-5.1 の 2 モデルを比較した.
アスペクト説明文の効果検証実験では,アスペクト説明文の有無による性能差を検証した.
COCO Retrieved Concepts 実験では,正解ラベルがない画像キャプションデータセットに対する対比因子生成を検証した.
各実験カテゴリの概要を表~\ref{tab:experiment_overview}に示す.
\begin{table}[htbp]
  \centering
  \caption{実験カテゴリの概要}
  \label{tab:experiment_overview}
  \begin{tabular}{ll}
    \toprule
    実験カテゴリ & 目的・検証内容 \\
    \midrule
    データセット別比較 & 基本性能評価 3データセット \\
    Few-shot実験 & 0/1/3-shot設定の比較 \\
    グループサイズ比較 & group\_size(50--300)の影響分析 \\
    モデル比較 & GPT-4o-mini vs GPT-5.1 \\
    アスペクト説明文比較 & 説明文の有無による性能差 \\
    COCO実験 & 正解ラベルなしデータセットでの検証 \\
    \bottomrule
  \end{tabular}
\end{table}

評価の観点として,以下の 3 つの指標を用いた(各指標の詳細な定義は\ref{subsec:evaluation_metrics}節を参照).
有効性の評価には,Sentence-BERT(SBERT)文埋め込みのコサイン類似度を 0.0〜1.0 に正規化した値(以降,SBERT類似度)を主要指標として使用し,生成ラベルと正解ラベルの意味的類似度を測定した.
汎用性の評価には,複数のドメイン(レビュー,感情分類,画像キャプション)と複数のアスペクトタイプ(具体的アスペクト,抽象的概念)に対する性能を測定した.
性能の評価には,BLEU スコアを参考指標として使用し,語彙レベルの一致度を補助的に確認した.
また,LLM による意味的類似度評価を補助指標として用い,GPT-4o-mini による 5 段階評価を実施した.
用いた評価指標の概要を表~\ref{tab:evaluation_metrics_overview}に示す.
\begin{table}[htbp]
  \centering
  \caption{評価指標の概要}
  \label{tab:evaluation_metrics_overview}
  \begin{tabular}{lll}
    \toprule
    評価指標 & 役割 & 位置づけ \\
    \midrule
    SBERT類似度 & 意味的類似度測定 & 主要指標 \\
    BLEU & 語彙レベル一致度確認 & 参考指標 \\
    LLM評価 & 意味的類似度評価(5段階) & 補助指標 \\
    \bottomrule
  \end{tabular}
\end{table}

\section{データセット}
\label{sec:dataset}

\subsection{データセットの選定理由}
本実験では,ドメインの多様性を確保するため,以下の 4 種類のデータセットを選定した.
SemEval-2014 ABSA は,アスペクトベース感情分析の標準的なベンチマークとして広く使用されており,正解ラベルが人手でアノテーションされている~\cite{pontiki-EtAl:2014:SemEval2014Task4}.
GoEmotions は,感情という抽象的な概念を扱うデータセットであり,具体的な物理的実体を持たない概念の命名精度を測るために選定した~\cite{demszky2020goemotions}.
Steam Review Aspect Dataset は,ゲームという特定の製品ドメインに特化したデータセットであり,専門性の高いテキスト集合間の意味的差分を抽出する能力を検証するために選定した~\cite{srec:steam-review-aspect-dataset}.
Retrieved Concepts(COCO Captions)は,画像キャプションデータセットであり,視覚的概念記述の生成能力を検証するために選定した~\cite{lin2014microsoft}.

ドメインの多様性として,レビューテキスト(SemEval-2014,Steam),感情分類テキスト(GoEmotions),画像キャプション(COCO)という異なるテキストタイプを網羅した.
また,具体的なアスペクト(Food,Price,Gameplay,Technical)と抽象的な概念(Atmosphere,Story,感情カテゴリ)を含むデータセットを扱うことで,概念の性質やデータセット特性と性能の関係を考察できるようにした.

以降の考察でデータセット特性を定量的に参照するため,テキスト統計の概要を表~\ref{tab:text_stats_overview}に示す(語数は小文字化後の空白分割による).
\begin{table}[htbp]
  \centering
  \caption{テキスト統計の概要(train/dev/test 合算)}
  \label{tab:text_stats_overview}
  \begin{tabular}{lrrrrrr}
    \toprule
    データセット & 件数 & 平均語数 & 中央語数 & 最大語数 & 平均ラベル数 & 複数ラベル率 \\
    \midrule
    GoEmotions & 54,263 & 12.83 & 12 & 33 & 1.18 & 16.25\% \\
    Steam Review Aspect & 1,100 & 244.70 & 156 & 1,473 & 3.27 & 90.27\% \\
    SemEval-2014 restaurant14 & 4,728 & 19.41 & 18 & 79 & 1.00 & 0.00\% \\
    SemEval-2014 laptop14 & 2,966 & 20.75 & 18 & 83 & 1.00 & 0.00\% \\
    \bottomrule
  \end{tabular}
\end{table}

\subsection{Steam Review Aspect Dataset}
Steam Review Aspect Dataset は,Steam ゲームレビューから収集されたテキストデータであり,特定のゲームアスペクトに関する言及を含むデータセットである~\cite{srec:steam-review-aspect-dataset}.
本データセットは英語の Steam ゲームレビュー 1,100 件(学習用 900 件,テスト用 200 件)で構成され,データ収集は SRec データベースのスナップショットに基づき行われた.レビューを特徴づける 8 種類のアスペクトが人手でアノテーションされており,Recommended(推奨),Story(物語),Gameplay(ゲームプレイ),Visual(視覚),Audio(聴覚),Technical(技術),Price(価格),Suggestion(提案・要望)からなる.ゲームという特定の製品ドメインに特化しており,特に \texttt{Gameplay} や \texttt{Technical} といったゲーム固有のメカニクスや技術的側面に関するアスペクトを含む.テストセットにおけるアスペクト件数の例として,Gameplay が 154 件,Recommended が 148 件,Story が 89 件である.
レビュー長は平均 244.70 語(中央値 156 語,最大 1,473 語)であり,語彙タイプ数は 29,056 語である.また,1 サンプルあたりの平均ラベル数は 3.27 で,複数アスペクトを併記するレビューは 90.27\% を占める.

実験での使用方法として,各アスペクトについて,そのアスペクトを含むテキスト群をグループ A,含まないテキスト群をグループ B として抽出した.
分割タイプは \texttt{aspect\_vs\_others} を用い,特定のアスペクトが含まれるテキストと含まれないテキストを比較した.
グループ A と B の抽出は,各アスペクトのラベルに基づいて行い,group\_size パラメータに応じてサンプル数を調整した.
データセット別比較では,group\_size = 100 を用い,各グループから最大 100 件のテキストを抽出した.

% データセット別比較で使用したアスペクト一覧は第\ref{sec:experiment_setup}節でまとめて述べる.

\subsection{SemEval-2014 ABSA(Restaurants)}
SemEval-2014 ABSA(Restaurants)は,アスペクトベース感情分析(ABSA)の標準的なベンチマークとして広く使用されるレストランレビューのデータセットである~\cite{pontiki-EtAl:2014:SemEval2014Task4}.
レストランレビューのテキストを含み,各文に対してアスペクト(観点)とそれに対する感情極性が人手でアノテーションされている.本研究では主に Food(食べ物),Service(サービス),Price(価格),Atmosphere(雰囲気)の 4 種類のアスペクトを用いる.Food や Price は具体的な名詞や数値に関連する言及が中心となる一方,Atmosphere は広範な文脈や比喩的表現からの推論を必要とし,データセット特性と命名性能の関係を考察する際の比較対象となる.
本データセットは 1 文あたり 1 アスペクトを前提としており,restaurant14 は平均 19.41 語,laptop14 は平均 20.75 語と短文である(表~\ref{tab:text_stats_overview}).

実験での使用方法として,SemEval-2014 データセットにおいて,Restaurant ドメインから Food と Service,Laptop ドメインから Battery と Screen の 4 アスペクトを用いた.
各アスペクトについて,そのアスペクトを含むテキスト群をグループ A,含まないテキスト群をグループ B として抽出し,split\_type = \texttt{aspect\_vs\_others} で分割した.

使用したアスペクトの選定理由として,Food は具体的な名詞や数値に関連する言及が中心となる具体的なアスペクトとして,Service,Battery,Screen は製品の属性に関する具体的なアスペクトとして選定した.
これにより,具体的なアスペクトにおける命名性能を評価できるようにした.

\subsection{GoEmotions}
GoEmotions は,細粒度感情分類タスクのために Reddit コメントから収集されたデータセットである~\cite{demszky2020goemotions}.
Demszky らによって構築された総レコード数 63,812 件から成るマルチラベル形式のデータセットであり,28 の感情カテゴリ(27 感情 + neutral)でラベル付けされている.主要なカテゴリには Joy(喜び),Anger(怒り),Admiration(称賛),Neutral(中立)などが含まれる.感情という抽象的概念をテキスト集合差分から推論する必要があるため,具体的な物理的実体を持たない概念の命名精度を検証する目的で用いた.実験では任意の感情アスペクト(例:joy)を指定し,その感情を含むテキスト群 $A$ とその他のアスペクトを含むテキスト群 $B$ を比較する設定とした.
平均 12.83 語の短文が中心であり,複数感情ラベルを併記するサンプルは 16.25\% である(表~\ref{tab:text_stats_overview}).

実験での使用方法として,GoEmotions データセットから全 28 感情カテゴリを用いた.
各感情カテゴリについて,その感情を含むテキスト群をグループ A,含まないテキスト群をグループ B として抽出し,split\_type = \texttt{aspect\_vs\_others} で分割した.

28 感情カテゴリの選定理由として,感情は物理的な実体を持たない高度に抽象的な概念であり,具体的なアスペクトと比較して命名精度が低下する傾向があるかを検証するために,全カテゴリを対象とした.
これにより,抽象的な概念における命名性能を包括的に評価できるようにした.

\subsection{Retrieved Concepts(COCO Captions)}
Retrieved Concepts(COCO Captions)は,視覚的概念記述の生成能力を検証するために,画像キャプションデータセット COCO に基づいて構築されたデータセットである~\cite{lin2014microsoft}.
MS-COCO 2017 train split の画像に対して,実験協力者のFarnoosh Javar によって訓練された非教師ありコンセプト発見モデルから得られた 300 個の潜在コンセプト埋め込みと,CLIP(ViT-B/32)による画像埋め込みとのコサイン類似度を計算し,各コンセプトについて類似度が高い画像 Top-100 と低い画像 Bottom-100 を取得している.この潜在コンセプト埋め込みは,UCBM~\cite{schrodi2024unsupervised} に代表される辞書学習型の非教師ありコンセプト発見手法と同様に,既存モデルの中間表現から自動的に概念ベクトルを抽出するタイプのモデルにより学習されているが,本論文では上流モデルの詳細には立ち入らず,得られた concept embeddings と,それに対して CLIP 類似度にもとづき取得された Top/Bottom 画像およびその COCO 由来キャプションのみを利用する.
各画像には COCO 由来の人手キャプションが 5 つ付与されており,本研究ではこれらのキャプションのみを集合 $A$, $B$ の要素として利用する.
非教師ありに学習された 300 の潜在コンセプト(concept\_0 ~ concept\_299)に対し,CLIP 類似度にもとづき取得された Top-100/Bottom-100 画像とそのキャプションからなる.各コンセプトに対して人手の正解ラベル(アスペクト名)は与えられておらず,潜在コンセプトとその Top/Bottom 例のみが提供されるため,正解アスペクト名が与えられない設定において,集合差分から対比因子ラベルを生成する挙動を補助的に検証できる.

実験では,300 コンセプトのうち concept\_0,concept\_1,concept\_2,concept\_10,concept\_50 の 5 コンセプトを用いた.
各コンセプトについて,潜在コンセプト埋め込みと画像埋め込みとの CLIP(ViT-B/32)コサイン類似度に基づき,類似度が高い順に Top-100,低い順に Bottom-100 の画像を選び,それらに付与されたキャプションをグループ A(Top 側),グループ B(Bottom 側)として用いた.
分割タイプは \texttt{aspect\_vs\_bottom100} を用いた.

正解ラベルが存在しないため,SBERT類似度 と BLEU スコアは参考値として記録するにとどめ,主に生成された対比因子と対応する画像群との整合性に基づく定性的評価を行った.

\section{実験設定}
\label{sec:experiment_setup}

\subsection{実験パイプラインの概要}
本実験では,コントラスティブ要約に基づく対比因子ラベル生成器として,GPT-4o-mini を含む複数の大規模言語モデルを用いた.
提案手法は,統一されたパイプラインとして構築され,その目的は,特定の概念に対応するテキスト集合 $A$(グループA)と,そうでないテキスト集合 $B$(グループB)の差分から,意味的な対比因子ラベル $L$ を LLM に生成させることである.本章の実験では,第\ref{sec:dataset}節で述べたとおり,SemEval,GoEmotions,Steam では人手アスペクトラベルにもとづき,COCO Retrieved Concepts では Top/Bottom 構造にもとづき,グループA/Bを構成する.

\begin{itemize}
  \item \textbf{タスク定式化}: 第3章で定義したとおり,集合 $A$ と $B$ の差分から自然言語ラベル $L$ を生成する写像 $\text{textContrastiveNaming}(A,B) \to L$ を用いる.本章の実験では,各データセットごとに定義されたグルーピング規則にもとづき構成したグループA/Bを入力として用いる.
  \item \textbf{グルーピング}: 各データセットごとに定義された規則に従い,ハイパーパラメータ $group\_size$ を用いて集合 $A$ と $B$ を抽出する.SemEval,GoEmotions,Steam ではアスペクトラベルの有無にもとづき,COCO Retrieved Concepts では各潜在コンセプトに対する Top-100/Bottom-100 画像のキャプションにもとづいてグループA/Bを構成する.
  データセット別比較では,プロンプトのコンテキスト長制限を考慮し,$group\_size = 100$ を用いた.
  \item \textbf{Few-shot ICL の検証}: LLM の出力形式の揺らぎや語彙の安定性を確保するための検証手段として,Few-shot インコンテキスト・ラーニング(ICL)のバリエーション(0-shot, 1-shot, 3-shot)を定量的に検証した.
  LLM は,プロンプト内でグループAとグループBを比較し,グループAに特徴的でグループBには欠如している意味的側面を推論するよう指示された.
\end{itemize}

実験パイプラインの全体フローを以下に示す.

\begin{enumerate}
  \item \textbf{データセット読み込み}:各データセットからテキストを読み込む.
\item \textbf{グループA/B抽出}:各データセットで定義されたグルーピング規則(アスペクトラベル,あるいは Retrieved Concepts における Top/Bottom 構造)に基づき,グループA(特定概念を含むテキスト群)とグループB(含まないテキスト群)を抽出する.
  \item \textbf{プロンプト生成}:グループAとグループBのテキストリストをプロンプトに組み込む.Few-shot例が設定されている場合は,プロンプトにFew-shot例を挿入する.
  \item \textbf{LLMによる対比因子ラベル生成}:GPT-4o-mini等のLLMにプロンプトを入力し,対比因子ラベルを生成する.
  \item \textbf{評価}:生成されたラベルと正解ラベルの意味的類似度をSBERT類似度,BLEU,LLM評価により測定する.
\end{enumerate}

データセット別比較で対象としたアスペクトは,SemEval-2014 から Food,Service,Battery,Screen の 4 種類,GoEmotions から全 28 感情カテゴリ,Steam から Gameplay,Visual,Story,Audio の 4 種類である.

\subsection{LLMモデルとパラメータ設定}
本実験では,対比因子ラベル生成器として GPT-4o-mini を主要モデルとして用いた.
モデル比較実験では,GPT-5.1 も使用した.
GPT-4o-mini を選択した理由は,コスト効率が高く,かつ十分な性能を発揮することが既存研究で確認されているためである.

各実験カテゴリでのモデル選択として,データセット別比較,Few-shot 実験,グループサイズ比較実験では GPT-4o-mini を使用した.
アスペクト説明文比較実験では GPT-4o を使用した.
モデル比較実験では,GPT-4o-mini と GPT-5.1 の 2 モデルを比較した.
COCO Retrieved Concepts 実験では,GPT-4o-mini を使用した.

温度パラメータ(temperature)の設定として,データセット別比較では temperature = 0.0 を用いた.
この設定により,決定論的な出力が得られ,実験の再現性が確保される.
Few-shot 実験,グループサイズ比較実験,モデル比較実験,アスペクト説明文比較実験においても temperature = 0.0 を用いた.
COCO Retrieved Concepts 実験においても temperature = 0.0 を用いた.

最大トークン数(max\_tokens)の設定として,データセット別比較では max\_tokens = 2000 に設定した.
Few-shot 実験,モデル比較実験では max\_tokens = 100 に設定した.
グループサイズ比較実験,アスペクト説明文比較実験,COCO Retrieved Concepts 実験では max\_tokens = 2000 に設定した.

その他の生成パラメータとして,top\_p や frequency\_penalty はデフォルト値を使用した.

\subsection{プロンプト設計}
プロンプトテンプレートは,以下の構造を持つ.
まず,タスクの説明として \texttt{2つのデータグループを比較して,グループAに特徴的でグループBには見られない表現パターンや内容の特徴を特定してください} という指示を提示する.
次に,Few-shot 例が存在する場合は \texttt{examples\_section} に挿入される.
Few-shot 例の形式は \texttt{【例題N】グループA: [...] グループB: [...] 回答: [正解ラベル]} である.
その後,実際のデータとして【グループA】と【グループB】のテキストリストが提示される.
各テキストは \texttt{- [テキスト内容]} の形式で列挙され,コンテキスト長制限を考慮して最大 100 件に制限される.
最後に,出力形式の指示として \texttt{英語で5-10単語程度で,グループAに特徴的でグループBには見られない主要な違いを簡潔に回答してください} が追加される.

グループA/Bの提示方法として,各テキストは \texttt{- [テキスト内容]} の形式で列挙され,グループAとグループBはそれぞれ \texttt{【グループA】} と \texttt{【グループB】} の見出しで区別された.
コンテキスト長制限を考慮し,各グループから最大 100 件のテキストを抽出した.

出力形式の指示方法として,\texttt{英語で5-10単語程度で,グループAに特徴的でグループBには見られない主要な違いを簡潔に回答してください} という指示をプロンプトの末尾に追加した.

Few-shot 例の挿入方法として,Few-shot 設定が 0 より大きい場合,プロンプトのタスク説明の後に \texttt{examples\_section} を挿入した.
Few-shot 例の形式は \texttt{【例題N】グループA: [...] グループB: [...] 回答: [正解ラベル]} であり,$N$ は例題番号である.

アスペクト説明文の使用方法として,アスペクト説明文比較実験では,アスペクトの説明文をプロンプトの冒頭に追加した.
例えば,Food アスペクトの場合,\texttt{Food refers to mentions of food quality, taste, menu items, or dining experience} といった説明文を挿入した.
説明文ありの条件と説明文なしの条件を比較することで,アスペクト説明文の効果を検証した.

\subsection{データの前処理と分割方法}
テキストの前処理手順として,各データセットからテキストを読み込み,アスペクトラベルに基づいてグループ A とグループ B に分割した.
テキストの前処理として,特殊文字の処理や正規化は行わず,データセットの生のテキストをそのまま使用した.

グループA/Bの抽出方法として,各アスペクトについて,そのアスペクトを含むテキスト群をグループA,含まないテキスト群をグループBとして抽出した.
分割タイプは \texttt{aspect\_vs\_others} を用い,特定のアスペクトが含まれるテキストと含まれないテキストを比較した.
COCO Retrieved Concepts 実験では,分割タイプとして \texttt{aspect\_vs\_bottom100} を用い,Top-100のキャプションをグループA,Bottom-100のキャプションをグループBとして抽出した.

グループサイズ(group\_size)の決定方法として,データセット別比較では group\_size = 100 を用いた.
この値は,プロンプトのコンテキスト長制限を考慮して決定された.
グループサイズ比較実験では,group\_size を 50,100,150,200,300 の 5 段階で変化させた.

サンプリングは,各グループの候補集合から group\_size 件になるようランダムにサブサンプリングし,候補数が group\_size 未満の場合は重複を許して補完した.重み付きサンプリングは使用しなかった.

コンテキスト長制限への対応として,各グループから最大 100 件のテキストを抽出し,プロンプトのコンテキスト長を制限内に収めた.
データセット別比較では group\_size = 100 を用い,グループサイズ比較実験では最大 300 まで検証したが,コンテキスト長超過エラーを回避するため,実際のプロンプトでは必要に応じてテキスト数を制限した.

\subsection{Few-shot例の作成方法}
Few-shot例の選定基準として,各データセットのアスペクトラベルに基づき,正解ラベルが明確な例を選定した.
Few-shot例は,グループAとグループBのテキストリストと,それに対応する正解ラベルで構成された.

例の品質管理方法として,Few-shot 例は,各データセットのアスペクトラベルに基づいて作成し,正解ラベルが明確であることを確認した.
Few-shot 例の形式は \texttt{【例題N】グループA: [...] グループB: [...] 回答: [正解ラベル]} であり,$N$ は例題番号である.

0-shot,1-shot,3-shot の違いと設定方法として,0-shot 設定では Few-shot 例を挿入せず,タスク説明のみを提示した.
1-shot 設定では,1 つの Few-shot 例を \texttt{examples\_section} に挿入した.
3-shot 設定では,3 つの Few-shot 例を \texttt{examples\_section} に挿入した.
Few-shot 実験では,0-shot,1-shot,3-shot の 3 つの設定を比較した.

\subsection{実験カテゴリの定義}
各実験カテゴリのパラメータ設定を表~\ref{tab:experiment_config}に示す.

temperature は全条件で 0.0 とし,top\_p や frequency\_penalty などその他の生成パラメータはデフォルト値を使用した.
LLM 評価は gpt-4o-mini(temperature = 0.0)で実施し,COCO 実験のみ無効とした.

\begin{table}[htbp]
\centering
\caption{実験カテゴリ別パラメータ設定(変動項目のみ)}
\label{tab:experiment_config}
\small
\begin{tabular}{lllll}
  \toprule
  実験カテゴリ & max\_tokens & few\_shot & group\_size & 生成モデル \\
  \midrule
  データセット別比較 & 2000 & 0 & 100 & \texttt{4o-mini} \\
  Few-shot実験 & 100 & 0/1/3 & 100 & \texttt{4o-mini} \\
  グループサイズ比較 & 2000 & 0 & 50--300 & \texttt{4o-mini} \\
  モデル比較 & 100 & 0 & 100 & \texttt{4o-mini, 5.1} \\
  アスペクト説明文比較 & 2000 & 0 & 100 & \texttt{4o} \\
  COCO実験 & 2000 & 0 & 100 & \texttt{4o-mini} \\
  \bottomrule
\end{tabular}
\end{table}

\subsection{比較手法とベンチマーク}
LLMによって生成された対比因子ラベルの品質を,人手アノテーションされた既存の ABSA ベンチマーク(SemEval-2014 Restaurant/Laptop,Steam レビューなど)の正解ラベルとの意味的類似性と比較することで評価した.

\section{評価指標}
\label{sec:evaluation_metrics}
生成された自然言語ラベル $L$ の品質を評価するために,SBERT類似度,BLEU,および LLM による意味的類似度評価を用いた.

\subsection{評価指標の選定理由}
\label{subsec:evaluation_metrics}
本実験では,SBERT類似度,BLEU,LLM 評価の 3 つの指標を用いた.
SBERT類似度 を選んだ理由は,生成ラベルと正解ラベルの意味的類似度を測定するためである.
本タスクでは,LLM が生成するラベルが \texttt{食べ物の品質に関する言及} のような説明的なフレーズとなるのに対し,正解ラベルは \texttt{food} のような単一の単語であるため,語彙レベルの一致度を超えたセマンティックな評価が必要である.
BLEU を選んだ理由は,語彙レベルの一致度を補助的に確認するためである.
LLM 評価を選んだ理由は,LLM による意味的類似度評価を補助指標として用いるためである.

各指標の役割と位置づけとして,SBERT類似度 は主要指標として位置づけられ,生成ラベルと正解ラベルの意味的類似度を測定した.
BLEU は参考指標として位置づけられ,語彙レベルの一致度を補助的に確認した.
LLM 評価は補助指標として位置づけられ,GPT-4o-mini による 5 段階評価を実施した.

\subsection{BERTスコア(Sentence-BERT類似度)}
BERT系モデルにより文を埋め込み表現に変換し,文間の意味的な近さを類似度として数値化する指標である.

本実験では,Sentence-BERT~\cite{reimers2019sentence} により生成ラベル $L$ と正解ラベル $L_{\mathrm{ref}}$(アスペクト説明文を用いる条件ではその説明文)を文埋め込みに変換し,コサイン類似度を計算する.実装では,SentenceTransformer(\texttt{all-MiniLM-L6-v2})の文埋め込みを用い,コサイン類似度($-1$〜$1$)を $(\mathrm{cos}+1)/2$ により 0.0〜1.0 に正規化した値を BERTスコアとして扱う.

本指標により,生成ラベルが正解ラベル(またはその説明文)と意味的にどの程度一致しているかを連続値で測り,語彙一致に依存しない有効性評価を行うことを狙う.

値が 1.0 に近いほど意味的に類似していることを示す.

\subsection{BLEU(Bilingual Evaluation Understudy)}
BLEU は,生成文と参照文の間の n-gram の重複にもとづき,語彙レベルの一致度を数値化する指標である~\cite{papineni-etal-2002-bleu}.

本実験では,NLTK~\cite{bird2009natural} の \texttt{sentence\_bleu} により,参照側に正解ラベル(またはその説明文),候補側に生成ラベルを与えて BLEU を算出し,SmoothingFunction.method1 を適用した.評価範囲は 0.0 から 1.0 であり,1.0 に近いほど一致度が高いことを示す.

本指標は,語彙レベルの一致度を補助的に確認するために用いた.ただし,本タスクでは正解ラベルが \texttt{food} や \texttt{price} のように 1 語の名詞句で与えられることが多い一方,生成ラベルは説明的フレーズになりやすく,n-gram が成立しにくいため BLEU が低値になりやすい.したがって,BLEU は参考値として解釈する.

\subsection{LLM評価スコア}
LLM 評価は,LLM を評価器として用い,参照テキストと候補テキストの意味的類似度を段階評価する指標である.

本実験では,参照側に正解ラベル(またはその説明文),候補側に生成ラベルを与え,GPT 系モデルに 5 段階(1--5)で評価させた.データセット別比較,Few-shot 実験,グループサイズ比較実験,アスペクト説明文比較実験では GPT-4o-mini を用い,temperature = 0.0 に設定した.モデル比較実験では GPT-4o を用い,temperature = 0.0 に設定した.COCO Retrieved Concepts 実験では LLM 評価を無効化した.

本指標により,埋め込み類似度では捉えにくい意味的一致/不一致を補助的に確認し,BERTスコアやBLEUの結果の解釈を支えることを狙う.ただし,単一モデルを評価器として用いる自動評価であるため,評価器バイアスが混入し得る点に留意し,補助指標として解釈する.

評価プロンプトの設計として,以下のプロンプトを使用した.
\begin{quote}
\ttfamily
参照テキストと候補テキストの意味的類似度を5段階(1-5)で評価してください.\par
参照テキスト: \{reference\_text\}\par
候補テキスト: \{candidate\_text\}\par
評価基準:\par
- 5: 完全に同じ意味\par
- 4: ほぼ同じ意味(細かい違いのみ)\par
- 3: 類似しているが一部異なる\par
- 2: 部分的に類似している\par
- 1: ほとんど異なる\par
出力形式(JSON形式):\par
\{\par
    "score": 4,\par
    "normalized\_score": 0.8,\par
    "reasoning": "評価理由を簡潔に説明"\par
\}\par
\end{quote}

評価基準(5段階評価の詳細)として,1 から 5 の整数で評価し,5 が最も類似度が高く,1 が最も類似度が低い.
正規化方法として,5 段階評価(1-5)を 0.0-1.0 に正規化し,normalized\_score = (score - 1) / 4 として計算した.

SBERT類似度 との関係として,LLM 評価スコアは SBERT類似度 を補完する補助指標として位置づけられ,両指標を併用することで生成ラベルの品質を多角的に評価した.

\section{統計的分析}
\label{sec:statistical_tests}
追加実験では,条件差がアスペクトに依らず一貫して観測されるかを補足的に確認するため,統計的検定を行った.有意水準は \( \alpha=0.05 \)(両側)とした.検定対象は,生成ラベルと参照ラベル(または説明文)との意味的一致を表す主要指標である SBERT類似度(BERTスコア)とした.検定統計量の計算には SciPy を用いた~\cite{virtanen2020scipy}.

\subsection{繰り返し測定(ブロック)と多水準比較}
Few-shot(0/1/3-shot)および group\_size(50/100/150/200/300)の比較では,Steam の 4 アスペクトをブロック(繰り返し測定の単位)とみなし,条件を要因とする Friedman 検定を実施した~\cite{friedman1937use,demsar2006statistical}.補足として条件ペアごとの対応のある Wilcoxon 検定(両側)も算出し~\cite{wilcoxon1945individual},多重比較の補正には Holm 法を用いた~\cite{holm1979simple}(主検定が非有意の場合,事後比較は参考として扱う).

\subsection{2条件比較}
モデル比較およびアスペクト説明文あり/なしの比較では,Steam の 4 アスペクトをブロックとする対応のある Wilcoxon 検定(両側)を用いた~\cite{wilcoxon1945individual}.

\subsection{指標間の整合(補足)}
SBERT類似度と LLM 評価の序列がどの程度一致するかを補足的に示すため,全 36 条件の順位に対して Spearman の順位相関(両側)を算出した~\cite{spearman1904association}.

\chapter{結果と考察}

第4章で述べた各実験について,得られた結果を示し,次いで実験ごとに個別の考察を行う.その上で,章末において全ての実験カテゴリを横断して総括的に考察する.

本章では (i) メイン実験(複数データセットの結果比較),(ii) 入力やプロンプト条件別の実験(Few-shot 設定,グループサイズ,使用モデル,アスペクト説明文の有無),(iii) 外部の正解ラベルを前提としないデータセットを用いた検証,(iv) 補足的な分析(概念の具体性,エラー分析,統計的分析)のカテゴリ別に,結果と考察を述べる構成で記述する.

\section{メイン実験}
\noindent
本節では,SemEval-2014,GoEmotions,Steam の 3 データセットに対してメイン実験を行った.以降では,データセット別の集計結果,主要アスペクト別の結果,全体統計の順に結果を示し,最後にこれらの結果にもとづく考察を述べる.
\subsection{実験結果}
実験設定の詳細(パラメータ一覧)として,temperature = 0.0,max\_tokens = 2000,few\_shot = 0,group\_size = 100,GPT モデル = gpt-4o-mini,LLM 評価 = 有効(gpt-4o-mini,temperature = 0.0),アスペクト記述 = 無効とした.


データセット別の結果を表~\ref{tab:main_dataset_results}に示す.
\begin{table}[htbp]
  \centering
  \caption{メイン実験:データセット別評価スコア}
  \label{tab:main_dataset_results}
  \begin{tabular}{lccc}
    \toprule
    データセット & SBERT類似度 & BLEU & LLM \\
    \midrule
    SemEval-2014 & & & \\
    \quad 平均 & 0.7531 & 0.0220 & 0.5500 \\
    \quad 最小 & 0.7181 & 0.0123 & 0.4000 \\
    \quad 最大 & 0.8012 & 0.0278 & 0.6000 \\
    \midrule
    GoEmotions & & & \\
    \quad 平均 & 0.7127 & 0.0073 & 0.4714 \\
    \quad 最小 & 0.5437 & 0.0 & 0.2 \\
    \quad 最大 & 0.8941 & 0.0408 & 0.8 \\
    \midrule
    Steam & & & \\
    \quad 平均 & 0.5403 & 0.0 & 0.3 \\
    \quad 最小 & 0.5164 & 0.0 & 0.2 \\
    \quad 最大 & 0.5612 & 0.0 & 0.6 \\
    \bottomrule
  \end{tabular}
\end{table}

表~\ref{tab:main_dataset_results}より,SBERT類似度および LLM スコアは SemEval-2014 が最も高く,次いで GoEmotions,Steam の順に低下した.一方で BLEU は全データセットで低く,特に Steam では 0.0 に張り付いている.

アスペクト別の結果(主要なアスペクト)を表~\ref{tab:main_aspect_results}に示す.
\begin{table}[htbp]
  \centering
  \caption{メイン実験:主要アスペクト別評価スコア}
  \label{tab:main_aspect_results}
  \begin{tabular}{lcccc}
    \toprule
    データセット & アスペクト & SBERT類似度 & BLEU & LLM \\
    \midrule
    \multirow{4}{*}{SemEval-2014} & Food & 0.7286 & 0.0123 & 0.4000 \\
    & Service & 0.7181 & 0.0240 & 0.6000 \\
    & Battery & 0.7646 & 0.0278 & 0.6000 \\
    & Screen & 0.8012 & 0.0240 & 0.6000 \\
    \midrule
    \multirow{4}{*}{GoEmotions} & Joy & 0.8192 & 0.0000 & 0.8000 \\
    & Anger & 0.7214 & 0.0000 & 0.6000 \\
    & Disgust & 0.8316 & 0.0240 & 0.8000 \\
    & Embarrassment & 0.8941 & 0.0408 & 0.8000 \\
    \midrule
    \multirow{4}{*}{Steam} & Gameplay & 0.5612 & 0.0000 & 0.2000 \\
    & Visual & 0.5164 & 0.0000 & 0.2000 \\
    & Story & 0.5383 & 0.0000 & 0.6000 \\
    & Audio & 0.5452 & 0.0000 & 0.2000 \\
    \bottomrule
  \end{tabular}
\end{table}

表~\ref{tab:main_aspect_results}より,SemEval-2014 では \textit{Screen} が最も高い SBERT類似度(0.8012)を示し,GoEmotions では \textit{Embarrassment} が最も高い(0.8941).Steam では SBERT類似度が全体として低い一方,LLM スコアは \textit{Story} のみ高く(0.6000),他のアスペクトは 0.2000 に留まった.

全体の統計を表~\ref{tab:main_overall_stats}に示す.
\begin{table}[htbp]
  \centering
  \caption{メイン実験:全体統計}
  \label{tab:main_overall_stats}
  \begin{tabular}{lccc}
    \toprule
    評価指標 & 平均 & 最小 & 最大 \\
    \midrule
    SBERT類似度 & 0.6980 & 0.5164 & 0.8941 \\
    BLEU & 0.0082 & 0.0000 & 0.0408 \\
    LLM & 0.4611 & 0.2000 & 0.8000 \\
    \bottomrule
  \end{tabular}
\end{table}

表~\ref{tab:main_overall_stats}より,SBERT類似度は 0.5164--0.8941 とばらつきがある一方,BLEU は平均 0.0082 と極めて低い.また,LLM スコアは 0.2000--0.8000 の範囲にあり,条件によって厳しめの評価になっている.

\subsection{考察}
\label{subsec:main_experiment_discussion}
メイン実験の結果から,対比因子タスクにおいて LLM は 一定の性能を示すことが確認できる.全 36 実験の SBERT類似度は平均 0.6980(最小 0.5164,最大 0.8941)であり,データセット別には SemEval-2014(平均 0.7531)と GoEmotions(0.7127)が高く,Steam(0.5403)が大きく低下した(表~\ref{tab:main_dataset_results}).LLM による 5 段階評価でも同様に,SemEval(0.5500)および GoEmotions(0.4714)が Steam(0.3000)を上回っており,データセット間の序列は概ね一致している(表~\ref{tab:main_dataset_results}).

この序列は,データセット特性の違いとして解釈できる.SemEval-2014 や GoEmotions では,アスペクト/感情カテゴリに対応する語彙や典型表現が比較的一貫しており,グループ $A/B$ の差分が特定の意味領域に集中しやすい.その結果,LLM は差分の中心となる要因を短いラベルとして抽出しやすく,安定した命名につながる.一方で Steam はレビューが長く雑多で(平均 251 語,中央値 163 語,最大 1,521 語;語彙タイプ数 15,740),皮肉や複合評価などのノイズも大きいため,アスペクトがレビューの文脈に依存しやすい.このとき $A/B$ の差分は複数の要素に分散しやすく,単一の要因を短いラベルに圧縮することが難しくなるため,平均スコアの低下として現れると考えられる.

評価指標間の関係として,BLEU は平均 0.0082 と極めて低く,0.0 となる条件も多い(表~\ref{tab:main_overall_stats}).この結果は,生成ラベルが正解ラベルと語彙的にはほとんど一致していないことを示している.一方で,SBERT類似度は平均 0.6980 と中程度の値を示しており(表~\ref{tab:main_overall_stats}),語彙が一致しなくても意味的には近い表現へ言い換えられているケースが多いと解釈できる.参考として,メイン実験再実行ログ(36 条件)の語数統計では,正解ラベルは平均 1.0 語,生成ラベルは平均 9.7 語であり,短い正解ラベルに対して生成ラベルが説明的フレーズになりやすいという性質が確認できる.したがって,BLEU の低さは,単純な失敗を意味するというより,本タスクでは言い換えが頻繁に生じるために n-gram 一致に基づく評価が低く出やすいことを反映している.

SBERT類似度と LLM 評価は,スコアの水準は異なるものの,全 36 条件における順位には強い正の相関が確認され(Spearman順位相関 $\rho = 0.73$, $p < 0.001$),どの条件が相対的に良いかという傾向を概ね共有している.ただし,SemEval-2014 と Steam は条件数が少ないため,データセット内での相関推定は不安定である.一方で,LLM 評価の平均は 0.4611 と低めであり,単なる意味的近さだけでなく,対比因子として焦点が合っているか,アスペクト名として簡潔に読めるかといった観点で,より厳しく妥当性を判定していると解釈できる.したがって,SBERT類似度は意味的一致の程度を広く把握する指標,LLM 評価は対比因子ラベルとしての品質を保守的に確認する指標として位置付けられ,両者は互いに補完関係にある.

さらにアスペクト別に見ると,高スコアになりやすい条件は,(1) 語彙的一貫性,(2) グループ $A/B$ の統計的差分の大きさ,(3) 抽象度の低さ(具象性)に依存する.例えば SemEval-2014 の \textit{Screen} は SBERT類似度 0.8012,GoEmotions の \textit{Embarrassment} は 0.8941 と高い値を示しており(表~\ref{tab:main_aspect_results}),これらは典型的表現が比較的明確である.一方で,Steam のようにアスペクト境界が曖昧でノイズが大きい条件では,$A/B$ 差分が単一アスペクトに対応しにくく,性能が低下しやすい.

以上より,本研究目的である対比因子タスクにおける LLM の性能は,条件が整ったデータセット/アスペクトでは 0-shot でも中程度以上の意味的一致を達成する一方,ドメインノイズや多義性が大きい条件では性能が低下しやすい,という形で整理できる.

\section{Few-shot設定による性能比較}
\subsection{実験結果}
実験設定として,Steam データセットを用いて,Few-shot 設定(0-shot,1-shot,3-shot)による性能差を検証した.
実験パラメータは,temperature = 0.0,max\_tokens = 100,group\_size = 100,GPT モデル = gpt-4o-mini,LLM 評価 = 有効(gpt-4o-mini,temperature = 0.0)とした.
総実験数は 12 実験(4 アスペクト × 3 Few-shot 設定)であり,全実験が成功した.

Few-shot 別の平均スコアを表~\ref{tab:fewshot_summary}に,アスペクト別の Few-shot 効果を表~\ref{tab:fewshot_aspect}に示す.
\begin{table}[htbp]
  \centering
  \caption{Few-shot設定による性能比較:平均スコア}
  \label{tab:fewshot_summary}
  \begin{tabular}{lccc}
    \toprule
    Few-shot設定 & SBERT類似度 & BLEU & LLM \\
    \midrule
    0-shot & & & \\
    \quad 平均 & 0.5526 & 0.0 & 0.3000 \\
    \quad 最小 & 0.5462 & 0.0 & 0.2 \\
    \quad 最大 & 0.5596 & 0.0 & 0.4 \\
    \midrule
    1-shot & & & \\
    \quad 平均 & 0.6530 & 0.0 & 0.3500 \\
    \quad 最小 & 0.5111 & 0.0 & 0.2 \\
    \quad 最大 & 0.8356 & 0.0 & 0.8 \\
    \midrule
    3-shot & & & \\
    \quad 平均 & 0.5754 & 0.0 & 0.4000 \\
    \quad 最小 & 0.5416 & 0.0 & 0.2 \\
    \quad 最大 & 0.6449 & 0.0 & 0.6 \\
    \bottomrule
  \end{tabular}
\end{table}

表~\ref{tab:fewshot_summary}より,SBERT類似度は 1-shot が平均 0.6530 と最も高く,0-shot(0.5526)および 3-shot(0.5754)を上回った.一方で LLM スコアは 0-shot から 3-shot にかけて増加し,BLEU は全設定で 0.0 であった.

SBERT類似度についてFriedman検定を実施し,$p=0.4724$で0/1/3-shot間に有意差は確認されなかった(Holm補正付き事後比較も全て非有意)。

\begin{table}[htbp]
  \centering
  \caption{Few-shot設定による性能比較:アスペクト別SBERT類似度}
  \label{tab:fewshot_aspect}
  \begin{tabular}{lccc}
    \toprule
    アスペクト & 0-shot & 1-shot & 3-shot \\
    \midrule
    Gameplay & 0.5462 & 0.6802 & 0.5644 \\
    Visual & 0.5562 & 0.5111 & 0.6449 \\
    Story & 0.5483 & 0.8356 & 0.5416 \\
    Audio & 0.5596 & 0.5850 & 0.5505 \\
    \bottomrule
  \end{tabular}
\end{table}

表~\ref{tab:fewshot_aspect}より,アスペクト別には 1-shot で \textit{Story} が 0.8356 と大きく上昇する一方,\textit{Visual} では 1-shot が 0.5111 と低下しており,Few-shot の効果は一様ではない.

\subsection{考察}
Few-shot 実験では,0-shot,1-shot,3-shot の 3 設定における性能の違いを検証した.
SBERT類似度 の平均は 0-shot で 0.5526,1-shot で 0.6530,3-shot で 0.5754 であり,1-shot で最も高い値を示したが,Friedman 検定の結果($p=0.4724$),統計的有意差は確認できなかった.
一方,LLM スコアは 0-shot で 0.3000,1-shot で 0.3500,3-shot で 0.4000 と単調増加しており,こちらも同様に統計的には有意な差は得られていない.

1-shot 設定では,プロンプト中に 1 つの例題 $(A_{\mathrm{ex}},B_{\mathrm{ex}},L_{\mathrm{ex}})$ を与えることで,出力が正解アスペクト名に近い語彙へと寄りやすくなる傾向が見られた.
この結果,SemEval 型のラベルに対して語彙的に近い短いラベルが生成されやすくなり,Few-shot 実験における SBERT類似度 が平均値として 0.5526(0-shot)から 0.6530(1-shot)へと上昇するなどの傾向が確認されたと解釈できる(表\ref{tab:fewshot_summary}参照).
他方で,3-shot 設定では,複数例のスタイルを平均したより説明的なラベル(複数側面をまとめたフレーズ)が出力されやすくなり,正解ラベルとの埋め込み類似度はやや低下するものの,LLM による「説明としての自然さ」「対比としての一貫性」は向上したと考えられる.

アスペクト別に見ると,\textit{Story} では 1-shot で SBERT類似度 0.8356,LLM スコア 0.8000 と特に高い値を示し,物語的要素を含むレビューに対して Few-shot ICL が有効であることが分かる.
一方,\textit{Visual} や \textit{Audio} では,SBERT類似度 は 0.5700 前後であるにもかかわらず LLM スコアは 0.2000 に留まり,表層的な類似度があっても「対比因子ラベルとして有用か」という観点では厳しく評価されている.
これは,これらのアスペクトでは,対比因子ラベルが「グラフィックス」「サウンド」という表層カテゴリに留まり,具体的な差分(解像度,アートスタイル,音の存在感など)を十分に表現できていないことを反映している.

これらの結果は,Few-shot ICL が単純に性能を単調改善させる仕組みではなく,ラベルのスタイル(キーワード指向か,説明指向か)と評価指標との間にトレードオフを生じさせることを示している.
統計的な有意差は得られていないものの,0-shot でも SBERT類似度 0.5500 程度,LLM スコア 0.3000 程度を達成していることから,Few-shot なしでも本タスクは一定程度機能し,Few-shot の設計により「正解ラベルに近い語彙を出させるか」「説明として自然なフレーズを出させるか」をある程度制御できることが示唆された.
したがって,本タスクにおいては,定量的一致(SBERT類似度)の最大化を重視する場合には 1-shot 程度の例示を,説明としての自然さや対比の一貫性を重視する場合には 3-shot によるスタイル誘導を選択するといったように,傾向レベルの結果を踏まえつつ最適化したい観点に応じた Few-shot 設計が求められる.

\section{グループサイズの影響分析}
\subsection{実験結果}
実験設定として,Steam データセットを用いて,group\_size(50,100,150,200,300)による性能差を検証した.
実験パラメータは,temperature = 0.0,max\_tokens = 2000,few\_shot = 0,GPT モデル = gpt-4o-mini,LLM 評価 = 有効(gpt-4o-mini,temperature = 0.0)とした.
総実験数は 20 実験(4 アスペクト × 5 group\_size)であり,全実験が成功した.

グループサイズ別の性能を表~\ref{tab:groupsize_results}に示す.
Friedman 検定では p=0.3309 となり,Holm 補正後の全ペア比較も含め有意差は確認されなかった.
\begin{table}[htbp]
  \centering
  \caption{グループサイズによる性能比較(各group\_sizeでの全アスペクト平均値)}
  \label{tab:groupsize_results}
  \begin{tabular}{lccc}
    \toprule
    group\_size & SBERT類似度 & BLEU & LLM \\
    \midrule
    50 & 0.5455 & 0.0 & 0.3000 \\
    100 & 0.5436 & 0.0 & 0.3500 \\
    150 & 0.5321 & 0.0 & 0.2500 \\
    200 & 0.5396 & 0.0 & 0.3000 \\
    300 & 0.5375 & 0.0 & 0.2000 \\
    \bottomrule
  \end{tabular}
\end{table}

表~\ref{tab:groupsize_results}より,SBERT類似度は 0.5321--0.5455 の範囲に収まり,group\_size に対する単調な改善は確認できない.また BLEU は全条件で 0.0 であり,LLM スコアも 0.2000--0.3500 の範囲で大きな差は見られない.

全体の統計を表~\ref{tab:groupsize_overall_stats}に示す.
\begin{table}[htbp]
  \centering
  \caption{グループサイズ比較実験:全体統計}
  \label{tab:groupsize_overall_stats}
  \begin{tabular}{lccc}
    \toprule
    評価指標 & 平均 & 最小 & 最大 \\
    \midrule
    SBERT類似度 & 0.5396 & 0.5019 & 0.5636 \\
    BLEU & 0.0000 & 0.0000 & 0.0000 \\
    LLM & 0.2800 & 0.2000 & 0.6000 \\
    \bottomrule
  \end{tabular}
\end{table}

表~\ref{tab:groupsize_overall_stats}より,全条件を通した SBERT類似度の範囲は 0.5019--0.5636 であり,group\_size の変更は平均値レベルでは限定的な影響に留まった.

コンテキスト長との関係として,group\_size が大きくなるほど,プロンプトのコンテキスト長が増加する.
メイン実験では group\_size = 100 を用い,コンテキスト長制限を考慮して各グループから最大 100 件のテキストを抽出した.

\subsection{考察}
Group Size 比較実験では,Steam データセットに対して group\_size を 50, 100, 150, 200, 300 と変化させた.
SBERT類似度 の平均は,それぞれ 0.5455, 0.5436, 0.5321, 0.5396, 0.5375 であり,全体として 0.53--0.55 の範囲に収まった.Friedman 検定の結果($p=0.3309$)および Holm 補正付きの事後比較では,いずれの group\_size 間にも統計的に有意な差は検出されておらず,これらの違いはあくまで平均値レベルの変動にとどまる.
一方,LLM スコアの平均は 0.2000--0.3500 の範囲にあり,group\_size 間で顕著な差は確認されなかった.
また,本評価で 5 段階評価を 0.0--1.0 に線形変換した際に 1 点が 0.2000 に対応するため,多くの条件で 0.2000 付近に集中しやすく,サンプル数の変更だけでは人手評価相当の水準で明確な改善が得られにくかったと解釈できる.

この結果は,50〜300 の範囲では,サンプル数の増加が対比因子ラベルの意味的類似度に与える影響は限定的であり,特定の group\_size が極端に有利または不利になることはないことを,統計的検定でも有意差が検出されなかったという事実とあわせて,平均値レベルの傾向として示唆しているにとどまる.
group\_size を増やすことでノイズの平均化が期待される一方で,本実験では SBERT類似度 の単調な向上は観測されず,差分は最大でも 0.01 程度に留まった.

アスペクト別に見ると,group\_size の変化よりも,\textit{Gameplay} や \textit{Story} といったアスペクト間の難易度差の方が SBERT類似度・LLM スコアの変動に大きく寄与している.
したがって,本タスクにおける主なボトルネックはサンプル数よりも,データセット特性やアスペクト定義の明瞭さであると解釈できる.
実運用上は,プロンプトのコンテキスト長や計算コストを考慮しつつ,本実験で観測された傾向の範囲では 100〜200 程度の group\_size を用いることが,性能とコストのバランスをとるうえで一つの妥当な目安になると考えられる.

\section{モデル比較実験}
\subsection{実験結果}
GPT-4o-mini vs GPT-5.1 の比較として,Steam データセットを用いて,2 モデルによる性能差を検証した.
実験パラメータは,temperature = 0.0,max\_tokens = 100,few\_shot = 0,group\_size = 100,LLM 評価 = 有効(gpt-4o,temperature = 0.0)とした.
総実験数は 8 実験(4 アスペクト × 2 モデル)であり,全実験が成功した.

モデル別の性能差を表~\ref{tab:model_comparison_summary}に,アスペクト別の性能差を表~\ref{tab:model_comparison_aspect}に示す.
対応のある Wilcoxon 検定では p=0.8750(中央値差 -0.0158,Holm 補正後も非有意)となり,統計的には差は確認されなかった.
\begin{table}[htbp]
  \centering
  \caption{モデル比較実験:平均スコア}
  \label{tab:model_comparison_summary}
  \begin{tabular}{lccc}
    \toprule
    モデル & SBERT類似度 & BLEU & LLM \\
    \midrule
    GPT-4o-mini & & & \\
    \quad 平均 & 0.5453 & 0.0 & 0.3000 \\
    \quad 最小 & 0.5214 & 0.0 & 0.2 \\
    \quad 最大 & 0.5600 & 0.0 & 0.4 \\
    \midrule
    GPT-5.1 & & & \\
    \quad 平均 & 0.5375 & 0.0 & 0.2500 \\
    \quad 最小 & 0.5167 & 0.0 & 0.2 \\
    \quad 最大 & 0.5621 & 0.0 & 0.4 \\
    \bottomrule
  \end{tabular}
\end{table}

表~\ref{tab:model_comparison_summary}より,平均 SBERT類似度は GPT-4o-mini が 0.5453,GPT-5.1 が 0.5375 と近い値であり,LLM スコアも 0.3000 と 0.2500 で大差はない.

SBERT類似度に対する対応のあるWilcoxon検定ではp=0.8750で、GPT-4o-miniとGPT-5.1の差は有意ではなかった(中央値差 5.1−4o-mini = -0.0158)。

\begin{table}[htbp]
  \centering
  \caption{モデル比較実験:アスペクト別SBERT類似度}
  \label{tab:model_comparison_aspect}
  \begin{tabular}{lcc}
    \toprule
    アスペクト & GPT-4o-mini & GPT-5.1 \\
    \midrule
    Gameplay & 0.5600 & 0.5423 \\
    Visual & 0.5425 & 0.5287 \\
    Story & 0.5573 & 0.5167 \\
    Audio & 0.5214 & 0.5621 \\
    \bottomrule
  \end{tabular}
\end{table}

表~\ref{tab:model_comparison_aspect}より,\textit{Gameplay},\textit{Visual},\textit{Story} では GPT-4o-mini が上回る一方,\textit{Audio} は GPT-5.1 が上回っており,優位性はアスペクトに依存している.

\subsection{考察}
モデル比較実験では,Steam データセットに対して gpt-4o-mini と gpt-5.1 を比較した.
SBERT類似度 の平均は gpt-4o-mini が 0.5453,gpt-5.1 が 0.5375,LLM スコアはそれぞれ 0.3000 と 0.2500 であり,新しいモデルである gpt-5.1 が一貫して優位とはならなかった.対応のある Wilcoxon 検定の結果($n=4, p=0.8750$),これらの差はいずれも有意水準 $5\%$ では統計的に有意とはいえない.
特に \textit{Gameplay}, \textit{Story}, \textit{Visual} のアスペクトでは gpt-4o-mini が SBERT類似度・LLM スコアともに高く,\textit{Audio} のみ gpt-5.1 が SBERT類似度 で上回った.

この結果は,0-shot・temperature=0 という決定論的条件では,モデルの世代よりもタスクとの整合性や事前学習分布が性能を左右しうる一方で,本実験の小規模データにおいては統計的に有意な優劣が確認できないことを示している.
すなわち,本実験の条件とサンプル数の範囲に限れば,「より新しいモデルを使えば常に良い対比因子が得られる」という前提を支持する明確な証拠は得られておらず,平均値レベルの傾向と統計的検定結果の両方を踏まえたうえで,対象ドメインとアスペクト特性に対する実測に基づくモデル選択が必要である.
同時に,ここでの知見はあくまでこの決定論的プロンプト条件におけるものであり,温度や Few-shot 例の設計を変えた場合にも同様の傾向が維持されるとは限らない点に注意が必要である.

\section{アスペクト説明文の効果検証}
\subsection{実験結果}
実験設定として,Steam データセットを用いて,アスペクト説明文の有無による性能差を検証した.
実験パラメータは,temperature = 0.0,max\_tokens = 2000,few\_shot = 0,group\_size = 100,GPT モデル = gpt-4o,LLM 評価 = 有効(gpt-4o-mini,temperature = 0.0)とした.
総実験数は 8 実験(4 アスペクト × 2 条件(説明文あり/なし))であり,全実験が成功した.

説明文の有無による性能差を表~\ref{tab:aspect_desc_summary}に,アスペクト別の効果を表~\ref{tab:aspect_desc_aspect}に示す.
対応のある Wilcoxon 検定では p=0.3750(Holm 補正後も非有意)となり,性能差は傾向レベルであった.
\begin{table}[htbp]
  \centering
  \caption{アスペクト説明文の効果検証:平均スコア}
  \label{tab:aspect_desc_summary}
  \begin{tabular}{lccc}
    \toprule
    条件 & SBERT類似度 & BLEU & LLM \\
    \midrule
    説明文なし & & & \\
    \quad 平均 & 0.5395 & 0.0 & 0.2500 \\
    \quad 最小 & 0.5311 & 0.0 & 0.2 \\
    \quad 最大 & 0.5544 & 0.0 & 0.4 \\
    \midrule
    説明文あり & & & \\
    \quad 平均 & 0.5496 & 0.0 & 0.3000 \\
    \quad 最小 & 0.5186 & 0.0 & 0.2 \\
    \quad 最大 & 0.5810 & 0.0 & 0.4 \\
    \bottomrule
  \end{tabular}
\end{table}

表~\ref{tab:aspect_desc_summary}より,説明文あり条件は説明文なし条件に比べて,SBERT類似度が 0.5395 から 0.5496 に増加し,LLM スコアも 0.2500 から 0.3000 に増加した.一方で BLEU は両条件で 0.0 であった.

SBERT類似度で対応のあるWilcoxon検定を行い,$p=0.3750$で説明文あり/なしの差は有意ではなかった(中央値差 $with\_desc - no\_desc = 0.0131$)。

\begin{table}[htbp]
  \centering
  \caption{アスペクト説明文の効果検証:アスペクト別SBERT類似度}
  \label{tab:aspect_desc_aspect}
  \begin{tabular}{lcc}
    \toprule
    アスペクト & 説明文なし & 説明文あり \\
    \midrule
    Gameplay & 0.5335 & 0.5523 \\
    Visual & 0.5311 & 0.5186 \\
    Story & 0.5392 & 0.5467 \\
    Audio & 0.5544 & 0.5810 \\
    \bottomrule
  \end{tabular}
\end{table}

表~\ref{tab:aspect_desc_aspect}より,説明文の効果はアスペクトで異なり,\textit{Audio} は 0.5544 から 0.5810,\textit{Gameplay} は 0.5335 から 0.5523 へと増加する一方,\textit{Visual} は 0.5311 から 0.5186 に低下した.

\subsection{考察}
アスペクト説明文比較実験では,同じ Steam データセットに対し,アスペクトの説明文をプロンプト先頭に付加する条件としない条件を比較した.
全体平均では,説明文なしで SBERT類似度 0.5395,LLM スコア 0.2500,説明文ありで SBERT類似度 0.5496,LLM スコア 0.3000 と,平均値レベルでは改善傾向が見られた.対応のある Wilcoxon 検定の結果($n=4, p=0.3750$),これらの差も統計的には有意ではなく,少数サンプルで観測された傾向として解釈するのが妥当である.
特に \textit{Audio} と \textit{Gameplay} では SBERT類似度 の上昇幅が大きく,LLM スコアも 0.2000 から 0.4000 へと改善しているが,これらも平均値レベルの改善として位置付けられる.
これは,アスペクトの意味領域を明示することで,LLM の注意がレビュー全般のスタイルではなく,アスペクト固有の差分(音響面,ゲーム性)に向かいやすくなった結果と解釈できる.

一方,\textit{Visual} では説明文あり条件で SBERT類似度 が 0.5311 から 0.5186 に低下しており,汎用的すぎる説明文が「見た目」以外の要素(レビューの詳細さや長さなど)への注意を誘導した可能性がある.
このように,説明文は一般に安定化に寄与する一方で,粒度が広すぎたり抽象的すぎたりすると LLM の注意がアスペクト固有の視覚的特徴からレビュー全体のスタイルなどへと拡散し,かえってアスペクト固有性を弱めるリスクもある.

評価指標間の関係に関しては,BLEU が全条件で 0.0 付近に張り付いているのに対し,SBERT類似度 と LLM スコアはアスペクトや設定の違いを敏感に反映している.
とりわけ,SBERT類似度 が近い場合でも LLM スコアに差が出るケースがあり,SBERT類似度 が「意味空間における距離」を測るのに対し,LLM スコアは「対比因子ラベルとしての焦点の合致」や「説明の妥当性」をより厳しく評価しているといえる.
この構造から,対比因子ラベリングの評価には,SBERT類似度 を主要指標としつつ,LLM 評価を補助指標として併用する多面的な評価設計が必要であることが確認された.

\section{COCO Retrieved Concepts実験}
\label{sec:coco_experiment}
\subsection{実験結果}
実験設定として,COCO Retrieved Concepts データセットを用いて,正解ラベルがない画像キャプションデータセットに対する対比因子生成を検証した.本データセットは,各画像に対して COCO データセット由来の複数のキャプションが付与されている一方で,画像をどのような基準でクラスタリングすべきかという正解となるクラスタラベルは与えられていない.本データセットは,300個の潜在概念(concept embeddings)が学習されたモデルから生成されており,各概念について,CLIP(ViT-B/32)を用いて計算した画像埋め込みとのコサイン類似度に基づいて,MS-COCO 2017 train splitから画像を取得している.具体的には,各概念に対して最も類似度が高い100枚の画像を Top-100,最も類似度が低い100枚の画像を Bottom-100 として抽出している.このとき,Top-100 と Bottom-100 の分割基準は潜在概念と CLIP 類似度により定まるが,それがどのような視覚的側面に対応しているかは外部からは明示されない,という状況を想定している.

この分割結果に対して,Top-100 をグループA,Bottom-100 をグループBとして抽出した.各グループについて,画像ごとに付与された COCO キャプション群を用いて対比因子生成を行うことで,各概念がどのような視覚的側面によって Top-100 と Bottom-100 に分割されているのかを,自然言語の対比因子として言語化できるかを検証することを狙いとした.

実験パラメータは,temperature = 0.0,max\_tokens = 2000,few\_shot = 0,group\_size = 100,GPT モデル = gpt-4o-mini,LLM 評価 = 無効とした.総実験数は 5 実験(5 コンセプト)であり,全実験が成功した.正解ラベルがないため,SBERT類似度 と BLEU スコアは参考値として記録するにとどめ,評価指標としては用いなかった.代わりに,生成された対比因子と,各コンセプトに対応する画像群とを見比べることで,対比因子が画像の視覚的特徴を適切に記述しているか,すなわち各概念に対する類似度ランキングの分割基準を妥当な形で言語化できているかを確認した.

視覚的概念記述の生成能力の検証結果を表~\ref{tab:coco_results}に示す.
各コンセプトについて,生成された対比因子と画像との整合性を確認するため,代表的な画像を図~\ref{fig:coco_concept0}--\ref{fig:coco_concept50}に示す.

\begin{table}[htbp]
  \centering
  \caption{COCO Retrieved Concepts実験:評価スコア}
  \label{tab:coco_results}
  \begin{tabular}{lcc}
    \toprule
    評価指標 & 平均 & 範囲 \\
    \midrule
    SBERT類似度 & 0.6173 & 0.5714--0.6537 \\
    BLEU & 0.0000 & 0.0000--0.0000 \\
    LLM & --- & (無効) \\
    \bottomrule
  \end{tabular}
\end{table}

表~\ref{tab:coco_results}に示す SBERT類似度と BLEU は,正解ラベルが存在しないため参考値であるが,少なくとも 5 コンセプトの範囲では SBERT類似度が 0.57--0.65 の範囲に収まり,BLEU は 0.0 であった.

\begin{figure}[htbp]
  \centering
  \begin{subfigure}[b]{0.45\textwidth}
    \centering
    \includegraphics[width=\textwidth,keepaspectratio,bb=0 0 500 332]{image/coco/concept_0_group_a.jpg}
    \caption{Group A (Top-100)}
  \end{subfigure}
  \hfill
  \begin{subfigure}[b]{0.45\textwidth}
    \centering
    \includegraphics[width=\textwidth,keepaspectratio,bb=0 0 640 415]{image/coco/concept_0_group_b.jpg}
    \caption{Group B (Bottom-100)}
  \end{subfigure}
  \caption{concept\_0の代表画像例(生成対比因子:「Group A features everyday scenes and objects, while Group B focuses on events and people in formal settings.」)}
  \label{fig:coco_concept0}
\end{figure}

\begin{figure}[htbp]
  \centering
  \begin{subfigure}[b]{0.45\textwidth}
    \centering
    \includegraphics[width=\textwidth,keepaspectratio,bb=0 0 640 427]{image/coco/concept_1_group_a.jpg}
    \caption{Group A (Top-100)}
  \end{subfigure}
  \hfill
  \begin{subfigure}[b]{0.45\textwidth}
    \centering
    \includegraphics[width=\textwidth,keepaspectratio,bb=0 0 640 425]{image/coco/concept_1_group_b.jpg}
    \caption{Group B (Bottom-100)}
  \end{subfigure}
  \caption{concept\_1の代表画像例}
  \label{fig:coco_concept1}
\end{figure}

\begin{figure}[htbp]
  \centering
  \begin{subfigure}[b]{0.45\textwidth}
    \centering
    \includegraphics[width=\textwidth,keepaspectratio,bb=0 0 500 332]{image/coco/concept_2_group_a.jpg}
    \caption{Group A (Top-100)}
  \end{subfigure}
  \hfill
  \begin{subfigure}[b]{0.45\textwidth}
    \centering
    \includegraphics[width=\textwidth,keepaspectratio,bb=0 0 612 612]{image/coco/concept_2_group_b.jpg}
    \caption{Group B (Bottom-100)}
  \end{subfigure}
  \caption{concept\_2の代表画像例}
  \label{fig:coco_concept2}
\end{figure}

\begin{figure}[htbp]
  \centering
  \begin{subfigure}[b]{0.45\textwidth}
    \centering
    \includegraphics[width=\textwidth,keepaspectratio,bb=0 0 640 427]{image/coco/concept_10_group_a.jpg}
    \caption{Group A (Top-100)}
  \end{subfigure}
  \hfill
  \begin{subfigure}[b]{0.45\textwidth}
    \centering
    \includegraphics[width=\textwidth,keepaspectratio,bb=0 0 480 480]{image/coco/concept_10_group_b.jpg}
    \caption{Group B (Bottom-100)}
  \end{subfigure}
  \caption{concept\_10の代表画像例}
  \label{fig:coco_concept10}
\end{figure}

\begin{figure}[htbp]
  \centering
  \begin{subfigure}[b]{0.45\textwidth}
    \centering
    \includegraphics[width=\textwidth,keepaspectratio,bb=0 0 565 640]{image/coco/concept_50_group_a.jpg}
    \caption{Group A (Top-100)}
  \end{subfigure}
  \hfill
  \begin{subfigure}[b]{0.45\textwidth}
    \centering
    \includegraphics[width=\textwidth,keepaspectratio,bb=0 0 640 456]{image/coco/concept_50_group_b.jpg}
    \caption{Group B (Bottom-100)}
  \end{subfigure}
  \caption{concept\_50の代表画像例(生成対比因子:「Group A focuses on electronics and mobile devices.」)}
  \label{fig:coco_concept50}
\end{figure}

\subsection{考察}
COCO Retrieved Concepts 実験では,非教師ありに学習された潜在コンセプトと CLIP 類似度に基づいて構成された Top-100/Bottom-100 画像群に付与されたキャプションを用い,5 つの概念に対して対比因子ラベルを生成した.
正解ラベルが存在しないため,テキストベースの SBERT類似度(平均 0.6173)と,実画像との整合性に基づく定性的評価を組み合わせて妥当性を検証した.
ここでの SBERT類似度 はあくまでキャプション間の意味類似度を測る参考指標であり,生成ラベルと画像との対応そのものを評価するものではない点に留意する必要がある.

concept\_0,concept\_10,concept\_50 では,「日常的な場面と物 vs フォーマルなイベントと人々」「子ども・家族・動物を含む活動」「電子機器・モバイル端末」といった対比因子ラベルが生成され,少なくともこれらのコンセプトに関しては,対応する画像群の視覚的特徴と整合的と解釈できる例が観察された.
これらのケースでは,LLM はキャプション差分から「イベント性」「家庭的活動」「電子機器」といった高レベル概念を抽出し,それが実際の画像に現れる被写体カテゴリや構図とおおむね一致していると評価された.
この結果は,キャプション集合の差分から視覚的概念を言語化し得る可能性を一部ケースで示唆するものであるが,正解ラベルが存在しないという制約の下での定性的評価に基づくものであり,より体系的な視覚整合性評価は今後の課題として残る.

一方,concept\_1 や concept\_2 では,「スポーツと屋外活動」「動物と自然風景」といった対比因子ラベルが生成されたものの,Top 側の画像群には時計や信号機などスポーツとは無関係な画像が多数含まれており,対比因子ラベルは集合の一部の特徴を過度に代表させたものになっていた.
これは,LLM がキャプション中で頻出する語(例:\textit{sports}, \textit{outdoor})に引きずられ,本質的な視覚パターン(時計,時間)を取り逃がすバイアスを持つことを示している.

これらの観察から,テキスト集合差分のみに基づく対比因子ラベル生成は,視覚的概念に対して一定の成功例を持つ一方で,キャプション側の偏りやノイズに敏感であり,画像の本質的特徴を必ずしも忠実に反映しない場合があることが分かる.
視覚ドメインへ応用する際には,画像から抽出した物体ラベルやシーン属性などの情報も併用し,テキストと視覚特徴の両方に整合する形で対比因子ラベルを検証する必要があることが示唆される.

\section{概念の具体性による性能比較}
\label{sec:concreteness}
\subsection{実験結果}
LLM による対比因子ラベル生成の性能は,対象となる概念の具体性(Concrete vs.\ Abstract)によって差異が観測された.
データセット別・概念タイプ別の性能比較を表~\ref{tab:concreteness_comparison}に,主要アスペクト別の詳細を表~\ref{tab:concreteness_aspects}に示す.

\begin{table}[htbp]
  \centering
  \caption{概念の具体性による性能比較:データセット別SBERT類似度}
  \label{tab:concreteness_comparison}
  \begin{tabular}{lcc}
    \toprule
    データセット & 概念タイプ & SBERT類似度平均 \\
    \midrule
    SemEval-2014 & 具体的アスペクト & 0.7531 \\
    GoEmotions & 抽象的概念 & 0.7127 \\
    Steam & ドメイン固有で抽象度の高いアスペクト & 0.5403 \\
    \bottomrule
  \end{tabular}
\end{table}

表~\ref{tab:concreteness_comparison}より,SBERT類似度の平均は SemEval-2014(0.7531)と GoEmotions(0.7127)で高く,Steam(0.5403)で低い.すなわち,概念の具体性や語彙的一貫性が高い設定ほど,高い類似度が得られる傾向が見られる.

\begin{table}[htbp]
  \centering
  \caption{概念の具体性による性能比較:主要アスペクト別SBERT類似度}
  \label{tab:concreteness_aspects}
  \begin{tabular}{lcc}
    \toprule
    データセット & アスペクト & SBERT類似度 \\
    \midrule
    \multirow{4}{*}{SemEval-2014(具体的)} & Food & 0.7286 \\
    & Service & 0.7181 \\
    & Battery & 0.7646 \\
    & Screen & 0.8012 \\
    \midrule
    \multirow{4}{*}{Steam(ドメイン固有・抽象度高)} & Gameplay & 0.5612 \\
    & Visual & 0.5164 \\
    & Story & 0.5383 \\
    & Audio & 0.5452 \\
    \midrule
    \multirow{4}{*}{GoEmotions(抽象的)} & Joy & 0.8192 \\
    & Disgust & 0.8316 \\
    & Embarrassment & 0.8941 \\
    & Neutral & 0.5437 \\
    \bottomrule
  \end{tabular}
\end{table}

表~\ref{tab:concreteness_aspects}より,SemEval-2014 では \textit{Screen} が 0.8012 と高い一方,Steam では \textit{Visual} が 0.5164 と低い.また GoEmotions では \textit{Embarrassment} が 0.8941 と高いが,\textit{Neutral} は 0.5437 と低く,カテゴリ間のばらつきが確認できる.

SemEval-2014 の具体的なアスペクト(SBERT類似度 平均 0.7531)は,Steam のアスペクト(SBERT類似度 平均 0.5403)と比較して高い値を記録した.
GoEmotions の感情カテゴリ(SBERT類似度 平均 0.7127)は,Steam のアスペクトと比較して高い値を記録したが,SemEval-2014 の具体的なアスペクトと比較して低い値を記録した.
GoEmotions の一部の感情カテゴリ(Disgust 0.8316,Embarrassment 0.8941,Joy 0.8192)では,高い SBERT類似度 を記録した一方,neutral(0.5437)では低い値を記録した.

\subsection{考察}
データセット別の性能比較から,概念の具体性とデータセット設計が対比因子ラベルの性能に強く影響することが分かる.
SemEval-2014 では SBERT類似度 平均 0.7531,GoEmotions では 0.7127,Steam では 0.5403 と,明確な性能差が観測された.

SemEval-2014 で扱う \textit{Food}, \textit{Service}, \textit{Battery}, \textit{Screen} などのアスペクトは,具体的な物理属性や機能に結びついた概念であり,レビュー中での語彙が比較的一貫している.
このため,グループAとグループBの差分には「食べ物の品質」「価格」「バッテリーの持続時間」「画面の解像度」といった安定した統計的パターンが現れやすく,LLM はそれらを単語レベルまたは短い説明フレーズとして抽出しやすい.
結果として,SemEval では SBERT類似度 が高く,LLM スコアも 0.5500 と比較的高い.

GoEmotions の 28 感情カテゴリは,物理的対象ではなく内在状態を表す抽象的概念であるが,データセット設計上,各感情に対応する典型的トリガー表現や感情語彙が豊富に含まれている.
そのため,\textit{disgust}, \textit{embarrassment}, \textit{joy} など一部の感情カテゴリでは SBERT類似度 が 0.8300 以上に達し,LLM スコアも 0.8000 と高い値を記録している.
一方で,\textit{neutral} のように境界が曖昧なカテゴリでは 0.5437 まで低下しており,概念の抽象度だけでなく,アノテーション方針と語彙的一貫性が性能に影響していることが分かる.

Steam Review Aspect Dataset では,\textit{Gameplay}, \textit{Visual}, \textit{Story}, \textit{Audio} といったアスペクトの SBERT類似度 平均が 0.5403 に留まり,LLM スコアも 0.3000 前後と低い.
Steam レビューは長文で雑多な記述が多く,複数アスペクトへの同時言及や皮肉的表現も頻出する.
その結果,グループAとグループBの差分が単一アスペクトにきれいに対応せず,「ゲームプレイ」「ストーリー」「雰囲気」「技術的問題」といった複数の要素が混在した対比因子が生成される傾向がある.
この構造的ノイズが,SemEval や GoEmotions と比較した際の性能低下の主因である.

これらの結果から,対比因子ラベリング性能は,(i) アスペクトの具体性,(ii) レビュー中の語彙的一貫性,(iii) グループA/Bの統計的差分の大きさ,の三要因と一定の関係を持つことが示唆された.
特に,SemEval のように人手ラベルとテキスト分布が整合的なベンチマークでは,集合差分からの自動命名が比較的容易であり,Steam のようにドメイン特有の多義性とノイズが大きい場合には難易度が高い傾向がある.
要するに,対比因子ラベリングの成否はアスペクト概念の明瞭さとレビュー言語の一貫性に大きく左右され,ノイズや多義性の大きいドメインでは人手アスペクトと整合したラベルを得ることが難しい.
次節では,これらのデータセット特性に対して Few-shot ICL による出力スタイルの制御がどの程度補償的に働くかを検証する.

\section{エラー分析と限界}
本実験では,全ての実験カテゴリにおいて実行上の失敗は発生せず,予定した 89 実験を完了した.
一方で,コンテキスト長制限を考慮し,各グループから最大 100 件のテキストを抽出する制約を設けており,大規模なテキスト集合に対する性能は十分に評価できていないという限界がある.
また,具体的なアスペクト(SemEval-2014 の Food,Service,Battery,Screen)では高い SBERT類似度 を記録した一方で,抽象度の高いアスペクト(Steam の Gameplay,Visual,Story,Audio)や一部の感情カテゴリ(GoEmotions の neutral など)ではスコアが低く,概念の抽象性が命名性能に影響することが示唆された.

\section{補足分析}
Few-shot ICL では,少数の例題をプロンプトの \texttt{examples\_section} に挿入し,モデルが出力形式や語彙選択を安定させることを狙った.
Steam データセットを用いた Few-shot 実験では,1-shot 設定が 0-shot および 3-shot と比較して最も高い SBERT類似度 を示し,過度な例示よりも代表的な例を少数与える方が有効である可能性が示された.
さらに,SemEval-2014,GoEmotions,Steam の結果を比較すると,具体的アスペクトを扱う SemEval-2014 で最も高いスコアが得られ,抽象度が高い Steam のアスペクトでスコアが低下するなど,概念の具体性が対比因子ラベル生成の難易度を規定する一因であることがうかがえた.

\section{統計的分析}
\label{sec:statistical_analysis}
Few-shot(0/1/3-shot)とグループサイズ比較について Holm 補正済み Friedman 検定を実施し,それぞれ p=0.4724,p=0.3309 でいずれも有意差は確認されなかった.
モデル比較(n=4)とアスペクト説明文有無(n=4)については,対応のある Wilcoxon 検定を実施し,p=0.8750(中央値差 -0.0158),p=0.3750 といずれも非有意であった(Holm 補正後も非有意).
詳細な統計量と検定結果は \texttt{stat\_tests.md} を参照されたい.

外れ値の分析を表~\ref{tab:outliers}に示す.
\begin{table}[htbp]
  \centering
  \caption{メイン実験における外れ値分析}
  \label{tab:outliers}
  \begin{tabular}{lcc}
    \toprule
    指標 & 最小値 & 最大値 \\
    \midrule
    SBERT類似度 & 0.5164(Steam Visual) & 0.8941(GoEmotions Embarrassment) \\
    BLEU & 0.0000 & 0.0408(GoEmotions Embarrassment) \\
    LLM & 0.2000 & 0.8000(GoEmotions Joy, Disgust, Embarrassment) \\
    \bottomrule
  \end{tabular}
\end{table}

表~\ref{tab:outliers}より,SBERT類似度と LLM スコアはいずれも条件によるばらつきが大きく,最良ケース(GoEmotions \textit{Embarrassment})と最悪ケース(Steam \textit{Visual})で大きな差がある.また BLEU は最大でも 0.0408 に留まり,多くの条件で 0.0 となる.

GoEmotions の感情カテゴリでは,neutral アスペクトで SBERT類似度 0.5437 と低い値を記録したが,これは他の感情カテゴリと比較して低い値であった.

実験間の一貫性の確認として,全実験カテゴリにおいて実験間の一貫性が確認された.

\chapter{まとめ}
\section{結論}

本研究は、大規模言語モデル(LLM)の強力な文脈理解能力と生成能力を活用し、ニューロン発火条件に対応するテキスト集合間の意味的差分を自然言語で記述する「対比因子命名」という新規タスクの実現可能性を検証した~[1--3]。従来の XAI 手法が抱えていた構造的なボトルネックを解消するため、LLM(GPT-4o-mini)をコントラスト生成器として利用するアプローチを提案し、その有効性を定量的に実証した~[1, 4, 5]。

その有効性を示すために、本研究では性質の異なる 4 つの多様なデータセット――SemEval-2014(レストランレビュー)、Steam レビュー、GoEmotions(感情分類)、Retrieved Concepts(視覚的概念記述)――を対象に検証を行った~[6--10]。これにより、本手法が製品レビュー、感情データ、視覚概念テキストといった、内容と粒度の異なるドメインに対しても一貫した枠組みとして適用可能であることを示した。

主要な定量評価では、生成ラベルと人手ラベルとの意味的類似性を BERTScore を用いて測定した結果、平均約 0.551 という中適度な意味的関連性を達成した~[1, 11--14]。この結果は、LLM がニューロンの発火パターンという集合的な差分から、その意味的な核を抽出できるという主要な仮説~[4] が、一定の条件の下で妥当であることを定量的に裏付けるものである~[1, 14]。

また、生成ラベルの品質は、「Food」「Price」のような語彙的に安定した具体的アスペクトにおいて特に高い傾向を示し、LLM が明示的なテキスト証拠に基づく「抽出」タスクに強みを持つことを確認した~[12, 13, 15, 16]。一方で、語彙的一致度を測る BLEU スコアは極めて低値(約 0.007)であり、本タスクにおいては語彙的重複よりも意味的妥当性を重視する BERTScore のような指標が不可欠であることが明らかになった~[1, 14, 18, 19]。

\section{本研究の貢献}

本研究の第一の貢献は、従来は人手に依存していた「概念の命名」プロセスを、LLM を用いて自動化するスケーラブルな基盤を提示した点である~[3, 20, 21]。LIME や SHAP~[22--24] に代表される局所的な特徴可視化手法や、非教師ありコンセプト抽出(UCBM/CCE)~[25--27] のように概念ベクトルの発見に留まっていた従来手法と異なり、本手法は「概念の発見」と「命名」を一体として扱う XAI の新たなワークフローを提案した~[3, 12, 28]。

第二に、本手法は人手アノテーションが困難あるいは高コストであるドメインにおいても適用可能な汎用性を有することを示した。具体的には、物理的実体を持たない抽象概念である Joy や Anger など 28 種類の感情カテゴリを含む GoEmotions データセット~[10, 29, 30]、および「Gameplay」「Technical」といった専門性の高いアスペクトを含む Steam レビューデータセット~[7, 8, 31] を対象とし、抽象的・専門的な概念に対しても一定水準の対比因子命名が可能であることを示した。

第三に、本研究は、メカニスティック解釈(MI)分野~[28, 32] における Attribution Graphs~[26, 33] などが発見した特徴量に対し、人間が理解できる自然言語ラベルを自動付与する命名モジュールとして機能し得ることを示した点である。これにより、「労働集約的 (labor-intensive)」と指摘されてきた手動ラベリングのボトルネック~[3, 32, 34] を緩和し、ブラックボックスモデルの構造理解をより実務的なレベルへと押し上げる足掛かりを提供した。

\section{今後の課題}

一方で、本研究にはいくつかの限界も存在する。まず、BERTScore における平均スコア 0.551 は中程度の意味的類似度を示すものの、高精度な命名が常に達成されているわけではなく、とりわけ抽象的な感情概念や、複数アスペクトが絡み合うケースにおいては、生成ラベルと人手ラベルの解釈がずれる事例も確認された~[10, 30]。また、評価指標として BERTScore と BLEU に依存しており、人間評価との整合性を直接的に検証できていない点も課題として残る~[1, 14, 18, 19]。

これらの課題に対して、本研究の結果は今後の具体的な改善方向性も示唆している。第一に、抽象概念命名の精度向上に向けて、Chain-of-Thought(CoT)プロンプティングを導入し、LLM による逐次的な思考過程の明示化を通じて、集合差分の解釈プロセスを段階的に構造化することが有望である~[35, 36]。第二に、BLEURT や BARTScore など、人間評価との整合性が高いとされる学習ベースの評価指標~[36, 37] を導入し、単一指標に依存しない多面的な評価設計へ移行することが求められる~[35, 37]。

最後に、本研究で提案した対比因子命名は、これまでの非教師あり概念抽出技術が広大な地層から「未知の鉱石(概念ベクトル)」を発見する作業であったとすれば~[38]、その鉱石の「成分と特性の差分」(集合 A vs.~集合 B)を LLM という高性能な分析装置にかけ、「希少な合金に関する言及」のような、人間に理解可能な名前(対比因子ラベル)を瞬時に自動付与する試みであると言える~[38, 39]。この枠組みは、ブラックボックスモデルの解釈可能性をスケーラブルに高めるための重要な一歩であり、今後の XAI 研究および実応用に向けた基盤として発展が期待される。

\chapter*{謝辞}
\addcontentsline{toc}{chapter}{謝辞}
ファルヌシュさんのデータセット提供の謝辞を述べる

% TODO: 謝辞の内容を記述



\chapter{付録}
\label{chap:appendix}

\section{全アスペクトの結果}
\label{sec:appendix_all_aspects}

本章では,紙幅の都合上本文に掲載しなかった全アスペクトの結果を示す.
なお,SemEval-2014 および Steam の「全アスペクト」は,各データセットに定義された全カテゴリではなく,本論文のデータセット別比較で用いた代表アスペクト集合(各4件)を指す.一方で GoEmotions については,本実験で扱った全 28 カテゴリを掲載する.

% データセット別比較(few-shot=0,group\_size=100,gpt-4o-mini)の全アスペクト結果

\begin{table}[htbp]
  \centering
  \caption{データセット別比較:本実験で扱った全アスペクト別評価スコア}
  \label{tab:appendix_all_aspects_dataset_comparison}
  \small
  \begin{tabular}{lcccc}
    \toprule
    データセット & アスペクト & SBERT類似度 & BLEU & LLM \\
    \midrule
    \multirow{4}{*}{SemEval-2014} & Food & 0.7526 & 0.0240 & 0.6000 \\
    & Service & 0.6898 & 0.0123 & 0.4000 \\
    & Battery & 0.7491 & 0.0240 & 0.6000 \\
    & Screen & 0.8008 & 0.0000 & 0.6000 \\
    \midrule
    \multirow{28}{*}{GoEmotions} & Admiration & 0.7297 & 0.0000 & 0.4000 \\
    & Amusement & 0.6826 & 0.0000 & 0.4000 \\
    & Anger & 0.7719 & 0.0000 & 0.6000 \\
    & Annoyance & 0.7455 & 0.0000 & 0.4000 \\
    & Approval & 0.5560 & 0.0000 & 0.2000 \\
    & Caring & 0.7370 & 0.0000 & 0.6000 \\
    & Confusion & 0.5875 & 0.0000 & 0.2000 \\
    & Curiosity & 0.5960 & 0.0000 & 0.2000 \\
    & Desire & 0.7384 & 0.0000 & 0.8000 \\
    & Disappointment & 0.6495 & 0.0000 & 0.2000 \\
    & Disapproval & 0.6275 & 0.0000 & 0.6000 \\
    & Disgust & 0.8782 & 0.0000 & 0.6000 \\
    & Embarrassment & 0.8441 & 0.0330 & 0.6000 \\
    & Excitement & 0.8566 & 0.0000 & 0.8000 \\
    & Fear & 0.7844 & 0.0000 & 0.8000 \\
    & Gratitude & 0.8301 & 0.0211 & 0.8000 \\
    & Grief & 0.7308 & 0.0000 & 0.6000 \\
    & Joy & 0.7253 & 0.0143 & 0.4000 \\
    & Love & 0.6792 & 0.0143 & 0.4000 \\
    & Nervousness & 0.6364 & 0.0000 & 0.4000 \\
    & Neutral & 0.6405 & 0.0000 & 0.4000 \\
    & Optimism & 0.7438 & 0.0170 & 0.6000 \\
    & Pride & 0.5490 & 0.0000 & 0.2000 \\
    & Realization & 0.6669 & 0.0143 & 0.4000 \\
    & Relief & 0.6751 & 0.0000 & 0.4000 \\
    & Remorse & 0.7231 & 0.0000 & 0.4000 \\
    & Sadness & 0.8255 & 0.0240 & 0.6000 \\
    & Surprise & 0.5435 & 0.0000 & 0.2000 \\
    \midrule
    \multirow{4}{*}{Steam} & Gameplay & 0.5424 & 0.0000 & 0.4000 \\
    & Visual & 0.5111 & 0.0000 & 0.2000 \\
    & Story & 0.5176 & 0.0000 & 0.2000 \\
    & Audio & 0.5603 & 0.0000 & 0.2000 \\
    \bottomrule
  \end{tabular}
\end{table}


% 参考文献(References)
\newpage
\addcontentsline{toc}{chapter}{参考文献}
\renewcommand{\bibname}{参考文献}

%% 参考文献に bibtex を使う場合
%\bibliographystyle{junsrt}
%\bibliography{hoge}

%% 参考文献を直接ファイルに含めて書く場合
\begin{thebibliography}{99}

\bibitem{pontiki-EtAl:2014:SemEval2014Task4}
M.~Pontiki, D.~Galanis, J.~Pavlopoulos, H.~Papageorgiou, I.~Androutsopoulos, and S.~Manandhar:
``SemEval-2014 Task 4: Aspect Based Sentiment Analysis,''
in \textit{Proceedings of the 8th International Workshop on Semantic Evaluation (SemEval 2014)}, Dublin, Ireland, August 23--24, 2014, pp.~27--35, doi:10.3115/v1/S14-2004.
% 研究概要: Aspect Based Sentiment Analysis(アスペクトベース感情分析)の標準ベンチマークタスクを提案。レビュー文からアスペクト(観点)とその極性(ポジティブ/ネガティブ/ニュートラル)を抽出するタスクを定義し、評価データセットを提供。
% データセット: Restaurant(レストラン)とLaptop(ノートPC)の2ドメインで構成。Restaurantドメインではfood, service, price, atmosphereなどのアスペクト、Laptopドメインではbattery, screen, keyboard, performanceなどのアスペクトが定義されている。
% タスク構成: 複数のサブタスク(アスペクト抽出、極性分類、アスペクトカテゴリ分類など)を含む包括的な評価フレームワークを提供。
% 本研究での使用: 対比因子生成実験の主要データセットとして使用。Restaurantドメインからfood/service、Laptopドメインからbattery/screenの4アスペクトを用い、アスペクトを含むテキスト群と含まないテキスト群の対比分析に活用。
% 意義: ABSA研究の標準ベンチマークとして広く用いられ,本研究における対比因子生成手法の評価基盤を提供。

\bibitem{ribeiro2016should}
M.~T. Ribeiro, S.~Singh, and C.~Guestrin:
``Why should I trust you?: Explaining the predictions of any classifier,''
in \textit{Proceedings of the 22nd ACM SIGKDD International Conference on Knowledge Discovery and Data Mining}, San Francisco, CA, USA, August 13--17, 2016, pp.~1135--1144, doi:10.1145/2939672.2939778.
% 研究概要: ブラックボックス機械学習モデルの予測を説明するための手法LIME(Local Interpretable Model-agnostic Explanations)を提案。任意の分類器に対して、個々の予測に対する局所的な説明を生成する。
% 手法: モデル非依存(model-agnostic)のアプローチで、対象インスタンスの近傍で擬似データを生成し、距離カーネルで重み付けしたLASSO回帰(線形モデル)を学習して各特徴量の重要度を算出。テキスト、画像、表形式データなど複数のモダリティに対応。
% 主な貢献: 説明可能性の評価指標としてlocal fidelity(忠実性:局所近傍で説明モデルが元のモデルをどれだけ再現できているか)とinterpretability(可読性)を導入し、人間実験により説明の有用性(予測精度向上や欠陥発見)を実証。SP-LIME(submodular pick)により、サブモジュラー最適化を用いて冗長性の少ない多様な代表的説明セットを選択する手法も提案。
% 課題: 局所的な説明に限定され、モデル全体の動作原理は説明しない。後続研究(Slack et al. 2020、Alvarez-Melis & Jaakkola 2018)により、敵対的攻撃に対する脆弱性やロバスト性の問題が指摘されている。なお、説明の安定性(stability)は本論文の中心的な評価指標ではなく、主に後続研究で議論される概念である。

\bibitem{lundberg2017unified}
S.~M. Lundberg and S.-I. Lee:
``A unified approach to interpreting model predictions,''
in \textit{Advances in Neural Information Processing Systems}, vol.~30, Long Beach, CA, USA, December 4--9, 2017, pp.~4765--4774, arXiv:1705.07874, doi:10.5555/3295222.3295230.
% 研究概要: 協力ゲーム理論のShapley値に基づき、機械学習モデルの予測に対する各特徴量の寄与度を説明するSHAP (SHapley Additive exPlanations) を提案。
% 手法: 既存の複数の説明手法(LIME、DeepLIFT、Layer-Wise Relevance Propagation、Shapley regression valuesなど)を統一的に解釈できる理論的フレームワークを提供。
% SHAP値は、特徴量のすべての可能な組み合わせに対する予測への寄与度を平均することで計算され、加法性(各特徴量のSHAP値の合計が予測値と基準値の差に等しい)を満たす。
% 主な貢献: 異なる説明手法間の比較を可能にし、モデル解釈の統一的な評価基準を確立。線形モデル、ツリーモデル、深層学習モデルなど、様々なモデルタイプに適用可能。
% 課題: 計算コストが高く、特に特徴量数が多い場合には近似手法が必要。また、後付け説明(post-hoc explanation)の限界として、モデル自体の解釈可能性を高めるわけではない。

\bibitem{wachter2017counterfactual}
S.~Wachter, B.~Mittelstadt, and C.~Russell:
``Counterfactual explanations without opening the black box: Automated decisions and the GDPR,''
\textit{Harvard Journal of Law \& Technology}, vol.~31, no.~2, pp.~841--887, 2017.
% 研究概要: GDPRの文脈で、ブラックボックスを開かずに個別の自動意思決定を説明する手段として反事実説明(counterfactual explanations)を提案。
% 提案内容: 望ましい結果を得るために何を最小限変更すべきかを示す無条件反事実説明を定義し、当事者の理解・異議申立て・将来の行動支援という目的に対応づける。
% 本研究での位置づけ: 対比因子(差分)を言語化する説明の理論的起点として参照し、内部特徴の寄与(feature attribution)とは異なる説明観を整理する。
% 注意: 法的要請と説明の定義が中心で、具体的な生成アルゴリズムや実装詳細は限定的。

\bibitem{dhurandhar2018cem}
A.~Dhurandhar, P.~Chen, R.~Luss, C.~Tu, P.~Shanmugam, K.~Das, Y.~Liu, and P.~Tambe:
``Explanations based on the Missing: Towards Contrastive Explanations with Pertinent Negatives,''
in \textit{Advances in Neural Information Processing Systems}, vol.~31, 2018 (NeurIPS 2018), Montr\'eal, Canada, December 3--8, 2018, arXiv:1802.07623.
% 研究概要: ブラックボックス分類器に対して、入力に含まれるべき最小限の特徴(Pertinent Positives)と含まれてはいけない最小限の特徴(Pertinent Negatives)を同時に求めるCEM(Contrastive Explanation Method)を提案し、対比的な説明を与える枠組みを示す。
% 手法: 元の入力に対してL1/L2正則化付き最適化問題を解き、決定クラスを維持しつつ不要な特徴を削除・必要な特徴の追加を行うことでPP/PNを導出する。MNIST、調達不正データ、脳活動データなどで有効性を検証。
% 本研究での位置づけ: 「何があるか/ないか」に基づく対比的説明の代表的先行研究として参照し、本研究で扱うテキストベースの対比因子ラベル生成との関係と違い(数値特徴前提・連続最適化ベース)を整理する際に用いる。

\bibitem{kim2018interpretability}
B.~Kim, M.~Wattenberg, J.~Gilmer, C.~Cai, J.~Wexler, F.~Viegas, et al.:
``Interpretability beyond feature attribution: Quantitative testing with concept activation vectors (TCAV),''
in \textit{Proceedings of the 35th International Conference on Machine Learning (ICML 2018)}, Stockholm, Sweden, July 10--15, 2018, pp.~2668--2677, arXiv:1711.11279.
% 研究概要: 高レベル概念(ストライプ、色、性別など)の重要度を、入力特徴量への寄与ではなく「概念」単位で定量評価する手法 TCAV(Testing with Concept Activation Vectors)を提案。
% 手法: ユーザーが用意した概念例(正例)と対照例から Concept Activation Vector(CAV)を学習し、内部表現における方向微分(directional derivative)で概念感度を測ることで、クラス予測が概念にどれだけ依存するかを評価する。
% 本研究での位置づけ: 固定語彙の特徴重要度ではなく「概念でテストする」解釈の代表例として参照し、本研究の対比因子ラベル生成(テキスト集合の差分を言語化)との共通点・差異(内部表現アクセス前提か否か)を整理する。
% 注意: 概念例の選び方・対照集合の設計に結果が依存しやすく、また内部表現(bottleneck)をどこに取るかで解釈が変わりうる。

\bibitem{luss2024cell}
R.~Luss, E.~Miehling, and A.~Dhurandhar:
``CELL your Model: Contrastive Explanations for Large Language Models,''
arXiv preprint arXiv:2406.11785, June 2024.
% 研究概要: 生成タスクでは「クラス予測」がない点に着目し、プロンプトを最小限変更したときに出力が(矛盾する/望ましくない等)別応答へ変わることを根拠に、LLMの応答理由を対比的に説明する CELL を提案。
% 手法: ブラックボックス(API呼び出し)前提で、プロンプトのマスク化とインフィリングにより候補の対比プロンプトを探索し、ユーザー定義のスコア関数(距離/好ましさ等)で「対比」が成立する変更を見つける。長文文脈向けにクエリ予算制約付き探索も提示。
% 本研究での位置づけ: 出力説明を「入力の最小変更で別出力になる」という対比で与える系譜として参照し、テキスト集合の差分を言語化する本研究の対比因子生成と、ブラックボックス前提・スコア関数設計という観点で接続する。
% 注意: 生成される対比はスコア関数・インフィラー・探索予算に強く依存し、得られる説明は因果の保証というより局所的な反実仮想の提示に近い。

\bibitem{bucinca2024contrastive}
Z.~Bu{\c{c}}inca, S.~Swaroop, A.~E. Paluch, F.~Doshi-Velez, and K.~Z. Gajos:
``Contrastive Explanations That Anticipate Human Misconceptions Can Improve Human Decision-Making Skills,''
arXiv preprint arXiv:2410.04253, October 2024.
% 研究概要: 人が持ちがちな誤概念(誤った直観)を予測し、AIの判断と人の判断の「差分」を説明する対比的説明(human-centered contrastive explanations)を提案し、意思決定スキルの向上(deskillingの抑制)を実験で検証。
% 提案内容: AIの選択そのものを正当化する一方向(unilateral)説明ではなく、AIの選択と「人が選びそうな選択」の対比を示すことで、なぜ自分の推論がずれたかを学べる設計枠組みを提示。
% 本研究での位置づけ: 対比説明を「人間側の推論(誤概念)」まで含めて設計する立場として参照し、本研究の対比因子ラベル生成が想定する利用者(比較・理解・学習)との接続点を整理する。
% 注意: 誤概念の推定(人が選びそうな選択の予測)モデルやタスク設定に依存し、汎化や運用時の失敗(誤推定による逆効果)のリスク評価が重要。

\bibitem{anthropic2025biology}
Anthropic:
``On the Biology of a Large Language Model,''
Transformer Circuits, 2025. Available at \url{https://transformer-circuits.pub/2025/attribution-graphs/biology.html} (accessed: 2025-12-13).
% 研究概要: Claude 3.5 Haikuを対象に、Circuit Tracing(回路トレース)手法を用いてモデル内部の計算プロセスを解析。
% 手法: 出力から逆方向に、各特徴量(features)がどの順序で使われたかを追跡し、最終出力への影響を数値化。
% 元の巨大モデルを直接可視化するのではなく、あるプロンプトに対する振る舞いを模倣するより単純な「置き換えモデル(replacement model)」を構築し、その上で特徴量間の相互作用をグラフとして可視化することで、影響の大きい計算経路(attribution graph)を抽出。
% 主な発見: 多言語処理における共通概念空間、詩生成における計画性、推論プロセスと自己説明の乖離など。
% 注意: 解析対象は主に「特徴量(features)」とその相互作用であり、個々の物理的ニューロンではなく人間が意味づけしやすい中間表現(superfeatures/human-interpretable features)を扱う。
% 課題: Attribution Graphが描くのは「replacement model上のfeaturesの流れ」であり、必ずしもオリジナルのすべての内部挙動を忠実に再現したものとは限らない。
% また、紹介されている回路や計算経路はモデルが出力を生成する過程のごく一部を捉えたものであり、複雑な長文や多段推論、リアルタイムの文脈変化などをすべてトレースするのは現状では困難。
% グラフのノード(特徴量)の意味付けは自動化されておらず、手動で「スーパーノード」としてグループ化する必要がある。また、attribution graphを因果(causal)な説明と見るか相関(correlational)な説明と見るかには慎重さが求められる。

\bibitem{demszky2020goemotions}
D.~Demszky, D.~Movshovitz-Attias, J.~Ko, A.~Cowen, G.~Nemade, and S.~Ravi:
``GoEmotions: A Dataset of Fine-Grained Emotions,''
in \textit{Proceedings of the 58th Annual Meeting of the Association for Computational Linguistics (ACL 2020)}, pp.~4040--4054, 2020.
% 研究概要: Redditコメント約5.8万件に対し、27感情カテゴリ+Neutralの多ラベル注釈を付与した大規模な英語感情データセット GoEmotions を提案。
% データセット: 会話的テキスト(Reddit)を対象とし、細粒度の感情分類や転移学習評価に利用できる設計。多ラベルである点が特徴(1文に複数感情が付与されうる)。
% 本研究での使用: 感情/主観表現を含むテキスト集合を扱う際の代表的公開データセットとして参照し、テキスト集合差分の言語化(対比因子生成)と親和性のあるタスク背景として位置づける。
% 注意: Reddit由来のドメイン偏りや注釈方針(多ラベル・Neutral)の影響があり、他ドメインへの一般化やラベル解釈には留意が必要。

\bibitem{srec:steam-review-aspect-dataset}
S.~Khosasi:
``Steam review aspect dataset,''
2024. Available at \url{https://srec.ai/blog/steam-review-aspect-dataset} (accessed: 2025-12-13).
% 研究概要: Steamレビューを対象に、レビュー本文に対するアスペクト(観点)情報を扱うためのデータセットを公開・紹介する記事。
% データセット: アスペクト抽出/アスペクト分類など、レビューの観点分析タスクでの利用を想定したリソースとして位置づけられている。
% 本研究での使用: Steamレビューを用いた対比因子生成・評価の背景データ(アスペクトという外部ラベル)として参照。
% 注意: Web記事由来の参照で、配布形式や最新版URLが変更されうるため、再現の際は取得日時やスナップショットの確保が望ましい。

\bibitem{papineni-etal-2002-bleu}
K.~Papineni, S.~Roukos, T.~Ward, and W.-J. Zhu:
``BLEU: a Method for Automatic Evaluation of Machine Translation,''
in \textit{Proceedings of the 40th Annual Meeting of the Association for Computational Linguistics (ACL 2002)}, pp.~311--318, 2002.
% 研究概要: 機械翻訳の自動評価指標 BLEU を提案。参照訳との n-gram 一致率(修正精度)を基礎に、短すぎる出力を抑制する brevity penalty を組み合わせてスコア化する。
% 特徴: 高速・言語非依存・大規模評価に適し、人手評価との相関を狙う指標として普及。主に表層一致(n-gram)に基づくため、同義表現には弱い。
% 本研究での使用: 生成した説明文/ラベル文の自動評価で、意味類似系(BERTScore)と併用する主要指標の一つとして用いる(表層的一致度の補助軸)。
% 注意: 語彙や言い換えに敏感で、単文・短文では不安定になりやすい。単独で品質判断せず、意味指標や定性確認と併用する。

\bibitem{devlin2018bert}
J.~Devlin, M.-W. Chang, K.~Lee, and K.~Toutanova:
``BERT: Pre-training of Deep Bidirectional Transformers for Language Understanding,''
arXiv preprint arXiv:1810.04805, 2018.

\bibitem{bau2017networkdissection}
D.~Bau, B.~Zhou, A.~Khosla, A.~Oliva, and A.~Torralba:
``Network Dissection: Quantifying Interpretability of Deep Visual Representations,''
in \textit{Proceedings of the IEEE Conference on Computer Vision and Pattern Recognition (CVPR 2017)}, Honolulu, HI, USA, July 2017, pp.~6541--6549, arXiv:1704.05796.
% 研究概要: CNN の各ユニットと Broden データセットの概念マスクとの IoU を用いて,「どのユニットがどの概念を検出しているか」を定量化する手法 Network Dissection を提案し,深層視覚表現の解釈可能性を測定。
% 本研究での位置づけ: 画像モデル内部の概念特徴に対する「固定語彙ベースの自動ラベリング」手法として参照し,対比因子生成タスクとの違い(内部アクティベーション前提・画像中心)を説明する際の代表例として用いる。

\bibitem{oikarinen2022clipdissect}
T.~Oikarinen and T.-W. Weng:
``CLIP-Dissect: Automatic Description of Neuron Visual Features with Language Models,''
arXiv preprint arXiv:2204.10965, 2022.
% 研究概要: CLIP の画像・テキスト埋め込み空間を用い,高活性化画像と多数のテキスト候補との類似度を計算することで,任意の視覚モデルのユニットにオープンエンドな概念ラベルを自動付与する CLIP-Dissect を提案。
% 本研究での位置づけ: CLIP を用いた検索ベースの自動概念命名の代表例として参照し,本研究のテキスト集合ベース・外部ラベル評価との対比に用いる。

\bibitem{schrodi2024unsupervised}
S.~Schrodi, M.~R{\"u}ckl, T.~Wirth, M.~B{\"o}hm, and D.~R{\"u}gamer:
``Concept Bottleneck Models Without Predefined Concepts,''
arXiv preprint arXiv:2407.03921, 2024.

\bibitem{oikarinen2023labelfree}
T.~Oikarinen, S.~Tripathi, T.~M. Mitchell, and D.~Alvarez\mbox{-}Melis:
``Label-Free Concept Bottleneck Models,''
in \textit{Proceedings of the 11th International Conference on Learning Representations (ICLR 2023)}, Kigali, Rwanda, May 2023, arXiv:2304.06129.
% 研究概要: GPT-3 や CLIP を用いてタスクに関連する概念候補リスト(概念バンク)を自動生成し,ラベル付き概念を用いずに Concept Bottleneck Model を構築する Label-Free CBM を提案。
% 本研究での位置づけ: 事前定義概念なしに概念ボトルネックを構築する既存手法として引用し,Discover-then-Name との関係や,本研究の「テキスト集合+外部アスペクトラベル」という評価設定との違いを説明する。

\bibitem{rao2024discoverthenname}
S.~Rao, Y.~Zhao, M.~Sachan, and A.~Gupta:
``Discover-then-Name: Task-Agnostic Concept Bottlenecks via Automated Concept Discovery,''
arXiv preprint arXiv:2407.14499, 2024.
% 研究概要: CLIP 特徴に対して Sparse Autoencoder や NMF を適用して概念方向を教師なしで発見し,高活性化画像と CLIP/LLM を用いて概念に名前を付ける Discover-then-Name (DN-CBM) を提案。
% 本研究での位置づけ: 「発見してから名付ける」コンセプトボトルネックの代表例として参照し,本研究の対比因子生成タスクとのタスク設計の違いを議論する。

\bibitem{ameisen2025attribution}
E.~Ameisen, J.~Lindsey, A.~Pearce, W.~Gurnee, N.~L. Turner, B.~Chen, C.~Citro, D.~Abrahams, S.~Carter, B.~Hosmer, J.~Marcus, M.~Sklar, A.~Templeton, T.~Bricken, C.~McDougall, H.~Cunningham, T.~Henighan, A.~Jermyn, A.~Jones, A.~Persic, Z.~Qi, T.~B. Thompson, S.~Zimmerman, K.~Rivoire, T.~Conerly, C.~Olah, and J.~Batson:
``Circuit Tracing: Revealing Computational Graphs in Language Models,''
\textit{Transformer Circuits}, 2025. Available at \url{https://transformer-circuits.pub/2025/attribution-graphs/methods.html} (accessed: 2025-12-13).
% 研究概要: Attribution Graphsの手法論を提案。LLMの内部計算プロセスをトレースし、特徴量間の相互作用をグラフとして可視化することで計算構造(回路)を発見する。
% 手法: ある出力が計算されるまでにネットワーク内部でどのニューロンや重みがどのような順序で使われたかを、出力側からさかのぼって追跡。
% 各ニューロンや結合が最終出力にどの程度影響したかを数値化し、影響の大きい経路どうしを結んだグラフとして可視化することで、計算グラフを抽出。
% 課題: グラフのノードとして発見される「特徴量(features)」は必ずしも自然言語的に意味のある概念(semantic concept)とは限らず、ノードが具体的に何を検出しているかを自然言語で特定するプロセスは自動化されていない。
% また、Attribution Graphはモデルの振る舞いの一部を可視化できるが、すべての振る舞いやグローバルなアルゴリズムを保証するものではない。特に注意(attention)による経路は追えない場合があり、「dark matter」と呼ばれる未説明部分が残る。
% 可視化された回路は多くのノード・エッジを含みうるため、手作業での解釈や簡潔な説明への落とし込みが困難で、feature-splitting/absorptionやsupernodesによる手動グルーピングが必要となる。
% 関連研究: anthropic2025biologyと同一シリーズの研究で、Attribution Graphsの手法を具体的に適用した実証研究。

\bibitem{bordt2022posthoc}
S.~Bordt, M.~Finck, E.~Raidl, and U.~von Luxburg:
``Post-Hoc Explanations Fail to Achieve their Purpose in Adversarial Contexts,''
in \textit{Proceedings of the 2022 ACM Conference on Fairness, Accountability, and Transparency (FAccT '22)}, Seoul, Republic of Korea, June 21-24, 2022, pp.~1495--1515, doi:10.1145/3531146.3533153.
% 研究概要: 機械学習の説明可能性に関する法的義務(説明要求)を念頭に,post-hoc説明アルゴリズムがその目的(透明性確保)を達成できないことを論じる。
% 視点: 法・哲学・技術の観点を統合し,説明の提供者と受け手の利害が対立する敵対的状況では,説明が操作されうる点を中心に議論する。
% 主張: 現実的な適用場面ではpost-hoc説明は多義的で解釈の余地が大きく,この曖昧性ゆえに根本的な対立を解消できず,規制が意図する透明性目的に不適合になりうる。
% 本研究での位置づけ: post-hoc説明の限界(特に敵対的文脈)を整理し,評価設計や説明の目的の明確化が必要であるという問題意識の根拠として参照する(arXiv:2201.10295 の要旨で主張を確認)。

\bibitem{slack2020fooling}
D.~Slack, S.~Hilgard, E.~Jia, S.~Singh, and H.~Lakkaraju:
``Fooling LIME and SHAP: Adversarial Attacks on Post hoc Explanation Methods,''
in \textit{Proceedings of the 2020 AAAI/ACM Conference on AI, Ethics, and Society (AIES 2020)}, New York, NY, USA, February 7--8, 2020, pp.~180--186, arXiv:1911.02508, doi:10.1145/3375627.3375830.
% 研究概要: 入力摂動に基づくpost-hoc説明(LIME/SHAP)が信頼できないことを示し,モデルの偏りを隠したまま無害に見える説明を出させる攻撃を提案する。
% 手法: 元の分類器の予測挙動(入力分布上の偏り)を維持しつつ,説明器が参照する摂動周辺の振る舞いを細工するscaffoldingにより,任意の望ましい説明を誘導できることを示す。
% 実験: COMPASを含む複数の実データで,極端に偏った分類器でもLIME/SHAPが無害な説明を返すようにできることを検証する。
% 本研究での位置づけ: post-hoc説明の敵対的脆弱性(説明が監査や偏り診断に使われる状況での限界)を示す代表例として参照する(arXiv:1911.02508 の要旨で主張を確認)。

\bibitem{alvarez2018robustness}
D.~Alvarez\mbox{-}Melis and T.~S. Jaakkola:
``On the Robustness of Interpretability Methods,''
in \textit{Proceedings of the 32nd Conference on Neural Information Processing Systems (NeurIPS 2018) Workshops}, 2018, arXiv:1806.08049.
% 研究概要: 解釈手法における重要要件として「説明のロバスト性」(入力が類似なら説明も類似すべき)を主張し,その定量指標を提案する。
% 手法: 説明のロバスト性を測るメトリクスを導入し,既存の説明手法がそれらの指標に照らして十分にロバストではないことを示す。
% 提案: 既存の解釈手法にロバスト性制約を組み込むことで改善する方向性(ロバスト性の強制)を提示する。
% 本研究での位置づけ: 説明(post-hoc含む)の品質要件としてロバスト性を明示し,本研究の評価観点・限界整理(小さな入力差で説明が大きく変わる問題)に接続する(arXiv:1806.08049 の要旨で主張を確認)。

\bibitem{mersha2024survey}
M.~Mersha, K.~Lam, J.~Wood, A.~AlShami, and J.~Kalita:
``Explainable AI: A Survey of Needs, Techniques, Applications, and Future Direction,''
arXiv preprint arXiv:2409.00265, 2024.
% 研究概要: 医療・金融・自動運転など安全クリティカル領域でのブラックボックス性の課題に対し,XAIが透明性・説明責任・公平性を支えるという問題設定を整理するサーベイ。
% まとめ内容: 用語と定義,XAIの必要性と受益者,手法のタクソノミー(分類),応用領域での適用例を包括的に整理し,信頼性向上の観点で俯瞰する。
% 本研究での位置づけ: XAI全体の背景整理(ニーズ・分類・応用)として参照し,本研究の対比因子ラベル生成をXAIの枠組みの中で位置づける際の導入・関連付けに用いる(arXiv:2409.00265 の要旨で主張を確認)。

\bibitem{rudin2019stop}
C.~Rudin:
``Stop explaining black box machine learning models for high stakes decisions and use interpretable models instead,''
\textit{Nature Machine Intelligence}, vol.~1, no.~5, pp.~206--215, 2019, doi:10.1038/s42256-019-0048-x.
% 研究概要: ブラックボックスモデルに対するpost-hoc説明(explainable AI)の流れに対し,高リスク意思決定(医療・刑事司法など)では,説明を後付けするよりも最初から解釈可能なモデルを用いるべきだと主張する。
% 主張: 説明生成でブラックボックス利用を正当化する方向ではなく,運用上の安全性・説明責任の要請が強い領域では「解釈可能性を内在化したモデル設計」へ研究と実装の重心を移すべきだと論じる。
% 本研究での位置づけ: 対比因子生成のような説明生成を議論する際に,post-hoc説明の限界と「解釈可能な仕組みの設計」志向(説明の目的・適用範囲の明確化)の背景として参照する(Nature公開descriptionで主旨を確認)。

\bibitem{vilone2020systematic}
G.~Vilone and L.~Longo:
``Explainable Artificial Intelligence: a Systematic Review,''
arXiv preprint arXiv:2006.00093, 2020, doi:10.48550/arXiv.2006.00093.

% 研究概要: 深層学習の普及に伴いXAIが急速に拡大している背景を整理し,既存の多様なXAI研究を体系的に俯瞰するシステマティックレビュー。
% 整理方法: 階層的な分類体系でXAI文献を4クラスタ(サーベイ,理論・概念,手法,評価)に整理し,研究領域の全体像をまとめる。
% 本研究での位置づけ: XAI研究の全体地図(手法と評価の位置づけ)を引用し,本研究の対比因子ラベル生成とその評価設計を関連研究の枠組みに接続する(arXiv:2006.00093 の要旨で主張を確認)。

% 研究概要: ローカル説明(例: 特徴重要度や根拠ハイライト)を多数集め、説明の類似性にもとづいて概念レベルのクラスタ(Concept Explanation Clusters)にまとめることで、個別説明の集合からグローバルな傾向や代表パターンを抽出する枠組みを提案。
% 本研究での位置づけ: ローカル説明を集約してデータ集合レベルの差異や偏りを読むという観点で関連し、対比因子ラベル生成におけるグループ差分の要約と接続できる。

\bibitem{haedecke2025conceptClusters}
E.~Haedecke, M.~Akila, and L.~von Rueden:
``Global Properties from Local Explanations with Concept Explanation Clusters,''
in \textit{World Conference on eXplainable Artificial Intelligence (xAI 2025)}, Springer, Cham, 2025, pp.~3--24, doi:10.1007/978-3-031-41510-0\_1.

% 研究概要: クラスタリング手法が精度・効率を優先するあまりブラックボックス化してきた背景を整理し,高リスク領域での利用に向けて「解釈可能なクラスタリング」の要件・評価観点・研究動向を体系化したサーベイ。
% タクソノミー: クラスタリングの流れを pre-clustering / in-clustering / post-clustering に分け,決定木・ルール・プロトタイプ等の解釈モデルの違いも含めて分類し,適用場面に応じた手法選択の指針を与える。
% 本研究での位置づけ: 本文(関連研究)では,クラスタ(グループ)レベルのパターン理解と個別インスタンスの所属理由の間に説明ギャップがあるという問題意識の根拠として参照し,本研究の対比因子ラベル生成を集合レベル説明の方向として位置づける。
% 実在確認: arXiv:2409.00743 の要旨で主張を確認(下記URL)。
\bibitem{hu2024interpretableClustering}
L.~Hu, M.~Jiang, J.~Dong, X.~Liu, and Z.~He:
``Interpretable Clustering: A Survey,''
arXiv preprint arXiv:2409.00743, 2024.

% 研究概要: 2つ以上の文書群を比較し,差分(相違点)を強調する対比的/比較要約(contrastive/comparative summarization)について,手法・データセット・評価指標・応用を系統的に整理したサーベイ。
% 整理ポイント: 既存研究を方法論の観点で俯瞰し,標準データセットや競争的タスクの不足など,今後の研究基盤整備上の課題も指摘する。
% 本研究での位置づけ: 本文(関連研究)では,対比的要約/グループ差分要約の問題設定(複数文書群の比較と差分ハイライト)を整理する基礎文献として参照し,本研究の集合レベル差分の自然言語記述タスクの背景付けに用いる。
% 実在確認: Springer(International Journal of Data Science and Analytics)に公開され,DOIが付与されている(下記URL)。
\bibitem{strohle2024contrastive}
T.~Str{\"o}hle, R.~Campos, and A.~Jatowt:
``Contrastive text summarization: a survey,''
\textit{International Journal of Data Science and Analytics}, vol.~18, pp.~353--367, 2024, doi:10.1007/s41060-023-00434-4.

% 研究概要: 2対象(A vs B)の比較に対して,LLMで差分を抽出・整理し,意思決定に有用な「属性付き・構造化された」対比要約を生成する手法 STRUM-LLM を提案。
% 手法: 入力ソースから差分となる属性(helpful contrast)を同定し,根拠テキストに紐づく形で整理して提示することで,ユーザ向け比較要約の可用性と検証可能性を高める。
% 本研究での位置づけ: 本文(関連研究)では,A vs B の差分を属性として構造化するLLMパイプラインの代表例として参照し,本研究の対比因子ラベル生成(集合間の意味差分を言語化)との接点と差分を示す。
% 実在確認: arXiv:2403.19710 の要旨で主張を確認(下記URL)。
\bibitem{saha2024strumllm}
A.~Saha, B.~P. Majumder, H.~Jhamtani, S.~Subramanian, S.~Sreedhar, S.~Chakrabarti, and P.~Kankar:
``STRUM-LLM: Attributed and Structured Contrastive Summarization for User-Oriented Comparison,''
arXiv preprint arXiv:2403.19710, 2024, doi:10.48550/arXiv.2403.19710.

% 研究概要: ABSA(アスペクトベース感情分析)に対して GPT-4 / GPT-3.5 の zero-shot / few-shot / fine-tuning を比較評価し,性能とコストのトレードオフを整理する。
% 結果: SemEval-2014 Task 4 の「アスペクト抽出+極性分類(joint)」で fine-tuned GPT-3.5 が高いF1(SOTA相当)を達成する一方,推論コスト(モデル規模・運用負荷)の増大も指摘し,プロンプト工夫と微調整の使い分け指針を提示。
% 本研究での位置づけ: 本文(関連研究)では,LLMを用いたABSAの代表的検討として参照し,「所与のアスペクト/極性ラベル体系への対応づけ(予測)」と,本研究の非教師あり・集合間コントラストの命名(差分記述)とのタスク設計上の違いを明確化するために用いる。
% 実在確認: arXiv:2310.18025 の要旨で主張を確認(下記URL)。
\bibitem{simmering2023llmabsa}
P.~F. Simmering and P.~Huoviala:
``Large language models for aspect-based sentiment analysis,''
arXiv preprint arXiv:2310.18025, 2023, doi:10.48550/arXiv.2310.18025.

% 研究概要: テキストクラスタリングを分類問題として再定式化し,LLMのin-context learningで,埋め込み微調整やK-means等のクラスタリング手順を介さずにクラスタ割当を行う枠組みを提案。
% 論文中の課題意識: 既存のLLM利用クラスタリングは外部埋め込みモデルと類似度計算,クラスタ数などのハイパーパラメータ調整に依存し,計算コストとドメイン適応負担が残る点,およびクラスタの解釈性が弱い点を課題とする。
% 手法: (1) データをミニバッチで逐次投入し候補ラベルを生成,(2) 近義ラベルをLLMで統合して粒度を調整,(3) 得られたラベル集合に対して各テキストをLLMで分類し,ラベル=クラスタとして出力する。
% 実験設定の要点: 5データセット(例: ArxivS2S, GoEmo, Massive-D/I, MTOP-I)でACC/NMI/ARIにより評価。実装ではGPT-3.5-turbo,ミニバッチサイズ15,例示ラベル数は真のラベル数の20%を使用。
% 本研究での位置づけ: 本研究は集合A/Bの差分に対応する対比因子の命名が主目的で,本論文は単一集合のクラスタ割当とクラスタラベル生成が主目的。ただし,ラベル候補生成と同義統合により粒度を調整する設計は,対比因子ラベルの候補生成・正規化の参照点となる。
% 特筆: 粒度分析,プロンプト差分分析,few-shotでのラベル生成,コスト比較を行い,クラスタリングをLLM分類へ寄せることで計算手順を単純化できる点を補強している。
% 実在確認: arXiv:2410.00927(下記URL)に加え,SIGIR-AP 2025 掲載でACM DOI 10.1145/3767695.3769519 が付与されている(arXiv HTML版で確認)。
\bibitem{huang2024textclustering}
C.~Huang and G.~He:
``Text Clustering as Classification with LLMs,''
in \textit{Proceedings of the 2025 Annual International ACM SIGIR Conference on Research and Development in Information Retrieval in the Asia Pacific Region (SIGIR-AP 2025)}, 2025, doi:10.1145/3767695.3769519.

% 研究概要: 生成文と参照文のペアを入力として,人手評価に近いスコアを回帰的に予測する学習型評価指標 BLEURT を提案。BERT表現を用い,合成データを用いた大規模事前学習の後に,人手スコアで微調整することで頑健性を高める。
% 論文中の課題意識: BLEU/ROUGE等の表層一致指標は,人手判断との相関が低い場合があり,ドメイン変化やモデル品質向上に伴う分布変化に弱い点を課題とする。
% 本研究での位置づけ: 本研究では説明文の自動評価としてBERTScore/BLEUを主要指標とする一方,学習型評価指標の代表例として,評価指標の限界と改善方向(学習により人手判断へ近づける)を示す背景文献として参照する。
% 特筆: 少量の(偏り得る)人手例でも学習可能で,合成例による事前学習によりout-of-distribution下でも性能が落ちにくいことを,WMT MetricsやWebNLGで実験的に示す。
\bibitem{sellam-etal-2020-bleurt}
T.~Sellam, D.~Das, and A.~Parikh:
``BLEURT: Learning Robust Metrics for Text Generation,''
in \textit{Proceedings of the 58th Annual Meeting of the Association for Computational Linguistics (ACL 2020)}, pp.~7881--7892, 2020, doi:10.18653/v1/2020.acl-main.704.

\bibitem{yuan2021bartscore}
W.~Yuan, G.~Neubig, and P.~Liu:
``BARTScore: Evaluating Generated Text as Text Generation,''
% 研究概要: 生成文評価を,事前学習済みseq2seqモデルによるテキスト生成確率(平均対数尤度)として捉える枠組みを提案し,BARTに基づく自動評価指標 BARTScore を提示する。参照文→生成文(または逆)の条件付き生成確率をスコアとし,観点(流暢性・情報性・事実性など)に応じた変種を構成できる。
% 論文中の課題意識: BLEU等の表層一致指標や,単純な埋め込み類似度は,評価観点の切替や意味的妥当性の捉え方に限界があり,生成タスク横断で頑健に使える自動評価が課題である。
% 本研究での位置づけ: 本研究は説明文の自動評価にBERTScore/BLEUを主要指標として用いるが,生成確率ベースの代替指標として位置づけ,評価指標選択の幅(回帰型: BLEURT,生成型: BARTScore)を整理する際の参照文献とする。
% 特筆: 多様なタスク・観点で既存指標より良い相関/性能を示すと主張し,ExplainaBoard上でメタ評価の可視化を提供して,指標間の補完関係の分析を促す。
in \textit{Advances in Neural Information Processing Systems (NeurIPS)}, vol.~34, pp.~27263--27277, 2021, doi:10.48550/arXiv.2106.11520.

\bibitem{reiter2018structured}
E.~Reiter:
``A Structured Review of the Validity of BLEU,''
% 研究概要: BLEUと人手評価の相関に関する既存研究を構造化レビューとして整理し,34本の論文から報告された284個の相関を集約して,BLEUの妥当性を検討する。
% 論文中の課題意識: BLEUを評価指標として使うには,人手評価や実運用上の有用性と結びつく検証が必要だが,文生成(NLG)や文単位評価,仮説検証での利用は根拠が弱い点を指摘する。
% 本研究での位置づけ: 本研究は説明文の自動評価にBERTScore/BLEUを用いるが,BLEUを主要指標として扱う際の注意点(用途限定・過信回避)を明確化するための根拠文献として参照する。
\textit{Computational Linguistics}, vol.~44, no.~3, pp.~393--401, 2018, doi:10.1162/COLI\_a\_00322.

\bibitem{holtzman2020curious}
A.~Holtzman, J.~Buys, L.~Du, M.~Forbes, and Y.~Choi:
``The Curious Case of Neural Text Degeneration,''
% 研究概要: 同一モデルでもデコーディング戦略次第で生成文が単調・反復的に劣化する現象(degeneration)を分析し,確率分布の上位確率質量の集合からサンプリングするnucleus sampling(top-p)を提案する。
% 論文中の課題意識: 最尤に基づくデコーディング(例: 貪欲・ビーム)が,人間文と異なる分布的性質を誘発し,反復や凡庸さを生む点を問題化する。
% 本研究での位置づけ: 対比因子ラベル生成や説明文生成ではLLM出力の品質が重要であり,温度・top-p等のデコーディング設定が結果に与える影響を議論する際の基礎文献として参照する。
in \textit{International Conference on Learning Representations (ICLR)}, 2020, doi:10.48550/arXiv.1904.09751.

\bibitem{zhang2019bertscore}
T.~Zhang, V.~Kishore, F.~Wu, K.~Q. Weinberger, and Y.~Artzi:
``BERTScore: Evaluating Text Generation with BERT,''
% 研究概要: 生成文と参照文の各トークン同士の類似度を,BERTの文脈化埋め込みに基づいて算出し,precision/recall/F1として集約する自動評価指標 BERTScore を提案する(表層一致ではなく意味的近さを捉える)。
% 論文中の課題意識: BLEU等のn-gram一致は言い換えに弱く,人手評価との相関やモデル選択に限界がある点を問題として,文脈表現によるロバストな評価を目指す。
% 本研究での位置づけ: 本研究の主要評価指標(BERTスコア)として用い,対比因子ラベル生成で得られた説明文が正解説明とどの程度意味的に整合するかを定量化するために用いる。
% 特筆: 多数のMT/キャプション生成システム出力で人手評価との相関改善を示し,難しい言い換え(adversarial paraphrase)に対する頑健性も報告する。IDF重み付けなどの実装上の選択肢も提示される。
arXiv preprint arXiv:1904.09675, 2019, doi:10.48550/arXiv.1904.09675.

\bibitem{openai2023neurons}
OpenAI:
``Language models can explain neurons in language models,''
OpenAI Blog, 2023. Available at \url{https://openai.com/index/language-models-can-explain-neurons-in-language-models/} (accessed: 2025-12-13).
% 研究概要: GPT-2 の多数のニューロンに対し,トップ発火トークン列を GPT-4 に入力して自然言語説明を生成し,Simulation Score により説明の妥当性を自動評価する「自動解釈可能性」パイプラインを提示。
% 本研究での位置づけ: LLM を用いたニューロン説明と自動スコアリングの代表例として参照し,本研究が扱うテキスト集合レベルのタスクとの違いを説明する。

\bibitem{bills2023automatedinterp}
S.~Bills, N.~Muennighoff, R.~Hoang, N.~Mu, et al.:
``Automatically Interpreting Millions of Features in Large Language Models,''
arXiv preprint arXiv:2310.13052, 2023.
% 研究概要: Sparse Autoencoder によって LLM の中間表現から数百万規模のスパース特徴を抽出し,各特徴に対して LLM による説明文生成と Simulation Score による自動評価を行うことで,大規模な自動概念命名・自動スコアリングを実現。
% 本研究での位置づけ: 「OpenAI/Anthropic 型」の自動解釈パイプラインの代表として参照し,自動命名の達成度と限界に関する議論の背景とする。

\bibitem{stein2024towards}
A.~Stein, A.~Naik, Y.~Wu, M.~Naik, and E.~Wong:
``Towards Compositionality in Concept Learning,''
in \textit{Proceedings of the International Conference on Machine Learning (ICML)}, 2024, doi:10.48550/arXiv.2406.18534.
% 研究概要: 概念ベース解釈(埋め込みを高水準概念へ分解)において,個々の概念が合成的に全体サンプルを説明できること(compositionality)が重要だと位置づけ,既存の教師なし概念抽出が非合成的な概念に陥りやすい点を示す。
% 論文中の課題意識: 概念抽出が得る概念が「部分の和として全体を説明する」性質を満たさず,解釈可能性や下流タスクでの有用性が限定される点を課題とする。
% 手法: 合成的概念表現が満たすべき性質を整理し,それらに従う概念を自動発見する Compositional Concept Extraction (CCE) を提案する。
% 本研究での位置づけ: 本研究は集合間差分を言語化して対比因子ラベルを生成するが,「概念(説明単位)がどの程度合成的に全体を説明できるか」という観点は,生成ラベルの解釈可能性・再利用可能性を議論する際の背景として参照できる。
% 特筆: 画像・テキストを含む複数データセットで評価し,ベースラインより合成性が高い概念を得られ,複数の下流分類で精度改善も報告する。

\bibitem{lin2014microsoft}
T.-Y.~Lin, M.~Maire, S.~Belongie, J.~Hays, P.~Perona, D.~Ramanan, P.~Doll{\'a}r, and C.~L. Zitnick:
``Microsoft COCO: Common Objects in Context,''
in \textit{Proceedings of the 13th European Conference on Computer Vision (ECCV 2014)}, Zurich, Switzerland, September 6--12, 2014, pp.~740--755, arXiv:1405.0312, doi:10.1007/978-3-319-10602-1\_48.
% 研究概要: 日常シーン画像に対して,物体カテゴリとインスタンス単位のセグメンテーション等を大規模に付与したデータセット COCO を提案し,物体認識を「シーン理解の文脈」で評価する基盤を提供する。
% 論文中の課題意識: 既存データセット(例: PASCAL/ ImageNet)では,文脈を含む複雑な日常シーンでの検出・分割を十分に評価しにくい点を課題として,より現実的なシーンと密なアノテーションを整備する。
% 本研究での位置づけ: 本研究で扱う自然言語説明・概念ラベルの評価や例示において,COCO由来の画像キャプション/概念資源を参照する場合のデータセット基盤として位置づける(必要に応じて)。
% 特筆: インスタンスセグメンテーションを含む多様なアノテーションと,PASCAL等との統計比較・ベースライン提示により,ベンチマークとしての設計意図を明確化している。

\bibitem{koh2020concept}
P.~W. Koh, T.~Nguyen, Y.~S. Tang, S.~Mussmann, E.~Pierson, B.~Kim, and P.~Liang:
``Concept Bottleneck Models,''
in \textit{Proceedings of the 37th International Conference on Machine Learning (ICML 2020)}, vol.~119, pp.~5338--5348, 2020, arXiv:2007.04612, doi:10.48550/arXiv.2007.04612.
% 研究概要: 入力→概念(人間可読な属性)→ラベルの二段階で予測する Concept Bottleneck Models (CBM) を提案し,概念予測値を編集して最終予測へ反映する「概念介入」を可能にする。
% 論文中の課題意識: 端から端まで学習する高性能モデルは,高水準概念での操作・デバッグが難しく,説明や介入の粒度が人間の理解単位と合わない点を課題とする。
% 本研究での位置づけ: 本研究は概念ボトルネックを設けるのではなく集合間差分を言語化するが,「中間の説明単位(概念/対比因子)を介して人間が介入可能にする」という解釈可能性の設計思想として関連づけられる。
% 特筆: テスト時に概念を人手で訂正できると精度が改善することを示し,人間-モデル協調の評価軸(介入可能性)を明確化している。

\bibitem{radford2021learning}
A.~Radford, J.~W. Kim, C.~Hallacy, A.~Ramesh, G.~Goh, S.~Agarwal, G.~Sastry, A.~Askell, P.~Mishkin, J.~Clark, G.~Krueger, and I.~Sutskever:
``Learning Transferable Visual Models From Natural Language Supervision,''
in \textit{Proceedings of the 38th International Conference on Machine Learning (ICML 2021)}, vol.~139, pp.~8748--8763, 2021, arXiv:2103.00020, doi:10.48550/arXiv.2103.00020.
% 研究概要: 大規模な画像-テキスト対に対して,キャプションと画像の対応を当てる対照学習(CLIP)により視覚表現を事前学習し,自然言語プロンプトを介したゼロショット転移で多様な下流タスクへ適用できることを示す。
% 論文中の課題意識: 固定カテゴリの教師あり学習に依存する従来CVは,新しい概念に対して追加ラベルが必要で汎用性が制限される点を課題とし,より広い言語 supervision による一般化を狙う。
% 本研究での位置づけ: 本研究はテキスト集合差分の言語化が中心だが,「自然言語で概念を参照し,表現空間上で評価・転移する」枠組みは,概念ベース手法やラベル語彙の扱いを議論する際の基盤として関連づけられる。
% 特筆: ImageNetを含む多数ベンチマークでゼロショット性能を報告し,プロンプト設計が性能に影響することも示唆する(モデルを固定したまま言語入力でタスク定義を切替)。


\bibitem{reimers2019sentence}
N.~Reimers and I.~Gurevych:
``Sentence-BERT: Sentence Embeddings using Siamese BERT-Networks,''
in \textit{Proceedings of the 2019 Conference on Empirical Methods in Natural Language Processing and the 9th International Joint Conference on Natural Language Processing (EMNLP-IJCNLP)}, Hong Kong, China, November 3--7, 2019, pp.~3982--3992, doi:10.18653/v1/D19-1410.
% 研究概要: BERTをクロスエンコーダ(文ペア同時入力)ではなくSiamese/Triplet構造で微調整し,単文から固定長の文埋め込みを生成する Sentence-BERT (SBERT) を提案。コサイン類似度での検索・クラスタリングを高速に実行できる。
% 論文中の課題意識: BERTの文ペア比較は高精度だが,全組合せ比較が必要な類似検索・クラスタリングでは計算量が爆発し実用的でない点,および単純な[CLS]/平均プーリングが有用な文埋め込みになりにくい点を課題とする。
% 本研究での位置づけ: 本研究の対比因子ラベル生成ではテキスト集合の分割・近傍検索・代表例抽出などで文埋め込みを用いる局面があり得るため,SBERTは「高速な意味類似度計算の標準的基盤」として位置づけられる。
% 特筆: 10,000文の類似ペア探索をクロスエンコーダの数十時間規模から,埋め込み計算+近似探索により秒〜ミリ秒オーダへ落とせる点を明確に示し,STS等で性能も維持/改善を報告する。

\bibitem{bird2009natural}
S.~Bird, E.~Klein, and E.~Loper:
\textit{Natural Language Processing with Python: Analyzing Text with the Natural Language Toolkit},
O'Reilly Media, 2009, ISBN:9780596516499. Available at \url{https://www.nltk.org/book/} (accessed: 2025-12-13).
% 研究概要: NLTKを用いた自然言語処理の入門書として,トークナイズ・品詞タグ付け・構文解析・情報抽出・コーパス/WordNet利用など,典型的NLP処理をPythonコードと例題で体系化する。
% 本(教材)が扱う課題意識: NLPの基本処理は理論だけでなく「再現可能な実装」として学ぶ必要がある一方,実データ(コーパス)とアルゴリズムを結びつけたハンズオン教材が不足しがちである点を背景に,標準ライブラリ(NLTK)と公開データで学べる形にまとめる。
% 本研究での位置づけ: 本研究の実験実装では,テキスト前処理・特徴抽出・データ確認などの基礎作業が不可避であり,その実装パターン(コーパス操作・前処理)を参照する実務的な基盤文献として位置づける。
% 特筆: オンラインで公開された付録/実行例(NLTK Bookサイト)により,環境構築後すぐ試せる教材として再利用性が高い。

\bibitem{patricio2025cbvlm}
C.~Patr{\'\i}cio, I.~Rio-Torto, J.~S. Cardoso, L.~F. Teixeira, and J.~C. Neves:
``CBVLM: Training-free explainable concept-based Large Vision Language Models for medical image classification,''
arXiv preprint arXiv:2501.12266, 2025, doi:10.48550/arXiv.2501.12266. (Accepted for publication in \textit{Computers in Biology and Medicine}.)
% 研究概要: 医用画像分類に対して,Concept Bottleneck Model (CBM) の「概念→診断」という説明様式を,学習なし(training-free)でLVLM/LLMにより実現する枠組み CBVLM を提案。概念ごとにLVLMへ有無判定を問い合わせ,得られた概念列に基づいて最終診断を生成/選択させることで,診断を概念に接地した説明として提示する。
% 論文中の課題意識: 医用領域では(1)アノテーション不足と(2)ブラックボックス性が導入障壁となる。CBMは解釈性を与えるが概念注釈コストが高く,新概念追加のたびに再学習が必要という運用上の負担がある。
% 手法: (1) 概念検出: 画像と概念テキストの照合/質問により概念の有無を推定,(2) 診断: 予測概念をプロンプトに埋め込み診断を出力。加えて retrieval による in-context example 選択でfew-shot性能を補強する。
% 本研究での位置づけ: 本研究はテキスト集合差分の言語化が主対象だが,「説明単位(概念/因子)を先に推定し,最終判断(ラベル生成/診断)をその説明単位に接地して出力する」という設計思想は,対比因子ラベル生成の根拠付け・人手介入可能性の議論と接続できる。
% 特筆: training-freeで概念の追加が容易であり,少数例のICL+retrievalで注釈コストを下げつつ,複数データセット・複数LVLMでCBMや教師あり法を上回ると報告する。


\bibitem{deng2019annotation}
Y.~Deng, Y.~Sun, Y.~Zhu, Y.~Xu, Q.~Yang, S.~Zhang, M.~Zhu, J.~Sun, W.~Zhao, X.~Zhou, and K.~Yuan:
``Efforts estimation of doctors annotating medical image,''
arXiv preprint arXiv:1901.02355, 2019, doi:10.48550/arXiv.1901.02355.
% 研究概要: 医用画像アノテーションに要する医師の作業量(effort)を定量化する指標を提案し,能動学習によるサンプル選択とU-Net系のセグメンテーションを組み合わせて,「枚数削減」だけでなく「実際の労力削減」を評価対象に据える。
% 論文中の課題意識: 医用画像の高精度アノテーションは専門医の時間を要し高コストである。既存の能動学習は候補数削減に寄りがちで,少数でも時間がかかる現実の労力を直接扱えていない点を課題とする。
% 本研究での位置づけ: 本研究の対比因子ラベル生成でも,人手作業(正解説明/概念の整備・検証)コストがボトルネックになり得るため,「コストを対象化して評価する」という問題設定の近さから背景文献として参照できる。
% 特筆: セグメンテーションを例に,候補数の削減率と労力削減率が一致しないことを示し,労力を明示的に最適化/報告する重要性を強調する。
% 実在確認(再調査): arXiv:1901.02355 として公開(上記URL)。arXivページ上に journal reference の記載はなく,IEEE Xplore 等でも同一タイトルの会議録を確認できなかったため,確実な arXiv 版で記載する。

\bibitem{nan2025deep}
T.~Nan, S.~Zheng, et al.:
``Deep learning quantifies pathologists' visual patterns for whole slide image diagnosis,''
\textit{Nature Communications}, vol.~16, 5493, 2025, doi:10.1038/s41467-025-60307-1.
% 研究概要: 病理医のWSI閲覧時の視線(eye-tracking)から視覚的レビュー・パターンを収集し,それを手がかりに病理医の専門知を復号して診断する深層学習システムPEAN(Pathology Expertise Acquisition Network)を提案。ピクセル単位の手動アノテーションに代わる低負担な知識獲得として位置づけ,5881件のWSI・皮膚病変5分類でAUC 0.992,診断精度96.3%を報告。視線ベースの収集によりアノテーション時間は手動の約4%まで削減可能とする(要旨より)。
% 論文中の課題意識: 高精度なWSI診断モデルは病理医によるピクセル単位アノテーションに依存しがちだが,WSIは超高解像度でアノテーションコストが極めて高く,大規模データセット構築の障壁になる。弱教師ありはコストを下げる一方で,病理医の事前知識による誘導が弱く,頑健性・解釈性の面で臨床要件を満たしにくい。
% 本研究での位置づけ: 本研究はテキスト集合A/Bの差分を対比因子ラベルとして言語化する立場だが,「専門家の暗黙知(どこを見るか)を低コストで外部化し,モデルの推論過程を人間の判断過程に近づける」という観点は,対比因子ラベルの根拠付け・説明可能性の議論(人間にとって妥当な注目領域/差分の抽出)と接続できる。
% 引用箇所での使われ方: 関連研究(人手ラベリングへの依存)で,専門家アノテーションの高コスト性を補足する具体例として引用(要旨では視線データにより手動の約4%へ削減可能と主張)。


\bibitem{tseng2025expertllm}
Y.-M.~Tseng, W.-L.~Chen, C.-C.~Chen, and H.-H.~Chen:
``Evaluating Large Language Models as Expert Annotators,''
arXiv preprint arXiv:2508.07827, 2025. (Accepted to COLM 2025).
% 研究概要: 一般領域ではLLMを人間アノテータの代替として使う動きがある一方で,専門知識が必要な領域での有効性は十分検証されていないという問題意識のもと,finance/biomedicine/law の3領域で,単体LLMとマルチエージェント(複数LLMの議論による合意形成)を比較評価する。推論時手法(CoT,self-consistency)や推論モデル(例:o3-mini)も含めて検証する(要旨より)。
% 論文中の課題意識: 専門領域アノテーションは高コストであり,LLM代替が期待されるが,ベンチマーク上の「専門性」と実タスクの専門知識要求は一致しない可能性がある。推論時の長いCoTが専門領域アノテーションに必ずしも効かず,議論環境でもモデルが頑固に初期判断を変えない等の挙動が出る点を課題として示す。
% 本研究での位置づけ: 本研究はLLMに対比因子ラベル生成を担わせ,評価にもLLMを用いるため,「LLMを(準)アノテータとして使う際の限界・安定性(推論時手法や推論モデルの効果が限定的)」という知見は,評価設計(人手ゴールドとの併用,条件統制,エラー分析)や議論による改善可能性の検討に直結する。
% 引用箇所での使われ方: 2025-12-13時点で本文(論文/chapters/)内に \cite{tseng2025expertllm} は存在せず,参考文献リストのみ。関連研究で使う場合は,「専門領域でLLMを専門家アノテータ代替として使う際,CoT等が効きにくい/推論モデル優位が明確でない」点を根拠に,LLM評価・自動ラベリングの限界を述べる位置が自然。

\bibitem{zhong2022describing}
R.~Zhong, C.~Snell, D.~Klein, and J.~Steinhardt:
``Describing Differences between Text Distributions with Natural Language,''
in \textit{Proceedings of the 39th International Conference on Machine Learning (ICML 2022)}, 
vol.~162, pp.~27099--27116, 2022.
% 研究概要: 2つのテキスト分布の違いを自然言語で説明するタスクを定式化。GPT-3をファインチューニングして候補説明を生成し、再ランキングにより最適な説明を選択するProposer-Verifierフレームワークを提案。54の実世界バイナリ分類タスクで評価し、人間アノテーションとの76%の類似度を達成。
% 本研究での位置づけ: ブラックボックスなLLMを用いたテキスト分布間の差分説明の先駆的研究として参照。本研究の対比因子生成タスクとの違い(ABSAスキーマとの意味的一致評価の欠如)を説明する際の代表例として用いる。

\bibitem{dunlap2024visdiff}
L.~Dunlap, Y.~Zhang, X.~Wang, R.~Zhong, T.~Darrell, J.~Steinhardt, J.~E. Gonzalez, and S.~Yeung-Levy:
``Describing Differences in Image Sets with Natural Language,''
in \textit{Proceedings of the IEEE/CVF Conference on Computer Vision and Pattern Recognition (CVPR 2024)}, Seattle, WA, USA, June 17--21, 2024, pp.~17952--17961, arXiv:2312.02974.
% 研究概要: 画像セット間の差分を自然言語で説明するSet Difference Captioningタスクを提案。VisDiffは2段階アプローチを用い、第1段階で画像キャプションとLLMを用いて候補説明を生成、第2段階でCLIPを用いて再ランキングを行う。VisDiffBenchデータセット(187組の画像セットペア)で評価。
% 本研究での位置づけ: Zhong et al. (2022)の画像領域への拡張として参照。本研究のテキスト集合ベース・ABSAスキーマ評価との違いを説明する際の代表例として用いる。

\bibitem{pham2024topicgpt}
C.~M. Pham, A.~Hoyle, S.~Sun, P.~Resnik, and M.~Iyyer:
``TopicGPT: A Prompt-based Topic Modeling Framework,''
in \textit{Proceedings of the 2024 Conference of the North American Chapter of the Association for Computational Linguistics: Human Language Technologies (NAACL-HLT 2024)}, Mexico City, Mexico, June 16--21, 2024, pp.~2956--2984.
% 研究概要: LLMを用いたトピックモデリングフレームワークTopicGPTを提案。従来のLDAのような曖昧な単語バッグ表現ではなく、自然言語ラベルと自由形式の説明を持つトピックを生成し、解釈可能性を向上。ユーザーが制約を指定し、モデルを再訓練せずにトピックを修正可能。Wikipediaトピックに対する調和平均純度0.74を達成。
% 本研究での位置づけ: LLMを用いたクラスタ・トピックラベリングの代表例として参照。本研究の対比的概念命名タスクとの違い(差分説明の欠如、ABSAスキーマ評価の欠如)を説明する際の代表例として用いる。

\bibitem{lam2024lloom}
M.~S. Lam, J.~Teoh, J.~Landay, J.~Heer, and M.~S. Bernstein:
``Concept Induction: Analyzing Unstructured Text with High-Level Concepts Using LLooM,''
in \textit{Proceedings of the 2024 CHI Conference on Human Factors in Computing Systems (CHI '24)}, Honolulu, HI, USA, May 11--16, 2024, Art.~No.~933, doi:10.1145/3613904.3642830.
% 研究概要: 非構造化テキストを,明示的な包含基準を伴う高レベル概念(concept)として抽出・整理する concept induction を提案し,LLMを用いてサンプルテキストの合成と概念の一般化を反復するLLooMアルゴリズムと,分析を支援する対話型ツールLLooM Workbenchを提示する(要旨より)。
% 論文中の課題意識: トピックモデルやクラスタリングは低レベルのキーワードに寄りやすく解釈作業が重い。そこで,理論駆動の分析に使える形で,高レベル概念を自然言語の包含基準として明示し,データ被覆も確保しながら抽出したい,という課題を扱う(要旨より)。
% 本研究での位置づけ: LLMを用いた高レベル概念帰納の代表例として参照。本研究の対比的概念命名タスクとの違い(単一コーパスからの概念帰納 vs 対比的テキスト集合間の差分説明)を説明する際の代表例として用いる。
% 引用箇所での使われ方: 序論で,LLMを用いたクラスタ/トピックの命名・概念帰納の代表例として関連研究列挙の中で引用(`01_introduction.tex`)。

\bibitem{anthropic2024monosemantic}
A.~Templeton, T.~Conerly, J.~Marcus, J.~Lindsey, T.~Bricken, B.~Chen, A.~Pearce, et al.:
``Scaling Monosemanticity: Extracting Interpretable Features from Claude 3 Sonnet,''
\textit{Transformer Circuits Thread}, 2024. Available at \url{https://transformer-circuits.pub/2024/scaling-monosemanticity/} (accessed: 2025-12-13).
% 研究概要: Claude 3 Sonnetの中間層活性をSparse Autoencoder (SAE)を用いて分解し、より解釈可能なコンポーネントに変換する手法を提案。SAEにより、モデル内から数百万の解釈可能な特徴を抽出し、言語やモダリティを超えた抽象的概念に対応する特徴を特定。Golden Gate Bridgeのような具体的実体から、「裏切り」「自己認識」などの高度に抽象的な概念までを含む特徴を発見。
% 本研究での位置づけ: LLM内部活性ベースの自動解釈可能性の代表例として参照。本研究のブラックボックス設定(内部活性アクセス不要)との違いを説明する際の代表例として用いる。

\bibitem{spearman1904association}
C.~Spearman:
``The Proof and Measurement of Association between Two Things,''
\textit{The American Journal of Psychology}, vol.~15, no.~1, pp.~72--101, 1904, doi:10.2307/1412159.
% 実在確認: The American Journal of Psychology 15(1):72-101 (1904). JSTOR stable id 1412159(https://www.jstor.org/stable/1412159)。
% 研究概要: 2変数の連関を順位にもとづく相関として定式化し,Spearmanの順位相関係数(rho)として知られる指標の導入・議論を行う古典。
% 本研究での位置づけ: 順位(ランキング)間の一致度を測る非パラメトリック相関として,結果の順位比較・相関分析の根拠文献になりうる。

\bibitem{friedman1937use}
M.~Friedman:
``The Use of Ranks to Avoid the Assumption of Normality Implicit in the Analysis of Variance,''
\textit{Journal of the American Statistical Association}, vol.~32, no.~200, pp.~675--701, 1937, doi:10.1080/01621459.1937.10503522.
% 実在確認: Journal of the American Statistical Association 32(200):675-701 (1937). DOI:10.1080/01621459.1937.10503522(doi.orgからTaylor\&Francisへ解決、Crossrefに登録あり)。
% 研究概要: 分散分析で暗黙に仮定される正規性等を避けるため,観測値を順位に置き換えて検定する枠組みを整理(いわゆるFriedman検定の古典的原典として参照される)。
% 本研究での位置づけ: 手法比較の順位データに基づく差の検定(ノンパラメトリック)を述べる際の基礎文献になりうる。

\bibitem{holm1979simple}
S.~Holm:
``A Simple Sequentially Rejective Multiple Test Procedure,''
\textit{Scandinavian Journal of Statistics}, vol.~6, no.~2, pp.~65--70, 1979.
% 実在確認: Scandinavian Journal of Statistics 6(2):65-70 (1979). Semantic Scholar掲載情報より確認(https://www.semanticscholar.org/paper/b0ebbcf713b3ddf3f94325bc58dc39ff76fdc412)。
% 研究概要: 多重検定で家族誤差率(FWER)を制御する逐次棄却(step-down)手順を提案(Holm--Bonferroni法)。Bonferroniより一様に強力(棄却しやすい)な補正として標準的に参照される。
% 本研究での位置づけ: 複数条件/複数モデルの比較で p 値を多数扱う際の多重比較補正(Holm補正)の根拠文献になりうる。

\bibitem{wilcoxon1945individual}
F.~Wilcoxon:
``Individual Comparisons by Ranking Methods,''
\textit{Biometrics Bulletin}, vol.~1, no.~6, pp.~80--83, 1945.
% 実在確認: Biometrics Bulletin 1(6):80-83 (1945). Crossref登録あり。DOI:10.2307/3001968(doi.orgからJSTOR stableへ解決)。
% 研究概要: 対応のある2群比較などを順位にもとづいて行うノンパラメトリック検定(Wilcoxonの順位和/符号付順位検定として定着する手法)の原典。
% 本研究での位置づけ: 正規性を仮定しにくいスコア分布の比較で,順位ベースの検定手法を参照する際の基礎文献になりうる。

\bibitem{demsar2006statistical}
J.~Dem\v{s}ar:
``Statistical Comparisons of Classifiers over Multiple Data Sets,''
\textit{Journal of Machine Learning Research}, vol.~7, pp.~1--30, 2006.
Available at \url{https://www.jmlr.org/papers/v7/demsar06a.html} (accessed: 2025-12-16).
% 実在確認: Journal of Machine Learning Research (JMLR) vol.7, pp.1-30 (2006). JMLR公式ページ(https://www.jmlr.org/papers/v7/demsar06a.html)で公開を確認。
% 研究概要: 複数データセット上で複数分類器を比較する統計的手順を整理し,平均順位にもとづくFriedman検定と事後比較(Nemenyi/Bonferroni-Dunn等)の推奨・注意点をまとめる定番文献。
% 本研究での位置づけ: 複数条件(モデル/設定)×複数データセットの性能比較における統計検定(順位ベース)と事後比較の根拠文献になりうる。

\bibitem{virtanen2020scipy}
P.~Virtanen et al.:
``SciPy 1.0: Fundamental Algorithms for Scientific Computing in Python,''
\textit{Nature Methods}, vol.~17, pp.~261--272, 2020, doi:10.1038/s41592-019-0686-2.
% 実在確認: Nature Methods 17:261-272 (2020). DOI:10.1038/s41592-019-0686-2(doi.orgからnature.comへ解決、Crossrefに書誌登録あり)。
% 研究概要: Python科学計算ライブラリSciPyの主要アルゴリズム群(最適化・線形代数・統計・信号処理等)とソフトウェア基盤、品質管理・貢献体制を総覧するソフトウェア論文。
% 本研究での位置づけ: 統計検定や数値計算(例: Friedman検定、順位処理、最適化等)をSciPyで実装・再現する際の基盤文献になりうる。

\end{thebibliography}



\end{document}
